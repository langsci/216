\documentclass[output=paper]{LSP/langsci} 
\author{
Nerida Jarkey\affiliation{School of {L}anguages and {C}ultures, University of Sydney}
}
\title{Bridging constructions in narrative texts in White Hmong (Hmong-Mien)}
%epigram
\abstract{This chapter examines bridging constructions in narrative texts in White Hmong (Hmong-Mien, Laos). Bridging constructions occur in all the texts examined for the study, with frequency and type of construction varying according to narrator and text type. Recapitulative linkage is far more common than either summary linkage, which is limited to first-person narratives and reported speech, or mixed linkage, which serves to summarize direct quotations in oral and written texts with a more literary character. In terms of function, the analysis shows that bridging constructions in White Hmong narrative texts work cohesively, linking one unit in the event line of the narrative to the next and thus serving to progress the main sequence of events. The event described by the bridging construction is constructed as a salient point in the event line, and becomes the base from which the next unit in the event line of the narrative proceeds.}
\maketitle
%-------------------------

\begin{document}\label{ch:5}

\section{Introduction} 
\label{Jasec:Introduction}

\subsection{White Hmong language}
\ili{White Hmong} (ISO code: mww) is a language of the Hmong-Mien (Miao-Yao) family, mainly spoken in the mountainous regions of northern Vietnam, Laos, and Thailand, and of southern China, as well as in some diasporic communities. \ili{White Hmong} is an analytic, isolating language; most words are monosyllabic, although compounding and borrowing result in some multisyllabic words. Syllable structure is basically open and every syllable carries one of seven phonemic tones, represented by syllable-final consonant letters in the orthography used here. Some consonants are quite complex, including combinations of features such as pre-nasalisation with both lateral and aspirated release.

Alignment in \ili{White Hmong} is nominative-accusative, and the syntactic function of core arguments is coded by constituent order: generally AVO for transitive clauses and SV for intransitive clauses. Presentative existentials are verb initial and copula clauses are CS copula CC. Topical elements can be fronted and ellipsis of arguments can occur when referents can easily be retrieved through the linguistic or extra-linguistic context. While head modifier order is most common, within the noun phrase some elements, including possessives, numerals and numeral classifiers, precede the head.

Like many other languages of Mainland Southeast Asia, \ili{White Hmong} is rich in \isi{serial verb constructions} (SVCs). These involve two or more distinct verbs, linked together in a single clause by virtue of the fact that they share one or more core arguments as well as all grammatical categories. Thus it is very common in \ili{White Hmong} for a single event to be expressed by multiple verbs, none of which is subordinate to any other \citep[][76--110]{jarkey15}. This phenomenon, along with the ellipsis of arguments, mentioned above, is illustrated in many of the examples in this chapter, such as in the sequence of verbs in (\ref{Jaex00}): 


\begin{exe}
\ex \label{Jaex00}
\gll muab coj mus los tas\\
take take.along go bury finish\\
\glt \sqt{after (they) took (him) (and) carried (him) away (and) buried (him),...}\\
\end{exe}

\subsection{Chapter overview}
%%\begin{styleStandard}
This chapter examines bridging constructions in \ili{White Hmong} \isi{narrative} discourse. In accordance with \citeauthor{guerin18} in this volume, a bridging construction is viewed here as a discourse \isi{cohesion} strategy linking two discourse units, often though not always immediately adjacent to one another. The \isi{final clause} of the first unit is referred to as the ``reference clause'' (underlined throughout) and initial clause of the second unit, as the ``bridging clause'' (bolded throughout). The bridging clause refers back to the reference clause by recapitulation or \isi{anaphora}.

The corpus for this study, comprising six \isi{narrative} texts (approximately 15,000 words in total), is presented in \refsec{JaData}. Section \ref{Jasec:Characteristics} deals with the topics of the frequency (\refsec{JaFrequency}), position (\refsec{JaPosition}), form (\refsec{JaForm}), and types of linkage (\refsec{Jatypes}) that occur in bridging constructions in this corpus. Section \ref{JaFunctions} examines their function. All bridging constructions in the data work cohesively, linking one unit in the \isi{event line} of the \isi{narrative} to the next \citep[][14--17]{longacre83}. This often involves a change in aspect between the reference clause and the bridging clause, which contributes to constructing that point as a salient one in the \isi{narrative} progression and highlights its function as a pivot between the preceding and following discourse units (\refsec{JaChange}). In other cases, the bridging construction simply serves to bring the \isi{narrative} back to the \isi{event line} after a brief digression (\refsec{JaReturn}).
%%\end{styleStandard}
%
%
\subsection{Data sources}
\label{JaData}
%\begin{styleStandard}
All of the six \isi{narrative} texts examined for this study are in linear \isi{narrative} form --four transcribed from recordings of oral narratives 
\citep{fuller85,johnson92} and two produced in written form from the outset \citep{vang90}. The texts and text types are shown in Table \ref{Jatable1}. 
%\end{styleStandard}


% Please add the following required packages to your document preamble:
% \usepackage{multirow}
\begin{table}[]
\small
\caption{Texts and text types}
\label{Jatable1}
\begin{tabular}{llll}
  \lsptoprule
Mode & Person                  & Narrative type                     & Source text                         \\
\midrule
 Oral & First  					& Personal  						& `Kee's story'       \\
   &                     &    account                            & \citep[][Appendix B]{fuller85}                                   \\
          &                                    &        &                           \\                    
           & Third 					& Traditional 						& `The beginning of the world'                            \\
            &            &  myth                                   & \citep[][Chapter 1]{johnson92}                                        \\
             &           &                                    & `The story of Ms Fine Flower I'                                   \\
            &            &                                    & \citep[][Chapter 5]{johnson92}                                  \\
           &             &                                   & `The story of Ms Fine Flower II'                                   \\
           &             &                                    & \citep[][Chapter 6]{johnson92}                                       \\
                           &                                    &        &                           \\                                           
 Written  &                     &  Semi-historical                                  & `The beginning'         \\
            &            &  account with                                  & \citep[][Chapter 1]{vang90}                                      \\
            &            &  legendary elements                                  & `God sends the Pahawh'                                      \\
            &            &                                    & \citep[][Chapter 2]{vang90}                                    \\
 \lspbottomrule
\end{tabular}
\end{table}



The first text shown in Table \ref{Jatable1} -- `Kee’s Story' -- is a first-person oral account of the narrator’s escape from Laos in 1975, after the end of the war \citep[][225--235]{fuller85}. The next three -- `The Beginning of the World' and two versions of `The Story of Ms Fine Flower' told by two different narrators -- are traditional \ili{White Hmong} myths \citep[][3--13, 120--140, 161--168]{johnson92}. These stories are told in the third person, but contain some first-person components in reported speech. The final two texts are also in the third person with some first-person, reported speech components. These are accounts of the life and teachings of a messianic figure, Shong Lue Yang, who was active in northern Laos from 1959 until his assassination in 1971. The story was written by two of his disciples using the Hmong writing system Shong Lue Yang himself had developed \citep[][11--37]{vang90}.

%
\section{Characteristics of bridging constructions in narrative texts} 
\label{Jasec:Characteristics}

%\begin{styleStandard}
Having introduced the language and data in \refsec{Jasec:Introduction} above, this section discusses the frequency, position, form, and types of bridging constructions found in the texts examined.
%\end{styleStandard}
%
\subsection{Frequency}
\label{JaFrequency}
%\subsection[]{Frequency}
%\begin{styleStandard}
On average, across all the texts used as data for this study, one bridging construction occurs roughly every 37 clauses. Although this gives a general idea of frequency, it must be noted that this figure is not particularly robust. This is due not only to the limited amount of data examined, but also to the fact that it is often quite challenging to determine the boundaries of a single clause in \ili{White Hmong}. A number of factors contribute to this challenge, including the range of paratactic strategies involving simple juxtaposition that occur (in addition to verb serialisation), as well as the frequent linkage of multiple \isi{serial verb constructions} \citep[][183--186, 237--241]{jarkey15}.

One of the texts examined -- `The Legend of Ms Fine Flower I' -- stands out from the others in that very few bridging constructions appear in it: only one in approximately 174 clauses. A second version of this same traditional myth, told by a different narrator, was also examined, and was found to use bridging constructions far more frequently: around one in every 28 clauses. This shows that the low frequency in some texts cannot be attributed to \isi{narrative} mode, person, or type, and probably relates simply to the style of the narrator.

%Bridging constructions appear more than twice as frequently in the first-person \isi{narrative} text examined (about one in every 19 clauses) than in the third person narratives (on average one in every 42 clauses). This may again be due simply to a difference between individual narrators. However, while only \isi{recapitulative linkage} occurs in the third person texts, both recapitulative and \isi{summary linkage} occur in the first-person text, and this may be partially responsible for this higher frequency. The discrepancy disappears if we look only at \isi{recapitulative linkage} in the first-person text: one in every 42 clauses.
%\end{styleStandard}
%
\subsection{Position}
\label{JaPosition}
%\begin{styleStandard}
To understand more about the functions of bridging constructions in \isi{narrative} texts, it will help to begin by looking at the positions in which they predominantly occur. Bridging constructions in the texts examined occur most commonly at the boundary between discourse episodes, at what might be thought of as major boundaries (``chapters'') or minor boundaries (``paragraphs'') (see \citeauthor{guerin18} in this volume). The only cases in which they appear other than at the junction of discourse episodes is where they are used simply to bring the \isi{narrative} back to the \isi{event line} after a diversion containing \isi{supportive material}. This minor type will be discussed later, in (\refsec{JaReturn}). Here the focus is on their most common position, at the boundary of discourse episodes.
 
The extract below from `The Story of Ms Fine Flower I' comes at the end of a series of paragraphs describing a plot to kill the character Mr Sultry Toad by having a snake bite him. We pick up the story in clause (\ref{Jaex:01a}), just after the snake has bitten his foot four times:
 

\begin{exe}
\ex \label{Jaex:01ae}
\begin{xlist}
\ex \label{Jaex:01a}
\gll Ces Nraug.Kub.Kaws mob{\textasciitilde}mob ko.taw\\
and.then Mr.Sultry.Toad \textsc{redup}{\textasciitilde}be.hurt foot\\
\glt \sqt{And then Mr Sultry Toad’s foot really hurt.}\\
\ex \label{Jaex:01b}
\gll ces \underline{\smash{Nraug.Kub.Kaws}} \underline{\smash{tuag}} \underline{lawm} \underline{lau}.\\
and.then Mr.Sutry.Toad die \textsc{prf} \textsc{ip}\\ 
\glt \sqt{and then Mr Sultry Toad died.}\\
\ex \label{Jaex:01c}
\gll \textbf{Nraug.Kub.Kaws} \textbf{tuag} \textbf{tas}, …\\		
Mr.Sultry.Toad die finish \\ 
\glt \sqt{After Mr Sultry Toad died, …}\\
\ex \label{Jaex:01d}
\gll muab coj mus los tas, \\		           
take take.along go bury finish,\\
\glt \sqt{(and) after (they) took (him) away (and) buried (him),}\\
\ex \label{Jaex:01e}
\gll ces Txiv.Nrau.Ntsuag thiaj.li mus coj Niam.Nkauj.Zuag.Paj rov los.\\     	      
and.then The.Young.Orphan so.then go take.along Ms.Fine.Flower return come.home\\
\glt \sqt{then The Young Orphan came back home, bringing Ms Fine Flower along.} \citep[][140]{johnson92}
\end{xlist}
\end{exe}

\noindent
The major episode concerning the murderous plot culminates in the death of Mr Sultry Toad, described in the reference clause (\ref{Jaex:01b}). The bridging clause (\ref{Jaex:01c}) then serves to pivot the \isi{narrative} to the final episode of the story, introducing the events after Mr Sultry Toad’s death, as the plotters bury him (\ref{Jaex:01d}), return home (\ref{Jaex:01e}), subsequently taking up their life together. 
 
In example (\ref{Jaex:01ae}), the bridging construction brings a relatively lengthy discourse episode, a whole series of paragraphs or a ``chapter'', to a close and creates a link to the next episode. In other cases, however, a bridging construction serves to introduce what might be thought of as simply a new minor episode, or ``paragraph''. In the myth of `The Beginning of the World', the first man and woman on earth have suffered the loss of their first crop, swept away by a windstorm. A new major episode begins:

\begin{exe}
\ex \label{Jaex:02af}
\begin{xlist}
\ex \label{Jaex:02a}
\gll Txiv.Nraug.Luj.Tub thiab Niam.Nkauj.Ntxhi.Chiv nyob{\textasciitilde}nyob\\
     Master.Lu.Tu       and   Ms.Ntxi.Chi \textsc{redup}{\textasciitilde}live\\
\glt \sqt{Master Lu Tu and Ms Ntxi Chi lived on}\\
\ex \label{Jaex:02b}
\gll ces ua.ciav ya mus poob rau puag nram kwj.ha\\
and.then how.is.it fly go fall to yonder place.down valley\\ 
\glt \sqt{and then – how can it be! – (the grains) flew way off (and) fell in yonder valley}\\
\ex \label{Jaex:02c}
\gll ces   pob.kws  xya  nplooj laus    txaus\\		
and.then  corn seven  leaf  be(come).old be.sufficient \\ 
\glt \sqt{and then the seven-leaf corn became fully matured}\\
\ex \label{Jaex:02d}
\gll ces \underline{nws} \underline{xub}  \underline{\smash{taug}}  \underline{kev} \underline{los}.\\		           
and.then 3\textsc{sg} initiate follow way come.home\\
\glt \sqt{and then it was the first to follow the path home.}\\
\ex \label{Jaex:02e}
\gll \textbf{Nws} \textbf{taug} \textbf{kev} \textbf{los} \textbf{txog} \\		           
 3\textsc{sg} follow way come.home  arrive\\
\glt \sqt{It followed the path right back home}\\
\ex \label{Jaex:02f}
\gll ces   nws hu  hais tias, ``Niam  thiab txiv,  quib qhov.rooj.''\\     	      
and.then 3\textsc{sg} call say \textsc{comp} mother and father open door\\
\glt \sqt{and then it called out, ``Mother and Father, open the door!''} \citep[][4]{johnson92}
\end{xlist}
\end{exe}

\noindent
After just three clauses of this new major episode (\ref{Jaex:02a}--\ref{Jaex:02c}), a bridging construction (\ref{Jaex:02d}--\ref{Jaex:02e}) is used to draw to a close the minor episode of what happened way off yonder, and to focus in again on the home scene. Here the bridging construction works to link two minor episodes (or ``paragraphs'') within a much longer major episode.
%\end{styleStandard}
%

\subsection{Form}
\label{JaForm}

The reference clause in a bridging construction in \ili{White Hmong} is always a \isi{main clause}. It can exhibit all the properties of a \isi{main clause}, including the expression of illocutionary force, as shown by the final illocutionary particle \textit{lau} in example (\ref{Jaex:01b}), and exclamatory topicalizers, such as \textit{ov} `oh!' in example (\ref{Jaex:08ab}). Another sign of the status of the reference clause as a \isi{main clause} is its ability to be preceded by a coordinating \isi{conjunction}. 
%\end{styleStandard}
%
%
By far the most common coordinating \isi{conjunction} used, not only before reference clauses but in general throughout \isi{narrative} texts, is \textit{ces} `and then'.\footnote{ Another two coordinating conjunctions that appear occasionally before reference clauses in the corpus are \textit{es} `so' in (\ref{Jaex:12ab})  and \textit{tab sis } `but'.} As shown in example (\ref{Jaex:03ad}), this \isi{conjunction} often appears both immediately before the reference clause and immediately after the bridging clause, bracketing the construction as it links one unit to the next in the \isi{narrative} sequence.
%\end{styleStandard}
%

\begin{exe}
\ex \label{Jaex:03ad}
\begin{xlist}
\ex \label{Jaex:03a}
\gll Mus txog tom kev,\\
go  arrive place.over.there road\\
\glt \sqt{(She) got to the road,}\\
\ex \label{Jaex:03b}
\gll \textsc{ces} \underline{txawm} \underline{mus} \underline{ntsib} \underline{\smash{nraug}} \underline{\smash{zaj}}.\\
and.then then go meet young dragon\\ 
\glt \sqt{and then (she) went (and) met a young dragon.}\\
\ex \label{Jaex:03c}
\gll \textbf{Ntsib} \textbf{nraug} \textbf{zaj},\\		
 meet young dragon \\ 
\glt \sqt{(She) met the young dragon}\\
\ex \label{Jaex:03d}
\gll \textsc{ces} nraug zaj txawm hais tias, …\\     	      
and.then  young dragon then  say \textsc{comp}\\
\glt \sqt{and then the young dragon said, …} \citep[][163]{johnson92}
\end{xlist}
\end{exe}



Example (\ref{Jaex:03ad}) also illustrates the fact that the coordinating \isi{conjunction} \textit{ces} is very often accompanied by another type of \isi{conjunction} indicating temporal sequence in relation to the preceding event, such as \textit{txawm} `then', as in (\ref{Jaex:03b}) and (\ref{Jaex:03d}), \textit{thiaj (li) } `so then', and \textit{mam (li) } `then next'. These \isi{sequential} conjunctions appear not before the whole clause, as the coordinating conjunctions do, but rather clause internally, after the subject (if it appears). In (\ref{Jaex:04ac}), we see both the clause external \textit{ces} `and then' and the clause internal \textit{mam li } `then next' in clauses (\ref{Jaex:04a}) and (\ref{Jaex:04c}). 

\begin{exe}
\ex \label{Jaex:04ac}
\begin{xlist}
\ex \label{Jaex:04a}
\gll \textsc{ces} \underline{nws} \underline{\textsc{mam.li}} \underline{tho} \underline{theem} \underline{hauv} \underline{no}.\\
and.then 3\textsc{sg} then.next pierce layer place.underneath this\\
\glt \sqt{And then next he pierced the layer underneath this.}\\
\ex \label{Jaex:04b}
\gll \textbf{Luj.Tub} \textbf{tho} \textbf{theem} \textbf{hauv} \textbf{no} \textbf{to}\\
Lu.Tu  pierce layer  place.underneath  this make.hole\\ 
\glt \sqt{Lu Tu pierced the layer underneath right through}\\
\ex \label{Jaex:04c}
\gll \textsc{ces} ib co coob{\textasciitilde}coob \textsc{mam.li} tawm hauv los.\\     	      
     and.then one \textsc{clf:coll} \textsc{redup}{\textasciitilde}be.many then.next emerge place.underneath come\\
\glt \sqt{and then next a great many (people) came out (from) underneath.} \citep[][12]{johnson92}
\end{xlist}
\end{exe}



While the reference clause is always a full \isi{main clause}, the bridging clause never is. It can be, first, a somewhat reduced \isi{main clause} or, second, a temporal subordinate clause. Both of these clause types freely occur sentence-initially in other contexts in \ili{White Hmong}; they are not restricted to bridging constructions. 
The bridging clauses in examples (\ref{Jaex:03ad}) and (\ref{Jaex:04ac}) are both cases of the first type: reduced main clauses. Clauses like this are reduced in that they cannot contain \isi{topic} markers or outer operators such as illocutionary force, nor the clause-external coordinating conjunctions or clause-internal \isi{sequential} conjunctions such as those so commonly occurring with the clauses that both precede and follow them. The bridging clause in example (\ref{Jaex:05ab}) exemplifies the second type: a temporal subordinate clause, that is, one that indicates the temporal relationship (\textit{when}, \textit{after}, etc.) between the event described by the bridging construction and that described by the following \isi{main clause}. The bridging clause in (\ref{Jaex:05b}) is introduced by the subordinating \isi{conjunction} \textit{thaum } `when'.
 
\begin{exe}
\ex \label{Jaex:05ab}
\begin{xlist}
\ex \label{Jaex:05a}
\gll ces \underline{\smash{nws}} \underline{\smash{poj.niam}} \underline{\smash{thiaj}} \underline{xauv.xeeb} \underline{tau} \underline{ob} \underline{\smash{leeg}} \underline{tub} \underline{ntxaib}.\\
and.then 3\textsc{sg} woman so.then give.birth get two \textsc{clf} son twin\\
\glt \sqt{...and so then his wife gave birth to twin boys.}\\
\ex \label{Jaex:05b}
\gll \textbf{Thaum} \textbf{xauv.xeeb} \textbf{tau} \textbf{nkawd}...\\     	      
    when  give.birth get 3\textsc{du}\\
\glt \sqt{When she had given birth to them...} \citep[][31]{vang90}
\end{xlist}
\end{exe}

Whether in the form of a reduced \isi{main clause} or a subordinate clause, bridging clauses cannot stand alone as independent sentences, and so are never followed by a sentence-final pause (indicated by a full stop in the orthography). No more than a brief, comma-like pause separates them from the following \isi{main clause}, which functions to introduce the next event in the \isi{event line}. Within the bridging construction itself the sentence-final break after the reference clause functions iconically as a signal of a momentary break in the temporal flow of the \isi{narrative}. The \isi{repetition} in the bridging clause reinforces this sense that the \isi{sequential} flow of events has halted briefly, before it takes off again with no more than a minor pause after the bridging clause.
 
\subsection{Types of linkage}
\label{Jatypes}
With only a small number of exceptions, bridging constructions in the \isi{narrative} data examined involve \isi{recapitulative linkage}. Exact recapitulation seems rare; in fact, considerable variation between the reference clause and the bridging clause is the norm. This is discussed and exemplified in \refsec{JaRecapitulative}. 
 
Examples of \isi{summary linkage} are limited to the first-person text (`Kee’s Story') and to reported speech components within third-person narratives, shown in \refsec{JaSummar}. A \isi{mixed linkage} type occurs with speech verbs introducing direct quotations. This is discussed in \refsec{JaMixed}.
 
\subsubsection{Recapitulative linkage}
\label{JaRecapitulative}
Most examples of \isi{recapitulative linkage} found in the data involve one or more than one of the types of variation identified by \citeauthor{guerin18} (in this volume): \isi{modification}, \isi{omission}, \isi{addition}, and \isi{substitution}. Below, each type of simple variation is illustrated in turn.
 
The high frequency of variation between the reference clause and the bridging clause relates to one of the key features of bridging constructions in this language: variation of the aspectual construal of the event described so that it can function as a pivot between the preceding and following discourse units. This is discussed in detail in §3.1. In addition to change in aspect, the examples below show a range of other kinds of variation.
 

\begin{itemize}
\item \textit{Almost exact recapitulation}
\end{itemize}

No example of exact recapitulation, in which the reference clause is simply repeated word-for-word in the bridging clause, occurs in the data. Example (\ref{Jaex:06ac}) came closest.


\begin{exe}
\ex \label{Jaex:06ac}
\begin{xlist}
\ex \label{Jaex:06a}
\gll ces  \underline{\smash{Txiv.Nraug.Ntsuag}} \underline{txawm} \underline{mus} \underline{\smash{pom}} \underline{nkawd}.\\
and.then The.Young.Orphan then go see 3\textsc{du}\\
\glt \sqt{… and The Young Orphan then went to see them.}\\
\ex \label{Jaex:06b}
\gll \textbf{Txiv.Nraug.Ntsuag} \textbf{mus} \textbf{pom} \textbf{nkawd}.\\
The.Young.Orphan go see 3\textsc{du}\\
\glt \sqt{The Young Orphan went to see them}\\
\ex \label{Jaex:06c}
\gll ces Txiv.Nraug.Ntsuag hais tias,...\\     	      
     and.then The.Young.Orphan say \textsc{comp}\\
\glt \sqt{and The Young Orphan said,...} \citep[][161]{johnson92}
\end{xlist}
\end{exe}

\noindent
Here the only difference is the \isi{sequential} \isi{conjunction} \textit{txawm } `then', which appears in the reference clause (\ref{Jaex:06a}), supporting the preceding coordinating \isi{conjunction} \textit{ces} `and then' in anchoring the reference clause in the \isi{sequential} flow of events. As noted above (\refsec{JaForm}) and as seen in (\ref{Jaex:06b}), \isi{sequential} conjunctions do not appear in bridging clauses, which offer a momentary break in this \isi{sequential} flow.

%
\begin{itemize}
\item \textit{Modification}
\end{itemize}

Example (\ref{Jaex:07ac}), illustrating \isi{modification}, is from a story about the first man and woman on the earth, who emerged from a rock fissure and initially survived by cooking the seeds of a magic flower.


\begin{exe}
\ex \label{Jaex:07ac}
\begin{xlist}
\ex \label{Jaex:07a}
\gll ces  \underline{nws} \underline{rauv} \underline{zeb.ntsuam} \underline{xwb}.\\
and.then 3\textsc{sg} burn pieces.of.coal only\\
\glt \sqt{… and he burned only pieces of coal.}\\
\ex \label{Jaex:07b}
\gll \textbf{Luj.Tub} \textbf{nkawd} \textbf{ob.niam.txiv} \textbf{rauv} \textbf{cov}  \textbf{ntawd}.\\
Lu.Tu 3\textsc{du} couple burn \textsc{clf:coll} that\\
\glt \sqt{Lu Tu and his wife burned those}\\
\ex \label{Jaex:07c}
\gll kib lub paj ntawd cov noob noj xwb.\\     	      
     fry \textsc{clf} flower that  \textsc{clf:coll} seed eat only\\
\glt \sqt{(to) fry the seeds of that flower to eat.} \citep[][3]{johnson92}
\end{xlist}
\end{exe}

\noindent
In (\ref{Jaex:07a}), the reference clause refers to the protagonist, \textit{Lu Tu}, with the third singular pronoun \textit{nws}, and to the pieces of coal with the full NP \textit{zeb ntsuam}. These nouns appear in modified form in the bridging clause (\ref{Jaex:07b}), as the full NP \textit{Luj Tub nkawd ob niam txiv } `the Lu Tu couple' and the pronominal phrase \textit{cov ntawd } `those', respectively. 

%
\begin{itemize}
\item \textit{Omission}
\end{itemize}

The bridging clause may represent a considerably reduced recapitulation of the reference clause by virtue of the \isi{omission} of one or more elements.

%
\begin{exe}
\ex \label{Jaex:08ab}
\begin{xlist}
\ex \label{Jaex:08a}
\gll ces  \underline{\smash{cua-daj-cua-dub}} \underline{ov} \underline{txawm} \underline{\smash{nplawm}} \underline{\smash{puag}} \underline{tim} \underline{\smash{qab}} \underline{\smash{ntug}} \underline{\smash{tuaj}}.\\
and.then wind-yellow-wind-black \textsc{ex} then beat long.way place.beyond behind boundary come \\
\glt \sqt{… and then a storm [lit. wind yellow wind black] oh! (it) then came whipping (from) way over the horizon.}\\
\ex \label{Jaex:08b}
\gll \textbf{Cua}  \textbf{nplawm}  \textbf{tuaj} ces,...\\     	      
     wind  beat   come  and.then\\
\glt \sqt{The wind came whipping and then,...} \citep[][4]{johnson92}
\end{xlist}
\end{exe}

\noindent
Here the locative phrase \textit{puag tim qab ntug } `(from) way over the horizon', which appears in the reference clause (\ref{Jaex:08a}), is completely omitted from the bridging clause (\ref{Jaex:08b}), as are the \isi{sequential} \isi{conjunction} \textit{txawm } `then' and the exclamatory particle \textit{ov} (which functions in (\ref{Jaex:08a}) as a topicaliser). Substitution also occurs to further reduce the length of the bridging clause, with the simple noun \textit{cua} `wind' replacing the four-part elaborate expression \textit{cua-daj-cua-dub }(wind-yellow-wind-black) `storm' (\citealt[][233--237]{jarkey15}, \citealt{johns82,mortensen03}).


\begin{itemize}
\item \textit{Addition}
\end{itemize}

While a locative phrase that occurs in the reference clause is omitted in the bridging clause in example (\ref{Jaex:08ab}), a temporal phrase is added in (\ref{Jaex:09ab}).


\begin{exe}
\ex \label{Jaex:09ab}
\begin{xlist}
\ex \label{Jaex:09a}
\gll ces  \underline{\smash{thiaj}} \underline{mam} \underline{xeeb} \underline{nws} \underline{tus} \underline{\smash{poj.niam}} \underline{rau} \underline{ntawm} \underline{nws} \underline{\smash{qhov.chaw}}.\\
and.then so.then  then.next be.born 3\textsc{sg} \textsc{clf} wife to place.nearby 3\textsc{sg} place \\
\glt \sqt{… and so then next his wife was born into his place [i.e., into the rock fissure from which the first man, Lu Tu, had emerged].}\\
\ex \label{Jaex:09b}
\gll \textbf{xeeb} \textbf{nws} \textbf{tus} \textbf{poj.niam} \textbf{rau} \textbf{ntawm} \textbf{nws} \textbf{qhov.chaw} \textbf{puv-hnub-puv-nyoog}  ces...\\     	      
     be.born 3\textsc{sg} \textsc{clf} wife to place.nearby 3\textsc{sg} place be.filled-day-be.filled-age and.then\\
\glt \sqt{His wife was born into his place (until her) time was fulfilled and then...} \citep[][3]{johnson92}
\end{xlist}
\end{exe}

\noindent
The elaborate expression \textit{puv-hnub-puv-hnoog} `fulfil one’s days', not found in the reference clause (\ref{Jaex:09a}), appears in the bridging clause (\ref{Jaex:09b}) to indicate the length of time that the protagonist’s wife remained behind before she followed her husband out to the earth.
 
\begin{itemize}
\item \textit{Substitution}
\end{itemize}
 
In some cases, rather than \isi{modification}, \isi{omission}, or \isi{addition} in the bridging clause, an element of the reference clause is substituted by an alternative in the bridging clause. Example (\ref{Jaex:10ab}) shows this kind of variation:
 
%
%

\begin{exe}
\ex \label{Jaex:10ab}
\begin{xlist}
\ex \label{Jaex:10a}
\gll thiab  \underline{tau} \underline{\smash{nyob}} \underline{tos},\\
and \textsc{pfv} stay wait\\
\glt \sqt{...and (he) stayed (there and) waited,}\\
\ex \label{Jaex:10b}
\gll \textbf{thaum} \textbf{nws} \textbf{tab.tom} \textbf{mus} \textbf{nyob} \textbf{tos} ces...\\     	      
     when 3\textsc{sg} just go stay wait and.then\\
\glt \sqt{(and) when he had just gone to stay (there) and wait, then...} \citep[][28]{vang90}
\end{xlist}
\end{exe}


\noindent
The morpheme \textit{tau}, marking perfective aspect, in the reference clause (\ref{Jaex:10a}) is substituted by the morpheme \textit{tab tom} `just (begin to)', functioning here to mark immediate inceptive aspect, in the bridging clause (\ref{Jaex:10b}). Inceptive aspect is reinforced by the \isi{addition} of the verb \textit{mus}, here meaning `go to do something'.
 
\subsubsection{Summary linkage}
\label{JaSummar}
As explained by \citeauthor{guerin18} (this volume), \isi{summary linkage} involves the use of a summarizing verb (such as a \isi{light verb}) in the bridging clause, which links anaphorically to the reference clause without lexical recapitulation. This kind of linkage occurs in the first-person text, `Kee’s Story', where it is roughly as frequent as \isi{recapitulative linkage}. In the third-person texts, on the other hand, it appears only occasionally, and then only in reported speech (both direct and indirect). This suggests that \isi{summary linkage} may be associated more with unplanned personal \isi{narrative} and conversation than with more literary style, third-person narration (the \isi{narrative} parts of the myths and written accounts examined).
 
Summary linkage is expressed in these texts with the copula verb \textit{yog} `be' followed by the adverbial \textit{li} `like, as' and, optionally, by a demonstrative pronoun, \textit{no} `this' or \textit{ntawd} `that'. This is illustrated from `Kee’s Story' in example (\ref{Jaex:11ac}):
 
\begin{exe}
\ex \label{Jaex:11ac}
\begin{xlist}
\ex \label{Jaex:11a}
\gll \underline{Lub} \underline{\smash{sij.hawm}} \underline{ntawm} \underline{\smash{neeg}} \underline{khiav} \underline{coob}     \underline{heev} \underline{mas}.        \\
\textsc{clf} time that person run be.many very \textsc{ip}\\
\glt \sqt{(At) that time there were very many people fleeing.}\\
\ex \label{Jaex:11b}
\gll   \textbf{Yog} \textbf{li} \textbf{ntawd},\\
\textsc{cop} like that\\
\glt \sqt{That being the case,}\\
\ex \label{Jaex:11c}
\gll lawv  thiaj hais tias ua  peb puas yog neeg nyob nram   tiag.\\     	      
     3\textsc{pl} then say \textsc{comp} do 1\textsc{pl} \textsc{q} \textsc{cop} person live place.down level.place\\
\glt \sqt{they then asked whether we were people (who) lived down (in Vientiane).} \citep[][227]{fuller85}
\end{xlist}
\end{exe}

\noindent
The expression \textit{yog li ntawd} `that being the case' in (\ref{Jaex:11b}) summarizes the information in the reference clause – that there were many people fleeing at the time – to explain why the officials asked the travellers where they came from. The narrator goes on to explain that only travellers who lived in Vientiane were allowed to go there.
 
In (\ref{Jaex:12ab}), a similar expression, \textit{yog li no} `this being the case', is used in an indirect speech report from `The Legend of Ms Fine Flower II':
 
\begin{exe}
\ex \label{Jaex:12ab}
\begin{xlist}
\ex \label{Jaex:12a}
\gll Niam.Nkauj.Zuag.Paj teb tias \underline{\smash{Txiv.Nraug.Ntsuag}} \underline{tsis} \underline{\smash{yuav}} \underline{\smash{Niam.Nkauj.Zuag.Paj}} \underline{es} \underline{\smash{Niam.Nkauj.Zuag.Paj}}     \underline{los} \underline{mus}.        \\
 Ms.Fine.Flower reply \textsc{comp} The.Young.Orphan \textsc{neg} marry Ms.Fine.Flower so Ms.Fine.Flower come  go\\
\glt \sqt{Ms Fine Flower replied that The Young Orphan (would) not marry her so she (had) left.}\\
\ex \label{Jaex:12b}
\gll Ces nraug  zaj txawm tias \textbf{yog} \textbf{li}  \textbf{no}  ces nraug  zaj   yuav   nws  no  ces …\\     	      
     and.then young dragon then \textsc{comp} \textsc{cop} like this and.then young dragon marry 3\textsc{sg} this and.then\\
\glt \sqt{And then the young dragon said that, this being the case, then he (would) marry her, and then …} \citep[][163]{johnson92}
\end{xlist}
\end{exe}


\noindent
In this example the expression, \textit{yog li no} `this being the case' is attributed to the dragon, summarizing the heroine’s explanation of her plight as the basis for his marriage proposal. Here the reference clause and the bridging clause function as a bridging construction within the reported conversation, rather than in the \isi{narrative} text that reports it.

\subsubsection{Mixed linkage}
\label{JaMixed}
There are other examples in the third-person \isi{narrative} parts of the more literary texts (both written and oral) which do not qualify as \isi{summary linkage}, but which are quite similar. They are characterized here as \isi{mixed linkage} because, while the verb of the reference clause is recapitulated in the bridging clause, the remainder of the bridging clause consists only of summarizing, anaphoric elements.

All examples found involve verbs of speech introducing a direct quotation in the reference clause, and it is the quotation only, not the whole of the reference clause, that is summarized anaphorically in the bridging clause. This is exemplified in (\ref{Jaex:13ab}).

\begin{exe}
\ex \label{Jaex:13ab}
\begin{xlist}
\ex \label{Jaex:13a}
\gll Ces \underline{\smash{Luj.Tub}} \underline{\smash{thiaj.li}} \underline{hais} \underline{tias} \underline{\smash{``Yog}}     \underline{\smash{tsaug{\textasciitilde}tsaug.zog}} \underline{thiab} \underline{\smash{nqhis{\textasciitilde}nqhis}} \underline{\smash{nqaij}} \underline{mas} \underline{\smash{yuav.tau}} \underline{rov} \underline{mus}...'' \\
 and.then Lu.Tu so.then say \textsc{comp} \textsc{cop} \textsc{redup}{\textasciitilde}be.sleepy and \textsc{redup}{\textasciitilde}crave  meat \textsc{top} must return go\\
\glt \sqt{And so then Lu Tu said, ``If (you) are very sleepy and are really craving meat, (I) must go back''...}\\
\ex \label{Jaex:13b}
\gll \textbf{Hais} \textbf{li}  \textbf{ntawd} \textbf{tag} ces...\\     	      
     say like that  finish  and.then\\
\glt \sqt{After saying that, then...} \citep[][8]{johnson92}
\end{xlist}
\end{exe}


\noindent
Rather than a copula or \isi{light verb} appearing in the bridging clause, as in \isi{summary linkage}, the speech verb of the reference clause, \textit{hais} `say', is repeated. It is accompanied by the  adverb \textit{li} `thus, like' and the demonstrative \textit{ntawd} `that, there', which serve to summarize the direct quotation.\footnote{ This is somewhat similar to the type of linkage reported by \citet[][128--129]{Guillaume2011} for \ili{Cavineña} (Tacanan, northern Bolivia), except that there is no restriction on the speech verbs that can be used in Hmong, while in \ili{Cavineña} the verbs used are limited to two summarizing verbs, which literally mean `be' and `affect'.} 

In other examples of this mixed type of linkage, \isi{substitution} is also involved:

\begin{exe}
\ex \label{Jaex:14ab}
\begin{xlist}
\ex \label{Jaex:14a}
\gll  \underline{\smash{Vaj.Leej.Txi}} \underline{tau} \underline{teb} \underline{tias} \underline{``tsis} \underline{tau}     \underline{\smash{txog}} \underline{\smash{caij}}, \underline{\smash{koj}} \underline{\smash{kav.tsij}} \underline{rov} \underline{\smash{qab}} \underline{mus} \underline{dua}.'' \\
  God \textsc{pfv} reply \textsc{comp} \textsc{neg} \textsc{pfv} arrive season 2\textsc{sg} hurry.to return back go again\\
\glt \sqt{God replied, ``The season has not come; you hurry back again''.}\\
\ex \label{Jaex:14b}
\gll \textbf{Vaj.Leej.Txi} \textbf{tau}  \textbf{txhib} \textbf{li} ces... \\     	      
     God \textsc{pfv} urge like and.then\\
\glt \sqt{God urged (him) like (that) and then...} \citep[][17]{vang90}
\end{xlist}
\end{exe}

\noindent
Here the narrator substitutes the speech verb \textit{teb} `reply' in the reference clause with a semantically more specific speech verb \textit{txhib} `urge', which describes the nature of God’s reply. 


\section{Functions of bridging constructions in White Hmong narratives} 
\label{JaFunctions}
As shown in \refsec{JaPosition}, bridging constructions in \ili{White Hmong} all play a role in enhancing discourse \isi{cohesion}, serving to progress the \isi{main event line}. Furthermore, the occurrence of a bridging construction often contributes to constructing a particularly salient point of progression – a point at which the \isi{narrative} moves forward to a new event, a new scene, a new episode, or a new ``chapter. \citet[][210]{kress06} describe the notion of salience as ''the degree to which an element draws attention to itself due to its size, its place in the foreground or its overlapping of other elements, its colour, its tonal values, its sharpness or definition and other features.'' 

The salience of the event described by the bridging construction is signaled linguistically in all cases by virtue of the simple fact that the clause describing that event is repeated in some way, whether by recapitulative, summary, or \isi{mixed linkage}. However, as will be shown in  \refsec{JaChange}, in many cases of \isi{recapitulative linkage} in \ili{White Hmong}, the salience of the event described is further enhanced by variation, not only due to the features of \isi{modification}, \isi{omission}, \isi{addition}, \isi{substitution}, and summary (\refsec{Jatypes}), but also involving a change in aspect between the reference clause and the bridging clause. This change allows the narrator to shift from a ``bird’s eye'' view of the event to a more engaged construal, as if pausing momentarily to observe the event as it is realized. This event then becomes a base from which the \isi{event line} of the \isi{narrative} moves forward. This \isi{aspectual variation} is the first main way in which bridging constructions serve to progress the \isi{narrative} sequence.

The second way in which a bridging construction can facilitate the \isi{narrative} progression is where \isi{supportive material} temporarily interrupts the flow of the \isi{event line}. This is discussed in \refsec{JaReturn}. In this case the bridging clause serves to pick up the \isi{action} exactly where it was left off, bringing the focus back to the \isi{main event line} and allowing it to proceed. These two ways in which bridging constructions are used to progress the \isi{narrative} sequence are not necessarily distinct; a single construction can serve to bring the \isi{narrative} back to the \isi{main event line} and also facilitate a change in aspectual construal.


\subsection{Change in aspect; change in construal}
\label{JaChange}
In the clear majority of cases of \isi{recapitulative linkage} in the \isi{narrative} texts examined, there is a change in aspect between the reference clause and the bridging clause. This not only enhances the salience of the event by adding to its temporal texture but also results in a change in its construal. It often allows the narrator to move from a more removed, ``bird’s eye'' perspective on the event to a more involved stance—to zoom in on the event and describe it as it unfolds. The narrator then uses this revised construal of the event as a point of departure, from which to move on to the next event in the \isi{narrative} sequence. 

Aspectual meaning is conveyed in \ili{White Hmong} in a variety of ways beyond the inherent aspectual meaning of the verb itself, including the use of pre-verbal aspectual morphemes, time adverbs, verbal \isi{reduplication}, and some types of \isi{serial verb constructions} (SVCs). The use of pre-verbal aspectual morphemes to change the way in which the same event is depicted between the reference clause and the bridging clause has been illustrated in example (\ref{Jaex:10ab}). Aspectual change from a simple verb in the reference clause to a SVC in the bridging clause occurs in example (\ref{Jaex:01ae}), while the opposite occurs in (\ref{Jaex:03ad}), which starts with verbs in series and changes to a simple verb. Variation in the type of SVC resulting in aspectual change is shown in examples (\ref{Jaex:02af}) and (\ref{Jaex:04ac}). The use of time adverbs in combination with \isi{reduplication} is illustrated in example (\ref{Jaex:15ac}), and that of \isi{reduplication} with SVCs in (\ref{Jaex:16ab}). 

Example (\ref{Jaex:15ac}) comes from the `Legend of Ms Fine Flower II'. Ms Fine Flower and her companion, Ms Sultry Toad, are introduced as being very poor. There follows a brief word picture that captures their poverty, describing how they go out every day to scavenge for wild nuts:


\begin{exe}
\ex \label{Jaex:15ac}
\begin{xlist}
\ex \label{Jaex:15a}
\gll  \underline{nkawd} \underline{\smash{niaj}} \underline{hnub} \underline{mus} \underline{khaws} \underline{\smash{txiv.ntseej}}     \underline{\smash{txiv.qhib}} \underline{\smash{noj}}.\\
 3\textsc{du} every day go pick chestnut  acorn eat\\
\glt \sqt{Every day the two of them went to pick chestnuts (and) acorns to eat.}\\
\ex \label{Jaex:15b}
\gll \textbf{Nkawd} \textbf{mus}  \textbf{khaws{\textasciitilde}khaws} \textbf{txiv.ntseej} \textbf{txiv.quib } \textbf{noj}, \\     	      
     3\textsc{du}  go  \textsc{redup}{\textasciitilde}pick   chestnut  acorn  eat\\
\glt \sqt{[One day] they went along picking (and) picking chestnuts (and) acorns to eat,} \\
\ex \label{Jaex:15c}
\gll ces Txiv.Nraug.Ntsuag txawm mus pom nkawd.\\     	      
     and.then The.Young.Orphan then go see 3\textsc{du}\\
\glt \sqt{and then The Young Orphan went to see them.} \citep[][161]{johnson92}
\end{xlist}
\end{exe}

\noindent
The young women’s \isi{action} is explicitly indicated as habitual with the use of the time adverb \textit{niaj hnub} `every day' in the reference clause (\ref{Jaex:15a}). The bridging clause (\ref{Jaex:15b}) then switches to continuous aspect, using the reduplicated verb \textit{khaws\~{}khaws} (`(be) picking (and) picking'). With this aspectual change, the narrator zooms in from an initial overview of their life circumstances to focus on a particular moment when, as they were busily engaged with their daily task, The Young Orphan entered their life (\ref{Jaex:15c}), and changed their fortunes completely.
 
The next example of aspectual change is from later in the same story, by which time the heroine, Ms Fine Flower, has married The Young Orphan. Her companion Ms Sultry Toad, enraged and jealous, devises a scheme to shame Ms Fine Flower.
 
\begin{exe}
\ex \label{Jaex:16ab}
\begin{xlist}
\ex \label{Jaex:16a}
\gll es \underline{\smash{Niam.Nkauj.Kub.Kaws}} \underline{txawm} \underline{muab} \underline{\smash{Niam.Nkauj.Kub.Kaws}} \underline{cov} \underline{\smash{niag}}     \underline{ntshav} \underline{\smash{pim}} \underline{\smash{coj}} \underline{mus}     \underline{\smash{pleev{\textasciitilde}pleev}} \underline{\smash{Niam.Nkauj.Zuag.Paj}} \underline{lub} \underline{\smash{qhov.ncauj}},\\
 so Ms.Sultry.Toad    then  take  Ms.Sultry.Toad \textsc{clf:coll} great blood vagina take.along go  \textsc{redup}{\textasciitilde}smear Ms.Fine.Flower \textsc{clf} mouth\\
\glt \sqt{so then taking her own menstrual blood, Ms Sultry Toad took (it) over (to) smear (and) smear (on) Ms Fine Flower’s mouth,}\\
\ex \label{Jaex:16b}
\gll \textbf{muab} \textbf{pleev{\textasciitilde}pleev}  \textbf{Niam.Nkauj.Zuag.Paj} \textbf{lub} \textbf{qhov.ncauj} \textbf{lo} \textbf{ntshav} \textbf{liab-vog}, \\     	      
     take \textsc{redup}{\textasciitilde}smear Ms.Fine.Flower \textsc{clf} mouth be(come).plastered blood  red-speckled\\
\glt \sqt{(she) took (it) (and) smeared (and) smeared (it) (on) Ms Fine Flower’s mouth (so that it) was plastered (with) red blood.} \citep[][162]{johnson92}
\end{xlist}
\end{exe}

\noindent
The reference clause (\ref{Jaex:16a}) uses a \isi{serial verb construction}, also involving \isi{reduplication}, to focus on the process of Ms Sultry Toad’s \isi{action} – \textit{muab … coj mus pleev{\textasciitilde}pleev }(take … take.along go \textsc{redup}{\textasciitilde}smear) – taking up the blood, carrying it over to her victim, and smearing it all over her mouth. The bridging clause (\ref{Jaex:16b}) retains some focus on this process – \textit{muab … pleev{\textasciitilde}pleev }(take … \textsc{redup}{\textasciitilde}smear) – but adds another verb in the series – \textit{lo }(become plastered with) – to also include the result of the \isi{action}, Ms Fine Flower’s mouth becoming plastered all over with blood. This is a point of great significance in the story, as Ms Sultry Toad then tells The Young Orphan that Ms Fine Flower’s red mouth is a sign that she has been drinking sheep’s blood, provoking him to drive his young wife out of their home.
 
In this section we have discussed the extremely common phenomenon of variation in aspect between the two clauses in a \isi{recapitulative linkage}. This variation in aspect results in a change in the construal of the event, giving a sense that the narrator moves to a closer focus and pauses briefly as the event unfolds, before moving on with the main line and thus progressing the \isi{narrative} sequence. In the next section we will look at the second way in which bridging constructions are used in \ili{White Hmong} to achieve this same broad function of moving the \isi{event line} of the \isi{narrative} forward.
 
\subsection{Return to the event line after supportive material}
\label{JaReturn}
 
In \ili{White Hmong} bridging constructions, the bridging clause generally follows the reference clause directly. Less commonly, one or more clauses intervene between the reference clause and the bridging clause. Their purpose is always to provide information that supports the \isi{narrative}, but which is not part of the \isi{event line}. The bridging clause then serves to bring the narration back to the \isi{event line}, as the narrator picks up the main sequence of events again following this parenthetical digression. In example (\ref{Jaex:17ae}) the \isi{event line} is describing the ceremonies associated with the birth of twins in the story of Shong Lue Yang.
%\end{styleStandard}
%
\begin{exe}
\ex \label{Jaex:17ae}
\begin{xlist}
\ex \label{Jaex:17a}
\gll \underline{lawv} \underline{\smash{thiaj}} \underline{muab} \underline{ob} \underline{\smash{leej}} \underline{\smash{me.nyuam}}     \underline{ntxaib} \underline{hu} \underline{\smash{plig}} \underline{thiab}     \underline{tis} \underline{\smash{npe}}.\\
3\textsc{pl} so.then take two \textsc{clf} child twins call spirit and assign name\\
\glt \sqt{… so then they took the two children (and) called (their) spirits and gave (them) names.}\\
\ex \label{Jaex:17b}
\gll Leej hlob muab hu.ua  Tsab.Yaj,\\
\textsc{clf} be.old take name Tsa.Ya\\ 
\glt \sqt{The older one (they) called Tsa Ya,}\\
\ex \label{Jaex:17c}
\gll leej yau  muab hu.ua Xab.Yaj.\\		
 \textsc{clf} be.young take name Xa.Ya \\ 
\glt \sqt{the younger one (they) called Xa Ya.}\\
\ex \label{Jaex:17d}
\gll \textbf{Tom.qab} \textbf{muab}  \textbf{nkawd} \textbf{hu} \textbf{plig} \textbf{tis} \textbf{npe} \textbf{tag},\\		           
 after take 3\textsc{du} call spirit assign name finish\\
\glt \sqt{After having taken those two, calling (their) spirits (and) giving (them) names,}\\
\ex \label{Jaex:17e}
\gll niam.tais thiab yawm.txiv  tau rov mus tsev lawm.\\     	      
mother-in-law and father-in-law \textsc{pfv} return go home \textsc{prf}\\
\glt \sqt{mother-in-law and father-in-law went back home.} \citep[][33]{vang90}
\end{xlist}
\end{exe}

\noindent
The reference clause (\ref{Jaex:17a}) introduces the ceremonies. The two juxtaposed main clauses in (\ref{Jaex:17b}) and (\ref{Jaex:17c}) follow, providing supportive information concerning the names given to the babies. The bridging clause (\ref{Jaex:17d}) then functions both to bring the \isi{narrative} back to the \isi{main event line} and to introduce the fact that the next event – (\ref{Jaex:17e}) the in-laws’ return home – occurred after the ceremonies were concluded.
 
While the intervening clauses in example (\ref{Jaex:17ae}) are main clauses, in example (\ref{Jaex:18ae}) non-main clauses intervene. This excerpt also comes from the story of Shong Lue Yang, whom the narrators believed to be one of the twelve sons of \textit{Vaj Leej Txi } `Sovereign Father, God'.
 
\begin{exe}
\ex \label{Jaex:18ae}
\begin{xlist}
\ex \label{Jaex:18a}
\gll ces \underline{nws} \underline{\smash{thiaj}} \underline{tau} \underline{muab} \underline{lub} \underline{tsho}     \underline{Soob}.\underline{\smash{Lwj}} \underline{hle} \underline{\smash{tseg}}     \underline{cia}\\
and.then 3\textsc{sg} so  \textsc{pfv} take  \textsc{clf} shirt Shong.Lue remove leave.behind set.aside \\
\glt \sqt{...and so he took (his) Shong Lue garb, removed (it) (and) left (it) behind}\\
\ex \label{Jaex:18b}
\gll tso rov qab mus nug Vaj.Leej.Txi dua\\
release return back go  ask God again\\ 
\glt \sqt{so (he) could go back [to heaven] to ask God again,}\\
\ex \label{Jaex:18c}
\gll seb  tim.li.cas  nkawd thiaj tsis lawv  qab los.\\		
find.out why 3\textsc{du} so \textsc{neg} follow back come\\ 
\glt \sqt{to find out why those two [his younger brothers] had not followed (him) back [to earth].}\\
\ex \label{Jaex:18d}
\gll \textbf{Nws} \textbf{tau}  \textbf{hle} \textbf{lub} \textbf{tsho} \textbf{Soob}.\textbf{Lwj} \textbf{tseg} \textbf{cia},\\		           
 3\textsc{sg} \textsc{pfv} remove \textsc{clf} shirt Shong.Lue leave.behind set.aside\\
\glt \sqt{He removed his Shong Lue garb (and) left (it) behind}\\
\ex \label{Jaex:18e}
\gll ces nws rov qab mus...\\     	      
and.then 3\textsc{sg} return back go\\
\glt \sqt{and then he went back...} \citep[][16]{vang90}
\end{xlist}
\end{exe}

\noindent
The digression in the non-main clauses (\ref{Jaex:18b}) and (\ref{Jaex:18c}) in this case serves to explain the purpose of the \isi{action} described in the reference clause (\ref{Jaex:18a}): the protagonist took off his human garb \textit{in order to return to heaven}. The \isi{action} of taking off his human garb is repeated in the bridging clause (\ref{Jaex:18d}), as the \isi{event line} is resumed.
 
In example (\ref{Jaex:19af}) from the first-person \isi{narrative} text `Kee’s Story', we see quite a lengthy diversion occurring between the reference clause (a) and the subsequent bridging clause (f). The narrator, along with his father and younger brother, managed to buy a letter giving permission to travel to Vientiane, so that they could then cross the Mekong River and flee war-torn Laos.
 
\begin{exe}
\ex \label{Jaex:19af}
\begin{xlist}
\ex \label{Jaex:19a}
\gll \underline{\smash{peb}} \underline{\smash{thiaj.li}}, \underline{\smash{peb}} \underline{txiv-tub}, \underline{\smash{peb}} \underline{\smash{thiaj.li}},     \underline{aws}, \underline{\smash{yuav}} \underline{lawv} \underline{ib}     \underline{\smash{daig}}  \underline{ntawv}.\\
1\textsc{pl} so.then 1\textsc{pl} father-son 1\textsc{pl} so.then \textsc{hesit} obtain 3\textsc{pl} one \textsc{clf} letter\\
\glt \sqt{So then we – we father and sons – so then we – um – bought their letter.}\\
\ex \label{Jaex:19b}
\gll Lawv daim ntawv ntawm yog ua Vientiane tuaj\\
 3\textsc{pl} \textsc{clf} letter that \textsc{cop} make  Vientiane come\\ 
\glt \sqt{That letter of theirs came from Vientiane}\\
\ex \label{Jaex:19c}
\gll hais tias tuaj xyuas  kwv.tij nyob rau pem Xieng.Khouang.\\		
say \textsc{comp} come visit  relative live to place.up Xieng.Khouang\\ 
\glt \sqt{(and it) said (they would) come (to) visit relatives up in Xieng Khouang.}\\
\ex \label{Jaex:19d}
\gll Lawv muaj peb  leeg  thiab\\
     3\textsc{pl} have three people also \\ 
\glt \sqt{They had three people too}\\
\ex \label{Jaex:19e}
\gll ces peb muaj peb leeg tab.tom phim  lawv daim ntawv ntawd\\
     and.then 1\textsc{pl} have three \textsc{clf} just  match 3\textsc{pl} \textsc{clf} letter  that\\ 
\glt \sqt{and then we had three people just matching that letter of theirs}\\
\ex \label{Jaex:19f}
\gll ces \textbf{peb} \textbf{thiaj}  \textbf{yuav} \textbf{lawv} \textbf{daim} \textbf{ntawv} \textbf{ntawm}, ces...\\		           
 and.then  1\textsc{pl} so.then obtain 3\textsc{pl} \textsc{clf} letter  that  and.then\\
\glt \sqt{and then we bought that letter of theirs, and then...} \citep[][227]{fuller85}
\end{xlist}
\end{exe}


This long diversion involving multiple clauses clearly supports the main line events of the \isi{narrative} – the story of flight from Laos – by explaining how the letter the travellers bought suited their needs and facilitated their journey. The length of this intervening material may be related to the informal, unplanned nature of this personal monologue.\footnote{ This use of recapitulation following a lengthy gap seems quite similar to some examples of self-\isi{repetition} used for \isi{cohesion} in \ili{Greek} conversations, given by \citeauthor{alvanoudi18} (this volume). In the \ili{Greek} examples, however, the \isi{repetition} connects a speaker’s previous and current turn, establishing contiguity after intervening turns by (an)other speaker(s).} When the \isi{event line} is picked up again in (\ref{Jaex:19f}), it is introduced by the \isi{sequential} \isi{conjunction} \textit{ces} `and then', which normally does not occur again until after a bridging clause. This clearly serves to reinforce the return to the \isi{sequential} \isi{event line} of the story.

The use of bridging clauses described here, to pick up the \isi{event line} after a parenthetical diversion, should not be thought of as completely separate from their use to modify the construal of the event (discussed in \refsec{JaChange}). In (\ref{Jaex:17ae}), for example, the bridging clause clearly serves both functions, not only returning the \isi{narrative} to the \isi{event line} but also shifting to completive aspect and thus explicitly asserting the ordered sequence of this event with the following one. Throughout the texts these two functions of bridging constructions can be seen to work together to progress the \isi{main event line} and to facilitate discourse \isi{cohesion}. 

\section{Conclusion} 
\label{Jasec:Conclusion}

This chapter has examined the position, form, frequency, and types of bridging constructions in \ili{White Hmong} \isi{narrative} texts, along with their discourse functions. 

Bridging constructions are commonly positioned at the boundary between discourse units that belong to the \isi{event line} of the \isi{narrative}. Here they serve to link both major episodes (``chapters'') and minor episodes (``paragraphs''). They can also occur in the absence of a discourse boundary, simply to bring the \isi{narrative} back to the \isi{event line} after a brief digression. 

In terms of form, reference clauses are all main clauses, and bridging clauses are either reduced main clauses, or temporal subordinate clauses serving to relate the event of the bridging construction to the next event in sequence (e.g., ``after'', ``when'', etc.). The construction as a whole is usually explicitly embedded in the \isi{sequential} \isi{event line} of the \isi{narrative} with coordinating, \isi{sequential}, or subordinating conjunctions.

The data show that the frequency and type of bridging constructions can vary in \ili{White Hmong} depending on narrator and text type. Recapitulative linkage is far more common than \isi{summary linkage}, which is limited to unplanned, spoken styles. A further mixed type of linkage involving a speech verb introducing a direct quotation occasionally occurs in more literary spoken and written texts.

The bridging constructions examined in this data from \isi{narrative} texts in \ili{White Hmong} serve to enhance the salience of the events they describe. This occurs in all cases by virtue of the fact that the clause describing that event ``draws attention to itself'' \citep[][210]{kress06} through \isi{repetition}. However, in \ili{White Hmong}, this salience is further enhanced in most cases by variation between the reference and bridging clause, including \isi{modification}, \isi{omission}, \isi{addition}, \isi{substitution}, and summary. A particularly common kind of variation involves a change in aspect. This change allows the narrator to shift from a ``bird’s eye'' view of the event concerned to a more engaged construal, as if pausing momentarily to observe the event as it unfolds. This momentary pause allows the narrator to use that event as a base from which the \isi{narrative} then moves forward. In these multiple ways, bridging constructions in \ili{White Hmong} work cohesively, linking one unit in the \isi{event line} to the next and serving to progress the main sequence of events. 

 \section*{Appendix}
 \setcounter{equation}{0}
 \exewidth{(A23)}
The excerpt below is the beginning of the story of the first man and woman on earth \citep[][3--4]{johnson92}. There are five bridging constructions in this excerpt, each of which helps to move the story forward in some way. The first bridging construction takes the story from the depiction of the man alone on the dark, barren earth, to the time when his wife is ready to join him. The second introduces a complication: the man has brought a magic flower with him to earth, but there is no wood to use to cook its seeds to eat. This dilemma is resolved, as the third bridging construction explains how they manage to burn coal to cook the seeds. When the seeds begin to run out, we see two bridging constructions in succession: the first resolving this complication, as they plant the remaining seeds, and the second introducing a new complication, as only one plant comes forth.

This excerpt illustrates well how bridging constructions function in \ili{White Hmong} as part of a wider phenomenon involving the strategy of \isi{repetition} with variation, to build up elements in a \isi{narrative} text as it moves forward in intricate, overlapping layers.

 \begin{exe}
\exi{(A1)} \label{JaexApp1}
\gll Thaum ub tsis muaj hnub tsis muaj hli,\\
time yonder \textsc{neg} have sun \textsc{neg} have moon,\\
\glt \sqt{Long ago, there was neither sun nor moon,}\\
\end{exe}
 
 \begin{exe}
\exi{(A2)} \label{JaexApp2}
\gll tsis muaj ib tug neeg nyob hauv lub ntiaj.teb no li.\\
 \textsc{neg} have one \textsc{clf} person be.located inside \textsc{clf} earth this at.all\\
\glt \sqt{(and) there were no people at all on this earth.}\\
\end{exe}




 \begin{exe}
\exi{(A3)} \label{JaexApp3}
\gll Muaj ib hnub, ib tug txiv.neej txawm tawm ntawm txoj sawv.toj los.\\
  have one day, one \textsc{clf} man then emerge place.nearby \textsc{clf} vein.in.hillside come\\
\glt \sqt{One day, a man emerged from a vein in the hillside.}\footnote{The terms \textit{sawv toj} and (a few lines further on) \textit{mem toj} both mean `vein/fissure in the hillside', and are related to the Hmong practices of geomancy.}\\
\end{exe}


 \begin{exe}
\exi{(A4)} \label{JaexApp4}
\gll  Nws lub npe hu.ua Txiv.Nraug.Luj.Tub.\\
   3\textsc{sg} \textsc{clf} name be.called Master.Lu.Tu\\
\glt \sqt{His name was Master Lu Tu.}\\
\end{exe}

 \begin{exe}
\exi{(A5)} \label{JaexApp5}
\gll  Nws tawm ntawm txoj mem.toj los xwb.\\
  3\textsc{sg} emerge place.nearby \textsc{clf} fissure.in.hillside come only\\
\glt \sqt{He just emerged from a fissure in the hillside.}\\
\end{exe}


 \begin{exe}
\exi{(A6)} \label{JaexApp6}
\gll Thaum nws tawm los txog saum yaj.ceeb no mas,\\
  when 3\textsc{sg} emerge come arrive place.above earth this \textsc{ip}\\
\glt \sqt{When he came out up onto the earth,}\\
\end{exe}


 \begin{exe}
\exi{(A7)} \label{JaexApp7}
\gll ntuj tsaus li qhov.paj teb tsaus li qhov.tsua.\\
     sky be.dark like cavern earth be.dark like cave\\
\glt \sqt{The sky was as dark as a cavern, the earth as dark as a cave.}\footnote{The expression \textit{qhov paj} (lit: `hole flower') does not, by itself, mean `cavern'. However here, in combination with \textit{qhov tsua} `cave' (lit: `hole rock'), it is probably functioning poetically to refer to limestone caves characterized by flower-like stalactite formations, more generally referred to as \textit{qhov tsua tawg paj} (lit: `hole rock bloom flower') or \textit{qhov tsua paj kaub} (lit: `hole rock flower crust') in Hmong.} 
\end{exe} 
 
 \begin{exe}
\exi{(A8)} \label{JaexApp8}
\gll Yeej tsis muaj hnub tsis muaj hli.\\
     originally \textsc{neg} have sun \textsc{neg} have moon\\
\glt \sqt{There was no sun (and) no moon.}
\end{exe}

\begin{exe}
\exi{(A9)} \label{JaexApp9}
\gll Nws cev tes xuas txawm tau ntuj nyob ntawd ntag.\\
     3\textsc{sg} raise.up hand touch then get sky be.located place.nearby \textsc{ip}\\
\glt  \sqt{He raised up his hand (and) was able to touch the sky there!}\footnote{In Hmong myths, the sky is often presented as a hemisphere that meets the earth at the horizon \citep[][14, fn.2]{johnson92}.}
\end{exe}

\begin{exe}
\exi{(A10)} \label{JaexApp10}
\gll Nws tawm ib.leeg ua.ntej los rau nraum yaj.ceeb no\\
     3\textsc{sg} emerge alone first come to place.outside earth this\\
\glt  \sqt{He came out first, all alone, to this earth}
\end{exe}

\begin{exe}
\exi{(A11)} \label{JaexApp11}
\gll ces \underline{\smash{thiaj}}  \underline{mam} \underline{xeeb} \underline{nws} \underline{tus}  \underline{\smash{poj.niam}} \underline{rau} \underline{ntawm} \underline{nws}  \underline{\smash{qhov.chaw}}.\\
     and.then so.then  then.next be.born 3\textsc{sg} \textsc{clf} wife to place.nearby 3\textsc{sg} place\\
\glt  \sqt{and so then next his wife was born in his place [i.e., in the fissure from which he had emerged].}
\end{exe}
 
\begin{exe}
\exi{(A12)} \label{JaexApp12}
\gll \textbf{Xeeb} \textbf{nws} \textbf{tus} \textbf{poj.niam} \textbf{rau} \textbf{ntawm} \textbf{nws} \textbf{qhov.chaw} \textbf{puv-hnub-puv-nyoog}\\
     be.born 3\textsc{sg} \textsc{clf} wife to place.nearby 3\textsc{sg} place be.filled-day-be.filled-age\\
\glt \sqt{His wife was born into his place (until her) time was fulfilled}
\end{exe}

\begin{exe}
\exi{(A13)} \label{JaexApp13}
\gll ces nws tus poj.niam thiaj mam tawm lawv qab los.\\
and.then 3\textsc{sg} \textsc{clf} wife so.then then.next emerge follow behind come\\
\glt \sqt{and so then next his wife came out after (him).}
\end{exe}

\begin{exe}
\exi{(A14)} \label{JaexApp14}
\gll Nws tus poj.niam mas hu.ua Niam.Nkauj.Ntxhi.Chiv no.\\
     \textsc{3sg} \textsc{clf} wife \textsc{top} be.called Ms.Ntxi.Chi this\\
\glt \sqt{His wife, (she) was called Ms Ntxi Chi.}
\end{exe}

\begin{exe}
\exi{(A15)} \label{JaexApp15}
\gll Ces nkawd ob tug niam.txiv thiaj los nyob ua.neej.\\
     and.then 3\textsc{du} two \textsc{clf} couple so.then come live prosper\\
\glt \sqt{And then the two of them came [to earth] to live and prosper.}
\end{exe}

\begin{exe}
\exi{(A16)} \label{JaexApp16}
\gll Tsis muaj hnub tsis muaj hli;\\
 \textsc{neg} have sun \textsc{neg} have moon\\
\glt \sqt{There was no sun (and) no moon;}
\end{exe}

\begin{exe}
\exi{(A17)} \label{JaexApp17}
\gll ntuj tsaus li qhov.paj teb tsaus li qhov.tsua xwb.\\
     sky be.dark like cavern earth be.dark like cave only\\
\glt \sqt{the sky was as dark as a cavern, the earth as dark as a cave.}
\end{exe}

\begin{exe}
\exi{(A18)} \label{JaexApp18}
\gll Thaum Txiv.Nraug.Luj.Tub tawm los\\
     time Master.Lu.Tu emerge come\\
\glt \sqt{When Master Lu Tu came out}
\end{exe}

\begin{exe}
\exi{(A19)} \label{JaexApp19}
\gll ces nws txawm tau ib lub paj Caus Ci uas nyob ntawm nws qhov.chaw nrog nws los.\\
     and.then \textsc{3sg} then get one \textsc{clf} flower Cau Ci \textsc{rel} be.located place.nearby \textsc{3sg} place be.with \textsc{3sg} come\\
\glt \sqt{then he got a Cau Ci flower, which had been in his place [i.e., in the fissure] with him.}
\end{exe}

\begin{exe}
\exi{(A20)} \label{JaexApp20}
\gll \underline{Nws} \underline{\smash{nqa}} \underline{tau} \underline{lub} \underline{\smash{paj}} \underline{tawm} \underline{los} \underline{rau} \underline{nraum} \underline{\smash{yaj.ceeb}} \underline{no}.\\
     \textsc{3sg} carry get \textsc{clf} flower emerge come to outside world this\\
\glt \sqt{He brought the flower out to this world.}
\end{exe}

\begin{exe}
\exi{(A21)} \label{JaexApp21}
\gll \textbf{Coj} \textbf{los}\\
     take.along come\\
\glt \sqt{[He] brought [it] along}
\end{exe}

\begin{exe}
\exi{(A22)} \label{JaexApp22}
\gll ces tsis muaj xyoob muaj ntoo, tsis muaj hluav.taws li\\
     and.then \textsc{neg} have bamboo have tree, \textsc{neg} have fire at.all\\
\glt  \sqt{and then (he) had neither bamboo [nor] trees, [so] (he) had no fire at all.}
\end{exe}

\begin{exe}
\exi{(A23)} \label{JaexApp23}
\gll Thaum ntawd nws txawm los nyob;\\
     time that \textsc{3sg} then come live\\
\glt  \sqt{At that time he came to live (here);}
\end{exe}

\begin{exe}
\exi{(A24)} \label{JaexApp24}
\gll \underline{ces} \underline{nws} \underline{rauv} \underline{zeb.ntsuam} \underline{xwb}.\\
     and.then 3\textsc{sg} burn coal only\\
\glt  \sqt{and then he burned only (pieces of) coal.}
\end{exe}

\begin{exe}
\exi{(A25)} \label{JaexApp25}
\gll \textbf{Luj} \textbf{Tub} \textbf{nkawd} \textbf{ob.niam.txiv} \textbf{rauv} \textbf{cov} \textbf{ntawd}\\
     Lu Tu 3\textsc{du} couple burn \textsc{clf:coll} that\\
\glt  \sqt{Lu Tu and his wife burned those}
\end{exe}

\begin{exe}
\exi{(A26)} \label{JaexApp26}
\gll kib lub paj ntawd cov noob noj xwb.\\
     fry \textsc{clf} flower that \textsc{clf:coll} seed eat only\\
\glt \sqt{(to) fry the seeds of that flower to eat.}
\end{exe}

\begin{exe}
\exi{(A27)} \label{JaexApp27}
\gll Nkawd nyob ces nyob{\textasciitilde}nyob,\\
     3\textsc{du} live and.then \textsc{redup}{\textasciitilde}live\\
\glt \sqt{The two of them lived on and on,}
\end{exe}

\begin{exe}
\exi{(A28)} \label{JaexApp28}
\gll kib{\textasciitilde}kib cov noob ntawm lub paj ntawd noj yuav tag;\\
     \textsc{redup}{\textasciitilde}fry \textsc{clf:coll} seed that \textsc{clf} flower that eat will finish\\
\glt \sqt{(and) kept frying the seeds of that flower to eat (until) (they) were going to run out;}
\end{exe}

\begin{exe}
\exi{(A29)} \label{JaexApp29}
\gll  \underline{ces}  \underline{nkawd}  \underline{\smash{thiaj}}  \underline{muab}  \underline{\smash{coj}}  \underline{mus}  \underline{\smash{cog}}.\\
     and.then 3\textsc{du} so.then take take.along go plant\\
\glt \sqt{So then they took (the seeds) and went to plant (them).}
\end{exe}

\begin{exe}
\exi{(A30)} \label{JaexApp30}
\gll \textbf{Cog} \textbf{tas} \textbf{na}\\
     plant finish \textsc{ip}\\
\glt \sqt{(They) finished planting (them), don’t you know,}
\end{exe}


\begin{exe}
\exi{(A31)} \label{JaexApp31}
\gll \underline{\smash{tuaj}} \underline{ib} \underline{tsob} \underline{xwb}.\\
     come one \textsc{clf} only\\
\glt \sqt{(and) there came forth only one plant.}
\end{exe}

\begin{exe}
\exi{(A32)} \label{JaexApp32}
\gll \textbf{Tuaj} \textbf{tau}...\\
     come get\\
\glt \sqt{There came forth (one plant)...}
\end{exe}

\section*{Abbreviations}
\textsc{1}		first person,
\textsc{2}		second person,
\textsc{3} 		third person,
\textsc{a}		transitive subject,
\textsc{cc}		copula complement,
\textsc{clf}		classifier,
\textsc{clf:coll}		collective classifier,
\textsc{comp}		complementizer,
\textsc{cop}		copula,
\textsc{cs}		copula subject,
\textsc{du}		dual,
\textsc{ex}		exclamative,
\textsc{hesit}		hesitation,
\textsc{ip}		illocutionary particle,
\textsc{neg}		negation,
\textsc{o}		transitive object,
\textsc{pfv}		perfective,
\textsc{pl}		plural,
\textsc{prf}		perfect,
\textsc{q}		question particle,
\textsc{redup}		reduplicated,
\textsc{rel}		relativizer,
\textsc{sg}		singular,
\textsc{svc}	serial verb construction,
\textsc{top}		topic,
\textsc{v}	verb

\section*{Acknowledgements}
I wish to express my sincere thanks to Valérie Guérin for inviting me to contribute to
this volume, and for her wonderful support as I prepared this chapter. Many thanks
also to Sasha Aikhenvald, Grant Aiton, and two anonymous reviewers for very helpful
comments on earlier versions. Although the data for the chapter come from published
sources, my analysis would not have been possible without the help of my dear Hmong
teachers and friends. Particular thanks go to Cua Lis, Thaiv Thoj, and Zoo Lis.

\sloppy

\printbibliography[heading=subbibliography,notkeyword=this] 
\end{document}

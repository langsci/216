\addchap{Preface}
\begin{refsection}


This volume is partly the result of a two-day workshop entitled \textit{Bridging Linkage in Cross-linguistic Perspective} organized at the Cairns Institute (James Cook University, Australia), on 25--26 February 2015 by Valérie Guérin and Simon Overall. Our intent at the time was two-fold: (i) to gather data from a variety of languages that would enable us to draw cross-linguistic generalisations about the formal and functional characteristics of bridging constructions, and (ii) to try and delimit the range of constructions that can be subsumed under the term bridging construction. In particular, we found it important to try and separate out bridging constructions from repetition. We aimed to cast our net as widely as possible in order to get a broad picture of bridging constructions and their instantiation across languages. For the workshop, we selected  nine genetically-unrelated languages: four languages spoken in South America; four languages of Oceania (Australia, Papua New Guinea, and Vanuatu); and Greek. Some of the presentations are reproduced in this volume in their most recent versions. Other chapters were invited by the volume’s editor. 

In preparation for the workshop (and subsequently, for this volume), we circulated among the authors (i) a list of core features defining bridging constructions extracted from the literature, reproduced below; (ii) a series of questions to address, if relevant, when describing bridging constructions. They are reproduced in the Appendix of \chapref{ch:1}; (iii) and an earlier version of \chapref{ch:1}, the introductory chapter. We asked that each author use these notions as a starting point to isolate typical and atypical instances of bridging constructions in their language of study. 



\section*{Characteristic features of bridging constructions}


\begin{itemize}
\item Bridging constructions are composed of a reference clause and a bridging clause. 
   \begin{itemize}
   \item The bridging clause is a \isi{non-main clause}. The dependency can be marked morphologically, syntactically, or prosodically.
   \item  Prototypically, the reference clause is a \isi{main clause}. 
   \item In the large majority of cases, the reference clause ends a discourse unit while the bridging clause appears at the beginning of a new discourse unit.
   \item  The bridging clause recapitulates at least one clause in the preceding discourse unit. 
   \end{itemize}
   \end{itemize}
   
 \begin{itemize}  
\item There are three types of bridging constructions, differentiated by the content of the bridging clause: 
\begin{itemize}
\item Recapitulative linkage: the bridging clause repeats the reference\linebreak clause more-or-less verbatim. 
\item  Summary linkage: the bridging clause does not repeat the reference clause but anaphorically refers to it with a \textit{summarizing} predicate as the bridging element (i.e., a \isi{demonstrative verb}, a \isi{pro-verb}, an auxiliary, or a \isi{light verb}). 
\item Mixed linkage: both types of linkage may co-occur in a single instance of bridging, where the bridging clause contains the same lexical verb as the reference clause in addition to a summarizing verb of the type typically found in \isi{summary linkage}.
\end{itemize}
\end{itemize}

 \begin{itemize} 
\item  In a stretch of discourse, bridging constructions enable:
\begin{itemize}
\item  Information backgrounding
\item  Referent tracking
\item  Event \isi{sequentiality} 
\item  Paragraph demarcation
\end{itemize}
\end{itemize}


\section*{The chapters}

\chapref{ch:1} takes a typological look at bridging constructions. After introducing the general concepts, Valérie Guérin and Grant Aiton review the three types of bridging constructions that are reported in the literature and in the current volume, and discuss the form and functions of bridging constructions across languages. 

In \chapref{ch:2}, Nick Emlen analyses \isi{recapitulative linkage} in \ili{Matsigenka}, a \ili{Kampan} (\ili{Arawak}) language, and shows how these constructions have been borrowed in \ili{Spanish}, but not in \ili{Quechua}, in a trilingual community in Peru. This chapter was presented in parts at \textit{Red Europea para el Estudio de las Lenguas Andinas} (REELA), Leiden, September 2015. 

Hannah Sarvasy presents bridging constructions in the \ili{Bantu} language \ili{Logoori} in \chapref{ch:3}. She argues that these constructions are rarely use in \ili{Logoori} discourse, restricted to \isi{procedural texts}, and as stylistic features, their presence in a text is highly dependent on the penchant of the speaker. 

Diana Forker and Felix Anker examine bridging constructions in the Nakh-Daghestanian language family in \chapref{ch:4}. They show that recapitulative and summary linkages both occur in narratives in the \ili{Tsezic} language group, and suggest that \isi{recapitulative linkage} can be found throughout the Nakh-Daghestanian language family. Forker and Anker additionally observe a regular shift in deixis between the reference clause and bridging clause, which results in a regular \isi{substitution} of an andative verb of motion for an equivalent venitive verb. 

In \chapref{ch:5}, Nerida Jarkey reveals that in \ili{White Hmong} \isi{recapitulative linkage} is more common than \isi{summary linkage} which is only found in first person narratives. The functions of these constructions are illustrated in the light of three \isi{text genres}. 

\chapref{ch:6} by Grant Aiton describes bridging constructions in \ili{Eibela}, a language of the Western Province of Papua New Guinea. Features of interest in \ili{Eibela} include three types of \isi{summary linkage} and discourse preferences relating \isi{summary linkage} to paragraphs and \isi{recapitulative linkage} to episodes. Parts of this chapter were published in the journal \textit{Language and Linguistics in Melanesia} in 2015. 

In \chapref{ch:7}, Lourens de Vries details bridging constructions in \ili{Korowai}, a Greater Awyu language of West Papua. Summary and recapitulative linkages are described in the wider context of clausal chains, their subtypes and functions clearly spelled out (whether they are marked or unmarked, carrying \isi{switch reference} marking or not, indicating \isi{thematic continuity} or discontinuity). 

Valérie Guérin analyzes \isi{recapitulative linkage} in \ili{Mavea}, an \ili{Oceanic} language of Vanuatu. In \chapref{ch:8}, she shows that bridging clauses are morphologically main clauses but phonologically marked as dependent and that their function in discourse is mostly to add emphasis. 

Finally, Angeliki Alvanoudi takes a conversation analytical framework to\linebreak study clausal repetitions in modern \ili{Greek} interactions in \chapref{ch:9}. She highlights similarities and differences between \isi{recapitulative linkage} and \isi{clause repetition} and hypothesizes that the former is a grammaticalized expression of the latter. 



\section*{The authors}

Grant Aiton's primary research interests are language variation and typology with a current emphasis on documentation and field linguistics. He completed a Master's degree in Linguistics at the University of Alberta where he conducted research on Athabaskan and Salish languages. His PhD project at James Cook University was the documentation of Eibela, a previously undescribed language spoken in Western Province and Southern Highlands Province, Papua New Guinea. \mailtolink{aiton.grant@gmail.com}

Angeliki Alvanoudi is Lecturer at the Aristotle University of Thessaloniki and Adjunct Lecturer in Linguistics at James Cook University, Australia. Her main interests are language and gender, language and cognition, and grammar and interaction. Her recent publications include \textit{Language contact, borrowing and code switching: A case study of Australian Greek} (Journal of Greek Linguistics, 2018) and \textit{The interface between language and cultural conceptualizations of gender in interaction: The case of Greek} (in Advances in cultural linguistics, ed. by F. Sharifian, Springer, 2017). She has written the books \textit{Grammatical gender in interaction: Cultural and cognitive aspects} (Brill, 2014) and \textit{Modern Greek in diaspora: An Australian perspective} (Palgrave, 2018). \mailtolink{alvanoudiag@yahoo.gr} 

Felix Anker studies General Linguistics at the University of Bamberg and will complete his studies with a Master’s thesis on topological relations in Tsova-Tush. His main research interests are languages of the Caucasus, language typology and morphosyntax. His prospective dissertation will be on various topics of Tsova-Tush syntax.
\mailtolink{felix.anker@gmx.de}

Lourens de Vries is professor of general linguistics at the Vrije Universiteit, Amsterdam, The Netherlands. His research focus is the description and typology of Papuan languages. He published grammatical descriptions of Wambon, Korowai, Kombai and Inanwatan. \mailtolink{l.j.de.vries@vu.nl}

Diana Forker teaches Caucasian Studies at the University of Jena. She completed her PhD at the Max Planck Institute for Evolutionary Anthropology. Her main interests are languages of the Caucasus, typology, and morphosyntax and sociolinguistics. She currently works on the documentation of Sanzhi Dargwa, a Nakh-Daghestanian language. Among her recent publications are \textit{A Grammar of Hinuq} (2013) and several articles on different aspects of Nakh-Daghestanian languages.  \mailtolink{diana.forker@uni-jena.de}

Nicholas Q. Emlen is a linguistic anthropologist (PhD University of Michigan, 2014) who has conducted extensive ethnographic research on multilingualism, language contact, and coffee production on the Andean-Amazonian agricultural frontier of Southern Peru. He also works on the reconstruction of Quechua-Aymara language contact in the ancient Central Andes, and on multilingualism among Quechua, Aymara, Puquina, and Spanish in the colonial Andes. He is currently a National Endowment for the Humanities Fellow at the John Carter Brown Library, and a visiting lecturer in anthropology at Brown University.\linebreak\mailtolink{nqemlen@gmail.com}

Valérie Guérin obtained a PhD from the University of Hawai'i at Mānoa for her work on Mavea, a moribund language of Vanuatu (grammar published by the University of Hawai'i Press in 2011).  She currently works on describing Tayatuk, a language spoken in the YUS conservation area, Morobe Province, Papua New Guinea. In 2013--2016, she was a postdoctoral research associate at the Language and Culture Research Centre, James Cook University, under the Australian Laureate Fellowship awarded to Professor Aikhenvald. She is currently affiliated with the Language and Culture Research Centre as an adjunct fellow researcher. \mailtolink{valerie.guerin@gmail.com}

Nerida Jarkey teaches Japanese Studies at the University of Sydney. Her research focuses on two Asian languages, Japanese and Hmong, and is concerned with the relationships between language, cognition, culture and the expression of social identity. She pays particular attention to multi-verb constructions, and is author of \textit{Serial Verb Constructions in White Hmong}, published by Brill in 2015. Address: School of Languages and Cultures (A18), University of Sydney, NSW 2006, Australia. \mailtolink{nerida.jarkey@sydney.edu.au} 

Hannah Sarvasy received her PhD in 2015 from James Cook University. She has conducted immersion fieldwork on Nungon (Papuan), Kim and Bom (Atlantic; Sierra Leone), and Tashelhit Berber and ran a pioneering longitudinal study of children’s acquisition of Nungon. Her publications include \textit{A Grammar of Nungon: A Papuan Language of Northeast New Guinea} (Brill, 2017), \textit{Word Hunters: Field Linguists on Fieldwork} (John Benjamins, 2018), and articles and book chapters on topics in Nungon grammar, fieldwork methodology, Bantu linguistics, and ethnobiology, as well as Kim and Bom language primers. She taught at UCLA, served as a Research Fellow at the Australian National University, and currently holds an Australian Research Council Discovery Early Career Researcher Award for the study of clause chains from typological, acquisition, and psycholinguistic angles. Address: MARCS Institute, Western Sydney University. \mailtolink{h.sarvasy@westernsydney.edu.au}


{\sloppy
\printbibliography[heading=subbibliography]}
\end{refsection}

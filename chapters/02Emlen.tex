\documentclass[output=paper]{LSP/langsci} 
\author{
   Nicholas Q. Emlen\affiliation{John Carter Brown Library, Brown University}
}
\title{The poetics of recapitulative linkage in Matsigenka and mixed Matsigenka-Spanish myth narrations}
\shorttitlerunninghead{The poetics of recapitulative linkage in Matsigenka}
%epigram
\abstract{In a small community in the Andean-Amazonian transitional zone of Southern Peru, speakers of Matsigenka use recapitulative linkages in myth narrations. These constructions establish a kind of rhythm, distinctive to the myth narration discourse genre, through which the events of the narrative unfold, information is introduced and elaborated, and suspense and surprise are achieved. This chapter describes the structural and discursive properties of these linking devices and their use in myth narrations. Bridging clauses generally recapitulate reference clauses verbatim or with minor modifications, and are usually linked to discourse-new information as simple juxtaposed clauses (though there is much variation in the structure and pragmatic functions of these constructions). Though the constructions contribute to discourse cohesion, their function is primarily poetic in nature. Furthermore, when Matsigenka speakers narrate the same myths in Spanish and in mixed Matsigenka-Spanish speech, they use the same kinds of linking constructions (which are otherwise uncommon in Spanish). Thus, the transfer of this kind of pattern from Matsigenka to Spanish is regimented by discourse genre, and offers an illustration of the cultural (i.e. metapragmatic) mediation of \isi{language contact}.}
\maketitle
%-------------------------

\begin{document}

\section{Introduction} 
\label{EmIntroduction}
This chapter describes a type of \isi{recapitulative linkage} used in \ili{Matsigenka} myth narrations in a small, multiethnic community on the Andean-Amazonian agricultural frontier of Southern Peru. It also briefly presents the use of this construction in \ili{Spanish} and mixed \ili{Matsigenka}-\ili{Spanish} myth narrations by the same speakers. The most common form of the construction is as follows: a proposition is uttered (the reference clause, indicated in underlined text throughout this chapter), followed by a pause (indicated in brackets). Then, the proposition in the reference clause is recapitulated in the bridging clause (indicated in boldface text) and followed immediately by discourse-new information, usually in the form of a simple juxtaposed clause without any subordinating morphology. A simple \ili{Matsigenka} example is given in (\ref{Em01ab}):\footnote{\ili{Matsigenka} morphemic analyses are adapted from \citet{michael08} and \citet{vargas13}.}



%
\begin{exe}
\ex \label{Em01ab}
\begin{xlist}
\ex \label{Emex:01a}
\glt \textit{Impogini maika \underline{\smash{oaigake}}}. [0.6]\\
\gll impogini maika o-a-ig-ak-i \\
 then now \textsc{3f-}go\textsc{-pl}-\textsc{pfv}-\textsc{real}\\
\glt \sqt{Then they went.}\\
\ex \label{Emex:01b}
\glt \textit{\textbf{Oaigake} agaiganake oviarena.}\\
\gll o-a-ig-ak-i o-ag-a-ig-an-ak-i o-piarena\\     	      
   \textsc{3f-}go\textsc{-pl}-\textsc{pfv-}\textsc{real} \textsc{3f-}get\textsc{-ep-}\textsc{pl}-\textsc{abl-}\textsc{pfv}-\textsc{real} \textsc{3f-}gourd\\
\glt \sqt{They went (and) they got their gourds.}
\end{xlist}
\end{exe}



These recapitulative linkages often express continuity between a single character’s simultaneous or immediately \isi{sequential} actions (as in \textit{oaigake} `they went' in (\ref{Emex:01a}) and \textit{agaiganake oviarena} `they got their gourds' in (\ref{Emex:01b})); for this reason, the recapitulated clause and the discourse-new clause usually have the same subject. However, there is substantial variation in the structure and pragmatic function of these constructions. For instance, in many cases the discourse-new information clarifies or elaborates the preceding proposition instead of offering a new one, and less frequently, the subject of the discourse-new clause is different from that of the recapitulated clause. More rarely, the recapitulated element does not contain a verb at all, but still follows the discursive patterns described here and thus must be considered part of the same phenomenon.

Among some speakers in the community, these linkages are employed very frequently in myth narrations – sometimes more than a dozen times over the course of a brief five- or ten-minute \isi{narrative}, and many more times in longer narratives. The frequent use of these pause/\isi{repetition} sequences to structure the events and introduce new information creates a particular kind of \isi{narrative} rhythm that is a salient poetic characteristic of the myth narration discourse genre. The association between myth narrations and recapitulative linkages is so close that the one is rarely found without the other – even personal narratives about one’s own life or family history, which are similar in other respects to myth narrations, do not include them. Thus, while recapitulative linkages certainly contribute to discourse \isi{cohesion} – a common function of such constructions (see \citeauthor{guerin18}, this volume) – their exclusive association with the myth narration discourse genre suggests that they should be understood primarily as a poetic or stylistic feature of that genre.

Linkage constructions similar to the kind described in this chapter (also known as head-tail linkages or tail-head linkages, among other terms) have been identified in a number of indigenous Amazonian languages, particularly in Western \isi{Amazonia}. These include \ili{Cavineña} \citep{Guillaume2011}, \ili{Tariana} \citep[][169--171]{aikhenvald02}, \ili{Yurakaré} \citep{vangijn14}, \ili{Aguaruna} \citep{overall14}, \ili{Murui} \citep[][515--522]{kasia17}, and \ili{Ese Ejja} \citep[][598--599]{vuill12}. Note, however, that my analysis differs from these cases in focusing on the poetic function of such constructions in \ili{Matsigenka} (and in both \ili{Spanish} and mixed \ili{Matsigenka}-\ili{Spanish} speech). The ubiquity of linkage constructions across Western \isi{Amazonia} suggests that they might be an areal phenomenon attributable to \isi{language contact} \citep[][916]{seifart10}, as indeed we see in the transfer of such a construction from \ili{Eastern Tucanoan} to \ili{Tariana} in the Vaupés region \citep[][169--171]{aikhenvald02}. This would certainly be consistent with the proposal of \citet{beieretal.2002} that \isi{Amazonia} constitutes a ``discourse area,'' in which particular ways of speaking have diffused broadly across languages and language families in that region (though this notion has usually been applied to contact between indigenous languages instead of between indigenous and European colonial languages). However, linkage constructions are a common enough discourse strategy among the languages of the world (for instance, in Papuan languages; see \citealt{devries.2005}) that it may be difficult to distinguish the effects of areal diffusion from chance except in very clear cases.

There is a more specific sense in which the \ili{Matsigenka} linkage constructions discussed in this chapter are relevant to the \isi{topic} of \isi{language contact} – namely, that their regimentation by the myth narration discourse genre is what licenses their portability between languages (I use the linguistic anthropological senses of the terms \textit{regimentation} and \textit{discourse genre}; see \citealt{briggs.1992}; \citealt{silverstein93}; and \refsec{Emmyth-narr}). As young \ili{Matsigenka}-\ili{Spanish} bilinguals in the community have taken up interest in myths, they have begun to perform such narrations in \ili{Spanish} and in mixed \ili{Matsigenka}-\ili{Spanish} speech (though this is not as common as \ili{Matsigenka} narrations). When this happens, they use the very same kinds of linkage constructions as in the \ili{Matsigenka} narrations, even though this creates utterances that are considered unusual in \ili{Spanish} (see \refsec{Emmixedspeech}). I argue that because these recapitulative linkages are regimented by the local metapragmatic conventions of myth narration, they are also used when that discourse genre is invoked in a different lexico-grammatical code. In other words, since such linkages are understood to be part of a well executed myth performance, they are transferred to another language when speakers perceive themselves to be engaged in the same myth performance discourse genre in that language. While these \ili{Spanish} and mixed \ili{Matsigenka}-\ili{Spanish} performances are not considered exemplary of \ili{Matsigenka} verbal art, they often draw on other poetic conventions of \ili{Matsigenka} myth performance as well, including (among others) the frequent use of ideophones, reported speech and special voices, and a common set of \isi{prosodic} features and facial expressions for the indication of surprise, apprehension, and intensity. This case thus gives one example of how the effects of \isi{language contact} can be culturally (i.e. metapragmatically) mediated. However, as I mentioned earlier, this case is different from the kind of inter-indigenous \isi{language contact} commonly associated with an Amazonian discourse area. Furthermore, since myth narration is not practiced much among the younger generations, and since many \ili{Matsigenka} speakers are shifting to \ili{Spanish}, this contact feature is not likely to persist.
 
This chapter begins with an introduction to \ili{Matsigenka}, Andean \ili{Spanish}, and the discourse genre of myth narration on the Andean-Amazonian frontier of Southern Peru (\refsec{Emmyth-narr}). Then, in \refsec{Emformal}, I give a formal characterization of recapitulative linkages (\refsec{Embasic}), including relations between the reference clause and the bridging clause (\refsec{Emrelations}), and the composition of the second discourse unit (\ref{Em2ndunit}). In \refsec{Ematypical}, I discuss some atypical cases. Next, in \refsec{Emmixedspeech}, I go on to describe how the \ili{Matsigenka} recapitulative linkages discussed thus far are borrowed in \ili{Spanish} and mixed \ili{Matsigenka}-\ili{Spanish} performances of the same discourse genre. \refsec{Emconclus} offers some concluding comments.
%
\section{ Matsigenka, Spanish, and myth narration on the Andean-Amazonian frontier}
\label{Emmayth.narration}
\subsection{Languages and communities}
\label{Emland.comm}
\ili{Matsigenka} is an \ili{Arawak} language, of the \ili{Kampan} sub-group, spoken by a few thousand people in the Amazonian lowlands adjacent to the Southern Peruvian \isi{Andes} (for more on the classification of \ili{Matsigenka}, see \citealt{aikhenvald1999}; \citealt[][212--219]{michael08}; \citealt{michael10}; and \citealt{payne81}). Most speakers of \ili{Matsigenka} have at least some exposure to \ili{Spanish}, and many people in the Andean contact zone (as in the community described in this chapter) also speak Southern Peruvian \ili{Quechua} \citep{emlen.2017}. \ili{Matsigenka} is a head-marking language with a rich polysynthetic structure, and it uses verbal suffixes and enclitics, as well as a few prefixes and proclitics, for most of its grammatical functions. For more on the typological profile of the \ili{Kampan} languages, see \citet{michael08} and \citet{mihas15}. This chapter also discusses Andean \ili{Spanish}, a set of contact varieties spoken by millions of people across Western South America. Andean \ili{Spanish} features notable phonological and structural influence from \ili{Quechua} (for more, see \citealt[][593--595]{adelaar2004}; \citealt{babel18}; \citealt{cerron03}; \citealt{escobar03}). For more information about the heterogeneous forms of \ili{Spanish} in this area, see \citet{emlenforth}.
 
The community where these recordings were made occupies a small, remote hillside in the  {Alto Urubamba} Valley of Southern Peru, part of traditional \ili{Matsigenka} territory that abuts the \isi{Andes}. This region has been a conduit for the movement of goods, people, and languages between the \isi{Andes} and \isi{Amazonia} since the Inka period and likely long before \citep{Gade1972,camino.1977}. Today the {Alto Urubamba} is an agricultural frontier, and as the road network has expanded into \isi{Amazonia} since the 1950s, tens of thousands of \ili{Quechua}-speaking migrants from the \isi{Andes} have come to \ili{Matsigenka} territory in search of land for the cultivation of coffee and other tropical crops. This migratory wave has displaced many \ili{Matsigenka} people to remote corners of the valley, while others have intermarried with Andean settlers and joined the multiethnic agrarian society.

The community where this research was conducted came together in the 1980s and 1990s through the intermarriage of \ili{Matsigenka} people from across the region and Andean settlers from the nearby highlands. These people come from a wide variety of sociolinguistic backgrounds, and many are trilingual in \ili{Matsigenka}, \ili{Quechua}, and \ili{Spanish}. \ili{Matsigenka} and \ili{Quechua} are associated with domestic life and kin relations (depending on the family background), while \ili{Quechua} is used in interactions relating to the coffee economy and rural agrarian society. \ili{Spanish} is the language of the community’s political and institutional life. Most people can speak, or at least understand, all three languages. For more about how the three languages are used in the community, see \citet{emlen14,emlen.2015, emlen.2017}.
 
\subsection{Myth narration}
\label{Emmyth-narr}
Myth narration is one of many locally recognized discourse genres in the community. I mean the term \textit{discourse genre} both in the formal sense of ``constellations of co-occurrent formal elements and structures that define or characterize particular classes of utterances''  \citep[][141]{briggs.1992}, and in the metapragmatic sense of culturally constructed ``orienting frameworks, interpretive procedures, and sets of expectations'' \citep[][670]{hanks.1987} that regiment the production and interpretation of speech  \citep[see also][]{bakhtin86,silverstein93}.  

Myth narration is something of a specialized discursive skill in the community, and the oldest members who grew up beyond the coffee frontier and the Dominican missionary sphere are considered to be its most authoritative performers. These performances are usually relatively monologic, unlike in other places where they tend to be more dialogic \citep[e.g. among speakers of the nearby and closely related \ili{Nanti} language;][44]{michael08}. This is due in part to the fact that many young \ili{Matsigenka} speakers are shifting to \ili{Spanish} and \ili{Quechua} and are increasingly directing their attention to the rural agrarian social world instead of the cultural practices of their parents and grandparents. The performances usually take place at the home in the evening, and can last for hours, depending on the stamina and skill of the speaker and the engagement of the audience. Others are briefer, and last only a few minutes. The best performances (as judged by local metapragmatic standards) are quite long, feature virtuosic displays of creativity and improvisation, and are “keyed” \citep[see][]{goffman74,bauman77} – that is, signaled as instances of a particular discourse genre – by special formal and \isi{narrative} features. These features include frequent ideophones and other iconic phenomena, reported speech (often with special voices), a particular set of \isi{prosodic} features and facial expressions, cameos by characters from other myths that create intertextual links across the dense web of \ili{Matsigenka} cosmology, and the kind of \isi{narrative} rhythm that emerges from the frequent use of the bridging constructions discussed here. \ili{Matsigenka} myth narration in the community has come to be constructed around a language ideology that conceives of such discourse as an exemplary model \citep{kroskrity98} of traditional \ili{Matsigenka} language, culture, and knowledge, and it is generally subject to a regime of purism in which code-switching is discouraged (a fact that distinguishes it from all other domains of \ili{Matsigenka} language use in the community). 

However, during my field work in 2009--2012, \ili{Matsigenka} myths were occasionally performed in \ili{Spanish} and in mixed \ili{Matsigenka}-\ili{Spanish} speech, particularly by younger people who were interested in traditional \ili{Matsigenka} culture and were not deterred by the ideology of linguistic purism. These narrations usually came with disclaimers about their non-authoritativeness, and tended to offer a brief, \textit{just the facts} versions of the stories rather than the kind of lengthy, virtuosic performances described above. Some of these \ili{Spanish} and mixed \ili{Matsigenka}-\ili{Spanish} performances were given upon my request (sometimes to the puzzled amusement of older and more authoritative narrators), but many speakers also performed them among their friends and families, and in spaces of explicit cultural exposition such as community festivals and visits from municipal officials. Note that I never witnessed or successfully elicited a \ili{Matsigenka} myth in \ili{Quechua}, a language that is associated with a different tradition of verbal art, and that is understood by the local ideologies of language to be incompatible with explicit expressions of \ili{Matsigenka} culture. This is part of a larger tension in the conflicted and contested space of the agricultural frontier, where \ili{Quechua} and \ili{Matsigenka} are connected to opposite sides of an ethnically-inflected struggle over land and legitimacy, and where \ili{Spanish} represents a (relatively) unmarked common ground \citep[see][]{emlen.2015,emlen.2017}.

Most \ili{Matsigenka} myths tell a story of ``cosmological transformism'' \citep[][471]{castro98}, an ontological principle common in indigenous South American societies by which many animals, plants, and supernatural beings were once human before taking their current form, in which they now retain their essentially human subjectivity. This phenomenon has been described among \ili{Matsigenka} people by \citet{rosengren06} and \citet{johnson03}, among others. These are origin stories, but since the moment of transformation often hinges on a moral transgression of one or another character in the myth, they also serve as ``morality tales'' \citep[][118--124,220]{johnson03}  that warn \ili{Matsigenka} speakers about particular types of dangerous emotions or behavior \citep{izquierdo07,johnson99,rosengren00, shepard02}. \ili{Matsigenka} stories have been collected in translation and in \ili{Matsigenka} by anthropologists \citep[e.g.][]{baer94,renard04, renard91} and by missionaries \citep[e.g.][]{cenitayoga44,davis99}, usually as source of information regarding \ili{Matsigenka} culture and ontology rather than as a representation of the language and verbal art per se. However, a thorough recent compilation of 170 written \ili{Matsigenka} texts \citep{vargas13} gives a closer look at \ili{Matsigenka} linguistic structure and the verbal artistry associated with myths, as well as a rich perspective on \ili{Matsigenka} culture. However, those myths do not appear to exhibit the recapitulative linkages discussed in this chapter, either because of the particular sociolinguistic circumstances of the narrators, or because those myths were collected in written rather than oral form.

The data used in this chapter come from audio and video recordings of 35 myth narrations in the community, performed by seven people from a range of different ages and sociolinguistic backgrounds. These were collected over the course of 19 months of field work in 2009--2012. Additionally, 11 myth performances from speakers in five other communities in the {Alto Urubamba} were included in the corpus as a basis of regional comparison; however, only data from the community of focus are presented in this chapter. Some myths were told for me in my house, while others were recorded in the narrators’ homes as they performed the myths for their families. Several recordings were also made by \ili{Matsigenka} speakers themselves, whom I had trained to use the equipment in my absence. The use of the bridging constructions appears to be consistent across these contexts, and does not vary by the age or gender of the narrator. The 35 performances each ranged from several minutes to nearly an hour in length, and I identified a total of around 300 bridging constructions in the myth corpus. Note that these constructions also appear, using the same structures and in roughly the same frequency, in my recordings from across the {Alto Urubamba}, though I do not know how widespread they are beyond that region. For instance, bridging constructions following this pattern do not appear in \ili{Nanti} (Lev Michael, p.c.) nor in \ili{Caquinte} (Zachary O’Hagan, p.c.), two of the nearest \ili{Arawak} languages, and I have not noted similar constructions in the local variety of \ili{Quechua}. 

\subsection{Recapitulative linkages in myth narrations}
\label{Emrecap.myth}
By way of an example of bridging constructions in \ili{Matsigenka} myth performances, consider a passage from the \textit{pakitsa} `harpy eagle' myth, told in November 2011 by one of the community’s most authoritative practitioners of \ili{Matsigenka} verbal art. She told the story one evening to me and several of her family members, and it featured all of the elements of virtuosic performance mentioned above. In this sequence the \textit{pakitsa} `harpy eagle,' who had recently been transformed from a man into an eagle, swoops down upon the house of his human wife, daughter, and son (the man mentioned in (\ref{Emex:02a})). He snatches up his daughter, who had been walking around outside the house, and carries her off to his nest across the river. The sequence contains two bridging constructions, in (\ref{Em02ab}) and (\ref{Em03ab}). The passages in (\ref{Em02ab}) and (\ref{Em03ab}) are directly \isi{sequential} in the \isi{narrative}. 

The narrator first sets the tone of this scene in (\ref{Emex:02a}) by describing the mother, who is occupied by routine domestic work inside the house and is unaware of the fate that is about to befall her daughter. In (\ref{Emex:02b}), this context is restated in the bridging clause and linked to a description of the daughter’s vulnerable position outside the house (note that this case is unusual in linking clauses with different subjects). In this case, the bridging construction serves to express the simultaneous unwitting actions of the mother and the daughter, a calm scene that will be interrupted by the violent arrival of the \textit{pakitsa} in (\ref{Em03ab}).
\pagebreak

\begin{exe}
\ex \label{Em02ab}
\begin{xlist}
\ex \label{Emex:02a}
\glt \textit{\underline{\smash{Impogini otarogavagetake iroro}} oga irotyo iriniro yoga matsigenka}. [1.1]\\
\gll impogini o-tarog-a-vage-t-ak-i iroro o-oga iro-tyo iriniro i-oga matsigenka\\
 then \textsc{3f}-sweep\textsc{-ep-dur-ep-pfv-real} she \textsc{3f-}that she\textsc{-affect} his.mother \textsc{3m-that} person\\
\glt \sqt{Then she was sweeping, she, the mother of the man.}\\
\ex \label{Emex:02b}
\glt \textit{\textbf{Impogini otarogavageti}, inti oga oshinto anuvagetakeroka oga oga sotsiku}. [1.0]\\
\gll impogini o-tarog-a-vage-t-i i-nti o-oga o-shinto o-anu-vage-t-ak-i-roka o-oga o-oga sotsi-ku\\     	      
   then \textsc{3f-}sweep\textsc{-ep-dur-ep-real} \textsc{3m-cop} \textsc{3f-}that \textsc{3f-}daughter \textsc{3f-}walk\textsc{-dur-eu-pfv-real-epis.wk} \textsc{3f-}that \textsc{3f-}that outside\textsc{-loc}\\
\glt \sqt{Then she was sweeping, [and] her daughter must have been walking around, um, outside.}
\end{xlist}
\end{exe}


Then, in (\ref{Em03ab}), the eagle-man dives in and grabs his daughter, an abrupt turn of events that the narrator punctuates with a stark and deliberate 1.3 second pause. Once this development has been introduced, the narrator restates it in the bridging clause in (\ref{Emex:03b}) and links it to the \textit{pakitsa’s} next act of carrying the girl across the river to his nest. Both events are related as witnessed by the mother, which invites the listeners to contemplate the horror of such an experience. In (\ref{Em03ab}), the bridging construction allows the eagle-man’s sudden attack to stand alone in dramatic tension before it is restated to express continuity with the girl’s removal to the nest. 

\begin{exe}
\ex \label{Em03ab}
\begin{xlist}
\ex \label{Emex:03a}
\glt \textit{Okemiri maika yarapaake \underline{\smash{yagapanutiro pe oga oshinto otyomiani}}}. [1.3]\\
\gll o-kem-i-ri maika i-ar-apa-ak-i i-ag-apanu-t-i-ro pe o-oga o-shinto o-tyomia-ni\\
 \textsc{3f-}listen\textsc{-real-3m} now \textsc{3m-}fly\textsc{-all-pfv-real} \textsc{3m-}get\textsc{-dir:dep-ep-real-3f} \textsc{emph} \textsc{3m-}that \textsc{3f-}daughter \textsc{3f-}small\textsc{-anim}\\
\glt \sqt{She heard him [as] he flew in and he grabbed her young daughter.}\\
\ex \label{Emex:03b}
\glt \textit{\textbf{Yagapanutiro}, opampogiavakeri koa yarakaganake anta yovetsikakera ivanko intati anta}.\\
\gll i-ag-apanu-t-i-ro o-pampogi-av-ak-i-ri koa i-ar-akag-an-ak-i anta i-ovetsik-ak-i-ra i-panko intati anta \\     	      
   \textsc{3m-}get\textsc{-dir:dep-ep-real-3f} \textsc{3f-}watch\textsc{-tr-pfv-real-3m} more \textsc{3m-}fly\textsc{-caus-abl-pfv-real} there \textsc{3m-}make\textsc{-pfv-real-sbd} \textsc{3m-}house other.side there\\
\glt \sqt{He grabbed her, [as] [the mother] watched him, [and] he quickly flew her away to where he had made his house on the other side [of the river].}
\end{xlist}
\end{exe}
 
The effect of these constructions is to establish a \isi{narrative} rhythm through which the plot unfolds and information is introduced and elaborated (for another extended example, see (\ref{Em13ad}) below). This rhythm creates tension, suspense, and surprise in the \isi{narrative}, and (in the best performances) holds the listeners in rapt attention. In some myth narrations these bridging constructions appear between every two or three clauses – sometimes twice a minute or more – and this \isi{narrative} rhythm is only heard within such performances. Note that these constructions are not communicatively necessary, strictly speaking, for the functional purposes of discourse \isi{cohesion}; indeed, the discourse would be perfectly intelligible and easy to follow without them. Instead, these bridging constructions are oriented toward the poetic function of language, which, by Jakobson’s definition (\citeyear{jakobson60}), prioritizes the form of the message above its purely referential ends (particularly through the co-occurrence of formal features in a given stretch of discourse). Thus, this analysis follows the long linguistic anthropological tradition of research on verbal art and ethnopoetics (\citealt{bauman77,hymes81}; for a recent review, see \citealt{webster13}).
%
\section{ Formal characterization}
\label{Emformal}
\subsection{Basic template}
\label{Embasic}
This section gives a formal characterization of recapitulative linkages in \ili{Matsigenka} myth performances in the Andean-Amazonian frontier community. The basic template for these constructions is given in (\ref{Em04}): 
 

\begin{exe}
\ex \label{Em04}
\glt [...[\underline{Reference clause}]]\textsubscript{discourse unit}

\glt[0.5--4.0 second pause] 

\glt[[\textbf{Bridging clause}] [Discourse-new information]]\textsubscript{discourse unit}\\
\end{exe}



%vvvvvvvvvvvvggggggggggggggggggggggggggggggg  review \ref
Here, discourse units are understood as stretches of discourse that present particular events in the \isi{narrative}, and that are marked off by pauses and special intonational contours. In \isi{addition} to a 0.5--4.0 second pause between the discourse units, speakers sometimes utter a validating \textit{mmhmm} or \textit{aha}, as in (\ref{Em05ab}), and in (\ref{Em12ab}) below. These pauses are seen as appropriate moments for backchannel. In some of the recordings in the corpus that were made by native speakers of \ili{Matsigenka} themselves, a listener supplied the validating \textit{mmhmm} or \textit{aha} instead of the narrator (however, there are no cases in my data in which a listener repeats a reference clause). The example in (\ref{Em05ab}) is from a different speaker’s performance of the \textit{pakitsa} `harpy eagle' myth, and refers to the same events in (\ref{Em02ab}) and (\ref{Em03ab}) above. Note that the emphatic particle \textit{pe} in (\ref{Emex:05a}) comes from Andean \ili{Spanish} (for more, see \ref{Emmixedspeech}).


\begin{exe}
\ex \label{Em05ab}
\begin{xlist}
\ex \label{Emex:05a}
\glt \textit{\underline{\smash{Yamanakero pe}.} } [2.4]\\
\gll i-am-an-ak-i-ro pe\\
 \textsc{3m-}carry\textsc{-abl-pfv-real-3f} \textsc{emph}\\
\glt \sqt{He carried her away.}\\
\ex \label{Emex:05b}
\glt \textit{mmhmm}. [0.5]\\
\ex \label{Emex:05c}
\glt \textit{\textbf{Yamanakero} imenkotakara imperitaku}.\\
\gll i-am-an-ak-i-ro i-menko-t-ak-a-ra imperita-ku \\     	      
   \textsc{3m-}carry\textsc{-abl-pfv-real-3f} \textsc{3m-}make.nest\textsc{-ep-pfv-real-sbd} cliff\textsc{-loc}\\
\glt \sqt{He carried her away [to] where he had made his nest in the cliff. }
\end{xlist}
\end{exe}



In \isi{addition} to bridging constructions that take place in the narrator’s voice, the phenomenon also appears in the reported speech of characters in the \isi{narrative}, as in (\ref{Em06ab}): 

\begin{exe}
\ex \label{Em06ab}
\begin{xlist}
\ex \label{Emex:06a}
\glt \textit{Okantiro maika, ``noshinto, \underline{\smash{gaigakite nia}.''}} [1.1]\\
\gll o-kant-i-ro maika no-shinto n-ag-a-ig-aki-t-e nia \\
 \textsc{3f-}say\textsc{-real-3f} now \textsc{1-}daughter \textsc{irr-}get\textsc{-ep-pl-trnloc.pfv-ep-irr} water\\
\glt \sqt{She said to her, ``my daughter[s], go get water.''}\\
\ex \label{Emex:06b}
\glt \textit{``\textbf{Gaigakite nia} maika nontinkakera ovuroki.''}\\
\gll n-ag-a-ig-aki-t-e nia maika no-n-tink-ak-e-ra ovuroki\\     	      
   \textsc{irr-}get\textsc{-ep-pl-trnloc.pfv-ep-irr} water now \textsc{1-irr-}mash\textsc{-pfv-irr-sbd} masato\\
\glt \sqt{``Go get water, I’m going to mash up masato.''}
\end{xlist}
\end{exe}

%
Within the template given in (\ref{Em04}), bridging constructions can take a variety of forms. Linkages between the reference clause and the bridging clause are discussed in \refsec{Emrelations}; relationships between the bridging clause and the discourse-new information in the second discourse unit are discussed in \refsec{Em2ndunit}; and some atypical cases are described in \refsec{Ematypical}.
%
\subsection{Reference clause/bridging clause relations}
\label{Emrelations}
Before discussing the relationship between the reference clause and the bridging clause, it is necessary to first characterize typical reference clauses. These units are usually simple clauses (e.g. \textit{oaigake} `they went' in (\ref{Emex:01a})). However, it bears mentioning that in some cases, the reference unit itself is a more complex construction, as in the example in (\ref{Em07ab}). This case comprises a reference unit of two juxtaposed clauses (\ref{Emex:07a}) that are both repeated verbatim in the bridging clause (\ref{Emex:07b}). Such juxtapositions are common in \ili{Matsigenka} (see \refsec{Em2ndunit}).
 
\begin{exe}
\ex \label{Em07ab}
\begin{xlist}
\ex \label{Emex:07a}
\glt \textit{\underline{\smash{Agake omonkigakero}.}} [1.4]\\
\gll o-ag-ak-i o-monkig-ak-i-ro \\
 \textsc{3f-}get\textsc{-pfv-real} \textsc{3f-}carry.in.clothing\textsc{-pfv-real-3f}\\
\glt \sqt{She caught [it] [and] carried it in her cushma.}\\
\ex \label{Emex:07b}
\glt \textit{\textbf{Agake omonkigakero} sokaitakero oga shitatsiku}...\\
\gll o-ag-ak-i o-monkig-ak-i-ro sokai-t-ak-i-ro o-oga shitatsi-ku\\     	      
   \textsc{3f-}get\textsc{-pfv-real} \textsc{3f-}carry.in.clothing\textsc{-pfv-real-3f} dump.out\textsc{-ep-pfv-real-3f} \textsc{3f-}that mat\textsc{-loc}\\
\glt \sqt{She caught [it] [and] carried it [in her cushma], [and then] she dumped it out onto the mat...}
\end{xlist}
\end{exe}

 
Bridging clauses are usually verbatim repetitions of the reference clause – that is, \textit{recapitulative linkages} – as in (\ref{Em07ab}) and in most of the other examples given in this chapter. Summary linkages, in which the reference clause is referred to anaphorically with a summarizing verb rather than repeated (\citeauthor{guerin18}, this volume), do not appear. This is apparently because the construction’s poetic function is built on \isi{repetition}. However, in some cases the bridging clause presents a modified order or form of the information, or information is omitted, added, or substituted. For instance, in the passage from the first \textit{pakitsa} `harpy eagle' myth given in (\ref{Em02ab}) and (\ref{Em03ab}) above, the reference clause \textit{yagapanutiro pe oga oshinto otyomiani} `he grabbed her young daughter' (\ref{Emex:03a}), with its full direct object noun phrase, is shortened to \textit{yagapanutiro} `he grabbed her' (\ref{Emex:03b}). Similarly, in (\ref{Em08ab}) the adverbial \textit{inkenishiku} `in the forest' in the reference clause is omitted in the bridging clause:

\begin{exe}
\ex \label{Em08ab}
\begin{xlist}
\ex \label{Emex:08a}
\glt \textit{Iaigake \underline{\smash{imagavageigi inkenishiku}.}} [2.0]\\
\gll i-a-ig-ak-i i-mag-a-vage-ig-i inkenishi-ku \\
 \textsc{3m-}go\textsc{-pl-pfv-real} \textsc{3m-}sleep\textsc{-ep-dur-pl-real} forest\textsc{-loc}\\
\glt \sqt{They went [and] they slept in the forest.}\\
\ex \label{Emex:08b}
\glt \textit{\textbf{Imagavageigi} ipokaigai okutagitanake ikantiri ``tsame''}...\\
\gll i-mag-a-vage-ig-i i-pok-a-ig-a-i o-kutagite-t-an-ak-i i-kant-i-ri tsame\\     	      
   \textsc{3m-}sleep\textsc{-ep-dur-pl-real} \textsc{3m-}come\textsc{-ep-pl-dir:reg-real} \textsc{3f-}be.dawn\textsc{-ep-abl-pfv-real} \textbf{3m-}say\textsc{-real-3m} let’s.go\\
\glt \sqt{They slept [and then] they came back the next day, and he said to him, ``let’s go.''}
\end{xlist}
\end{exe}


%
Some information is omitted in the bridging clauses in (\ref{Emex:03b}) and (\ref{Emex:08b}), though they both retain enough similarity to the reference clauses to serve the poetic function of \isi{repetition}. Similarly, in (\ref{Em09ab}), the \ili{Spanish} reportative evidential particle \textit{dice} in the reference clause is omitted in the bridging clause, because it is unnecessary to mark the evidential status of the same information more than once in the same stretch of discourse \citep[for a similar case in \ili{Sunwar}, see][392]{schulze73}.\footnote{This reportative evidential particle, which has been borrowed from \ili{Spanish} into \ili{Matsigenka} in some parts of the {Alto Urubamba}, is common in some varieties of Andean \ili{Spanish} \citep[as well as its variant \textit{dizque}; see][]{babel.2009}.} 


\begin{exe}
\ex \label{Em09ab}
\begin{xlist}
\ex \label{Emex:09a}
\glt \textit{ \underline{\smash{Itentaigari dice}.}} [1.8]\\
\gll i-tent-a-ig-a-ri dice \\
 \textsc{3m-}accompany\textsc{-ep-pl-real-3m} \textsc{evid.rep}\\
\glt \sqt{He brought him along, they say.}\\
\ex \label{Emex:09b}
\glt \textit{\textbf{Itentaigari} ya itasonkake...}\\
\gll i-tent-a-ig-a-ri ya i-tasonk-ak-i\\     	      
   \textsc{3m-}accompany\textsc{-ep-pl-real-3m} at.that.point \textsc{3m-}blow.on\textsc{-pfv-real}\\
\glt \sqt{He brought him along, and then he blew [on him]...}
\end{xlist}
\end{exe}


A case of \isi{substitution} can be seen in the \ili{Spanish} example in (\ref{Em15ab}) below, whereby the reference clause \textit{sigue caminando} `she kept walking' is restated in the bridging clause as \textit{sigue avanzando} `she kept moving forward'. Such lexical substitutions, however, are uncommon. 
%
\subsection{Relations within the second discourse unit}
\label{Em2ndunit}
Relations within the second discourse unit – that is, between the bridging clause and the discourse-new information that follows it – can take a number of forms. As discussed above, the second discourse unit often expresses simultaneity or immediate temporal continuity between the \isi{action} in the reference/bridging clause and a discourse-new proposition, as in `he flew away' and `he went into the forest in order to hunt' in (\ref{Em10ab}): 

\begin{exe}
\ex \label{Em10ab}
\begin{xlist}
\ex \label{Emex:10a}
\glt \textit{Oneiri \underline{\smash{yaranake}.}} [2.1]\\
\gll o-ne-i-ri i-ar-an-ak-i\\
 \textsc{3f-}see\textsc{-real-3m} \textsc{3m-}fly\textsc{-abl-pfv-real}\\
\glt \sqt{She saw him [as] he flew away.}\\
\ex \label{Emex:10b}
\glt \textit{\textbf{Yaranake} iatake inkenishiku anta inkovintsatera iriro aikiro irityo pakitsa.}\\
\gll i-ar-an-ak-i i-a-t-ak-i inkenishi-ku anta i-n-kovintsa-t-e-ra iriro aikiro iri-tyo pakitsa\\     	      
   \textsc{3m-}fly\textsc{-abl-pfv-real} \textsc{3m-}go\textsc{-ep-pfv-real} forest\textsc{-loc} there \textsc{3m-irr-}hunt\textsc{-ep-irr-sbd} he also he\textsc{-affect} harpy.eagle\\
\glt \sqt{He flew away [and] went into the forest in order to hunt, the harpy eagle too.}
\end{xlist}
\end{exe}



Often, the bridging clause and discourse-new clause are simply linked as juxtaposed (or apposite) clauses, with no subordinating morphology. This is a common means of clause-linking in \ili{Matsigenka} and other \ili{Kampan} languages \citep[e.g.,][435]{michael08}. This can be seen in several of the examples given so far, including (\ref{Emex:10b}).

The expression of continuity and immediate temporal succession between two actions most often refers to the actions of a single character; for this reason, the subject of the reference/bridging clause and the subject of the discourse-new clause in the second discourse unit are usually the same. However, speakers sometimes express such a link between the actions of two different characters, as in sentence (\ref{Emex:03a}) above: \textit{impogini otarogavageti, inti oga oshinto anuvagetakeroka oga oga sotsiku} `Then she was sweeping, [and] her daughter must have been walking around, um, outside'. \ili{Matsigenka} does not mark \isi{switch reference} morphologically, and the change in subjects is simply expressed through person marking. 

But while the \ili{Matsigenka} bridging constructions described here usually express continuity and quick temporal succession between two actions, in other cases the discourse following the bridging clause instead offers an additional clarification or elaboration of the first \isi{action}. For instance, in example (\ref{Em11ac}), the discourse-new information in the second discourse unit is the reported utterance \textit{ipokai piri} `your father came back' (\ref{Emex:11c}), which clarifies what one man called out to another man in the reference clause (\ref{Emex:11a}):

\begin{exe}
\ex \label{Em11ac}
\begin{xlist}
\ex \label{Emex:11a}
\glt \textit{\underline{\smash{Ikaemakotapaakeri}.} } [1.8]\\
\gll i-kaem-ako-t-apa-ak-i-ri\\
 \textsc{3m-}call\textsc{-appl-ep-all-pfv-real-3m}\\
\glt \sqt{He called out to him.}\\
\ex \label{Emex:11b}
\glt \textit{mmhmm}. [0.3]\\
\ex \label{Emex:11c}
\glt \textit{\textbf{Ikaemakotapaakeri} ``ipokai piri.''}\\
\gll i-kaem-ako-t-apa-ak-i-ri i-pok-a-i piri\\     	      
   \textsc{3m-}call\textsc{-appl-ep-all-pfv-real-3m} \textsc{3m-}come\textsc{-dir:reg-real} your.father\\   
\glt \sqt{He called out to him, ``your father came back.''}
\end{xlist}
\end{exe}
%
Similarly, in (\ref{Em05ab}) discussed above, the clause \textit{yamanakero} `he carried her away' (\ref{Emex:05a}) is clarified by the additional discourse-new information \textit{imenkotakara imperitaku} `[to] where he had made his nest in the cliff' (\ref{Emex:05c}), marked with the subordinator \textit{-ra}. In such cases, the discourse-new information is linked to the reference/bridging clauses through a broader range of constructions than just the simple juxtapositions described above; however, this is less common.
%
\subsection{Some atypical cases}
\label{Ematypical}
It is important to note here two related variations of this poetic phenomenon that do not fall under the category of inter-clausal bridging constructions per se. First, in some cases a reference clause is simply repeated in a second discourse unit, within the same stylistic parameters described above, but is not linked to any discourse-new information at all, as in (\ref{Em12ab}). Such cases are therefore not bridging construction at all, but since they follow the same poetic structure, they thus must be considered in the same analysis. Note that the second discourse unit (\ref{Emex:12b}) differs from the reference clause (\ref{Emex:12a}) only by fronting the object, creating a pre-verbal focus construction \citep[][385]{michael08}.


\begin{exe}
\ex \label{Em12ab}
\begin{xlist}
\ex \label{Emex:12a}
\glt \textit{ \underline{\smash{Yagaigake aryopaturika chakopi}.}} [1.3]\\
\gll i-ag-a-ig-ak-i aryopaturika chakopi\\
 \textsc{3m-}grab\textsc{-ep-pl-pfv-real} large.(sheaf) arrow\\
\glt \sqt{They grabbed a big sheaf of arrows.}\\
\ex \label{Emex:12b}
\glt \textit{\textbf{Aryopaturika chakopi yagaigake.}}\\
\gll aryopaturika chakopi i-ag-a-ig-ak-i \\     	      
   large.(sheaf) arrow \textsc{3m-}grab\textsc{-ep-pl-pfv-real}\\
\glt \sqt{A big sheaf of arrows, they grabbed.}
\end{xlist}
\end{exe}

A second variation is a kind of construction in which the reference unit does not contain a verb at all, but is still an instance of the same poetic pattern discussed in this chapter. For instance, passage (\ref{Em13ad}) includes an ideophone \textit{kong kong} `whistle sound' that serves as a reference unit linking (\ref{Emex:13a}) and (\ref{Emex:13c}). The linkage in (\ref{Emex:13c}) reestablishes the flow of the \isi{narrative} after it is interrupted by a clarifying digression in (\ref{Emex:13b}). Note that the bridging discourse unit is followed by another, canonical bridging construction (\ref{Emex:13c} and \ref{Emex:13d}).


\begin{exe}
\ex \label{Em13ad}
\begin{xlist}
\ex \label{Emex:13a}
\glt \textit{ Okemake isonkavatapaake \underline{\smash{kong kong}.}} [1.0]\\
\gll o-kem-ak-i i-sonkava-t-apa-ak-i kong kong\\
 \textsc{3f-}hear\textsc{-pfv-real} \textsc{3m-}whistle\textsc{-ep-all-pfv-real} whistle.sound whistle.sound\\
\glt \sqt{She heard him whistle, kong kong.}\\
\ex \label{Emex:13b}
\glt \textit{ Tera iravise ampa ipokapaake aka pankotsiku.} [3.6]\\
\gll tera i-r-avis-e ampa i-pok-apa-ak-i aka panko-tsi-ku\\
 \textsc{neg.real} \textsc{3m-irr-}approach\textsc{-irr} bit.by.bit \textsc{3m-}come\textsc{-all-pfv-real} here house\textsc{-alien-loc}\\
\glt \sqt{He didn’t approach [the house], he came slowly to the house.}\\
\ex \label{Emex:13c}
\glt \textit{ \textbf{Kong kong} \underline{\smash{yogonketapaaka}.}} [2.4]\\
\gll kong kong i-ogonke-t-apa-ak-a\\
 whistle.sound whistle.sound \textsc{3m-}arrive\textsc{-ep-all-pfv-real}\\
\glt \sqt{Kong kong, [and] he arrived.}\\

\pagebreak

\ex \label{Emex:13d}
\glt \textit{\textbf{Yogonketapaaka} ikaemakotapaakero.}\\
\gll  i-ogonke-t-apa-ak-a i-kaem-ako-t-apa-ak-i-ro \\     	      
   \textsc{3m-}arrive\textsc{-ep-all-pfv-real} \textsc{3m-}call\textsc{-appl-ep-all-pfv-real-3f}\\
\glt \sqt{He arrived [and] he called out to her.}
\end{xlist}
\end{exe}


%
\section{ Spanish and mixed Spanish-Matsigenka speech}
\label{Emmixedspeech}
As I discussed in \refsec{Emmayth.narration}, \ili{Matsigenka} myths are usually performed in \ili{Matsigenka} with very little code-switching in \ili{Spanish} (though a number of other \ili{Spanish} discourse features, including the reportative evidential particle \textit{dice} (\ref{Emex:09a}), and the emphatic particle \textit{pues} or \textit{pe} (\ref{Emex:03a}), (\ref{Emex:05a}), often pass below the threshold of a speaker’s awareness). However, because of the community’s complex sociolinguistic constitution, ongoing language shift, and uneven distribution of discursive skills, the narration of \ili{Matsigenka} myths in \ili{Spanish} or in mixed \ili{Matsigenka}-\ili{Spanish} speech has become more common. This is particularly true among young people who wish to engage with traditional \ili{Matsigenka} culture, but who do not feel that they possess the requisite \ili{Matsigenka} language competence. These performances are strictly distinguished from the monolingual \ili{Matsigenka} performances discussed so far in this chapter, which are considered authoritative and culturally exemplary.

What is interesting about these \ili{Spanish} and mixed \ili{Spanish}-\ili{Matsigenka} performances is that they usually employ the same poetic and stylistic features that ``key'' the discourse genre of \ili{Matsigenka} myth performance \citep[in the sense of][]{goffman74}, including ideophones, \isi{prosodic} and facial expressions, reported speech, and bridging linkages. That is, once a narrator ``breaks through'' into full performance \citep{hymes75}, the metapragmatic conventions of \ili{Matsigenka} myth narration – that is, the local cultural expectations about what makes a ``good story'' – can be applied in \ili{Spanish} as well. 

For instance, consider the mixed \ili{Matsigenka}-\ili{Spanish} example in (\ref{Em14ab}). This young narrator acquired a great deal of cultural information while listening to his mother perform \ili{Matsigenka} myths over the course of his childhood, and he enjoys listening to such performances for hours on end; but while he cares deeply about \ili{Matsigenka} stories, he is not comfortable performing them entirely in \ili{Matsigenka}. He recorded himself recounting the story of the \textit{oshetoniro} demon to his wife one evening in their home while I rested outside: 
 
\pagebreak
\begin{exe}
\ex \label{Em14ab}
\begin{xlist}
\ex \label{Emex:14a}
\glt \textit{ Al medio se ha ido la canoa y \underline{\smash{se ha hundido pe ese oshetoniro}.}} [1.3]\\
\gll al medio se ha ido la canoa y se ha hundido pe ese oshetoniro\\
 \textsc{prep+det.def.m.sg} center \textsc{refl} have\textsc{.3sg.prs} go\textsc{.pst.ptcp} \textsc{det.def.f.sg} canoe and \textsc{refl} have\textsc{.3sg.prs} sink\textsc{.pst.ptcp} \textsc{emph} that\textsc{.adj.dem.m.sg} oshetoniro.demon\\
\glt \sqt{The canoe went out into the center (of the river) and that oshetoniro demon sank.}\\
\ex \label{Emex:14b}
\glt \textit{\textbf{Se ha hundido pe} mataka ya está maika yokaataka.}\\
\gll se ha hundido pe mataka ya está maika i-okaa-t-ak-a\\     	      
    \textsc{refl} have\textsc{.3sg.prs} sink\textsc{.pst.ptcp} \textsc{emph} that’s.it already be\textsc{.3sg.prs} now \textsc{3m-}drown\textsc{-ep-pfv-real}\\
\glt \sqt{He sank, that’s it, that’s it, he drowned.}
\end{xlist}
\end{exe}

Here, the reference clause in (\ref{Emex:14a}), \textit{se ha hundido pe ese oshetoniro} `that oshetoniro demon sank', is in \ili{Spanish} (except for the name of the demon itself), and it is recapitulated in the bridging clause with the subject omitted: \textit{se ha hundido pe} `he sank'. The code switch to \ili{Matsigenka} appears at the beginning of the discourse-new information in the second discourse unit in (\ref{Emex:14b}) (\textit{mataka ya está maika yokaataka} `that’s it, that’s it, he drowned'), directly after the bridging clause. It is significant that the reference clause and the bridging clause are the parts of the discourse that coincide in language choice: the poetic function of the constructions discussed in this chapter depends on the latter’s similarity with the former, so we would expect them to be in the same language. It is not until immediately after the \isi{repetition} of the reference clause that the narrator switches to \ili{Matsigenka}.

Another example comes from a performance by the same man’s wife (\ref{Em15ab}): 

\begin{exe}
\ex \label{Em15ab}
\begin{xlist}
\ex \label{Emex:15a}
\glt \textit{\underline{\smash{Sigue caminando}.}} [2.1]\\
\gll sigue caminando\\
 continue\textsc{.3sg.prs} walk\textsc{.prs.ptcp}\\
\glt \sqt{She kept walking.}\\
\ex \label{Emex:15b}
\glt \textit{\textbf{Sigue avanzando} oneapaakeri timashitake grande ya pe imaarane.}\\
\gll sigue avanzando o-ne-apa-ak-i-ri timashi-t-ak-i grande ya pe i-maarane\\     	      
    continue\textsc{.3sg.prs} go.forward\textsc{.prs.ptcp} \textsc{3f-}see\textsc{-all-pfv-real-3m} sneak.up.on\textsc{-ep-pfv-real} big already \textsc{emph} \textsc{m-}big\\
\glt \sqt{She kept going forward [and] she saw [it] sneaking up on her, a big one, a really big one.}
\end{xlist}
\end{exe}


Again here, the code switch from \ili{Spanish} to \ili{Matsigenka} in (\ref{Emex:15b}) takes place after the reference clause is recapitulated in the bridging clause, with the introduction of the discourse-new information. Note also that just as in most of the \ili{Matsigenka} examples given so far, the two propositions in the second discourse unit are linked as simple juxtaposed clauses (\textit{sigue avanzando oneapaakeri} `She kept going forward [and] she saw [it]'), which would be considered unusual in \ili{Spanish}. However, unlike in (\ref{Em14ab}), the verb \textit{caminar} `to walk' in the reference clause is substituted with the verb \textit{avanzar} `to go forward'. This \isi{substitution}, in a parallel construction following \textit{sigue}... `she kept...', was similar enough to serve the poetic purposes of the linkage.\footnote{When \ili{Matsigenka}/\ili{Spanish} bilinguals speak \ili{Spanish}, they often use present tense marking to express past events.}
 
 
In \isi{addition} to these examples of bridging linkages that feature \ili{Matsigenka}-\ili{Spanish} code-switching, we also find examples in myths performed entirely in \ili{Spanish}. For instance, one woman told a story to a group of family members, children, and visitors who did not speak \ili{Matsigenka} (\ref{Em16ab}):


\begin{exe}
\ex \label{Em16ab}
\begin{xlist}
\ex \label{Emex:16a}
\glt \textit{La había cogido y  \underline{la había tetado.}} [0.9]\\
\gll la había cogido y la había tetado\\
 her\textsc{.pn.obj.f.3sg} have\textsc{.3sg.pst} pick.up\textsc{.pst.ptcp} and her\textsc{.pn.obj.f.3sg} have\textsc{.3sg.pst} nurse\textsc{.pst.ptcp}\\
\glt \sqt{She picked up [the baby] and she nursed her.}\\
\ex \label{Emex:16b}
\glt \textit{\textbf{La había tetado} entonces la ha empezado a coger...}\\
\gll la había tetado entonces la ha empezado a coger\\     	      
    her\textsc{.pn.obj.f.3sg} have\textsc{.3sg.pst} nurse\textsc{.pst.ptcp} then it\textsc{.pn.obj.f.3sg} have\textsc{.3sg.prs} begin\textsc{.pst.ptcp} \textsc{to} take\textsc{.inf}\\
\glt \sqt{She nursed her, and then [the baby] began to take [the breast]...}
\end{xlist}
\end{exe}


%
As in many of the examples given so far in this chapter, the reference clause in (\ref{Emex:16a}) is repeated verbatim in the bridging clause; however, in this case the bridging clause is linked to the discourse-new information in the second discourse unit (\ref{Emex:16b}) by a \isi{conjunction} \textit{entonces} `then', a more familiar construction in \ili{Spanish} than the simple juxtaposed clauses above. As in other cases throughout this chapter, the reference clause in (\ref{Emex:16a}) was produced with falling \isi{intonation}, and the bridging clause was produced with rising \isi{intonation} to signal that the proposition would be followed by discourse-new information.


Another example from a \ili{Spanish} performance of a \ili{Matsigenka} myth comes from the same narrator (\ref{Em17ab}). More information about the variety of Andean \ili{Spanish} spoken in the community is available in \citet{emlenforth}.
%

\begin{exe}
\ex \label{Em17ab}
\begin{xlist}
\ex \label{Emex:17a}
\glt \textit{Así se habrá echado pues así,  \underline{\smash{y de su pie le ha empezado a tragarle pe}.}} [1.0]\\
\gll así se habrá echado pues así y de su pie le ha empezado a tragarle pe \\
 like.that self\textsc{.pn.refl.3} have\textsc{.3sg.fut} lie.down\textsc{.pst.ptcp} \textsc{emph} like.that and from her foot him\textsc{.pn.obl.3sg} have\textsc{.3sg.prs} begin\textsc{.pst.ptcp} to swallow\textsc{.inf+pn.3sg} \textsc{emph}\\
\glt \sqt{She must have laid down like that, and it began swallowing her from her foot.}\\
\ex \label{Emex:17b}
\glt \textit{\textbf{De su pie le ha empezado a tragar}, ha llegado hasta acá.}\\
\gll de su pie le ha empezado a tragar ha llegado hasta acá\\     	      
    from her foot him\textsc{.pn.obl.3sg} have\textsc{.3sg.prs} begin\textsc{.pst.ptcp} to swallow\textsc{.inf} have\textsc{.3sg.prs} arrive\textsc{.pst.ptcp} until here\\
\glt \sqt{It began swallowing her from her foot, [and] it got this far.} [Points to leg with finger.] 
\end{xlist}
\end{exe}


Here, the reference clause in (\ref{Emex:17a}) is repeated nearly verbatim in (\ref{Emex:17b}), with the exception of the emphatic particle \textit{pe}, which is omitted in the bridging clause, and the object enclitic \textit{le} `her' at the end of the infinitive verb \textit{tragar} `to swallow'. However, in this case the speaker does not use a \isi{conjunction} between the bridging clause and the discourse-new information, but rather uses the typically \ili{Matsigenka} juxtaposed verb construction in (\ref{Emex:17b}).
%

\section{ Conclusion}
\label{Emconclus}
This chapter presented a type of bridging construction that is ubiquitous in the narration of \ili{Matsigenka} myths in a small community on the Andean-Amazonian agricultural frontier of Southern Peru. The construction appears primarily in \ili{Matsigenka} language discourse, but it is also heard in \ili{Spanish} and in mixed \ili{Spanish}-\ili{Matsigenka} performances of the same genre. While these constructions surely contribute to discourse \isi{cohesion}, they must be understood primarily as a poetic feature distinctive to the discourse genre of myth narration.

The fact that these constructions are a property of the myth narration discourse genre – rather than of a particular lexico-grammatical code – means that they can be transferred from one language to another (in this case, \ili{Spanish}) when that genre is invoked. In fact, they must be transferred, to the extent that they are considered by the local metapragmatic standards to be an essential part of successful myth performance. In other words, because these constructions are limited to the genre of myth narration but cross-cut languages, they should be understood not as a property of the \ili{Matsigenka} language per se, but rather of the myth narration discourse genre – which may also cross-cut languages. The fact that the metapragmatic regimentation of discourse genres enables the circulation of features across languages shows how discourse areas might emerge from local cultures of language \citep[as in \isi{Amazonia};][]{beieretal.2002}, and it also illustrates how contact-induced language change can be mediated by locally meaningful categories of discursive behavior \citep[i.e., ‘culture’;][]{silverstein76}. This case thus supports the proposition that \textit{\isi{language contact} is culturally mediated}. However, this contact effect is only as stable as the community’s multilingualism, and it will likely not long outlast the language shift from \ili{Matsigenka} to \ili{Spanish} currently under way in the community.
%



%
\section*{Appendix}
 \setcounter{equation}{0}
Excerpt of Pakitsa (Harpy Eagle) story, {Alto Urubamba} \ili{Matsigenka}, November 2011. Analyzed by Nicholas Q. Emlen and Julio Korinti Piñarreal.\\

This narration of the \ili{Matsigenka} \textit{pakitsa} `harpy eagle' story was recorded in November 2011 in the {Alto Urubamba} region of Southern Peru. The narrator (whose name is withheld per the arrangement with the community) grew up speaking \ili{Matsigenka} and, to a lesser degree, \ili{Spanish}. She lived in various places across the {Alto Urubamba} Valley as \ili{Quechua}-speaking coffee farmers gradually colonized the region since the 1950s, and she lived for a brief time as an adult in a nearby Dominican mission. More information about this history and sociolinguistic situation can be found in \citet{emlen14,emlen.2015,emlen.2017,emlenforth}.

The \textit{pakitsa} story is popular across the region, and deals with themes of incest and cannibalism. The harpy eagle is a renowned hunter, which is a recurrent part of this story. A summary of this version of the story is excerpted from \citet[][255--256]{emlen14}: ``a man requests fermented yuca beer from his wife before going out to burn his \textit{chacra} for planting. However, the night before his son had had a dream that his father would become too drunk and be killed in the fire, so he warned his mother not to give him too much beer. But the man drank too much and was burned up in the fire. The son reprimanded his mother and instructed her to wake him up if the man appeared at the door of the house during the night—his body would be composed of ash, and a small amount of water would restore him. When the man appeared, the mother did not wake up her son, but rather threw an excessive quantity of water on her husband, disintegrating him into a puddle of ash on the ground. The ash that remained became the \textit{pakitsa} `harpy eagle' (with its distinctive puffy, ash-like white feathers around its neck).'' 

The excerpt below picks up at this point in the story. Here, the \textit{pakitsa}-man abducted his daughter and impregnated her. After this excerpt, \citet[][256]{emlen14} continues, the man and his daughter ``lived together in his nest and became cannibals. The \textit{pakitsa}-man was eventually killed while hunting for humans, and upon hearing of his death, his daughter ate their newborn son and disappeared into a river to join the mythical tribe of cannibalistic female \textit{maimeroite} warriors.'' 

The story, which lasted about sixteen minutes in total, was considered an exemplary instance of myth narration. Recapitulative linkages are indicated with underlined and bolded text, as in the accompanying chapter. The morpheme glossing conventions mostly follow \citet{vargas13}, which is the most complete accounting of {Alto Urubamba} \ili{Matsigenka} morphology to date. However, a full descriptive grammar of \ili{Matsigenka} remains to be written, and some of the morphemic analyses are preliminary.


\begin{exe}
\ex \label{Emapp01}
\glt \textit{Impo ikimotanake yoga pakitsa aryompa aryompa yantavankitanake.}\\
\gll impo i-kimo-t-an-ak-i i-oga pakitsa aryompa aryompa i-anta-vanki-t-an-ak-i\\
then \textsc{3m-}grow\textsc{-ep-abl-pfv-real} \textsc{3m-}that harpy.eagle gradually gradually \textsc{3m-}mature-\textsc{ni:}wing\textsc{-ep-abl-pfv-real}\\
\glt \sqt{Then the eagle grew bit by bit, [and] his wings matured.}\\
\end{exe}


\begin{exe}
\ex \label{Emapp02}
\glt \textit{Impogini maika iatake ikovintsavagetakera otomi anta iaigake yanuvageigakitira.}\\
\gll impogini maika i-a-t-ak-i i-kovintsa-vage-t-ak-i-ra o-tomi anta i-a-ig-ak-i i-anu-vage-ig-aki-t-i-ra\\
then now \textsc{3m-}go\textsc{-ep-pfv-real} \textsc{3m-}hunt\textsc{{}-dur-ep-pfv-real-sbd} \textsc{3f-}son there \textsc{3m-}go\textsc{{}-pl-pfv-real} \textsc{3m-}walk\textsc{{}-dur-pl-assoc.mot:dist-ep-real-sbd}\\
\glt \sqt{Then her sons went to hunt, they went on hunting trips.}\\
\end{exe}
 
\begin{exe}
\ex \label{Emapp03}
\glt \textit{Iatake yagaigi komaginaro inti iriro kishiatanatsi anta pankotsiku.}\\
\gll i-a-t-ak-i i-ag-a-ig-i komaginaro i-nti iriro kishia-t-an-ats-i anta panko-tsi-ku\\
\textsc{3m-}go\textsc{{}-ep-pfv-real} \textsc{3m-}get\textsc{{}-ep-pl-real} monkey.species \textsc{3m-cop} \textsc{3m.pro} comb\textsc{{}-ep-abl-subj.foc-real} there house\textsc{{}-alien-loc}\\
\glt \sqt{He went and caught monkeys, and [the eagle] kept combing [his feathers] at the house.}\\
\end{exe}
 
\begin{exe}
\ex \label{Emapp04}
\glt \textit{Okantiri maika ``kishiatanatsivi maika pinkovintsatakitera pinkovintsatakitera komaginaro anta  onkimotanakera pinampina irokona irokona pashi'' okantakerira.}\\
\gll o-kant-i-ri maika kishia-t-an-ats-i-vi maika pi-n-kovintsa-t-aki-t-e-ra pi-n-kovintsa-t-aki-t-e-ra komaginaro anta o-n-kimo-t-an-ak-i-ra pi-nanpina iro-kona iro-kona pi-ashi o-kant-ak-i-ri-ra\\
\textsc{3f-}say\textsc{{}-real-3m} now comb\textsc{{}-ep-abl-subj.foc-real-2} now \textsc{2-irr-}hunt\textsc{{}-ep-assoc.mot:dist-ep-irr-sbd} \textsc{2-irr-}hunt\textsc{{}-ep-assoc.mot:dist-ep-irr-sbd} monkey.species there \textsc{3f-irr-}grow\textsc{{}-ep-abl-pfv-irr-sbd} 2-side \textsc{3f.pro-incr} \textsc{3f.pro-incr} \textsc{2-poss} \textsc{3f-}say\textsc{{}-pfv-real-3m-sbd}\\
\glt \sqt{Then she said to him, ``you keep on combing yourself, today you have to go hunting, you have to go hunt a monkey, so that your partner will grow a little bit'' she said to him.}\\
\end{exe}
 
\begin{exe}
\ex \label{Emapp05}
\glt \textit{Ipotevankitanake}\\
\gll i-pote-vanki-t-an-ak-i\\
\textsc{3m-}flap\textsc{{}-ni:}wing\textsc{{}-ep-abl-pfv-real}\\
\glt \sqt{He flapped his wings.}\\
\end{exe}
  
\pagebreak
\begin{exe}
\ex \label{Emapp06}
\glt \textit{Oneiri \underline{\smash{yaranake}.}}\\
\gll o-ne-i-ri i-ar-an-ak-i \\
\textsc{3f-}see\textsc{{}-real-3m} \textsc{3m-}fly\textsc{{}-abl-pfv-real}\\
\glt \sqt{She saw him [as] he flew away.}\\
\end{exe}
 
\begin{exe}
\ex \label{Emapp07}
\glt \textit{\textbf{Yaranake} iatake inkenishiku anta inkovintsatera iriro aikiro irityo pakitsa.}\\
\gll i-ar-an-ak-i i-a-t-ak-i inkenishi-ku anta i-n-kovintsa-t-e-ra iriro aikiro iri-tyo pakitsa\\
\textsc{3m-}fly\textsc{{}-abl-pfv-real} \textsc{3m-}go\textsc{{}-ep-pfv-real} forest\textsc{{}-loc} there \textsc{3m-irr-}hunt\textsc{{}-ep-irr-sbd} he also he\textsc{{}-affect} harpy.eagle\\
\glt \sqt{He flew away [and] went into the forest in order to hunt, the harpy eagle too.}\\
\end{exe}
 
\begin{exe}
\ex \label{Emapp08}
\glt \textit{Iaigi itomiegi aikiro ikovintsaigi yagaigi yamaigi komaginaro ikanti ``neri ina komaginaro kote  sekataigakempara.''}\\
\gll i-a-ig-i i-tomi-egi aikiro i-kovintsa-ig-i i-ag-a-ig-i i-am-a-ig-i komaginaro i-kant-i neri ina komaginaro n-onko-t-e Ø-n-sekat-a-ig-ak-empa-ra\\
\textsc{3m-}go\textsc{{}-pl-real} \textsc{3m-}son\textsc{{}-pl} also \textsc{3m-}hunt\textsc{{}-pl-real} \textsc{3m-}get\textsc{{}-ep-pl-real} \textsc{3m-}bring\textsc{{}-ep-pl-real} monkey.species \textsc{3m-}say\textsc{{}-real} take.it my.mother monkey.species \textsc{irr-}cook\textsc{{}-ep-irr} \textsc{1.incl-irr-}eat\textsc{{}-ep-pl-pfv-irr-sbd}\\
\glt \sqt{His sons also went to hunt, they caught and brought a monkey, they said ``take the monkey, mother, cook it so that we can eat.''}\\
\end{exe}
 
\begin{exe}
\ex \label{Emapp09}
\glt \textit{Inti iriro yami yovuokiri en kapashipankoku yoginoriiri yoga yashiriapaaka.}\\
\gll i-nti iriro i-am-i i-ovuok-i-ri en kapashi panko-ku i-ogi-nori-i-ri i-oga i-ashiri-apa-ak-a\\
\textsc{3m-cop} \textsc{3m.pro} \textsc{3m-}carry\textsc{{}-real} \textsc{3m-}drop\textsc{{}-real-3m} in palm.species house\textsc{{}-loc} \textsc{3m-caus-}lie.down\textsc{{}-real-3m} \textsc{3m-}that \textsc{3m-}fall\textsc{{}-adl-pfv-real}\\
\glt \sqt{He brought it, he dropped it on top of the thatched-roof house and made it stand up there, he made it fall down on top.}\\
\end{exe}
 
\begin{exe}
\ex \label{Emapp10}
\glt \textit{Agiri onkotakeri aikiro iriro iriro aikiro iati ikovintsatira iriro aikiro pakitsa.}\\
\gll o-ag-i-ri o-onko-t-ak-i-ri aikiro iriro iriro aikiro i-a-t-i i-kovintsa-t-i-ra iriro aikiro pakitsa\\
\textsc{3f-}get\textsc{{}-real-3m} \textsc{3f-}cook\textsc{{}-ep-pfv-real-3m} again \textsc{3m.pro} \textsc{3m.pro} also \textsc{3m-go-ep-real} \textsc{3m-}hunt\textsc{{}-ep-real-sbd} \textsc{3m.pro} also harpy.eagle\\
\glt \sqt{She took it in order to cook it, and the eagle went out to hunt again.}\\
\end{exe}
 
\begin{exe}
\ex \label{Emapp11}
\glt \textit{Onkotakeri impo oka onianiatakeri okisavitakerira itomi.}\\
\gll o-onko-t-ak-i-ri impo o-oka o-nia-nia-t-ak-i-ri o-kis-a-vi-t-ak-e-ri-ra i-tomi\\
 \textsc{3f-}cook\textsc{{}-ep-pfv-real-3m} then \textsc{3f-}this \textsc{3f-}speak-speak\textsc{{}-ep-pfv-real-3m} \textsc{3f-}make.angry\textsc{{}-ep-mot.obl-ep-pfv-real-3m-sbd} \textsc{3m-}son\\
\glt \sqt{She cooked it later, and she made his son mad by talking to [the eagle].}\\
\end{exe}
 
\begin{exe}
\ex \label{Emapp12}
\glt \textit{``Pinianiatanakeri maika pakitsa inkaontake matsigenka nianianiataerini.''}\\
\gll pi-nia-nia-t-an-ak-i-ri maika pakitsa i-n-kaont-ak-e matsigenka n-nia-nia-nia-t-a-e-ri-ni\\
 \textsc{2s-}speak-speak\textsc{{}-ep-abl-pfv-real-3m} now harpy.eagle \textsc{3m-irr-}be.like\textsc{{}-pfv-irr} person \textsc{irr-}speak-speak-speak\textsc{{}-ep-dir:reg-irr-3m-recp}\\
\glt \sqt{[He said], ``you keep on talking to the eagle as if he were a person that you could talk to.''}\\
\end{exe}
 
\begin{exe}
\ex \label{Emapp13}
\glt \textit{Impogini tataka isuretaka iriro irityo yoga pakitsa?}\\
\gll impogini tata-ka i-sure-t-ak-a iriro iri-tyo i-oga pakitsa\\
 then what\textsc{{}-indef} \textsc{3m.}think\textsc{.ep.pfv.real} \textsc{3m.pro} \textsc{3m.pro-affect} \textsc{3m-}that harpy.eagle\\
\glt \sqt{What must the eagle have thought?}\\
\end{exe}
 
\begin{exe}
\ex \label{Emapp14}
\glt \textit{\underline{\smash{Iatake intati anta itinkaraakero oga yovetsikakera imenko ivanko yoga}} \underline{\smash{pakitsa}}.}\\
\gll i-a-t-ak-i intati anta i-tinkara-ak-i-ro o-oga i-ovetsik-ak-i-ra i-menko i-panko i-oga pakitsa\\
 \textsc{3m-}go\textsc{{}-ep-pfv-real} other.side there \textsc{3m-}snap\textsc{{}-pfv-real-3f} \textsc{3f-}that \textsc{3m-}make\textsc{{}-pfv-real-sbd} \textsc{3m-}nest \textsc{3m-}house \textsc{3m-}that harpy.eagle\\
\glt \sqt{The eagle went across to break off [sticks] to build his nest, his house.}\\
\end{exe}
 
\pagebreak
\begin{exe}
\ex \label{Emapp15}
\glt \textit{\textbf{Itinkaraake itinkaraake} terong terong yovetsikake aryomenkorika kara.}\\
\gll i-tinkara-ak-i i-tinkara-ak-i terong terong i-ovetsik-ak-i aryo-menko-rika kara\\
 \textsc{3m-}snap\textsc{-pfv-real} \textsc{3m-}snap\textsc{-pfv-real} snapping.sound snapping.sound \textsc{3m-}make\textsc{-pfv-real} truly\textsc{-ni:}nest\textsc{-indef} there\\
\glt \sqt{He snapped off more and more [sticks] `terong terong' and made his big nest there.}\\
\end{exe}
  
\begin{exe}
\ex \label{Emapp16}
\glt \textit{\underline{\smash{Impogini otarogavagetake iroro}} oga irotyo iriniro yoga matsigenka.}\\
\gll impogini o-tarog-a-vage-t-ak-i iroro o-oga iro-tyo iriniro i-oga matsigenka\\
 then \textsc{3f-}sweep\textsc{{}-ep-dur-ep-pfv-real} she \textsc{3f-}that she\textsc{{}-affect} his.mother \textsc{3m-}that person \\
\glt \sqt{Then she was sweeping, she, the mother of the man.}\\
\end{exe}
 
\begin{exe}
\ex \label{Emapp17}
\glt \textit{\textbf{Impogini otarogavageti}, inti oga oshinto anuvagetakeroka oga oga sotsiku.}\\
\gll impogini o-tarog-a-vage-t-i i-nti o-oga o-shinto o-anu-vage-t-ak-i-roka o-oga o-oga sotsi-ku\\
 then \textsc{3f-}sweep\textsc{{}-ep-dur-ep-real} \textsc{3m-cop} \textsc{3f-}that \textsc{3f-}daughter \textsc{3f-}walk\textsc{{}-dur-eu-pfv-real-epis.wk} \textsc{3f-}that \textsc{3f-}that outside\textsc{{}-loc}\\
\glt \sqt{Then she was sweeping, [and] her daughter must have been walking around, um, outside.}\\
\end{exe}
 
\begin{exe}
\ex \label{Emapp18}
\glt \textit{Okemiri maika yarapaake \underline{\smash{yagapanutiro pe oga oshinto otyomiani}.}}\\
\gll o-kem-i-ri maika i-ar-apa-ak-i i-ag-apanu-t-i-ro pe o-oga o-shinto o-tyomia-ni \\
 \textsc{3f-}listen\textsc{{}-real-3m} now \textsc{3m-}fly\textsc{{}-all-pfv-real} \textsc{3m-}get\textsc{{}-dir:dep-ep-real-3f} \textsc{emph} \textsc{3m-}that \textsc{3f-}daughter \textsc{3f-}small\textsc{{}-anim}\\
\glt \sqt{She heard him [as] he flew in and he grabbed her young daughter.}\\
\end{exe}
 
\begin{exe}
\ex \label{Emapp19}
\glt \textit{\textbf{Yagapanutiro} opampogiavakeri koa yarakaganake anta yovetsikakera ivanko intati anta.}\\
\gll i-ag-apanu-t-i-ro o-pampogi-av-ak-i-ri koa i-ar-akag-an-ak-i anta i-ovetsik-ak-i-ra i-panko intati anta\\
 \textsc{3m-}get\textsc{{}-dir:dep-ep-real-3f} \textsc{3f-}watch\textsc{{}-tr-pfv-real-3m} more \textsc{3m-}fly\textsc{{}-caus-abl-pfv-real} there \textsc{3m-}make\textsc{{}-pfv-real-sbd} \textsc{3m-}house other.side there\\
\glt \sqt{He grabbed her, [as] [the mother] watched him, [and] he quickly flew her away to where he had made his house on the other side [of the river].}\\
\end{exe}
 
\begin{exe}
\ex \label{Emapp20}
\glt \textit{Okanti ``yamanakeroni noshinto.''}\\
\gll o-kant-i i-am-an-ak-i-ro-ni no-shinto\\
 \textsc{3f-}say\textsc{{}-real} \textsc{3m-}bring\textsc{{}-abl-pfv-real-3f-recp} 1-daughter\\
\glt \sqt{She said, ``he took away my daughter.''}\\
\end{exe}
 
\begin{exe}
\ex \label{Emapp21}
\glt \textit{Ipokapaake itomi ikantiro ``virotakani maika kantage- kantagetakovagetanatsivi.''}\\
\gll i-pok-apa-ak-i i-tomi i-kant-i-ro viro-takani maika kant-a-ge kant-a-ge-t-ako-vage-t-an-ats-i-vi\\
\textsc{3m-}come\textsc{{}-adl-pfv-real} \textsc{3m-}son \textsc{3m-}say\textsc{{}-real-3f} you\textsc{{}-culp} now do\textsc{{}-ep-dstr} do\textsc{{}-ep-dstr-ep-appl:indr-dur-ep-abl-subj.foc-real-2}\\
\glt \sqt{His son came [and] said to her, ``it’s your fault, you keep on doing it [i.e. talking].''}\\
\end{exe}
 
\begin{exe}
\ex \label{Emapp22}
\glt \textit{``Pine gara yagapanutiro incho''}\\
\gll  pi-ne gara i-ag-apanu-t-i-ro incho\\
\textsc{2-}see \textsc{neg.irr} \textsc{3m-}get\textsc{{}-dir:dep-ep-real-3f} my.sister\\
\glt \sqt{``Otherwise he wouldn’t have taken my sister away.''}\\
\end{exe}
 
\begin{exe}
\ex \label{Emapp23}
\glt \textit{Impo aryompa aryompa anta yogimonkanakero iriro anta intati anta ipegakagakero ikovintsavageti  komaginaro}\\
\gll  impo aryompa aryompa anta i-ogimonk-an-ak-i-ro iriro anta intati anta i-peg-akag-ak-i-ro i-kovintsa-vage-t-i komaginaro\\
then gradually gradually there \textsc{3m-}raise\textsc{-abl-pfv-real-3f} \textsc{3m.pro} there other.side there \textsc{3m-}turn.into\textsc{-caus.soc-pfv-real-3f} \textsc{3m-}hunt\textsc{-dur-ep-real} monkey.species\\
\glt \sqt{But little by little he raised her there on the other side of the river, he hunted monkey.}\\
\end{exe}
 
\begin{exe}
\ex \label{Emapp24}
\glt \textit{Aryompa aryompa oneiro iriniro antarotanake ya iroro irishinto antarotanake ya.}\\
\gll  aryompa aryompa o-ne-i-ro iriniro o-antaro-t-an-ak-i ya iroro iri-shinto o-antaro-t-an-ak-i ya\\
gradually gradually \textsc{3f-}see\textsc{{}-real-3f} their.mother \textsc{3f-}be.adult\textsc{{}-ep-abl-pfv-real} already \textsc{3f.pro} \textsc{3m-}daughter \textsc{3f-}be.adult\textsc{{}-ep-abl-pfv-real} already\\
\glt \sqt{And bit by bit her mother saw her, she was already grown up.}\\
\end{exe}
 
\begin{exe}
\ex \label{Emapp25}
\glt \textit{Okantiro maika ``noshinto aryo oga antarotanake'' okantiro ``hehe''.}\\
\gll  o-kant-i-ro maika no-shinto aryo o-oga o-antaro-t-an-ak-i o-kant-i-ro hehe\\
\textsc{3f-}say\textsc{{}-real-3f} now \textsc{1-}daughter truly \textsc{3f-}that \textsc{3f-}be.adult\textsc{{}-ep-abl-pfv-real} \textsc{3f-}say\textsc{{}-real-3f} yes\\
\glt \sqt{She said ``my daughter, you’ve grown up'', and she said, ``yes.''}\\
\end{exe}
 
\begin{exe}
\ex \label{Emapp26}
\glt \textit{\underline{\smash{Aryompa aryompa onamonkitanake}}.}\\
\gll  aryompa aryompa o-onamonki-t-an-ak-i\\
gradually gradually \textsc{3f-}be.pregnant\textsc{{}-ep-abl-pfv-real}\\
\glt \sqt{Little by little, her belly began to grow.}\\
\end{exe}
 
\begin{exe}
\ex \label{Emapp27}
\glt \textit{\textbf{Yonamonkitagakero irityo pakitsa oga tsinane.}}\\
\gll  i-onamonki-t-ag-ak-i-ro iri-tyo pakitsa o-oga tsinane\\
\textsc{3m-}be.pregnant\textsc{{}-ep-caus.soc-pfv-real-3f} \textsc{3m.pro-affect} harpy.eagle \textsc{3f-}that woman\\
\glt \sqt{The eagle had impregnated the woman [lit. made her belly grow].}\\
\end{exe}

\begin{exe}
\ex \label{Emapp28}
\glt \textit{\textbf{Yonamonkitagakero.}}\\
\gll  i-onamonki-t-ag-ak-i-ro\\
\textsc{3m-}be.pregnant\textsc{{}-ep-caus.soc-pfv-real-3f} \\
\glt \sqt{He had impregnated her.}\\
\end{exe}


\section*{ Abbreviations}
\textsc{1.incl}   First person inclusive,
1      First person,
2      Second person,
3      Third person,
\textsc{abl}      Ablative,
\textsc{adj}      Adjective,
\textsc{adl}      Adlative,
\textsc{affect}      Affect,
\textsc{alien}      Alienable possession,
\textsc{all}      Allative,
\textsc{anim}      Animate,
\textsc{appl}      Applicative,
\textsc{appl:indr}    Indirective applicative,
\textsc{assoc.mot:dist}  distal associated motion,
\textsc{caus}      Causative,
\textsc{caus.soc}    Sociative causative,
\textsc{cop}      Copula,
\textsc{culp}      Culpable,
\textsc{def}      Definite,
\textsc{dem}      Demonstrative,
\textsc{dep}      Departative,
\textsc{det}      Determiner,
\textsc{dir:dep}    Directional: departative,
\textsc{dir:reg}    Directional: regressive,
\textsc{dstr}      Distributive,
\textsc{dur}      Durative,
\textsc{emph}      Emphasis,
\textsc{ep}      Epenthesis,
\textsc{epis.wk}    Weak epistemic modality,
\textsc{f}      Feminine,
\textsc{incr}      Incremental,
\textsc{indef}      Temporally indefinite,
\textsc{inf}      Infinitive,
\textsc{irr}      Irrealis,
\textsc{loc}      Locative,
\textsc{m}      Masculine,
\textsc{neg}      Negation,
\textsc{neg.irr} Irrealis negation,
\textsc{ni:nest}   Incorporated noun: nest,
\textsc{ni:wing}    Incorporated noun: wing,
\textsc{obj}      Object,
\textsc{obl}      Oblique,
\textsc{pfv}      Perfective,
\textsc{pl}      Plural,
\textsc{pn}      Pronoun,
\textsc{prep}      Preposition,
\textsc{pro}      Pronoun,
\textsc{prs}      Present,
\textsc{pst}      Past,
\textsc{ptcp}      Participle,
\textsc{real}      Realis,
\textsc{recp}      Recipient,
\textsc{refl}      Reflexive,
\textsc{sbd}      Subordinate,
\textsc{sg}      Singular,
\textsc{subj.foc}    Subject focus,
\textsc{tr}      Transitive,
\textsc{trnloc}     Translocative

\section*{ Acknowledgments}
Thanks to my \ili{Matsigenka} friends and colleagues in the {Alto Urubamba}, Julio Korinti Piñarreal, Valérie Guérin, Simon Overall, and two anonymous reviewers. This research was supported by a Fulbright-Hays Doctoral Dissertation Research Abroad (DDRA) Fellowship and an NSF Doctoral Dissertation Improvement Grant (1021842). Any opinions, findings, conclusions, or recommendations expressed in this material are those of the author and do not necessarily reflect the views of the National Science Foundation. The research leading to these results also received funding from the European Research Council under the European Union’s Seventh Framework Programme (FP7/2007–2013)/ERC grant agreement number 295918. Thanks also to the John Carter Brown Library at Brown University.
%

\sloppy

\printbibliography[heading=subbibliography,notkeyword=this] 
\end{document}
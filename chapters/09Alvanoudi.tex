\documentclass[output=paper]{LSP/langsci} 
\ChapterDOI{10.5281/zenodo.2563694}
\author{Angeliki Alvanoudi\affiliation{Aristotle University of Thessaloniki and James Cook University}}
\title{Clause repetition as a tying technique in Greek conversation}
%epigram
\abstract{This chapter targets a language in which bridging constructions are not grammaticalized, that is, Greek. It examines instances of same-speaker and cross-speaker clause repetition in informal Greek conversation. The analysis demonstrates that the basic function of clause repetition is to display connectedness between what the current speaker says or does and what the same or previous speaker said or did immediately before. It is argued that clause repetition displays some similarities with recapitulative linkage and it is hypothesized that recapitulative linkage constructions have emerged from repetition practices in conversation.}
\maketitle
%-------------------------

\begin{document}
\label{ch:9}
\section{Introduction} 
\label{AlIntroduction}
Recapitulative linkage is a type of bridging construction in which the bridging clause repeats at least the predicate of the reference clause. Recapitulative linkage is not an integral part of the \ili{Greek} grammar. Yet \isi{clause repetition} is one of the \isi{cohesive} or tying techniques employed in \ili{Greek} conversation. It consists of a main or \isi{non-main clause} that repeats a prior main or \isi{non-main clause}, as in example (\ref{Alex1}), lines 4--6, and example (\ref{Alex2}), lines 1 and 3.  

% \clearpage
\begin{Transcript}{Ale}{‘Of course compared to quitting smoking coffee was (0.5)’}%
\label{Alex1}%
\TLine[{>}{>}]{Pol}{%
                 E\underline{γ}ó ton ékopsa.
                 \glt `I stopped drinking coffee.'}\\
\TLine{}{(1.3)}\\
\TLine{Pol}{Vévea éxodas kópsi to tsiɣáro o kafés °ítane: (0.5)
            \glt `Of course compared to quitting smoking coffee was (0.5)'}\\
\TLine[{>}{>}]{Ale}{%
                  \gll \textbf{$\uparrow$}\textbf{\underline{Ðe}} \textbf{su} \textbf{kostízi} \textbf{pá{[}ra} \hspace{0.4in} \textbf{polí.{]}}\\
                      \textsc{neg} 2\textsc{sg}.\textsc{gen} cost\textsc{.3sg.prs}  very \hspace{0.3in} much\\
                  \glt `It doesn't cost you very much.'\\
                  }\\    
\TLine[{>}{>}]{Pol}{
				 \gll  \hspace{1.7in}  \textbf{{[}\underline{Ðe}}  \textbf{su}	    \textbf{ko{]}stízi}\\
				 \hspace{1.7in} \textsc{neg} 2\textsc{sg}.\textsc{gen} cost\textsc{.3sg.prs}\\
				  \glt  \hspace{1.7in}`It doesn't cost you'
				}\\         
\TLine[{>}{>}]{}{%
				\gll °\textbf{pára}  \textbf{polí.=}\\
						very much\\
           \glt `very much.'\\
           }               
\end{Transcript}

\begin{Transcript}{Ale}{‘And if you are alone what are you supposed to do.=’}%
\label{Alex2}%
\TLine[{>}{>}]{Mar}{%
                 \gll \textbf{Ci} \textbf{áma} \textbf{íse} \textbf{ce} \textbf{mónos} \textbf{su}  \underline{ti} na \\
                 and if \textsc{cop.2sg.prs}  and by yourself what \textsc{sbjv} 	 \\
                 }\\
\TLine{}{%
 \gll ká\underline{\t{ts}}is na	kánis.=\\
 sit.\textsc{2sg.pfv} \textsc{sbjv} 	do.\textsc{2sg}\\
 \glt `And if you are alone what are you supposed to do.='\\
}\\

\TLine[{>}{>}]{Our}{%
\gll =\textbf{Áma} \textbf{íse}  \textbf{mónos} \textbf{su}   {[}ðen éçis  ce{]}\\
if  \textsc{cop.2sg.prs} by yourself 	\textsc{neg} have.\textsc{2sg.prs}   and\\
            \glt `=I think that if you are alone you are not'}\\
\TLine{Vag}{%
                 \hspace{2.2in} {[}°M: \hspace{0.7in} ne.{]}
                 \glt  \hspace{2.2in}   `Mm yes.'  }\\    
\TLine{Our}{%
				ti ðiáθesi pistévo tin íðʝa. ektós an íse me á{[}lus.{]}
				\glt `in the mood. unless you are together with others.'	}\\         
\TLine{Mar}{%
				\hspace{2.6in} {[}Ma{]}: ne. 
				 \glt \hspace{2.6in}`But of course.' }\\               
\end{Transcript}


Clause \isi{repetition} in \ili{Greek} conversation shares some of the formal and discursive properties of \isi{recapitulative linkage} constructions (\citealt[cf.][]{guerin18}, this volume). First, \isi{recapitulative linkage} involves \isi{repetition} of at least the verb of the reference clause; not all elements accompanying the verb of the reference clause are necessarily repeated. Clause repetitions in \ili{Greek} conversation involve \isi{repetition} of at least the verb of the first saying and some of the elements accompanying the verb. Second, \isi{recapitulative linkage} is a discourse strategy that achieves \isi{cohesion}, by establishing \isi{thematic continuity} or referential coherence \citep{devries.2005}, backgrounding the proposition of the reference clause and prefacing discourse-new information that is usually sequentially ordered (\citealt[][]{guerin18}, this volume). As we will see in this chapter, the basic function of \isi{clause repetition} in \ili{Greek} conversation is to display connectedness with the ongoing talk. Yet the two phenomena are not identical. Clause \isi{repetition} in \ili{Greek} conversation is a \isi{practice} widely distributed between speaker and recipient, whereas \isi{recapitulative linkage} occurs in same speaker’s utterance.\footnote{Valérie Guérin pointed out to me that this feature of \isi{recapitulative linkage} may be an artefact of the data rather than a pattern found in conversation, given that previous studies on bridging constructions did not analyse data from talk-in-interaction.} Unlike bridging clauses that prototypically consist of clauses, which are morphologically, syntactically or intonationally marked as dependent on the reference clauses, repeated clauses in \ili{Greek} conversation can be grammatically and pragmatically complete utterances. Moreover, \isi{recapitulative linkage} usually expresses a temporal semantic relation that involves the transition between linked events. Clause \isi{repetition} in \ili{Greek} conversation does not express temporal \isi{sequentiality} or any fixed semantic relationships; it carries different functions in different contexts.  

The aim of the present study is to examine instances of \isi{clause repetition} in two contexts: self-\isi{repetition} (same-speaker) and \isi{repetition} of prior turn at talk (cross-speaker) in \ili{Greek} conversation, and demonstrate that the basic function of \isi{clause repetition} is to display \textit{connectedness} between what the current speaker says or does and what the same or previous speaker said or did immediately before. It is hypothesized that across languages there is a continuum between \isi{repetition} as a generic linguistic \isi{practice} and more or less conventionalized forms of bridging constructions. The outline of the chapter is as follows. In \refsec{Alrole.repetition}, I review previous studies on the forms and functions of \isi{repetition} in conversation. In \refsec{Alrepetition.tying}, I approach the \isi{cohesive} function of \isi{repetition} in conversation through the lens of \isi{conversation analysis}. In \refsec{Alclause.repetition}, I analyse self-\isi{repetition} (\refsec{Alself.repetition}) and \isi{repetition} of prior turn (\refsec{Alrepetition.turn}) in naturally occurring conversations, focusing on the use of \isi{clause repetition} as a tying technique. \sectref{Alcontinuum} contains a discussion of the findings. 

\section{The role of repetition in conversation}
\label{Alrole.repetition}
Repetition, in Brown's words (\citeyear[][225]{brown.2000}), is ``a grammatical, stylistic, poetic, and cognitive resource associated with attention.'' It constitutes part of everyday human conduct and is found in social life, rituals, events, conversation, and grammar \citep{johnstone94,brown.2000, wong10}. In conversation, \isi{repetition} distinguishes \textit{self-repetition} and \textit{\isi{repetition} of a prior turn at talk} \citep{brown.2000}. In terms of form, \isi{repetition} can be exact or modified. Exact \isi{repetition} involves the exact duplication of words, that is, a ``perfect copy'' of a first saying, while modified \isi{repetition} involves a modified replication of words through \isi{addition} or \isi{omission}, that is, a ``near copy'' of a first saying (\citealt[][368]{couper96}, \citealt[][224]{brown.2000}). Repetition carries multiple functions that depend on the context of use of repeated elements. As \citet[][368]{couper96} observes, ``replication of form does not necessarily mean replication of function.'' A similar point is made by \citet[][12]{johnstone94}, who claims that although the referential meaning of repeated elements remains the same, non-referential aspects of their meaning change, given that the context of use of repeated elements changes. 

In general, \isi{repetition} is a mode of focusing the addressee’s attention to something. This generic function of \isi{repetition} can be particularized in different contexts of usage. For instance, speakers use \isi{repetition} to achieve discourse \isi{cohesion} \citep{goodwin87,norrick87,tannen87,tannen89,johnstone94,tyler94,sacks95,brown.2000}, and implement various social actions in talk-in-interaction, such as:

\begin{itemize}
\item Answering a question \citep{norrick87,raymond03,stiversHayashi10,stivers11}
\item Agreeing or disagreeing with prior speaker \citep{pomeratz84,goodwin87,norrick87,tannen87}
\item Claiming more agency with respect to the \isi{action} they are implementing \citep{stivers10,heritage2012,lee12}
\item Confirming an allusion \citep{schegloff96a}
\item Registering receipt of a prior turn \citep{tannen89,schegloff97,kim02}
\item Initiating repair \citep{schegloff77,sorjonen96,kim02}
\item Sustaining a particular topical focus \citep{tannen89,kim02}
\item Resuming a story \citep{wong10}
\end{itemize}

Repetition is also used for delivering recycled turn beginnings \citep{schegloff87} and dealing with interruption and overlapping talk \citep{norrick87,johnstone94}, and it can serve as a stylistic feature used for emphasis or clarification \citep{norrick87,johnstone94}. There is often an interrelation between the interactional functions of \isi{repetition} and ``the placement of \isi{repetition} in the turn-taking metric'' \citep[][411]{wong10}. For instance, self-repetitions may deal with overlapping talk, whereas repetitions of prior turn may initiate repair \citep{wong10}. 
The \isi{cohesive} function of \isi{repetition} in conversation is the \isi{topic} of the next section. 

\section{Repetition as a tying technique in conversation}
\label{Alrepetition.tying}
Discourse \isi{cohesion} is achieved through a variety of linguistic resources, such as \isi{repetition}, reference, ellipsis or \isi{omission}, \isi{substitution}, \isi{conjunction}, synonymy and collocation \citep{martin01}. Cohesion is usually understood through the lens of systemic functional linguistics \citep{halliday73,hasan76}, as a relation of dependence between the interpretation of some element and another element in discourse. This study, however, approaches \isi{cohesion} from a conversation analytic perspective. 

Speakers always deal with the problem of \isi{cohesion} or connectedness with\linebreak ongoing talk, when they design their turns, given that talk-in-interaction involves contingencies between prior, current and next turns. In talk-in-interaction, speakers take turns, which consist of turn-constructional units (TCUs), i.e., claus-\linebreak es, phrases and lexical items that constitute at least one \isi{action} \citep[][3--4]{schegloff07}. Turns form sequences, that is, courses of actions implemented through talk. The unit of sequence organization is the adjacency pair, which is composed of two turns produced by different speakers, adjacently placed and relatively ordered as first pair part and second pair part \citep[][13]{schegloff07}. First pair parts initiate some exchange, such as a question, a request or an offer. Second pair parts respond to the \isi{action} of the first pair parts: they deliver an answer to the question, a rejection or an acceptance of the request, or the offer. First pair parts project the relevance of specific second pair parts; they set powerful constraints on what the recipient should do, and on how the \isi{action} accomplished by the recipient should be understood \citep[][21]{schegloff07}. Thus, next turns are understood by co-participants to display an understanding of the just prior turn, and to embody an \isi{action} responsive to the just prior turn so understood \citep[][15]{schegloff07}. According to \citet[][131]{drew12}, interaction consists of ``contingently connected sequences of turns in which we each `act', and in which the other’s -- our recipient’s -- response to our turn relies upon, and embodies, his/her understanding of what we were doing and what we meant to convey in our (prior) turn.'' 

When speakers design their current turn, they need to display how their turn is connected with what came immediately before \citep[][134]{drew12}, namely how their turn is connected with the prior turn produced by a different speaker, or with the prior TCU within the same speaker’s turn. For example, in the beginning of turns speakers may display whether their current turn takes a different stance from the prior turn produced by another speaker \citep{schegloff96b}. In the beginning of non-initial TCUs within multi-unit turns, speakers may display whether the current TCU continues the project of the preceding TCU, or whether the current TCU launches or projects another \isi{action} \citep[][481]{maze12}. Repetition is one of the practices that speakers use to display connectedness with ongoing talk. \citet{couper17} describe \isi{repetition} as a generic linguistic \isi{practice} that ``depends on the establishment of a relation of formal similarity between a set of forms in one (current) turn and another set of forms in a prior turn''. In conversation, speakers use \isi{repetition} as a ``tying technique'' \citep{sacks95} or ``format tying'' (following the terminology of \citealt{goodwin87} and \citealt{goodwin90}) to create a relation between a current turn and a prior turn and, thus, achieve \isi{cohesion}. According to \citet[][177]{goodwin90}, format tying involves participants’ strategic use of phonological, syntactic or semantic surface structures of prior turns for tying talk between turns; \isi{repetition} is an instance of the format tying apparatus. The use of \isi{clause repetition} as a tying technique in \ili{Greek} conversation is analyzed in \refsec{Alclause.repetition}.  

\section{Clause repetition in Greek conversation}
\label{Alclause.repetition}

\subsection{Data}
\label{Alclause.repetition.data}
The data analyzed in this study stem from 33 fully transcribed audio-recorded naturally occurring face-to-face conversations among friends and relatives from the Corpus of Spoken \ili{Greek} of the Institute of Modern \ili{Greek} Studies.\footnote{Conversations have been transcribed according to the conventions of \isi{conversation analysis}. A list of transcription symbols is in the Appendix.} The total duration of the conversations examined is about 22 hours and 23 minutes, and the total number of words is 324,994.

Before moving to the analysis of the data, some basic information on the language profile is required. Modern \ili{Greek} belongs to the Indo-European group of languages, and is spoken by about 13 million speakers, with approximately 10 million of them living in Greece, and the rest in Cyprus and parts of the \ili{Greek} diaspora (detailed descriptions of the language can be found in \citealt{joseph87} and \citealt{Mackridge85}). \ili{Greek} is a fusional, highly inflecting language, in which several grammatical categories are marked morphologically. For instance, nouns inflect for gender, number and case, and verbs inflect for person, number, tense, aspect, voice, and mood. \ili{Greek} is a pro-drop language with a flexible word order.

Approximately 130 instances of \isi{clause repetition} were found in the data examined: 73 self-repetitions and 57 repetitions of a prior turn. In terms of form, the large majority of clause repetitions are modified. Most of the modified repetitions involve a change in \isi{intonation} that contributes to the change in meaning expressed by the repeated clause. Modified repetitions often involve \isi{addition} or \isi{omission} whereby speakers go beyond the initial version of the clause or omit something in the repeated clause. In terms of function, most clause repetitions are used as tying techniques. However, variation is found within each category of \isi{repetition}. Unlike repetitions of prior turn at talk, which are routinely used as tying techniques implementing various actions, self-repetitions do not always have a \isi{cohesive} function. 

Following Wong’s \citeyear{wong10} terminology, I refer to the antecedent of the \isi{repetition} as first saying, and the \isi{repetition} of the whole clause or part of the clause (at least of the verb) as second saying. I avoid using the terms \textit{reference clause} and \textit{bridging clause} (cf. \citeauthor{guerin18}, this volume), since the phenomenon examined here is not a typical bridging construction (cf. \refsec{AlIntroduction}). First saying and second saying are codified in the excerpts below as FS and SS respectively. Although \isi{clause repetition} may occur in various turns in each excerpt, special focus is given only to certain usages (marked with bold face). The turns in which these usages occur are followed by glossing. 

\subsection{Self-repetition}
\label{Alself.repetition}
28 out of the 73 self-repetitions found in the data have a \isi{cohesive} function, as shown in examples (\ref{Alex3}) to (\ref{Alex5}).  In the lines preceding (\ref{Alex3}), participants argue about whether \ili{Greek} taxi drivers drive safely or not. In lines 1--2, Thanos implies that they do not know how to drive through the form of a rhetorical question, and Petros disagrees in lines 3 and 5--7. He uses the negative particle \textit{óçi} `no' to express his disagreement with the previous speaker, in turn-initial position. In the next TCU, he offers an account for his disagreement: he claims that there are certain standards (\textit{ipárxun meriká stádars} `there are certain standards'), uttering the noun with emphasis due to increased loudness or higher pitch. The speaker seeks confirmation of understanding by the recipient, and offers another account for his disagreement in the next TCU. He starts the TCU with the discourse particle \textit{ðilaðí} `that is', and repeats the clause from the previous TCU (\textit{>ipárxun meriká< stádars} `there are certain standards'). The second saying is modified. The speaker utters part of the clause in a rushed way, with no emphasis on the noun. By repeating the clause, the speaker shows that the current TCU continues the project of the prior one. In this case, \isi{clause repetition} links different TCUs within the same speaker’s turn, and displays connectedness with ongoing talk. 


\begin{Transcript}[FS {>}{>}]{Tha}{=eh what eh hey. who knows how to drive?}%
\label{Alex3}%
\TLine{Tha}{% 
		=e \underline{ti}: e}\\
\TLine{}{%
  {[}moré. \underline{pços}  kséri  na \hspace{1.3in} oðiɣái?{]}
 \glt `=eh what eh hey. Who knows how to drive?'}\\
\TLine[{FS>}{>}]{Pet}{%
                  \gll {[}{[}Oçi. ap\underline{lá}{]} θélo             na    po          	   óti     \textbf{ipá{]}rxun}\\
                    no  just   want.\textsc{1sg.prs}  \textsc{sbjv} tell.\textsc{1sg.pfv}  \textsc{conj}   \textsc{cop.3pl.prs}\\
                  \glt `No. I just want to say that there are' }\\  
\TLine{Nef}{% 
		{[}(.............){]}}\\
\TLine[{FS}{>}{>}]{Pet}{%
\gll \textbf{meriká}                   \textbf{\underline{stá}dars}  >vre    peðí  mu.< \\
certain.\textsc{neut.acc.pl} standards \textsc{part} child my \\
            \glt `certain standards hey you man.'}\\
\TLine[{SS>}{>}]{}{%
\gll katálaves? ðilaðí, >\textbf{ipárxun}    \textbf{meriká}<                  \\
understand.\textsc{2sg.pst} that.is  \textsc{cop.3pl.prs} certain.\textsc{neut.acc.pl} \\ \vspace{-0.1in}
            \glt `Do you understand? That is, there are certain'}\\
\TLine{}{%
				\textbf{stáda{[}rs}, ta  opía    i   taksi\underline{d͡z}íðes ðen  da  sévode.{]} 
				\glt `standards that taxi drivers do not respect.'}\\              
\TLine{Tha}{%
                 \hspace{0.3in} {[}Eh $\uparrow$ ti?    pça    íne      ta-     $\uparrow$ e\underline{mís}   ta   \hspace{0.4in}  ká{]}nume\\ \vspace{-0.2in}
                  \glt  \hspace{0.3in} `what? What are the- we make' }\\      
\TLine{}{%
				 ta stádar. 
				 \glt `the standards' }\\               
\end{Transcript}

In (\ref{Alex4}), participants talk about carnival celebrations in the city of Patra in\linebreak Greece. In lines 1--2, Vagelis informs his co-participants about volunteers forming groups for the carnival parade (\textit{kánune:::grup}, `they form groups') and in line 4, Maria interrupts Vagelis before his turn reaches possible completion. In line 8, Vagelis continues the turn that was interrupted. He repeats an almost perfect copy (\textit{kánune grup}, `they form groups') of his previous clausal TCU (in line 2): the only difference between the first and second saying is the vowel lengthening in the first saying. The turn continues the \isi{action} of informing that was suspended. The speaker uses \isi{clause repetition} in turn-initial position. As \citet[][72]{schegloff87} argues, turn beginnings are ``sequence-structurally important places'' in conversation, because they project the turn type or shape, and the relation between the current turn and the prior one. The repeated clause prefacing the turn in line 8 conveys that what follows is part of the speaker’s prior activity, and connects the same speaker’s previous and current turn.
%vvvvvvvvvvvvvvvggggggggggggggggggggggggggggglaughing cant get aligned. i think it is ok-- main problem is page width alignment...

\begin{Transcript}[FS {>}{>}]{Vag}{And they all have the right to participate, whoever wants to participate,}%
\label{Alex4}%
\TLine{Vag}{% 
		=Ci éxun ðicéoma na katevúne ó\underline{:}li, ósi θélune,
		\glt `And they all have the right to participate, whoever wants to participate,'}\\
\TLine[{FS >}{>}]{}{%
  \gll .hh \textbf{kánu{[}ne:{]}:} \hspace{0.05in} \textbf{{[}:grup},{]}\\
 .hh make.\textsc{3pl.prs} \hspace{0.05in} group\\
 \glt `.hh they form groups,'}\\
 \TLine{Our}{%
           \hspace{0.5in}       {[}((giggle)){]}
                  }\\ 
\TLine{Mar}{%
                   \hspace{1.12in} {[}.h Fad{]}ázese na min íçes ce ði\underline{cé}oma
                   \glt \hspace{1.12in} ((laughing...........................................
                  \glt \hspace{1.12in} `Imagine if you didn’t even have the right' }\\  
 
\TLine{}{%
                    na katévis séna karnaváli.\\
                 .........................................))
                  \glt `to participate in the carnival.' }\\   
  \TLine{Our}{%
                  ((gig{[}gle.............)){]}
                  }\\ 
   \TLine{Mar}{%
                  \hspace{0.23in} {[}((she laughs{]}{[}...........)){]}

                  }\\ 
\TLine[{SS >}{>}]{Vag}{%
  \gll \hspace{1.1in} \textbf{{[}kánune{]}} \textbf{grup},  ci  éçi: \\
 \hspace{1.1in}  make.\textsc{3pl.prs} group and have.\textsc{3sg.prs}\\
 \glt \hspace{1.1in} `they form groups, and it is'}\\  
                 
\TLine{}{%
                    e- ci éxun polí pláka:. ʝatí {[}parusiázune po{]}lí protótipa 
                    \glt `eh- and they are very funny. Because they present very innovative' }\\                   
    \TLine{Mar?}{%
                  \hspace{1.6in}{[}°Α::(h) \hspace{0.5in}{]}
                  }\\                 
\TLine{Vag}{% 
		\underline{á}rmata::: me tin {[}\hspace{0.13in}epi{]}\underline{ce}rótita,=
		\glt `floats related to current affairs,'}\\
\TLine{Our}{% 
		\hspace{1in}{[}°(Ne){]}
		\glt \hspace{1in}`Yes.'}\\    
\end{Transcript}

In (\ref{Alex5}), in line 3, Katia suggests that she and her co-participants cook something. She uses a negative question in the subjunctive (\textit{>ðen báme na maʝirépsume?<} `Shall we go and cook?' or `Why don’t we cook?'), that expects a positive answer. Before recipients respond, and without an expected micro-pause after the delivery of the question, Katia initiates a new sequence by asking Eirini if she wants to eat (line 3), and does a subtopic shift. This sequence is closed down in line 8. In line 10, Katia returns to the initial \isi{action} that was suspended: she uses the discourse marker \textit{lipón} `so' to express exhortation, and repeats the clausal TCU that she initially employed, in line 3, to implement the suggestion. The repeated clause is modified (\textit{na \underline{\smash{pá}}me na maʝirépsume?} `shall we go and cook?'): the speaker uses the subjunctive without negation, utters the verb \textit{\underline{\smash{pá}}me} with emphasis, and does not deliver the clause in a rushed way. The speaker repeats the clausal TCU in the same turn (lines 12, 14) with modifications (\textit{ðen báme stin guzína na maʝirépsume?} `shall we go in the kitchen and cook?'). She uses the negative polar question format, and refers to the kitchen, where the activity will take place. In lines 13, 15--16, Eirini and Zoi accept the suggestion. In this excerpt, \isi{clause repetition} links the same speaker’s current and prior turn. 

\begin{Transcript}[FS {>}{>\hspace{0.1in}}]{Eir}{and I will get fat. .h o- but I will help with your cooking.}%
\label{Alex5}%
\TLine{Kat}{% 
		\underline{Pí}nasa.
		\glt `I am hungry.'}\\
\TLine{}{(1.1)}\\
\TLine[{FS>}{>}]{Kat}{%
  \gll \textbf{>Ðen} \textbf{báme}      \textbf{na}   \textbf{maʝirépsume?<} \\
 \textsc{neg} go.\textsc{1pl.prs} \textsc{sbjv} cook.\textsc{1pl.pfv}        \\
 \glt `Shall we go and cook?'}\\
 \TLine{}{%
  \gll  =Rináci      θa   fa::s?=\\
   Eirini.\textsc{dim} \textsc{fut} eat.\textsc{2sg.pfv}\\
 \glt `Eirini will you eat?'}\\
 \TLine{Eir}{% 
		=\underline{O}çi. alá θa voiθí{[}so $\uparrow$sti maʝirikí sa:s.{]}=
		\glt `No. but I will help you with the cooking.'}\\
 \TLine{Kat}{% 
		 \hspace{1.05in} {[}<ʝatí ðe θa fa:s?>\hspace{0.2in}{]}=
		 \glt \hspace{1.05in} `Why won’t you eat?'}\\
 \TLine{Eir}{%
                   =[.h $\uparrow$ʝatí \underline{é}faɣa \hspace{0.22in}sí]mera:. ðe boró álo. éxo \underline{ská}:si.\\
                     \hspace{3in} ((noise starts))
                   \glt `.h because I ate today. I cannot eat any more. I am full.' }\\ 
  \TLine{Kat}{% 
		 ={[}avɣá me patá\underline{:}(tes).{]}
		 \glt `eggs with potatoes.'}\\                 
 
  \TLine{Eir}{%
                   >ce θa ʝíno< xo\underline{dró}. .h ο- θa voi\underline{θi}so ómos sti maʝirikí sas.\\
                     ((\textit{laughing}..........................................................................))
                   \glt `and I will get fat. .h o- but I will help with your cooking.' }\\ 
 \TLine{}{(.)}\\
\TLine[{SS>}{>}]{Kat}{%
  \gll Lipón. \textbf{na} \textbf{\underline{pá}me}      \textbf{na}   \textbf{maʝirépsume?} \\
 so      \textsc{sbjv} go.\textsc{1pl} \textsc{sbjv} cook.\textsc{1pl.pfv}   \\
   \glt `So. Shall we go and cook?'}\\
 \hspace{1.5in} ((noise ends))\\
 
\TLine[{SS+>}{>}]{}{%
                   \gll θé{[}lete?       \textbf{=ðen}  \textbf{báme}    \textbf{stin}                  \textbf{ guzí{]}na}   \\
 want.\textsc{2pl.prs} \textsc{neg}  go.\textsc{1pl}  in kitchen\textsc{(f).acc.sg} \\
 \glt `Do you want? Shall we go in the kitchen'}\\
   \TLine{Zoi}{% 
		 \hspace{0.1in} {[}Ade. \underline{pá}me. páme.\hspace{0.9in}{]}
		 \glt `Come on. let’s go. let’s go.'}\\   
 \TLine[{SS+>}{>}]{Kat}{%
  \gll \textbf{na}   \textbf{maʝirépsu{[}me?{]}} \\
  \textsc{sbjv} cook.\textsc{1pl.pfv}   \\
   \glt `and cook?'}\\
   \TLine{Eir}{% 
		 \hspace{1in}{[}Ne.{]} =\underline{pá}me stin guzí{[}na.{]}
		 \glt  \hspace{1in}`Yes. Let’s go in the kitchen.'}\\  
    \TLine{Zoi}{% 
		 \hspace{2.35in}{[}Pá{]}me.
		 \glt \hspace{2.35in} `Let’s go.'}\\    
\end{Transcript}

In the examples examined above, self-\isi{repetition} is a tying technique that establishes contiguity between current and prior units or turns. Moreover, in (\ref{Alex3}) and (\ref{Alex4}), the repeated clause is followed by discourse-new information. Yet \isi{cohesion} is not the only function associated with self-\isi{repetition}. 45 of the self-repetitions found in the data have a non-\isi{cohesive} function: they deal with overlapping talk, pursue a response, initiate and deliver repair, and add emphasis. These functions are illustrated with examples (\ref{Alex6}) to (\ref{Alex8}). In (\ref{Alex6}), in line 2, Yorgos asks Sotiris a question. His first TCU (\textit{Αftό ði\underline{ló}nete?} `Is this announced?') overlaps with the talk by Sotiris (line 1), and Yorgos repeats the question (\textit{aftό ðilónete::?} `is this announced?'), in line 3. The second saying differs from the first saying, as the verb \textit{ðilónete::} is delivered with vowel prolongation and no emphasis. Sotiris answers Yorgos’ question in lines 4--5 (\textit{To ma\underline{θé}nis °siníθos}. `Usually you find out about it.'). His first TCU overlaps with Yorgos’s prior turn, and Sotiris repeats the answer in the next TCU (\textit{>siníθos< to maθénis.} `Usually you find out about it.'). The second saying is modified. The order of clause constituents is different, as the adverb precedes the verb phrase, plus the adverb is delivered in a rushed way, and the verb with no emphasis. In this excerpt, clause repetitions compensate for recipient’s possible trouble in hearing and understanding, and do not have a \isi{cohesive} function. 

\begin{Transcript}[FS {>}{>\hspace{0.1in}}]{Sot}{Is this announced? how do you find out about it?}%
\label{Alex6}%
\TLine{Sot}{% 
		{[}(benun)                \underline{ðiá}fori.\hspace{0.7in}{]}
		\glt `Various people come.'}\\
\TLine[{FS >}{>}]{Yor}{%
  \gll \textbf{{[}Αftό }                  \textbf{ði\underline{ló}nete?{]}} \\
 this.\textsc{neut.nom.sg} announce.\textsc{3sg.pass.prs}\\
 \glt `Is this announced?'}\\
 \TLine[{SS >}{>}]{}{%
  \gll \textbf{aftό}                  \textbf{ðilónete:{[}:?} \underline{pos} to maθénis?{]}\\
 this announce.\textsc{3sg.pass.prs} how it learn.\textsc{2sg.prs}\\
 \glt `Is this announced? How do you find out about it?'}\\
  \TLine[{FS >}{>}]{Sot}{%
  \gll  \hspace{0.8in} \textbf{{[}To} \textbf{ma\underline{θé}nis} \hspace{0.55in} \textbf{°siní{]}θos}.\\
 \hspace{0.8in} it learn.\textsc{2sg.prs} \hspace{0.55in} usually\\
 \glt \hspace{0.8in}`Usually you find out about it'}\\
   \TLine[{SS >}{>}]{}{%
  \gll  \textbf{>siníθos<} \textbf{to} \textbf{maθénis}.°\\
 usually it learn.\textsc{2sg.prs} \\
 \glt  `Usually you find out about it.'}
\end{Transcript}

In example (\ref{Alex7}), in line 2, Thanasis makes a statement (\textit{°Εsí ti \underline{\smash{ɣnó}}rises aftín.} `You met her.') that operates as a confirmation-seeking question, and in line 4, Telis initiates repair to resolve trouble in understanding Thanasis’s turn due to overlapping talk. In line 5, Thanasis completes the repair by repeating the clause that he used in his prior turn (\textit{°Τi ɣnórises.} `You met her'). Telis answers the question in line 6. Thanasis’s second saying is modified: the speaker utters the verb without emphasis, omits the second and third person singular pronouns, while keeping the clitic pronoun \textit{ti} (such omissions are common in \ili{Greek} conversation). The speaker uses \isi{clause repetition} to offset the recipient’s problem in understanding or hearing. 

\begin{Transcript}[FS {>}{>\hspace{0.1in}}]{Tel}{‘.hh No, but it feels like I know her.}%
\label{Alex7}%
\TLine{Chr}{% 
		{[}Νe:,{]} mu ta pe {[}>°eména.°       mu           \underline{ta}        pe.< \hspace{0.27in}{]}
		\glt `Yes, he told me. He told me.'}\\
\TLine[{FS >}{>}]{Th}{%
  \gll \hspace{0.95in} \textbf{{[}°Εsí}            \textbf{ti}                    \textbf{{[}ɣ\underline{nó}{]}ri{]}ses} aftín.\\ 
 \hspace{0.5in} \textsc{2sg.nom} \textsc{3sg.f.acc}   meet.\textsc{2sg.pst}   \textsc{3sg.f.acc}\\
 \glt  \hspace{0.95in} `You met her.'}\\
\TLine{Tel}{% 
		 \hspace{2.15in}{[}°(Νe,){]}
		 \glt \hspace{2.15in} `(Yes,)'}\\    
 \TLine{Tel}{% 
		 Eh?=
		 }\\ 
\TLine[{SS >}{>}]{Th}{%
  \gll    \textbf{=°Ti}                    \textbf{ɣnó{[}rises{]}} .\\
   \textsc{3sg.f.acc}   meet.\textsc{2sg.pst}   \\
 \glt  `You met her.'}\\
 \TLine{Tel}{% 
		\hspace{0.8in}  {[}.hh \hspace{0.1in} {]} >\underline{O}çi, alá mu ne san na din gzéro.
		 \glt \hspace{0.8in} `.hh No, but it feels like I know her.'}\\
\end{Transcript}

In example (\ref{Alex8}), \isi{clause repetition} is a \isi{practice} for pursuing the recipient’s response \citep{pomeratz84}. In line 4, Linos asks Mara when she and the others will leave (\textit{Mára, póte févʝete (...)} `Mara, when are you leaving (...)'). His turn overlaps with Mara’s answer (line 5) to Roza’s question. Mara does not respond, and Linos repeats his question in line 6 (\textit{>\underline{Pó}te θa fíʝete.< } `When are you leaving?'), with modifications. He delivers the turn in a rushed way, with emphasis on the interrogative word, and he uses future tense. His question receives no answer, and Linos delivers the same question again in line 8 (\textit{>\underline{Pó}te θa fíʝete esís?<} `When are you leaving?'), with a few modifications. He repeats what he said in his previous turn, adds the second person plural pronoun, and uses rising \isi{intonation}. Mara ignores him, and Linos reacts with frustration in line 11. His turn functions as a summons \citep{schegloff68} that aims to secure Mara’s attention and availability. Μara responds to the summons by displaying her attentiveness in line 12. Linos repeats his question in line 13 (\textit{>\underline{Pó}te tha fíʝete e\underline{sís}.<} `When are you leaving.'), with emphasis on the interrogative word and the second person plural pronoun, and falling \isi{intonation}. Mara answers the question in line 15. In this excerpt, the speaker asks a question that anticipates a response by the recipient but the recipient does not respond. The speaker pursues an articulated response by repeating the clause that he used to implement his question, and thus uses \isi{repetition} as an attention-getting device. 

\begin{Transcript}[FS {>}{>\hspace{0.1in}}]{Mar}{to the market, she wants to buy a T-shirt.}%
\label{Alex8}%
\TLine{Mar}{% 
		\underline{Pé}mpti íne anixtá. ci i Kalirói éç faɣoθí na páme.
		\glt `It is open on Thursday. and Kaliroi insists that we go.'}\\
\TLine{}{% 
		stin aɣorá na psonís{[}i:  \hspace{0.2in}  blú{]}za.=
		\glt `to the market, she wants to buy a T-shirt.'}\\
\TLine{Roz}{% 
		\hspace{1.3in}{[}\underline{Sí}mera?{]}
		\glt \hspace{1.3in} `Today?'}\\
\TLine[{FS >}{>}]{Lin}{%
  \gll  \textbf{=Mára,} \textbf{póte} \textbf{{[}févʝete}(.....){]}\\
 Mara     when leave.\textsc{2pl.prs}\\
 \glt `Mara, when are you leaving (...)'}\\
 \TLine{Mar}{% 
		\hspace{0.85in} {[}ðen   $\uparrow$báo     sí{]}:mera.=
		\glt \hspace{1in} `I am not going today.'}\\
\TLine[{SS >}{>}]{Lin}{%
  \gll  \textbf{=}\textbf{>Pó{[}te} \textbf{θa} \textbf{fiʝete}<{]}\\
 when \textsc{fut} leave.\textsc{2pl.pfv}\\
 \glt `When are you leaving?'}\\   
 \TLine{Mar}{% 
		\hspace{0.3in} {[}$\uparrow$Alá:  \hspace{0.2in}     áma{]}vɣo na psoníso ap ti má:na,=
		\glt \hspace{0.3in} `But if I go shopping for mum,'}\\
\TLine[{SS+ >}{>}]{Lin}{%
  \gll  \textbf{=}\textbf{>\underline{Pó}te} \textbf{θa} \textbf{fiʝete} \textbf{es{[}ís?<{]}}\\
 when \textsc{fut} leave.\textsc{2pl.pfv} \textsc{2pl.nom} \\
 \glt `When are you leaving?'}\\ 
  \TLine{Mar}{% 
		\hspace{1.7in} {[}pu \hspace{0.1in} {]} \underline{θé}l
		\glt \hspace{1.7in} `she wants'}\\
\TLine{}{% 
		 patá{[}es, \hspace{0.4in} \underline{θél}{]} 
		\glt  `potatoes, she wants'}\\
\TLine{Lin}{% 
		  \hspace{0.25in} {[}>Re su mi$\uparrow$lá{]} o re Dalára.<
 		\glt  \hspace{0.25in} `Hey I am talking to you.'}\\
\TLine{Mar}{% 
		  {[}Ne.  \hspace{0.3in}{]} 
 		\glt  `Yes.'}\\
\TLine[{SS+ >}{>}]{Lin}{%
  \gll  \textbf{{[}>\underline{Pó}te} \textbf{θa}{]} \textbf{fiʝete} \textbf{e\underline{sís}.}<\\
 when \textsc{fut} leave.\textsc{2pl.pfv} \textsc{2pl.nom} \\
 \glt `When are you leaving?'}\\ 	
 \TLine{}{(0.8)}\\	
\TLine{Mar}{% 
		  \underline{ðen} gzéro, \underline{Sá}:vato?
		\glt  `I don’t know, on Saturday?'}\\
		\hspace{1in}(.)\\
\TLine{Lin}{% 
		   Α\underline{::} >tha fíʝete \underline{Sá}vato.<
 		\glt  `Ah:: you are leaving on Saturday.'}\\
\end{Transcript}

Finally, self-\isi{repetition} operates as a stylistic feature used for emphasis. In example (\ref{Alex9}), participants assess positively a movie they watched. In lines 4--5, Yannis refers to a scene of \isi{action} that he found exciting, and he uses the interrogative clause \textit{zi i péθane?} `is he alive or dead?' to express the audience’s suspense during the screening. In line 6, he repeats the clause twice with non-falling \isi{intonation} (\textit{zi i péθane,} `is he alive or dead') in order to intensify the suspense. This self-\isi{repetition} is semantically based and iconically motivated \citep[cf.][]{norrick87}; it indicates the speaker’s emotional involvement, and has a clear emphatic function. 

\begin{Transcript}[FS {>}{>\hspace{0.3in}}]{Yan}{The movie creates such a suspense when nothing is happening,}%
\label{Alex9}%
\TLine{Yan}{% 
		 ={[}Το pos kata{]}férni {[}i tenía xorís na{]} simví {[}<$\uparrow$\underline{tí}po{]}ta,> \vspace{-0.2in}
		\glt `The movie creates such a suspense when nothing is happening,'}\\
\TLine{Ama}{% 
		 ={[}Polí  oréo.\hspace{0.1in}{]} \hspace{0.3in} {[}Po\underline{lí}  oréo.  \hspace{0.25in}{]}
		\glt `Very nice. Very nice.'}\\
\TLine{Nik}{% 
		\hspace{2.6in} {[}(Foveró.){]} 
		\glt \hspace{2.6in}`Fantastic.'}\\		
\TLine[{FS >}{>}]{Yan}{%
  \gll esí  na  se  \underline{é\t{ts}}i.  \textbf{zi}  \textbf{i} \\
 \textsc{2sg.nom} \textsc{sbjv} \textsc{cop.2sg.prs} like.that  live.\textsc{3sg.prs} or \\
 \glt `you are wondering. Is he alive or'}\\
 \TLine{}{% 
	 \gll	\textbf{péθane?} \\
		die.\textsc{3sg.pst}\\
	\glt `dead?'}\\
\TLine[{SS/SS+}{>}{>}]{}{%
  \gll \textbf{zi}  \textbf{i} \textbf{péθane,} \textbf{{[}zi}  \textbf{i} \textbf{pé{]}θane,} \\
 live.\textsc{3sg.prs} or die.\textsc{3sg.pst} live.\textsc{3sg.prs} or die.\textsc{3sg.pst} \\
 \glt `he alive or dead, he alive or dead'}\\
 \TLine{}{% 
	 \gll	.hh ce: \\
		hh and\\
	\glt `.hh and'}\\
    \TLine{Nik}{% 
		 \hspace{1.6in}{[}(Oréo.) \hspace{0.5in} {]}
		 \glt \hspace{1.6in} `(Nice.)'}\\    
\end{Transcript}

We now turn to repetitions that build on the prior turn produced by a different speaker.  

\subsection{Repetition of a prior turn at talk}
\label{Alrepetition.turn}
In next turns, speakers display how their current turn is connected with the prior turn produced by another speaker. Clause \isi{repetition} is among the resources that speakers employ to display this connectedness. In all 57 instances of \isi{repetition} of a prior turn at talk found in the data, speakers repeat clauses from prior turns produced by different speakers in order to embody their understanding of what the previous speakers did, and implement actions that respond to the just prior turn. In these cases, \isi{clause repetition} is a \isi{practice} that connects speaker’s current turn with prior talk. 

Answers to polar questions are a common interactional context in which repetitions of prior turn occur, as shown in examples (\ref{Alex10}) and (\ref{Alex11}). In this \isi{sequential} position, \isi{repetition} connects the speaker’s current turn with prior talk and allows the speaker to claim more agency with respect to the \isi{action} she is implementing \citep[cf.][]{heritage2012}. In (\ref{Alex10}), in lines 3--4, Roza asks Mara a question (\textit{Ce ðe- ci íne \underline{tό}so ʝelío epiçírima?} `And not- and is it such a ridiculous argument?'). In line 5, Mara replies with the confirmation particle \textit{ne} `yes', and repeats the clause that Roza used in her prior turn (\textit{íne ʝe\underline{lí}o epiçírima} `it is a ridiculous argument'), with modifications. She omits the adverb and adds emphasis on the adjective. 

\begin{Transcript}[FS {>}{>\hspace{0.1in}}]{Mar}{Five thousand Jews are said to have been working there,}%
\label{Alex10}%
\TLine{Mar}{% 
		 \underline{té}tça práɣmata. 
		\glt `such things.'}\\
\TLine{}{% 
		\hspace{0.1in} {[}aftό to len<           diá:fori. \hspace{0.3in}{]}
		\glt\hspace{0.1in} `many people say this.'}\\		
\TLine[{FS >}{>}]{Roz}{%
  \gll  {[}Ce   ðe-  ci   \textbf{íne}             \textbf{t\underline{ό}{]}so} \textbf{ʝelío} \\
 and \textsc{neg} and \textsc{cop.3sg.prs} so ridiculous  \\
 \glt  `And not- and is it such a ridiculous'}\\
 \TLine[{FS >}{>}]{}{%
  \gll   \textbf{{[}epiçírima?} \hspace{0.32in}\textbf{{]}}  \\
  argument\textsc{(neut).nom.sg}  \\
 \glt `argument?'}\\
\TLine[{SS >}{>}]{Mar}{%
  \gll  {[}Ne   \textbf{íne}             \textbf{ʝe\underline{lí}o} \textbf{epi{]}çírima?}\\
   yes \textsc{cop.3sg.prs}  ridiculous argument\textsc{(neut).nom.sg} \\
 \glt   `Yes it is a ridiculous argument,'}\\
  \TLine{}{% 
		 {[}alá (...){]}
		 \glt `but (...)'}\\ 
\TLine{Roz}{% 
		 {[}$\uparrow$Pé:de{]} çili\underline{á}ðes Εvr\underline{é}i $\downarrow$ítan léi:, ecí pu ðúlevan,
		\glt `Five thousands Jews are said to have been working there,'}\\   
\end{Transcript}

In example (\ref{Alex11}), Ourania replies (lines 3--4) to Chrysanthi’s polar question (lines 1--2). The question is implemented via the interrogative clause \textit{Itan- efiméreve to °Xad͡zikósta?} `Was- was the Hatzikosta hospital open?', and the answer is implemented via \isi{repetition} of the clause with falling \isi{intonation} (\textit{Efiméreve to Xatzikósta.} `The Hatzikosta hospital was open.').\footnote{A declarative or subjunctive \isi{main clause} in \ili{Greek} can be turned into a polar question through rising \isi{intonation} toward the end of the utterance.} The \isi{clause repetition} in this excerpt is modified: the speaker adds emphasis on the verb, and uses falling \isi{intonation} that turns the clause into a statement. 

%cant get the DZ plus tie on top to work. works with ts [see ex2], not dz
\begin{Transcript}[FS {>}{>\hspace{0.1in}}]{Chr}{Was- was the Hatzikosta hospital open?}%
\label{Alex11}%
\TLine[{FS >}{>}]{Chr}{%
  \gll  Itan-           \\
 \textsc{cop.3sg.pst}  \\
 \glt `Was-'}\\
      \TLine{}{% 
		\gll \textbf{efi{[}méreve }          \textbf{to}                       \textbf{°Xa\underline{d͡zi}kósta?{]}} \\
		be.on.duty.\textsc{3sg.pst} \textsc{def.neut.nom.sg} Hatzikosta \\
		 \glt  `was the Hatzikosta hospital open?'}\\  
\TLine[{SS >}{>}]{Our}{%
  \gll    \hspace{0.15in}   {[}.h      \textbf{Efi\underline{mé}reve} \hspace{1.5in} {]}                           \\
\hspace{0.15in} .h be.on.duty.\textsc{3sg.pst}   \\
 \glt }\\ 
 \TLine{}{%
  \gll      \textbf{to}                       {\bfseries Xa\underline{d͡zi}kósta.} \\
 \textsc{def.neut.nom.sg} Hatzikosta   \\
 \glt `The Hatzikosta hospital was open.'} 
\end{Transcript}

Clause repetitions are also found in agreement or disagreement with a prior turn. In example (\ref{Alex12}), lines 1--2, Aleka assesses the neighborhood (\textit{Αplós íne períerʝi i perioçí.} `It’s just a weird neighborhood.'), and in lines 3--4, Polychronis agrees with the assessment (\textit{Ine pe\underline{rí}erʝi i perioçí.} `It’s a weird neighborhood.'). He repeats the copula clause that Aleka used in her previous turn, with emphasis on the adjective, and he omits the adverb. This slightly modified \isi{repetition} is a \isi{practice} for implementing an agreement with the prior turn from an ``independent agentive position'' \citep[][285]{Thompson15}.  

\begin{Transcript}[FS {>}{>\hspace{0.1in}}]{Ale}{It’s a weird neighborhood. that’s why.}%
\label{Alex12}%
\TLine[{FS >}{>}]{Ale}{%
  \gll =Αplós \textbf{íne}              \textbf{períerʝi}              \textbf{i}                 \\
 just     \textsc{cop.3sg.prs}  weird.\textsc{f.nom.sg} \textsc{def.f.nom.sg}  \\
 \glt `It’s just a weird'}\\
 \TLine{}{%
  \gll      {[}\textbf{perioçí.}   {]} \\
 area\textsc{(f).nom.sg}   \\
 \glt `neighborhood.'}\\   
\TLine[{SS >}{>}]{Pol}{%
  \gll       \textbf{{[}Ine}              \textbf{pe\underline{rí}{]}erʝi}                            \\
 \textsc{cop.3sg.prs}  weird.\textsc{f.nom.sg}   \\
 \glt }\\ 
\TLine[{SS >}{>}]{}{%
  \gll     \textbf{i}                \textbf{perioçí.}  ʝaftó. \\
  \textsc{def.f.nom.sg} area\textsc{(f).nom.sg}  this\\
 \glt `It’s a weird neighborhood. That’s why.'}\\ 
\end{Transcript}

In example (\ref{Alex13}), \isi{clause repetition} is a \isi{practice} for disagreeing with the previous speaker. In line 3, Aleka makes a claim (\textit{ta: ta riθímzi $\uparrow$tóra mɲa xará}. `he keeps things- things in moderation very well.'), and in line 5, Polychronis contradicts the claim (\textit{°\underline{ðe} ta riθmízi.} `He doesn’t keep things in moderation)'). Polychronis utters the negated proposition expressed in the previous claim, by repeating the clause that Aleka used, omitting the adverbs and adding the negative particle before the clause.

\begin{Transcript}[FS {>}{>\hspace{0.2in}}]{Ale}{No. look. that is, he has kept things in moderation compared to}%
\label{Alex13}%
\TLine{Ale}{% 
		 $\uparrow$\underline{O}çi. cítakse. ðilaðí, ta çi riθmísi ta práɣmata se sçési 
		\glt `No. Look. That is, he has kept things in moderation compared to'}\\
\TLine{}{% 
		me to: pos ítan >(ótan eɣó)< to- to ɣ\underline{nó}risa,
		\glt `how things were (when) I met him,'}\\	
\TLine[{FS >}{>}]{}{%
  \gll       \textbf{ta:}    \textbf{ta}     \textbf{riθímzi}              \textbf{$\uparrow$tóra}  \textbf{mɲa} \textbf{xará}. \\
 them them  regulate.\textsc{3sg.prs}  now   very well \\
 \glt `he keeps things- things in moderation very well.'}\\ 
  \TLine{}{(1.2)}\\
\TLine[{SS >}{>}]{Pol}{%
  \gll     °\textbf{(\underline{ðe}}   \textbf{ta}      \textbf{riθmízi.)} \\
  \textsc{neg} them regulate.\textsc{3sg.prs}\\
 \glt `He doesn’t keep things in moderation.'}\\ 
\end{Transcript}

Clause \isi{repetition} is also used in next turns that confirm what the previous speaker said (\ref{Alex14}), receive information given by the previous speaker (\ref{Alex15}), or deliver repair within a story telling (\ref{Alex16}). In (\ref{Alex14}), participants are engaged in conversational arguing \citep{Muntigl98}. In the lines preceding the excerpt, Nionios claims that he and his peers never cooked when they were teenagers. Yannis contradicts the previous claim (lines 1--2), and asserts that he and his peers cooked (\textit{emís to \underline{ká}name.} `we did it.'). In line 4, Nionios initially confirms Yannis’s claim by repeating the clause that Yannis used in his previous turn (\textit{Το \underline{ká}name:.} `we did it'). The second saying that implements the confirmation is modified: the first person plural pronoun is omitted. In the next TCU, Nionios delivers a counterclaim that does not directly contradict nor challenge the addressee’s claim. 

\begin{Transcript}[FS {>}{>\hspace{0.2in}}]{Yan}{No. Why didn’t we do the same? We did it.}%
\label{Alex14}%
\TLine[{FS >}{>}]{Yan}{%
  \gll       \underline{O}çi. ʝa\underline{ti}   ðen  do káname    emís. =\textbf{emís} \textbf{to} \textbf{\underline{ká}name}. \\
 no    why \textsc{neg}  it   do.\textsc{1pl.pst} \textsc{1pl}     \textsc{1pl}    it  do.\textsc{1pl.pst} \\
 \glt `No. Why didn’t we do the same? We did it.'}\\
\TLine{}{% 
		 =e\underline{ɣó} ðe ma{[}ʝ\underline{í}{]}reva?=
		\glt `Wasn’t I the one cooking?'}\\
\TLine{Nio}{ \hspace{0.7in}{[}T-{]}}\\
\TLine[{SS >}{>}]{}{%
  \gll     \textbf{To} \textbf{\underline{ká}name:.} sa\underline{fós}       to káname\\
   it  do.\textsc{1pl.pst}  certainly it  do.\textsc{1pl.pst}\\
 \glt `We did it. We certainly did it'}\\ 
 \TLine{}{% 
		 allá:: ðen do kánan óla ta peðʝá::.
		\glt `but not all kids were doing the same thing.'}
\end{Transcript}


In example (\ref{Alex15}), line 2, Erato asks Yorgos if he switched the kitchen stove off, assuming that the food is ready, and in lines 4--5, Yorgos replies that he didn’t because the food is not ready (\textit{maʝi\underline{ré}°vete (akόma)}. `the food is (still) cooking.'). In line 6, Erato proposes the possible end of the sequence by claiming information receipt. Her turn is composed by three TCUs. The first TCU consists of the free-standing particle \textit{\underline{α}}, uttered with emphasis, which marks a change from not-knowing to now-knowing (similar to the \ili{English} particle \textit{oh}, \citealt{heritage1984}). In the second TCU, the speaker reuses elements from Yorgos’s prior turn to express receipt of information. She repeats the adverb \textit{tόra} `now' and the clause that delivers the informing \textit{maʝirévete °akόmi.} (`it is still cooking.'), with no emphasis on the verb. In the third TCU, the speaker accepts the information via the positive token particle \textit{ne} `yes'. 


\begin{Transcript}[FS {>}{>\hspace{0.2in}}]{Yor}{No. I don’t need to switch it off now, the food is still cooking.}%
\label{Alex15}%
\TLine{Yor}{% 
		 ti {[}faʝitá       íçe,\hspace{0.3in}{]}
		\glt `What kind of food they served,'}\\
\TLine{Era}{% 
		\hspace{0.15in} {[}Eklises to má{]}ti?
		\glt \hspace{0.15in}`Did you switch the stove off?'}\\
\TLine{}{(.)}\\		
\TLine[{SS >}{>}]{Yor}{%
  \gll    Oçi. ðe >xriázete               \textbf{tόra:,} \\
   no    \textsc{neg} need.\textsc{3sg.pass.prs} now \\
 \glt `No. I don’t need to switch it off now,'}\\ 
 \TLine{}{% 
		\gll \textbf{maʝi\underline{ré}°vete}        \textbf{(akόma).}<=\\
		cook.\textsc{3sg.pass.prs} still\\
		\glt `the food is still cooking.'}\\ 
 \TLine[{SS >}{>}]{Era}{%
  \gll    ={[}\underline{A}. >\textbf{tόra} \textbf{maʝiréve}{]}\textbf{{[}te} \textbf{akό{]}mi.} ne.<\\
   part  now   cook.\textsc{3sg.pass.prs} still    yes\\
 \glt `Ah. now it’s still cooking. yes.'}\\ 
 \TLine{Sot}{={[}°(...........................){]}}\\	
 \TLine{Yor}{% 
		 \hspace{1.3in} {[}°Ne \hspace{0.4in}{]}
		\glt  \hspace{1.3in} `Yes'}\\
\end{Transcript}

In example (\ref{Alex16}), Polychronis tells a story about a funny incident (lines 1--3, 5). He refers to the protagonists in the story via first person plural verbs \textit{ksecinísame} `we started', \textit{na páme} `to go', \textit{\underline{ðe} vríkame} `we didn’t find', \textit{ʝirnáɣame} `we were wandering around', and the pronoun \textit{mas} `us'. The collectivity introduced includes the speaker and one of the co-participants. Aleka’s participation in the story events establishes her as a story consociate that shares knowledge of the story events \citep{Lerner92}. Story consociates can participate in the course of story delivery by continuing the story or by repairing aspects of the story and its delivery, such as trouble in the event sequencing of the story, in the delivery of the story, in story elaboration, and in the facts of the story \citep{Lerner92}. In line 2, Polychronis reports with uncertainty that he, Aleka and the others went to Zythos restaurant (\textit{°ksecinísame na páme sto Zíθo°} `were we going to Zithos?'). In lines 6--7, Aleka repairs trouble in this fact of the story. She starts her turn with the negative particle \textit{\underline{o}çi} `no' that expresses her disagreement with what Polychronis said immediately before. She delivers the repair by repeating a clause that Polychronis used to refer to the specific fact of the story (\textit{ksecinísame, h na- na páme} `we were going'), and she adds the phrase \textit{ʝa kafé} `for coffee'. 

\begin{Transcript}[FS {>}{>\hspace{0.2in}}]{Pol}{And it just happened to us because we started heading to another place,}%
\label{Alex16}%
\TLine{Pol}{% 
		 = >Ce mas proécipse cόlas< ʝatí ʝa\underline{lú} ksecinísame,
		\glt `And it just happened to us because we started heading to another place,'}\\
\TLine[{FS >}{>}]{}{%
  \gll  °\textbf{ksecinísame} \textbf{na}    \textbf{páme}   sto                       Zíθo?°\\
   begin.\textsc{1pl.pst}  \textsc{sbjv} go.\textsc{1pl} to Zitho\textsc{(m).acc.sg}\\
 \glt `were we going to Zithos?'}\\ 
 \TLine{Pol}{% 
		 \underline{pú} ítane. {[}\underline{ðe}{]} vríkame trapézi °ecí péra >ce metá,°<
		\glt `where was it? We didn’t find a table over there and afterwards,'}\\
 \TLine{Ale}{% 
		  \hspace{0.5in} {[}Ne{]} 
		\glt  \hspace{0.5in} `Yes.'}\\
\TLine{Pol}{% 
		(.) \underline{ká}pos ʝirnáɣame, (ékane-) íçe po\underline{lí} krío °ecíni {[}ti méra,{]} \vspace{-0.2in}
		\glt `we were wandering around, it was- it was a very cold day,'}\\
\TLine{Ale}{% 
		  \hspace{2.9in} {[}\underline{O}çi. \hspace{0.1in} {]} 
		\glt  \hspace{2.9in} `No.'}\\
\TLine[{SS >}{>}]{}{%
  \gll  \textbf{ksecinísame}, h \textbf{na-}   \textbf{na} \textbf{páme}  \textbf{ʝa}   \textbf{kafé.} \\
   begin.\textsc{1pl.pst}  h \textsc{sbjv} \textsc{sbjv} go.\textsc{1pl} for coffee\\
 \glt `were we going for coffee.'}\\ 
\end{Transcript}

In the examples examined in this section, \isi{repetition} of a prior turn at talk is a \isi{practice} for responding to what the previous speaker did immediately before. Therefore, it displays the relevance between first and second pair part, and the fit between current and prior turn, and it operates as a tying technique. 

\subsection{Summary}
\label{AlSummary}
To recapitulate, the analysis of clause repetitions in \ili{Greek} conversation shows that the basic function of \isi{clause repetition} is \isi{cohesive}. Speakers often repeat clauses to display the connectedness between their current unit/turn and prior talk. Being an instance of format tying, \isi{clause repetition} is deployed in various \isi{sequential} contexts to carry out different social actions that respond to the just prior turn, such as answer, agreement/disagreement, confirmation, receipt of information, and repair. Moreover, the analysis demonstrates that the \isi{sequential} position of \isi{clause repetition} shapes the interactional functions of \isi{repetition}. Self-\isi{repetition} achieves \isi{cohesion} in conversation as well as other interactional tasks, such as dealing with overlapping talk, pursuing a response, initiating and delivering repair and adding emphasis. On the other hand, \isi{repetition} of a prior turn is routinely associated with a \isi{cohesive} function. Thus, \textit{who repeats} seems to be important for \textit{what \isi{repetition} does}. Overall, the findings reported in this study align with the findings reported by previous studies on the functions of \isi{repetition} in conversation (discussed in \refsec{Alrole.repetition}).

\section{From repetition to bridging constructions: Language diversity as a continuum}
\label{Alcontinuum}
Although \isi{clause repetition} and \isi{recapitulative linkage} differ in substantial ways (cf. \refsec{AlIntroduction}), they display certain analogies: like \isi{recapitulative linkage}, \isi{clause repetition} in \ili{Greek} conversation involves \isi{repetition} of at least the verb of the first saying and some of the elements accompanying the verb, and achieves \isi{cohesion}. Moreover, both \isi{recapitulative linkage} and \isi{repetition} practices are discourse practices. I suggest that these analogies point to a \textit{continuum} extending from \isi{clause repetition} at one extreme to \isi{recapitulative linkage} at the other extreme. In languages situated at the one extreme of the continuum \isi{clause repetition} has not been conventionalized, while in languages situated at the other extreme of the continuum \isi{clause repetition} has grammaticalized into \isi{recapitulative linkage}. 

It is possible that \isi{recapitulative linkage} constructions have emerged from \isi{repetition} practices in talk-in-interaction. The hypothesis about the discourse origin of \isi{recapitulative linkage} aligns with research that examines how discourse or interaction shapes grammar \citep{givon79,hopper80,ochs96,Selting01}. In Bybee’s words (\citeyear[][730]{bybee.2006}), ``grammar cannot be thought of as pure abstract structure that underlies language use''; grammar emerges in language use and it is ``epiphenomenal to the ongoing creation of new combinations of forms in interactive encounters'' \citep[][26]{hopper11}. As a number of studies \citep{couper11,gipper11,blythe.2013} demonstrate, discourse contexts motivate the \isi{grammaticalization} of specific constructions. For instance, \citet{couper11} argues that certain grammatical constructions, such as left dislocation, concession and extraposition, have emerged from the \isi{sequential} routines of mundane conversational interaction, whereby a succession of (cross-speaker) actions has been ``collapsed into'' a single speaker’s turn. This integrated construction can be said to grammaticalize from the conversational routine. For example, \citet{geluykens92}, cited in \citet{couper11}, suggests that left dislocation, in which a noun phrase is positioned initially and a reinforcing pronoun stands proxy for it in the relevant position in the sentence, has emerged from the recognition search sequence. This sequence consists of three moves in which the speaker introduces a new referent, the hearer acknowledges recognition of the referent, and the speaker elaborates upon the referent. According to \citet[][429]{couper11}, left dislocation is found in \ili{English} conversation both in its independent and integrated form (layering, cf. \citealt{hopper91}). In its independent form, the two component parts accomplish two different actions, i.e., they establish referents and elaborate upon them. In its integrated form, the two component parts are coalesced with no intervening turn or pause separating them, and they deliver one single \isi{action}, that is, they are specialized for listing and contrast. 

In line with these views, I suggest that \isi{recapitulative linkage} emerged from conversational routines: at some point, in certain languages, \isi{repetition} practices aiming at \isi{cohesion} were conventionalized and became part of grammar, that is, they grammaticalized into specific resources or patterns with a productive formal representation and a consistent and predictable semantic contribution (cf. \citealt{guerin18}, this volume). Although it is difficult to provide diachronic evidence for such a hypothesis, given that we lack records of talk-in-interaction in languages with bridging constructions, we have access to some synchronic evidence that point to the discourse origin of \isi{recapitulative linkage} constructions. 

The first type of evidence comes from languages in which \isi{repetition} is conventionalized to some extent. For example, in Tojolabal Mayan conversation, \isi{repetition} has become the default backchannel response to turns delivered by other speakers \citep[][260--261]{brody.1986}. As \citet[][224]{brown.2000} claims, ``this conversational \isi{practice} makes Mayan conversations strike the outside observer as extraordinarily repetitive, drawing attention to the fact that tolerance for \isi{repetition} in speech is \textit{culturally}, as well as contextually, quite variable'' (emphasis added). Clause \isi{repetition} is a rather common conversational \isi{practice} among speakers in certain languages. Due to its frequency \citep{bybee03} and cultural salience, \isi{clause repetition} crystallizes into specific grammatical constructions in these languages.\footnote{\citeauthor{jarkey18} (this volume) shows that \isi{summary linkage} in \ili{White Hmong} (Hmong-Mien, Laos) is limited to first person narratives and reported speech; this finding further points to the conventionalization of linkage constructions.} 

The second type of evidence comes from languages that employ \isi{recapitulative linkage} constructions. \citet[][112--113]{Guillaume2011} reports that languages vary in terms of the functions of \isi{recapitulative linkage}. Most languages use \isi{recapitulative linkage} to achieve coherence in context of high \isi{thematic continuity}, that is, within individual paragraphs. Yet some languages employ additional \isi{recapitulative linkage} constructions specialized for major thematic breaks, that is, between distinct paragraphs. Thus, languages develop formally distinct types of \isi{recapitulative linkage} for carrying out different tasks in discourse. This variation further discloses the interactionally motivated and emerging nature of \isi{recapitulative linkage}. More specifically, it shows that the development of \isi{recapitulative linkage} constructions involves the emergence of new forms that coexist and interact with the older forms (layering, \citealt{hopper91}), and the specialization of meanings attached to the forms in particular discourse contexts. Both layering and specialization are distinctive characteristics of \isi{grammaticalization} \citep{hopper93}. 

The third type of evidence for the discourse origin of \isi{recapitulative linkage} can be found in universal abstract principles governing linguistic practices in talk-interaction: nextness and progressivity \citep{schegloff06}. Nextness is a relation between current and immediately following position. The production of talk is a succession of next elements, such as words, parts of words or sounds. As \citet[][86]{schegloff06} argues, ``absent any provision to the contrary, any turn will be heard as addressed to the just prior, that is, the one it is next after''. Progressivity refers to the \isi{sequential} progress of interaction. Recipients orient to each next element as ``a next piece in the developing trajectory of what the speaker is saying or doing'' \citep[][86]{schegloff06}. These two principles operate in \isi{clause repetition} and bridging constructions: (a) repeats establish a relation between current and prior turn or TCU (nextness); (b) in reusing prior sayings, repeats disrupt the linguistic progressivity in talk-in-interaction, and, thus, they are examinable for their pragmatic import. That is, universal principles governing talk-in-interaction can function as constraints on ``what systems can evolve'', and ``selectors'' generating structures \citep[][446]{evans.2009}.  

By bringing together findings from languages with bridging constructions and a language in which bridging constructions are not grammaticalized, this paper demonstrates the fuzzy boundaries between bridging constructions and verbal \isi{repetition} and makes a case for the discourse origin of \isi{recapitulative linkage}. 



\section*{Abbreviations}
\begin{multicols}{3}
\begin{tabbing}
\textsc{prep}\hspace{1em} \= Diminutive\kill
1             \>  First person\\
2             \>  Second person\\
3             \>  Third person\\
\textsc{acc} \> Accusative\\
\textsc{clit} \> Clitic\\
\textsc{conj} \> Conjunction\\
\textsc{cop} \> Copula\\
\textsc{dim} \> Diminutive\\
\textsc{f} \>     Feminine\\
\textsc{fut} \> Future\\
\textsc{neg} \> Negation\\
\textsc{neut} \> Neuter\\
\textsc{nom} \> Nominative\\
\textsc{part} \> Particle\\
\textsc{pass} \> Passive\\
\textsc{pst} \> Past\\
\textsc{pfv} \> Perfective\\
\textsc{pl} \>     Plural\\
\textsc{prep} \> Preposition\\
\textsc{prs} \> Present\\
\textsc{sg} \>     Singular\\
\textsc{sbjv} \> Subjunctive\\
\end{tabbing}
\end{multicols}


\section*{Appendix: Transcription Symbols}
\begin{description}
\item 	The left bracket [ is the point of overlap onset between two or more utterances (or segments of them).
\item  The right bracket ] is point of overlap end between two or more utterances (or segments of them).
\item   The equal sign = is used either in pairs or on its own. A pair of equals signs is used to indicate the following:\\
(i) If the lines connected by the equals signs contain utterances (or segments of them) by different speakers, then the signs denote ``latching'' (that is, the absence of discernible silence between the utterances). 

(ii) If the lines connected by the equals signs are by the same speaker, then there was a single, continuous utterance with no break or pause, which was broken up in two lines only in order to accommodate the placement of overlapping talk. 
The single equals sign is used to indicate latching between two parts of the same speaker’s talk, where one might otherwise expect a micro-pause, as, for instance, after a turn constructional unit with a falling \isi{intonation} contour. 
\item  Numbers in parentheses (0.8) indicate silence, represented in tenths of a second. Silences may be marked either within the utterance or between utterances. 
\item  (.) indicates a micro-pause (less than 0.5 second).
\item A period indicates falling/final \isi{intonation}.
\item A question mark indicates rising \isi{intonation}.
\item A comma indicates continuing/non-final \isi{intonation}. 
\item Colons : are used to indicate the prolongation or stretching of the sound just preceding them. The more colons, the longer the stretching.
\item Underlining is used to indicate some form of emphasis, either by increased loudness or higher pitch. 
\item The degree sign ° is used to indicate the onset of talk that is markedly quiet or soft. When the end of such talk does not coincide with the end of a line, then the symbol is used again to mark its end. 
\item A hyphen - after a word or part of a word indicates a cut-off or interruption.
\item Combinations of underlining and colons are used to indicate \isi{intonation} contours. If the letter(s) preceding a colon is underlined, then there is prolongation of the sound preceding it and, at the same time, a falling \isi{intonation} contour. If the colon itself is underlined, then there is prolongation of the sound preceding it and, at the same time, a rising \isi{intonation} contour. 
\item The arrows mark sharp \isi{intonation} contours. The upper arrow $\uparrow$ indicates sharp \isi{intonation} rises, whereas the down arrow $\downarrow$  indicates sharp \isi{intonation} falls. 
\item The combination of the symbols  > and < indicates that the talk between them is compressed or rushed. 
\item The combination of the symbols < and > indicates that the talk between them is markedly slowed or drawn out. 
\item Hearable aspiration is shown with the Latin letter h. Its \isi{repetition} indicates longer duration. The aspiration may represent inhaling, exhaling, laughter, etc. 
\item If the aspiration is an inhalation, then it is indicated with a period before the letter h. 
\item Double parentheses and italics are used to mark meta-linguistic, para-linguistic and non-conversational descriptions of events by the transcriber. For example: ((laughs))
\item  Parentheses with dots (...) indicate that something is being said, but no hearing can be achieved. 
\item Words in parentheses represent a likely possibility of what was said. 
\end{description}



\section*{ Acknowledgments}
The first version of this paper was presented at the workshop ``Bridging linkage in cross-linguistic perspective'' at the Language and Culture Research Center, James Cook University, on 25--26 February 2015. I am grateful to the participants of this event for their feedback. I would like to thank the volume editor and two anonymous reviewers for providing insightful comments that helped to improve my arguments. 

\sloppy

\printbibliography[heading=subbibliography,notkeyword=this] 
\end{document}

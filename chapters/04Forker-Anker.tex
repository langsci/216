\documentclass[output=paper]{LSP/langsci} 
\ChapterDOI{10.5281/zenodo.2563684}
\author{Diana Forker\affiliation{University of Jena}\lastand Felix Anker\affiliation{University of Bamberg}}
\title{Bridging constructions in Tsezic languages} 
%\epigram{Change epigram in chapters/01.tex or remove it there }
\abstract{This paper treats bridging constructions in the Tsezic languages (Bezhta, Hunzib, Khwarshi, Hinuq, and Tsez) of the Nakh-Daghestanian language family. We describe the syntactic and semantic properties of bridging constructions based on corpus data from all five Tsezic languages. Bridging constructions are defined as bipartite constructions that consist of a finite reference clause, which is followed by a non-main adverbial clause that functions as the bridging clause. The adverbial clause contains a variety of temporal converbs with general perfective converbs being more common than other types of temporal converbs. Reference and bridging clauses are both a target for additions, omissions, modifications and substitutions. Bridging constructions are primarily found in traditional oral narratives such as fairy tales where they index the genre and function as stylistic devices to express parallelism. Within the narratives they are often used to indicate episode changes and can be accompanied by switches of subject referents or locations.}
\maketitle
%-------------------------

\begin{document}\label{ch:4}

\section{Introduction} 
\label{sec:Introduction}
The \ili{Tsezic} languages form one branch of the \ili{Nakh-Daghestanian} (or North-East Caucasian) language family and are traditionally grouped into two sub-families, the East \ili{Tsezic} languages comprising \ili{Bezhta} and \ili{Hunzib} and the West \ili{Tsezic} languages comprising \ili{Hinuq}, \ili{Khwarshi} and \ili{Tsez}. \ili{Tsezic} languages are mainly spoken in the northern part of the Caucasus in the Republic of Daghestan in the Russian Federation. \ili{Tsezic} languages are dependent marking and morphologically ergative. They are famous for their rich case systems, especially in the spatial domain, and their gender systems. For most of the \ili{Tsezic} languages there are grammatical descriptions or at least sketch grammars (see \citealt{Forker.2013a} for \ili{Hinuq}; \citealt{Khalilova.2009} for \ili{Khwarshi}; \citealt{vandenBerg.1995} for \ili{Hunzib};  \citealt{Comrie.et.al.2015} for \ili{Bezhta}; \citealt{Kibrik.Testelets.2004} for a sketch grammar of \ili{Bezhta} and \citealt{Alekseev.Radzhabov.2004} for a sketch grammar of \ili{Tsez}). Further syntactic descriptions of \ili{Tsez} are \citet{Radjabov.1999} and \citet{Polinsky.InPreparation}.

We assume that bridging constructions can be found in all \ili{Nakh-Daghestanian} languages. We will, however, concentrate on the \ili{Tsezic} languages in this paper because for this subgroup we have more data at our disposal than for any of the other subgroups. The most common type of bridging construction in \ili{Tsezic} is \isi{recapitulative linkage}, while \isi{summary linkage} is only used rarely and \isi{mixed linkage} is not found at all (for a definition and classification of the three possible bridging constructions see the introductory chapter to this volume and \refsec{sec:Types of bridging constructions} below).

The paper is structured as follows: in \refsec{sec:Formal characteristics} we will outline formal properties of bridging constructions in \ili{Tsezic} languages, i.e., syntactic properties of the reference clause and the bridging clause. \refsec{sec:Types of bridging constructions} deals with the two types of bridging constructions, \isi{recapitulative linkage} and \isi{summary linkage}. In \refsec{sec:Discourse functions of bridging constructions} we discuss the discourse functions of bridging constructions, and in \refsec{sec:Bridging constructions in other Nakh-Daghestanian languages} we look at further strategies of bridging constructions in other languages of the \ili{Nakh-Daghestanian} language family. 

Because bridging constructions are a strategy of natural discourse they cannot be easily elicited. The data analyzed in this paper originate from texts gathered by various researchers. For \ili{Tsez}, \ili{Hunzib} and \ili{Khwarshi} published corpora exist (\citealt{vandenBerg.1995}; \citealt{Abdulaev.Abdullaev.2010}; \citealt{Karimova.2014}). Around 42,500 words of the \ili{Tsez} corpus have been glossed by André Müller, and have been employed for this paper. Most of the \ili{Khwarshi} examples cited in this paper originate from texts gathered, glossed and translated by Zaira Khalilova. The \ili{Hinuq} corpus is currently unpublished. It has been gathered by Forker and contains around 43,000 words. The \ili{Bezhta} corpus (around 38,000 tokens) consists of the memories of Šeyx Ramazan, written down by himself at the end of the 20th century (thus they were composed in the written medium), translated and edited by Madžid Khalilov and glossed by Forker. In sum, all data used in this paper originate from written corpora, but the majority of them were oral narrations originally. Only for some of the \ili{Hinuq} texts we have audio recordings at our disposal. For the \ili{Tsez}, \ili{Khwarshi} and \ili{Hunzib} texts we do not have the relevant recordings and therefore cannot judge how much the texts have been edited and changed when the written versions were prepared.

\section{Formal characteristics}
\label{sec:Formal characteristics} 
Bridging constructions consist of two parts, the reference clause and the bridging clause. Reference clauses are main clauses that express an \isi{action} or an event. The bridging clause immediately follows the reference clause and recapitulates the events given in the reference clause while being syntactically dependent on the following clause, i.e., bridging clauses are non-main clauses. An example for this kind of construction is given in \refex{ex:1ab} from \ili{Hunzib}. Note that the bridging clause in \refex{ex:1b} contains the postposition \textit{muɣaƛ}, which follows the \isi{converb}. We are not in the position to judge whether the postposition functions as a complementizer in this example; its use in combination with the \isi{converb} is optional. 

\begin{exe}
	\ex	\label{ex:1ab}
\langinfo{Hunzib}{}{\citealt[234]{vandenBerg.1995}}
	\begin{xlist}
		\ex	\label{ex:1a}
		\gll	 \underline{uhu-n}		 \underline{lo}				 \underline{αbu}\\
			die-\textsc{cvb}	be.\textsc{prs.i}		father(\textsc{i})\\
		\glt	\sqt{Father died.}

		\ex	\label{ex:1b}
		\gll	\textbf{αbu}		\textbf{uhu-n}	muɣaƛ	biššu		ɨq'q'u		ɨs			e͂ƛe-n		lo q'arawulɬi		r-uw-a		diya\\
			father		die-\textsc{cvb}	after		very		big			sibling	go-\textsc{cvb}	be.\textsc{prs.i} guard(\textsc{v})			\textsc{v}-do-\textsc{inf}		\textsc{ben}\\
		\glt	\sqt{After father died, the eldest son went to guard the grave.}
	\end{xlist}
\end{exe}

It is also possible for another clause to intervene between the reference clause and the bridging clause but this does not seem to be very common, see example \refex{ex:2ab}.


\begin{exe}
	\ex	\label{ex:2ab} 
	\langinfo{Hinuq}{}{Forker, unpublished data}
	\begin{xlist}
		\ex	\label{ex:2a}
		\gll	 \underline{hoboži}  \underline{\smash{y}-i\smash{q}-no}   \underline{obu-zo}    \underline{baru-s}     \underline{ked}. hayɬu   kede-s    iyo    y-uh-en  zoqʼe-n\\
			now     \textsc{ii}-become-\textsc{pst.uw} father-\textsc{gen2} wife-\textsc{gen1}	daughter(\textsc{ii}) this.\textsc{obl} girl.\textsc{obl-gen1} mother(\textsc{ii}) \textsc{ii}-die-\textsc{cvb} be-\textsc{pst.uw}\\
		\glt	\sqt{Then the daughter of the stepmother was born. The mother of this girl had died.}

		\ex	\label{ex:2b}
		\gll	\textbf{obu-zo}   \textbf{baru-s}    \textbf{ked}      \textbf{y-iq-no},  haw idu  y-iči-r-ho  zoqʼe-n\\
			father-\textsc{gen2} wife-\textsc{gen1} daughter(\textsc{ii})  \textsc{ii}-become-\textsc{cvb} she home \textsc{ii}-be-\textsc{caus}-\textsc{icvb} be-\textsc{pst.uw}\\
		\glt	\sqt{After the daughter of the stepmother was born, the (other) girl had to stay at home.}
	\end{xlist}
\end{exe}
%

\subsection{Syntactic properties of the reference clause}
\label{ssec:Syntactic properties of the reference clause} 
Reference clauses are always main clauses and the majority of them are in the declarative mood. Theoretically, there are no restrictions concerning tense, aspect, modality and negation but since bridging constructions are very frequent in narratives, the most common strategy is the use of the unwitnessed past tense \refex{ex:3ab}, the present tense \refex{ex:13ab} and the perfect tense as illustrated in \refex{ex:1ab} above, since those are the preferred tenses found in \ili{Tsezic} narratives \footnote{\ili{Hinuq}, \ili{Tsez} and \ili{Khwarshi} formally and semantically distinguish between the unwitnessed past and the perfect. By contrast, in \ili{Hunzib} and \ili{Bezhta} (with some restrictions) there is only one such tense-aspect form that functions as indirect evidential (unwitnessed past) or as perfect depending on the context \citep{Khalilova.2011}.}.  

\begin{exe}
	\ex	\label{ex:3ab}
\langinfo{Khwarshi}{}{Z. Khalilova, p.c.}
	\begin{xlist}
		\ex	\label{ex:3a}
		\gll	 kʼutʼidin 	a͂qʼˤwa=n  		b-oq-un, 			\underline{l-ekʼ-x-un}\\
			suddenly 	mouse(\textsc{iii})=\textsc{add}		\textsc{iii}-catch-\textsc{cvb} 	\textsc{iv}-fall-\textsc{caus-pst.uw}\\
		\glt	\sqt{He took the mouse quickly and made her throw it (the ring).}

		\ex	\label{ex:3b}
		\gll		\textbf{l-ekʼ-x-uč}  				l-oq-un   			ise\\
			\textsc{iv}-fall-\textsc{caus-imm.ant} 	\textsc{iv}-catch-\textsc{pst.uw} 	\textsc{3sg.erg}\\
		\glt	\sqt{When he made her drop it, he took it (the ring).}
	\end{xlist}
\end{exe}

Occasionally, the reference clause is a non-declarative clause. The reference clause in example \refex{ex:4ab} from \ili{Hunzib} is an interrogative clause, marked by the interrogative marker \textit{-i} and as opposed to the typical use of the perfect tense it is in the simple future tense. The interrogative clause in \refex{ex:4ab}, however, is a kind of rhetorical question that the speaker asks after implying that somebody tried to frighten the cock by shooing it and the speaker immediately gives the answer by recapitulating the verbal predicate of the interrogative clause. It therefore rather functions as a declarative clause within the \isi{narrative}. The form of the clause as a question has probably been chosen to raise the interest of the addressee in the continuation of the story and to involve her/him more intensively in the narration.

\begin{exe}
	\ex	\label{ex:4ab}
\langinfo{Hunzib}{}{\citealt[157]{vandenBerg.1995}}
	\begin{xlist}
		\ex	\label{ex:4a}
		\gll	 bed	ħeleku	\underline{deno}	\underline{m-u\smash{q}'-o\smash{y}s-i?}\\
			then	cock(\textsc{iv})	back	\textsc{iv}-turn-\textsc{fut.neg-int}\\
		\glt	\sqt{Would not the cock then turn around?}

		\ex	\label{ex:4b}
		\gll	\textbf{bed} 	\textbf{deno}		\textbf{m-uq'e-n}		ʕali-ɬ-do			nuu-n			lo\\
			then	back		\textsc{iv}-turn-\textsc{cvb}	Ali-\textsc{cont-dir}	come-\textsc{cvb}	be.\textsc{prs.iv}\\
		\glt	\sqt{Then having turned, it went to Ali.}
	\end{xlist}
\end{exe}

Examples of this kind, i.e., non-declarative reference clauses, are scarce in our data and therefore won't be treated further. 

Since our data stem from written corpora, it is not possible to determine any \isi{prosodic} differences between the reference clause and the bridging clause and therefore the \isi{prosodic} properties of \ili{Tsezic} bridging constructions must be left for future research.


\subsection{Syntactic properties of the bridging clause}
\label{ssec:Syntactic properties of the bridging clause}
The only possible strategy to express bridging clauses in \ili{Tsezic} languages is the use of converbs. Converbs are defined as a \dqt{nonfinite verb form whose main function is to mark adverbial subordination} (\citealt[3]{Haspelmath.1995}). Converbs are the main strategy to express subordinate clauses with adverbial function in \ili{Tsezic} languages (for in-depth analyses of converbs see \citealt{Comrie.Forker.Khalilova.2012} and \citealt{Forker.2013b}). From a syntactic point of view the adverbial clauses in bridging constructions do not differ from other adverbial clauses.

\ili{Tsezic} languages have a large number of converbs that can be divided into the following groups based on their semantics and their morphosyntactic properties (\citealt{Comrie.Forker.Khalilova.2012}):

\begin{itemize}
\item general converbs 
\item specialized temporal converbs 
\item non-temporal converbs
\item local \isi{converb}/participle
\end{itemize}

General converbs can be characterized as contextual converbs that are semantically vague, in contrast to all other converbs that express particular semantic links. All \ili{Tsezic} languages have at least two general temporal converbs: a perfective \isi{converb} and an imperfective \isi{converb}. They can be used together with copulas as auxiliaries for the formation of periphrastic verb forms that head main clauses. In this case, they form a single predicate together with a copula-auxiliary. In particular, in all \ili{Tsezic} languages perfective converbs are used in periphrastic verb forms with the meaning of perfect or indirect evidential past (see footnote 1 in Section \refsec{ssec:Syntactic properties of the reference clause} above) as in \refex{ex:1a} and \refex{ex:2a}. In \ili{Hunzib} and \ili{Khwarshi},  the imperfective converbs are identical to the simple present. In \ili{Hinuq} and \ili{Tsez}, they are used for the formation of periphrastic present tenses (by adding the copula as \isi{finite} auxiliary) as in \refex{ex:2b}.

The specialized temporal converbs express the major temporal meanings of posteriority, simultaneity, and anteriority. Each language in the \ili{Tsezic} subgroup has several simultaneous and anterior converbs, but only one posterior \isi{converb}. Non-temporal converbs form the largest group and include local, causal, conditional (realis and irrealis), concessive, and purposive converbs. In addition, all \ili{Tsezic} languages have some local participle or \isi{converb} that denotes locations where actions or situations take place.

Converbal clauses do not express their own absolute time reference, evidentiality, or illocutionary force. For these features, they are dependent on the form of the \isi{main clause}. Applying \citeauthor{Bickel.2010}'s \citeyear{Bickel.2010} terminology we can describe them as ``non-finite'' and ``asymmetrical'' because they express fewer categories than main clauses. Temporal converbs express relative temporal reference whereby the event or situation referred to in the \isi{main clause} serves as temporal anchor. Illocutionary force markers, i.e., imperative and interrogative suffixes, exclusively occur in main clauses. Their scope can be restricted to the \isi{main clause} or extended to the converbal clause, depending on the construction in question. Evidentiality is only expressed in main clauses with past time reference and the scope of the evidential markers always extends to converbal clauses.

There are hardly any strict requirements of coreferentiality between converbal and main clauses. The most common way of expressing coreferential arguments between converbal clause and \isi{main clause} is through zero arguments in at least one of the clauses. Coreferential overt nouns and pronouns are possible, but rather uncommon, and the precise restrictions are not fully understood.

\ili{Tsezic} languages are predominantly head-final and converbal clauses commonly precede the \isi{main clause}. However, center-embedding or a position after the \isi{main clause} are also allowed. A few converbs such as posterior converbs or purposive converbs have a stronger tendency to occur after the \isi{main clause}, which can be explained by their semantics and iconicity. Perfective converbs, anterior converbs, and to a somewhat smaller degree simultaneous converbs occur in the vast majority of examples before the \isi{main clause}. This also has a semantic explanation: anterior \isi{converb} clauses and most perfective \isi{converb} clauses refer to situations that happened before the situation in the \isi{main clause}. Therefore, if they precede the \isi{main clause} their linear ordering reflects the temporal ordering of the situations, and the opposite ordering would sound rather unnatural. In the bridging constructions discussed in this paper the converbal clauses always precede the main clauses.

\reftab{tab:1:frequencies} shows the converbs that we found so far in our data. When we compare the range of converbs used in bridging constructions in the texts at our disposal, \ili{Tsezic} languages differ to some extent. Because we did not elicit bridging constructions we cannot judge if more converbs can be used (although this is very likely). The converbs listed in \reftab{tab:1:frequencies} belong to the general and specialized temporal converbs. Non-temporal converbs and the local \isi{converb}/participle are not found in our data, although such constructions seem theoretically possible. All converbs in \reftab{tab:1:frequencies} express temporal simultaneity (\sqt{when, while}) or anteriority/immediate anteriority (\sqt{after, immediately after}). Anterior converbs are used when the event expressed in the bridging clause takes place before the event in the following \isi{main clause}. The immediate anterior \isi{converb} serves the same purpose although the time span between the two events is shorter (\sqt{immediately after}). The simultaneous \isi{converb} is used to express that the two events, the one in the bridging clause and the one in the following \isi{main clause}, happen at the same time. The reason why predominantly (or exclusively) simultaneous and anterior converbs are used lies in their semantics, i.e., the iconicity of linear order of the clauses and temporal order of the events as explained above. The bridging clause is a converbal clause that normally precedes the \isi{main clause}, and this syntactic ordering fits well the simultaneous and anterior semantics of the converbs given in \reftab{tab:1:frequencies}.    

\begin{table}
\caption{Converbs in Tsezic bridging constructions}
\label{tab:1:frequencies}
 \begin{tabular}{lllllll} 
  \lsptoprule
           &\ili{Hinuq}  & \ili{Khwarshi} & \ili{Tsez} & \ili{Bezhta} & \ili{Hunzib}\\ 
  \midrule
  \textsc{pfv.cvb} & \textit{-n(o)} &   \textit{-un} &    \textit{-n(o)}  &    \textit{-na}     & \textit{-(V)n}\\
 \textsc{sim.cvb} & \textit{-(y/o)ƛ'o} &   \textit{ -q'arƛ'a} &  \textit{-ƛ'orey }    &         & \\
 \textsc{ant.cvb} & \textit{-nos}  &    &  \textit{ -nosi } &         & \\
 \textsc{ant.cvb} &  \textit{-aɬi} &   \textit{-aƛa} &      &         & \textit{-oɬ}\\
 \textsc{imm.sim} &  &   \textit{-uč} &    \textit{-run}  &         & \\
  
  \lspbottomrule
 \end{tabular}
\end{table}

As can be seen in \reftab{tab:1:frequencies}, the only \isi{converb} that is found in bridging constructions in all \ili{Tsezic} languages is the perfective \isi{converb}. This \isi{converb} is also used for the formation of complex \isi{finite} verb forms (e.g., perfect, pluperfect). The general meaning of the perfective \isi{converb} is anteriority, but it can also express simultaneity and occasionally manner of \isi{action}. It is typically found in \isi{narrative} sequences in chaining constructions as can be illustrated by means of examples \refex{ex:4ab} and \refex{ex:5a} (see also \refex{ex:25ab}). In \refex{ex:5a}, the \isi{main clause} (containing the verb \textit{b-acʼ-} `eat') is preceded by two adverbial clauses which contain perfective converbs (\textit{kʼoƛ-} `jump' and \textit{ƛux-} `remain') that refer to events that took place before the event described in the \isi{main clause}. 

In addition to the converbal suffixes, the dependent clauses often contain some argument or modifier marked with an additive enclitic enhancing \isi{cohesion} in a \isi{narrative} sequence, e.g., \textit{lači=n} \sqt{clothes(\textsc{v)=add}} and \textit{ho͂go-li-i-n} \sqt{coat-\textsc{obl-in=add}} in \refex{ex:25ab}. The additive enclitic also occurs in the converbal clauses in bridging constructions that are formed with the perfective \isi{converb}, e.g., \refex{ex:5b}, \refex{ex:13ab}, and \refex{ex:18ab}. In example \refex{ex:5ab} from \ili{Khwarshi} the \isi{action} expressed in the reference clause is almost identically repeated in the bridging construction and the only expressed argument in the bridging clause bears the additive enclitic (\textit{kad-ba=n} `girl-\textsc{pl=add}').


\begin{exe}
	\ex	\label{ex:5ab}
\langinfo{Khwarshi}{}{Z. Khalilova, p.c.}
	\begin{xlist}
		\ex	\label{ex:5a}
		\gll	 cʼodora-y   	bala-l    	kʼoƛ-un, 		y-acʼ-bič 		ƛux-un \underline{ƛux-u-so} 				\underline{\smash{ɡ}olluč}      	\underline{kad-ba}   \underline{b-acʼ-un}\\
			clever-\textsc{ii} 	corner-\textsc{lat}  	jump-\textsc{cvb},	\textsc{ii}-eat-\textsc{proh}  	remain-\textsc{cvb} stay-\textsc{pst.ptcp-def}		all 			girl-\textsc{pl}   	\textsc{hpl}-eat-\textsc{pst.uw}   \\
		\glt	\sqt{In order not to be eaten the clever one jumped into the wooden trunk, (the wolf) ate the 				rest of girls.}

		\ex	\label{ex:5b}
		\gll	\textbf{kad-ba=n}    	\textbf{b-acʼ-un}, 		m-okʼ-še 		b-eč-un bocʼo 	ɣon-o-ɬ-ɣul \\
			girl-\textsc{pl=add}  	\textsc{hpl}-eat-\textsc{cvb}, 	\textsc{iii}-go-\textsc{icvb}	\textsc{iii}-be-\textsc{pst.uw} wolf(\textsc{iii})   	forest-\textsc{obl-inter-all} \\
		\glt	\sqt{Having eaten the girls, the wolf went to the woods.}
	\end{xlist}
\end{exe}


If we take a look at the reference clause in \refex{ex:5ab} we notice that the unwitnessed past and the perfective \isi{converb} are formally identical (-\textit{un}). Despite the homopho\-ny, they are functionally different, e.g., the perfective \isi{converb} is not used to express evidentiality. The same homophony applies to \ili{Hinuq}, \ili{Tsez}, and partially to \ili{Bezhta} (cf. \citealt[244]{Forker.2013a}; \citealt[391]{Khalilova.2009}; \citealt{Khalilova.2011}; \citealt{Comrie.et.al.2016}). 



\section{Types of bridging constructions}
\label{sec:Types of bridging constructions}
In \ili{Tsezic} languages we find two types of bridging constructions. The first and most common construction is \isi{recapitulative linkage} that will be discussed in \refsec{ssec:Recapitulative linkage}. In these constructions, the \isi{action} expressed in the reference clause is repeated immediately in the bridging clause. Strictly verbatim \isi{repetition} is rare  and bridging constructions are frequently a target for \isi{modification}, i.e., we have omissions, additions and substitutions that distinguish the bridging clause from the reference clause. 

The second possibility is \isi{summary linkage}, i.e., the use of a dedicated verb to recapitulate the events expressed in the reference clause. This strategy is commonly used to summarize the content of direct speech. It will be treated in \refsec{sec:Summary linkage}.



\subsection{Recapitulative linkage}
\label{ssec:Recapitulative linkage}
(Almost) verbatim \isi{repetition} is occasionally found and \refex{ex:3ab} provides an example. Generally, reference clauses and bridging clauses slightly differ in terms of formal make-up and consequently usually also in content. As mentioned in \refsec{sec:Introduction}, there are four subtypes of \isi{recapitulative linkage}. All four are found in \ili{Tsezic} languages:

\begin{description}  
\item[Modifications:] reference clause and bridging clause contain the same information, i.e., there are no omissions or additions but word order might be changed or lexical \isi{NPs} can be replaced by corresponding pronouns in either the reference clause or the bridging clause
\item[Omissions:] reference clause and bridging clause differ in terms of content, i.e., the bridging clause contains less information than the reference clause
\item[Additions:] reference clause and bridging clause differ in terms of content, i.e., the reference clause withholds information which is then provided in the bridging clause
\item[Substitutions:] information given in the reference clause is substituted in the bridging clause by (near) synonyms in order to broaden or narrow the semantics of the verbal predicate or in order to change the point of view  
\end{description}

\subsubsection{Modifications}
\label{ssec:Modifications}
Modifications are not as common as omissions and additions and are often accompanied by those. Possible modifications are different word order or replacement of lexical \isi{NPs} by pronouns in the bridging clause and vice versa. The reference clause in \refex{ex:6ab} differs from the bridging clause in some aspects. The subject of the reference clause is encoded by a pronoun \textit{iɬe} in the ergative case whose referent, \textit{ɣʷade} \sqt{raven} was introduced by a lexical NP in the preceding clause. In the bridging clause, the subject is repeated as a lexical NP. Furthermore, reference clause and bridging clause differ in their constituent order due to the diverging position of the verb: VOS (verb-initial reference clause) vs. OSV (verb-final bridging clause). A similar example with changed constituent order from verb-initial to verb-final is \refex{ex:1ab}.  The constituent order in the clause preceding the reference clause (VS) and in the reference clause itself (VOS) is typical for introducing new referents into the discourse in the position of subject and object respectively. Both noun phrases denoting new referents (`raven' and `chicken') occur after the verb. In the bridging clause the constituent order has been changed to verb-final since the clause does not serve to introduce a new referent.

\begin{exe}
	\ex	\label{ex:6ab}
\langinfo{Khwarshi}{}{Z. Khalilova, p.c.}
	\begin{xlist}
		\ex	\label{ex:6a}
		\gll	 	šari 		coƛ-še 		idu  	eč-u-qʼarƛʼa, 				b-otʼqʼ-un    		ɣʷade \underline{y-ez-un}    		\underline{hos}  \underline{huho}    	\underline{iɬe}\\
			butter 	stir-\textsc{icvb} 		this 	be-\textsc{pst.ptcp-sim.cvb}    	\textsc{iii}-come-\textsc{pst.uw} 	raven(\textsc{iii}) \textsc{v}-take-\textsc{pst.uw} 	one  chicken(\textsc{v}) 	\textsc{3sg.erg}\\
		\glt	\sqt{When he was sitting and stirring the butter, a raven came and took one chicken.}

		\ex	\label{ex:6b}
		\gll	\textbf{hos}  	\textbf{huho}    	\textbf{ɣʷad-i}    		\textbf{y-ez-aƛa}, 			l-oc-un očʼe-č      	huho    	o͂ču-lo   		kʼakʼa-qa-l\\
			one  	chicken(\textsc{v}) 	raven.\textsc{obl-erg} 	\textsc{v}-take-\textsc{ant.cvb}  	\textsc{npl}-tie-\textsc{pst.uw} nine-\textsc{ints} 	chicken hen-\textsc{gen2} 	leg-\textsc{cont-lat} \\
		\glt	\sqt{When the raven took one chicken, he tied all nine chickens to the leg of the hen.} 
	\end{xlist}
\end{exe}
The opposite can be observed as well, i.e., the reference clause contains a lexical NP that is pronominally repeated in the bridging clause as in \refex{ex:7ab}. As mentioned above, modifications regularly go hand in hand with additions, omissions and substitutions. Thus, in \refex{ex:7ab} not only the linguistic form of the subject differs, but the goal expression in the referent clause has been omitted in the bridging clause. 

\begin{exe}
	\ex	\label{ex:7ab}
\langinfo{Hunzib}{}{\citealt[164]{vandenBerg.1995}} 
	\begin{xlist}
		\ex	\label{ex:7a}
		\gll	 bədaː	eče-r-α-α				koro		r-oχ-on=no, 		\underline{č'e\smash{q}}		\underline{\smash{g}ič'-en} \underline{lo}				\underline{kα-ƛ'o}\\
			so		stay-\textsc{pst.ptcp-obl-in}	hand(\textsc{v})	\textsc{v}-take-\textsc{cvb=add}	bird(\textsc{iv})	sit.down-\textsc{cvb}  be.\textsc{prs.iv} hand.\textsc{obl-spr} \\
		\glt	\sqt{While he was sitting, holding his hand out like this, a bird alighted in his hand.}

		\ex	\label{ex:7b}
		\gll	\textbf{ogu}		\textbf{gič'-oɬ},					rara-a=n				gul-un,	e͂ƛ'e-n lo			humutkurα-α 	hobolɬi-lα-α \\
			that(\textsc{iv})	sit.down-\textsc{ant.cvb}		bosom-\textsc{in=add}		put-\textsc{cvb}	go-\textsc{cvb} be.\textsc{prs.i}	Garbutli\textsc{-in}		hospitality-\textsc{obl-in}\\
		\glt	\sqt{When it alighted, he put it in his bosom and went to Garbutli as a guest.}
	\end{xlist}
\end{exe}


The \isi{repetition} of a lexical NP as pronoun in the bridging clause is only rarely found in \ili{Tsezic} languages. The preferred strategy is to leave the referent unexpressed in the bridging clause. This is not surprising because in clause linkage coreferent arguments are usually omitted in adverbial clauses. More generally, in \ili{Tsezic} languages arguments that are retrievable from the context are often not overtly expressed, not even in main clauses.  

\subsubsection{Omissions}
\label{ssec:Omissions}

Omissions are found in a vast amount of \isi{recapitulative linkage} constructions. Typical targets for \isi{omission} are lexical \isi{NPs} and adjectives as in \refex{ex:5ab} and \refex{ex:8ab}, numerals in \refex{ex:10ab}, pronouns, adverbs, locative arguments in \refex{ex:11ab} and other verbal complements like purposive clauses in \refex{ex:9ab} or infinitival clauses.

\begin{exe}
	\ex	\label{ex:8ab}
\langinfo{Hunzib}{}{\citealt[207]{vandenBerg.1995}} 
	\begin{xlist}
		\ex	\label{ex:8a}
		\gll	əg			buƛii		loder					\underline{iʔer}	\underline{ože}	\underline{ɨ\smash{q}'lə-n}			\underline{lo}\\
			that.\textsc{i}		home		be.\textsc{prs.ptcp}		small	boy(\textsc{i})	grow.up-\textsc{cvb}	be.\textsc{prs.i}\\
		\glt	\sqt{Now, that little boy who was at home had grown up.}

		\ex	\label{ex:8b}
		\gll	\textbf{ɨq'l-oɬ}					iyu-g			nɨsə-n	li				``diye αbu		niyo		e͂ƛ'e-r?" \\
			grow.up-\textsc{ant.cvb}	mother-\textsc{ad}	say-\textsc{cvb}	be.\textsc{prs.v}		\textsc{1sg.gen} father		where	go-\textsc{pst}\\
		\glt	\sqt{When he had grown up, he said to his mother, \dqt{Where did my father go?}} 
	\end{xlist}
\end{exe}

The example in \refex{ex:8ab} displays the most radical type of \isi{omission}, i.e., only the most important information given in the reference clause is repeated in the bridging clause, namely the verbal predicate, and all other information expressed by the lexical argument and the modifying adjective in the reference clause have been omitted. Example \refex{ex:9ab} from \ili{Tsez} shows further possibilities of \isi{omission}. Almost all information of the reference clause (adverb, lexical \isi{NPs} and the purposive clause) has been left out in the bridging clause. 

\begin{exe}
	\ex	\label{ex:9ab}
\langinfo{Tsez}{}{\citealt[211]{Abdulaev.Abdullaev.2010}}
	\begin{xlist}
		\ex	\label{ex:9a}
		\gll	\underline{nełƛ’osi} 			\underline{kʷaxa=tow}  	\underline{habihan=n}   \underline{ziru=n}   	\underline{xan-s}    		\underline{kid} \underline{esir-anix}      \underline{b-ik’i-n} \\
			of.that.time 	soon=\textsc{emph} miller=\textsc{add} 	fox=\textsc{add} khan-\textsc{gen1} 	daughter ask-\textsc{purp.cvb} 	\textsc{hpl}-go-\textsc{pst.uw}\\
		\glt	\sqt{Soon after that, the miller and the fox went to ask for the king's daughter.}

		\ex	\label{ex:9b}
		\gll		\textbf{ele-aɣor}   		\textbf{b-ik’i-ƛ’orey} 		ziru-de       		dandir   	ixiw 	bˤeƛ’e-s      reqen=no   	žeda-ɬ       				teɬ=gon  		b-ik’i-x        	ixiw ɣˤʷay=no  keze 		b-oq-no\\
			there-\textsc{in.vers}	\textsc{hpl}-go-\textsc{sim.cvb}	fox-\textsc{apud} 	together big  	flock.of.sheep-\textsc{gen1}  herd=\textsc{add} 	\textsc{dem.obl-cont}		inside=\textsc{cntr} 	\textsc{iii}-go-\textsc{icvb}	big  dog(\textsc{iii})=\textsc{add} meet 		\textsc{iii}-become-\textsc{pst.uw} \\
		\glt	\sqt{When they went there, the fox met a big flock of sheep and a large dog walking 	among them.} 
			\end{xlist}
\end{exe}


Omission of subject-like arguments is common. In example \refex{ex:10ab}, not only the ergative pronoun is absent from the bridging clause but also the numeral \sqt{three}. Note that this changes the gender agreement prefix in the bridging clause; the \isi{omission} of the numeral requires the P argument to be marked by the plural and thus the verb bears the neuter plural agreement prefix.

\begin{exe}
	\ex	\label{ex:10ab}
\langinfo{Tsez}{}{\citealt[92]{Abdulaev.Abdullaev.2010}}
	\begin{xlist}
		\ex	\label{ex:10a}
		\gll	zaman-ƛ’ay   	\underline{neɬa}      		\underline{ɬˤono} 		\underline{xexo\smash{y}}        		\underline{b-o\smash{ɣ}-no}\\
			time-\textsc{spr.abl} 	it.\textsc{obl.erg}  	three   	young.animal(\textsc{iii}) 	\textsc{iii}-hatch-\textsc{pst.uw}\\
		\glt	\sqt{After a while, it hatched three nestlings.}

		\ex	\label{ex:10b}
		\gll		\textbf{xexoy-bi}   			\textbf{r-oɣ-no}    			kʷaxa=tow  	ɣun-xor=no   		b-ay-n ziru-a    	aɣi-qor      		qˤaƛi-n   \\
			young.animal-\textsc{pl}   \textsc{npl}-hatch-\textsc{cvb}  	soon=\textsc{emph} tree-\textsc{ad.lat=add}  	\textsc{iii}-come-\textsc{cvb} fox-\textsc{erg}  bird-\textsc{poss.lat} 	shout-\textsc{pst.uw}  \\
		\glt	\sqt{Very soon after the nestlings hatched, a fox came to the tree and shouted to the bird.} 
			\end{xlist}
\end{exe}


In \ili{Hunzib}, the copula, which forms together with the perfective \isi{converb} the periphrastic perfect tense as in \refex{ex:1ab}, \refex{ex:7ab}, and \refex{ex:8ab}, is dropped in many bridging clauses and although this looks formally like an \isi{omission} such constructions are morphosyntactically substitutions and will be treated in \refsec{ssec:Substitutions}.  

\subsubsection{Additions}
\label{ssec:Additions}
Sometimes the bridging clause in \isi{recapitulative linkage} expresses more information than the reference clause. Additional information that is given in the bridging clause is not new or doesn't crucially alter the event described in the reference clause but rather provides additional background information in the form of adverbs or spatial arguments. 
The bridging construction in \refex{ex:11ab} contains more information about the manner of movement of the group (\sqt{happily}) and adds a locative argument (\sqt{on their way}), but there are also some omissions like the deletion of the locative adverb that expresses the place of origin. Furthermore, the bridging clause is introduced by the clause-initial manner adverb \textit{hemedur} \sqt{so}. Manner adverbials of this and similar types as well as temporal adverbials with a very general meaning are frequently used in \isi{narrative} discourse to establish boundaries between individual episodes and at the same time link the episodes together. It comes thus naturally to add them in bringing constructions (see also \ref{ex:4ab}).


\begin{exe}
	\ex	\label{ex:11ab}
\langinfo{Tsez}{}{\citealt[138]{Abdulaev.Abdullaev.2010}}
	\begin{xlist}
		\ex	\label{ex:11a}
		\gll	ža=n   			hemedur=tow  	ešur-no       			yizi-a yizi-ɬ       				r-oq-no       			\underline{ele-a\smash{y}}     		\underline{bitor}  \underline{u\smash{y}no=n}   	\underline{sada\smash{q}}    			\underline{r-ik’i-n} \\
			\textsc{dem.sg=add} 	so=\textsc{emph} 		take.along-\textsc{cvb} 	\textsc{dem.pl.obl-erg}  \textsc{dem.pl.obl-cont} 	\textsc{pl}-become-\textsc{cvb} 	there-\textsc{in.abl} 	thither four=\textsc{add} 	together 			\textsc{pl}-go-\textsc{pst.uw}\\
		\glt	\sqt{So they took him along with them as well and from there the four went further 					together.}

		\ex	\label{ex:11b}
		\gll		\textbf{hemedur}   \textbf{uyno=n}   	\textbf{rok’uɣʷey-ƛ’}    	\textbf{huni-x}      	\textbf{r-ik’i-ƛ’orey}  žeda-r     			b-exur-asi     	boc’i 		b-esu-n \\
			so        		four=\textsc{add} 	fun-\textsc{spr} 			way-\textsc{ad} \textsc{pl}-go-\textsc{sim.cvb} \textsc{dem.obl-in.lat} 		\textsc{iii}-kill-\textsc{res.ptcp} 	wolf    	\textsc{iii}-find-\textsc{pst.uw}  \\
		\glt	\sqt{So when the four of them went on their way happily, they found a wolf who was 	killed.} 
	\end{xlist}
\end{exe}


In the bridging clause in \refex{ex:12ab} there are no omissions but only additions that slightly alter the content. The predicate in the reference clause is a causative verb that expresses an \isi{action} carried out by the fox. In the bridging clause the predicate occurs in its bare intransitive form and consequently there is no agentive argument. Instead, the result of the \isi{action} is described and the predicate is further modified by an adverbial phrase expressing quality/evaluation.

\begin{exe}
	\ex	\label{ex:12ab}
\langinfo{Khwarshi}{}{Z. Khalilova, p.c.}
	\begin{xlist}
		\ex	\label{ex:12a}
		\gll	zor-i		ɬo   		ɡutʼ-un, 			\underline{ɬu\smash{ɣ}-kʼ-un}     			\underline{bocʼo} 	\underline{bolo-\smash{q}a-l}\\
			fox-\textsc{erg} 	water 	pour-\textsc{cvb}, 	stick-\textsc{caus-pst.uw} 	wolf   	ice-\textsc{cont-lat}\\
		\glt	\sqt{The fox poured out the water and the wolf froze to the ice.}

		\ex	\label{ex:12b}
		\gll		\textbf{b-oɡ}  		\textbf{b-oɬu}  	\textbf{bolo-qa-l}    	\textbf{bocʼo} 	\textbf{ɬuɣ-aƛa},  ɡoƛʼ-un    		zor-i \\
			\textsc{iii}-well 		\textsc{iii}-alike 	ice-\textsc{cont-lat}  	wolf(\textsc{iii})   	stick-\textsc{ant.cvb} call-\textsc{pst.uw} 		fox-\textsc{erg}  \\
		\glt	\sqt{When the wolf was good frozen to the ice, the fox called (the witch).} 
	\end{xlist}
\end{exe}



\subsubsection{Substitutions}
\label{ssec:Substitutions}
Substitutions in bridging clauses can be formal and/or semantic. The most common kind of \isi{substitution} concerns the verbal predicate of the reference clauses. Bridging clauses in \ili{Tsezic} languages are generally subordinate clauses and therefore require different marking than the preceding reference clause. Verbs in reference clauses occur in ``\isi{finite} verb forms'', most commonly present tense or unwitnessed past/perfect in our data (\refsec{ssec:Syntactic properties of the reference clause}) and are replaced by a suitable \isi{converb} in the bridging clause. The most frequent \isi{substitution} strategy found in all \ili{Tsezic} languages involves the verb form in the \isi{main clause} being replaced by the perfective \isi{converb}, indicating temporal anteriority with respect to the situation in the following \isi{main clause}. In most examples presented so far in this paper, the verb form in the \isi{main clause} is the unwitnessed past (\ref{ex:4ab}--\ref{ex:12ab}). This is due to the fact that the vast majority of texts analyzed for this paper are traditional fairy tales and legends that are almost exclusively narrated in the unwitnessed past. By contrast, example \refex{ex:13ab} from \ili{Bezhta} belongs to an autobiographical narration that also contains other tenses such as the present (used as historical present in the example) or the witnessed past. In \refex{ex:13ab} it is the present tense that occurs in the \isi{main clause} (reference clause). Regardless, \refex{ex:13ab} still illustrates the common \isi{substitution} strategy within the bridging clause.  

\begin{exe}
	\ex	\label{ex:13ab} 
	\langinfo{Bezhta}{}{unpublished data, courtesy of M. Khalilov}
	\begin{xlist}
		\ex	\label{ex:13a}
		\gll	holɬo-s     		kʼetʼo 	ɡemo=na   	y-iqʼe-na  			\underline{holco}  	\underline{huli}   \underline{\smash{y}-ü͂\smash{q}-ča}\\
			\textsc{dem.obl-gen1} 	good     taste=\textsc{add}  	\textsc{iv}-know-\textsc{cvb} 	he.\textsc{erg}	\textsc{dem} 	\textsc{iv}-eat-\textsc{prs}\\
		\glt	\sqt{Knowing its good taste, he eats it.}

		\ex	\label{ex:13b}
		\gll			\textbf{huli=na}   	\textbf{y-ü͂q-na}  	saala 	ničdiya 		box-a-ƛʼa 		a͂ko e͂ƛʼe-š 	huli \\
			\textsc{dem=add}  	\textsc{iv}-eat-\textsc{cvb} 	one   	green.\textsc{obl} 	gras-\textsc{obl-spr} 	release go-\textsc{prs} 	\textsc{dem}  \\
		\glt	\sqt{Having eaten it he lays down on the green grass.}
	\end{xlist}
\end{exe}

Besides the perfective \isi{converb}, we find the anterior \isi{converb} (as in \ref{ex:7ab}, \ref{ex:8ab}, and \ref{ex:12ab}), the immediate anterior \isi{converb} in \refex{ex:3ab} and the simultaneous \isi{converb} in \refex{ex:9ab} in bridging clauses. 

Sometimes we find \isi{substitution} by means of (near) synonymy, i.e., one of the verbs in either the reference clause or the bridging clause has a more general meaning than the other one. The verb \textit{-u͂če} \sqt{run} that is used in the reference clause in \refex{ex:14ab} provides a more precise description  of the kind of movement that is used to return home (namely fast movement by foot), while the more general verb  \textit{-e͂ƛe} \sqt{go} used in the bridging clause is a default verb to express movement. Note also that the locative adverb \textit{deno} \sqt{back} is substituted by \textit{buƛii} \sqt{home} which provides, in contrast to \textit{deno}, a more specific description of the goal of the motion. 



\begin{exe}
	\ex	\label{ex:14ab}
\langinfo{Hunzib}{}{\citealt[234]{vandenBerg.1995}} 
	\begin{xlist}
		\ex	\label{ex:14a}
		\gll	e͂ƛe-n=no			``r-uwo-r		q'arawulɬi" 	ƛe			nɨsə-n		šima-ƛ'o=n ƛ'-it'o				\underline{deno}	\underline{u͂če-n}		\underline{lo}			\underline{bəd}\\
			go-\textsc{cvb=add}		\textsc{v}-do-\textsc{pst}		guard(\textsc{v})		\textsc{quot}		say-\textsc{cvb}	grave-\textsc{spr=add} go-\textsc{cvb.neg}		back	run-\textsc{cvb}	be.\textsc{prs.i}	\textsc{3sg.i}\\
		\glt	\sqt{He went and without having gone to the grave, he said \dqt{I have guarded it} and he 				ran back (home).}

		\ex	\label{ex:14b}
		\gll	\textbf{e͂ƛe-n}		\textbf{buƛii}		ut'-un			lo				ɬαnα		wədə \\
			go-\textsc{cvb}	home		sleep-\textsc{cvb}	be.\textsc{prs.i}		three		day\\
		\glt	\sqt{Having gone home he slept for three days.}
	\end{xlist}
\end{exe}

Another kind of \isi{substitution} we find regularly is the replacement of one verb of motion by another one with a different deictic meaning, e.g., \sqt{go} is replaced by \sqt{come} in \refex{ex:15ab}. The reference clause contains a verb of motion that expresses movement away from the deictic center (\sqt{go}) where previous events took place while the verb in the following bridging clause changes the perspective and expresses movement to the new deictic center (\sqt{come}). This strategy is almost always used when the event expressed in the following \isi{main clause} takes place at a new location. Additionally, in example \refex{ex:15ab} the goal of the movement, namely the king's whereabouts, is replaced by the spatial adverb \textit{elo} \sqt{there}, similar to example \refex{ex:9ab}.



\begin{exe}
	\ex	\label{ex:15ab}
\langinfo{Tsez}{}{\citealt[74]{Abdulaev.Abdullaev.2010}}
	\begin{xlist}
		\ex	\label{ex:15a}
		\gll	aɣi=n   		b-is-no    		adäz=gon  	b-oc’-no    		t’eka=n \underline{kid}  	\underline{xan-dä\smash{ɣ}or}   		\underline{\smash{y}-ik’i-n}\\
			bird(\textsc{iii})=\textsc{add} 	\textsc{iii}-take-\textsc{cvb}  	ahead=\textsc{cntr} 	\textsc{iii}-drive-\textsc{cvb}  	he.goat(\textsc{iii})=\textsc{add} girl(\textsc{ii}) 	khan-\textsc{apud.vers} 	\textsc{ii}-go-\textsc{pst.uw} \\
		\glt	\sqt{Having taken a bird and chased a goat ahead, the girl went to the king.}

		\ex	\label{ex:15b}
		\gll			\textbf{elo-r}    	\textbf{y-ay-nosi}   			yiɬa    			xan-qor      		aɣi teƛ-xo    			zow-no \\
			there-\textsc{lat}	\textsc{ii}-come-\textsc{ant.cvb} 	she.\textsc{obl.erg}  	khan-\textsc{at.lat} 	bird give-\textsc{icvb}  		be\textsc{-pst.uw}   \\
		\glt	\sqt{After she arrived there, she wanted to give the bird to the king.}
	\end{xlist}
\end{exe}

Further \isi{substitution} can be found in the nominal domain, i.e., a lexical NP can be replaced by another lexical NP with a similar meaning. In \refex{ex:16ab} one word to express \sqt{time}, \textit{meχ}, is replaced in the bridging clause by another word \textit{zaban} expressing roughly the same meaning. Note again that gender agreement on the verb \textit{-e͂ƛe} \sqt{go} changes because the two words belong to different genders. 



\begin{exe}
	\ex	\label{ex:16ab}
\langinfo{Hunzib}{}{\citealt[202]{vandenBerg.1995}} 
	\begin{xlist}
		\ex	\label{ex:16a}
		\gll	a͂q'-oɬ					boɬu-l		lač'i			n-ɨza:-n			li,				həs=no		q'αm n-ɨza:-n			li				həs=no		bəʔi-d		əgi-d				tiq-en \underline{me\smash{χ}}		\underline{m-eƛ'e-n}		\underline{lo} \\
			come-\textsc{ant.cvb}	this-\textsc{erg}	clothes(\textsc{v})	\textsc{v}-wash-\textsc{cvb}	be.\textsc{prs.v}		one=\textsc{add}		head(\textsc{v}) \textsc{v}-wash-\textsc{cvb}	be.\textsc{prs.v}		one=\textsc{add}		here-\textsc{dir}	there-\textsc{dir}		be.busy-\textsc{cvb} time(\textsc{iv})	\textsc{iv}-go-\textsc{cvb}	be.\textsc{prs.iv}\\
		\glt	\sqt{After he had come, time passed while she washed clothes, washed her head, 				keeping busy with this and that.}

		\ex	\label{ex:16b}
		\gll	\textbf{zaban}		\textbf{n-eƛ'-oɬ},			b-u<wα>t'-a			anta				m-aq'-oɬ nɨsə-n	li				``b-u<wα>t'-a"		ƛe			nɨsə-n	li				ɣurdelo-l\\
			time(\textsc{v})	\textsc{v}-go-\textsc{ant.cvb}		\textsc{hpl}-sleep<\textsc{pl}>-\textsc{inf}	 moment(\textsc{iv})	\textsc{iv}-come-\textsc{ant.cvb} say-\textsc{cvb}	be.\textsc{prs.v}		\textsc{hpl}-<\textsc{pl}>sleep-\textsc{inf}	\textsc{quot}		say-\textsc{cvb}	be.\textsc{prs.v}		mullah-\textsc{erg}
\\
		\glt	\sqt{And when the time had passed, when the moment came to go to bed, the mullah 	said \dqt{Let's go to bed.}} 
	\end{xlist}
\end{exe}

\subsection{Summary linkage}
\label{sec:Summary linkage}
In \isi{summary linkage} the reference clause is replaced by a dedicated verb which summarizes its content. This kind of bridging construction is not very common in \ili{Tsezic} languages since \isi{recapitulative linkage} is the preferred bridging construction, but nevertheless can occasionally be found.
In example \refex{ex:17ab} from \ili{Hunzib} \isi{summary linkage} is achieved by using the dedicated verb \textit{-αq} \sqt{happen}. In this example, the verb ‘happen’ has scope over two reference clauses and is used to summarize both events.

\begin{exe}
	\ex	\label{ex:17ab}
\langinfo{Hunzib}{}{\citealt[160]{vandenBerg.1995}} 
	\begin{xlist}
		\ex	\label{ex:17a}
		\gll	\underline{e͂ƛ'e-n}		\underline{lo}				\underline{oɬu-dər}			\underline{k'arƛe-n}			\underline{lo}			\underline{oɬu-\smash{ɣur}}\\
			go-\textsc{cvb} be.\textsc{prs.i}		\textsc{3sg.obl-all}	wander-\textsc{cvb}		be.\textsc{prs.i}	\textsc{3sg.obl-com}\\
		\glt	\sqt{And he went down to her and went for a walk with her.}

		\ex	\label{ex:17b}
		\gll	\textbf{αq-oɬ}					bəd	ƛ'i		u͂χe-n			χoχ-ƛ'o		e͂ƛ'e-n	lo			bəd \\
			happen-\textsc{ant.cvb}	\textsc{3sg.i}	back	turn-\textsc{cvb}	tree-\textsc{spr}		go-\textsc{cvb}	be.\textsc{prs.i}	\textsc{3sg.i}\\
		\glt	\sqt{Having done this, he returned and went back into the tree.} 
	\end{xlist}
\end{exe}

Another type of \isi{summary linkage} that is relatively common is given in \refex{ex:18ab} and \refex{ex:19ab}. The reference clauses in \refex{ex:18ab} and \refex{ex:19ab} consist of quotes whose contents are summarized by a demonstrative pronoun that is used together with a verb of speech. 


\begin{exe}
	\ex	\label{ex:18ab}
\langinfo{Tsez}{}{\citealt[87]{Abdulaev.Abdullaev.2010}} 
	\begin{xlist}
		\ex	\label{ex:18a}
		\gll	``\underline{di}    \underline{mi}   \underline{\smash{ɣ}uro-x}     		\underline{e\smash{g}ir-an=ƛin} 				\underline{odä-si}  	\underline{zow-č’u} \underline{ži} 		\underline{r-od-a}   		 \underline{šebin} 	\underline{anu=ƛin}"\\
			\textsc{1sg}  \textsc{2sg}  	cows-\textsc{ad} 	send-\textsc{fut.def=quot} 	do-\textsc{res}  	be\textsc{-neg.pst.wit} now 	\textsc{iv}-do-\textsc{inf}  	thing  	be.\textsc{neg=quot}\\
		\glt	\sqt{{}``I didn't give birth to you to have you pasture the cows but now there is nothing to do."}

		\ex	\label{ex:18b}
		\gll	 \textbf{ža=n}   		 \textbf{eƛi-n}    	hemedur=tow  	ozuri-ƛay    	gugi-n \\ 
			this=\textsc{add} 		say-\textsc{cvb} 	so=\textsc{emph} 			eye-\textsc{sub.abl} 	escape-\textsc{pst.uw}\\
		\glt	\sqt{Having said this, he flew out of sight.} 
	\end{xlist}
\end{exe}


\begin{exe}
	\ex	\label{ex:19ab} 
	\langinfo{Hinuq}{}{Forker, unpublished data}
	\begin{xlist}
		\ex	\label{ex:19a}
		\gll	\underline{hiba\smash{y}ɬu} \underline{minut-ma}    \underline{b-a\smash{q}ʼ-a}   \underline{\smash{g}oɬ} \underline{dew-de}       \underline{aldo\smash{ɣ}o-r}    \underline{debe}        \underline{\smash{g}oɬa}   \underline{murad} \underline{tʼubazi}         \underline{b-uw-a\smash{y}az}\\
			that.\textsc{obl} minute-\textsc{in} \textsc{iii}-come-\textsc{inf} be   you.\textsc{sg.obl-aloc} in.front-\textsc{lat}  you.\textsc{sg.gen1} be.\textsc{ptcp} wish(\textsc{iii}) fulfill  \textsc{iii}-do-\textsc{purp}\\
		\glt	\sqt{(The horse said:) In that minute I will be in front of you to fulfill your wish.}

		\ex	\label{ex:19b}
		\gll	 \textbf{hag=no}   \textbf{eƛi-n}   gulu  kʼoƛe-n   hawa-ƛʼo b-iƛʼi-yo \\
			that=\textsc{add} say-\textsc{cvb} horse(\textsc{iii}) jump-\textsc{cvb} air-\textsc{spr} \textsc{iii}-go-\textsc{prs}\\
		\glt	\sqt{Having said that the horse goes away jumping through the air.} 
	\end{xlist}
\end{exe}

\section{Functions of bridging constructions}
\label{sec:Discourse functions of bridging constructions}
\subsection{Discourse functions}
\label{ssec:Discourse functions}
Cross-linguistically, bridging constructions are used to keep the discourse \isi{cohesive} and ease tracking of characters and events. Therefore, bridging constructions are regularly found in languages that employ \isi{switch reference}. Although there are no \isi{switch reference} constructions in \ili{Tsezic} languages, bridging constructions, or to be more precise \isi{recapitulative linkage}, can sometimes be found when the subject of the clause that follows the bridging clause deviates from the one in the reference and bridging clause. In \refex{ex:21ab}, the reference clause contains a lexical NP that is omitted in the following bridging clause but still serves as subject. The \isi{main clause} that follows the bridging clause switches the subject to another character of the \isi{narrative}. 

\begin{exe}
	\ex	\label{ex:21ab}
\langinfo{Hunzib}{}{\citealt[209]{vandenBerg.1995}} 
	\begin{xlist}
		\ex	\label{ex:21a}
		\gll	\underline{e͂du}		\underline{m-a\smash{q}'e-n}			\underline{lo}				\underline{ʕaždah}\\
			inside	\textsc{iv}-come-\textsc{cvb}	be.\textsc{prs.iv}	dragon(\textsc{iv})\\
		\glt	\sqt{The dragon went inside.}

		\ex	\label{ex:21b}
		\gll	\textbf{e͂du}		\textbf{m-aq'-oɬ}					boɬu-l		bodu		ʕaždah		b-iƛ'e-n		gαč' \\
			inside	\textsc{iv}-come-\textsc{ant.cvb}	\textsc{3sg.i-erg}	this(\textsc{iv})	dragon(\textsc{iv})	\textsc{iv}-kill-\textsc{cvb}	be.\textsc{prs.neg}\\
		\glt	\sqt{When it went inside, the boy did not kill the dragon.} 
			\end{xlist}
\end{exe}

Example \refex{ex:22ab} is another instance of subject switching. The reference clause and the following bridging clause share the subject \sqt{girl}, but the following clause changes to another subject (see also \refex{ex:23ab} below). 

\begin{exe}
	\ex	\label{ex:22ab}
\langinfo{Khwarshi}{}{Z. Khalilova, p.c.} 
	\begin{xlist}
		\ex	\label{ex:22a}
		\gll	akal-un    			ɡollu    			kad 	zamana-č   	m-okʼ-šehol 		\underline{ƛus-un}\\
			be.tired-\textsc{cvb} 	be\textsc{.prs.ptcp}  	girl 	time(\textsc{iii})-\textsc{ints} 	\textsc{iii}-go-\textsc{post.cvb} 	sleep-\textsc{pst.uw}\\
		\glt	\sqt{The girl who has been tired fell asleep as some time passed.}

		\ex	\label{ex:22b}
		\gll	\textbf{kad}  	\textbf{ƛus-uč}, 				abaxar-i 			m-oc-un    		iɬe-s   kode=n    ɣon-o-qo-l  \\
			girl 	sleep-\textsc{imm.ant}  	neighbour-\textsc{erg} 	\textsc{iii}-tie-\textsc{pst.uw} 	\textsc{3sg.obl-gen1} hair(\textsc{iii})=\textsc{add}  tree-\textsc{obl-cont-lat}\\
		\glt	\sqt{As soon as the girl fell asleep the neighbor tied her hair to the tree.} 
	\end{xlist}
\end{exe}

In many instances the switched subject occurs in the immediately preceding discourse. For instance, in example \refex{ex:6ab} above the clause preceding the reference clause has a demonstrative pronoun \sqt{he} as subject, referring to a male human being. The reference clause and the bridging clause share the subject \sqt{raven}. The next clause after the bridging clause switches back to the previous subject \sqt{he}. Other examples of this type are \refex{ex:7ab} and \refex{ex:12ab}.

However, in most of the examples the clause following the bridging constructions describes a new episode. An episode is a brief unit of \isi{action} in a \isi{narrative}. Consecutive episodes in narratives can but need not share some or all of the characters. They can take place in the same or in distinct locations. Therefore, a new episode can be accompanied by a change of the subject referent in comparison to the previous episode. This can mean that an entirely new referent is introduced in the clause after the referent clause as in \refex{ex:9ab}, \refex{ex:10ab} and \refex{ex:11ab}, or the previous subject-referent is taken up again as in \refex{ex:6ab}, \refex{ex:7ab}, or \refex{ex:12ab}. It is also possible to switch back to a protagonist who was not a subject referent in the bridging clause, but is not entirely new to the narration as in \refex{ex:1ab} and \refex{ex:22ab}.
Similarly, in a number of the examples the utterance following the reference clause moves the string of narration to a new spatial goal or location. For instance, in \refex{ex:5a} the situation takes place at the home of the protagonist. In \refex{ex:5b} the clause following the bridging construction describes that the place of the \isi{action} has changed from inside the house to outside. Comparable examples are \refex{ex:18ab} and \refex{ex:19ab} in which the clause after the bridging construction describes how one of the protagonists disappears from the scene.

A change of the protagonists or location more clearly indicates that a new episode follows and thus the bridging construction helps to structure the narration by demarcating episodes. As mentioned above, new episodes do not necessarily have new protagonists or new locations, but are defined by new actions. Therefore, the bridging construction can also mark the end of an episode and thus the beginning of a new episode in which the subject referent is just the same such that we have subject/\isi{topic} continuity as in \refex{ex:11ab}, \refex{ex:14ab}, and \refex{ex:15ab}. More specifically, in \refex{ex:11ab}, the episode in the bridging construction describes the joint walk of the protagonists. The new episode refers to how the protagonists found a dead wolf. The bridging construction in \refex{ex:14ab} describes the walk back home of the protagonist and the following clause his lying down to sleep.

Similarly, a change in the location is not obligatory, e.g., \refex{ex:16ab}, \refex{ex:21ab}, and \refex{ex:22ab}. For example, in \refex{ex:21ab} the bridging construction narrates that the girl fell asleep. This episode is followed by a new one in which the neighbor tied her hair to a tree.

Furthermore, bridging constructions may be used to express the chaining of events, i.e., consecutive events can be recapitulated. The reference clause in \refex{ex:23ab} actually consists of two clauses that express consecutive events, the drinking and the sleeping afterwards. Both events are recapitulated in the bridging clause that consists of two converbal clauses.


\begin{exe}
	\ex	\label{ex:23ab}
\langinfo{Hunzib}{}{\citealt[216]{vandenBerg.1995}} 
	\begin{xlist}
		\ex	\label{ex:23a}
		\gll	wedra			ɣino		χuƛ-un		lo,				\underline{\smash{χ}ura:-n}			\underline{lo}, \underline{ut'-un}			\underline{lo}				\underline{bəd}\\
			bucket(\textsc{iv})	wine(\textsc{iv})	drink-\textsc{cvb}	be.\textsc{prs.i}		get.drunk-\textsc{cvb}	be.\textsc{prs.i} sleep-\textsc{cvb}	be.\textsc{prs.i}		\textsc{3sg.i}\\
		\glt	\sqt{He drank a bucket of wine, got drunk and went to bed.}

		\ex	\label{ex:23b}
		\gll	\textbf{χura:-n}			\textbf{ut'-oɬ}				\textbf{bəd}		eže-n			lo			boɬu-l \\
			get.drunk-\textsc{cvb}	sleep-\textsc{ant.cvb}	\textsc{3sg.i}		take-\textsc{cvb}		be.\textsc{prs.i}	this.\textsc{obl-erg}\\
		\glt	\sqt{When he got drunk and went to bed, the dragon took him outside.} 
	\end{xlist}
\end{exe}

\subsection{Genre}
\label{ssec:Genre}
In the corpora of \ili{Tsezic} languages, bridging constructions are primarily found in fictional narratives, that is, fairy tales, sagas and legends. We do not have examples of bridging constructions from historical narratives except for a single instance in the autobiographical narration in \refex{ex:13ab}. In \isi{procedural texts}, we also find occasional occurrences of bridging constructions, but they cannot often be unambiguously separated from repetitions (see Section \refsec{ssec:Bridging constructions, repetition, and predicate doubling} for a discussion). 

Therefore, it seems that bridging constructions are stylistic devices of traditional narrations together with other stylistic markers such as unwitnessed past tenses and \isi{narrative} formulae. For instance, traditional narratives are characterized by use of special introductory formulae which index the genre. In \ili{Tsezic} languages as well as in many other languages of the wider area the introductory formulae consist of a \isi{repetition} of the verb \sqt{be}, i.e., \sqt{There was, there was not...}

Bridging constructions in \ili{Tsezic} represent a particular instance of \isi{parallelism}. Parallelism, i.e., recurring patterns in successive sections of the text, is one of the most common framing devices of ritual language, to which the genre of traditional narratives belongs (see \citealt{Frog.Tarkka.2017} for a short introduction). Parallelism has extensively been studied in poetry, including songs, epics, proverbs and other forms of ritual language, where it is used to express emphasis, and to provide authority or significance (e.g., \citealt{Jakobson.1966}; \citealt{Fox.2014}; among many others). Formulaic \isi{parallelism} as instantiated by the bridging constructions in \ili{Tsezic} help the narrator buy time while s/he mentally prepares the next sentences, and are a hallmark of oral performance (\citealt{Fabb.2015}).

Another criterion for the occurrence of bridging constructions seems to be the medium, i.e., if texts are written or originate from oral narrations. Oral narrations seem to have more bridging constructions than written texts (though, as in \refsec{sec:Introduction} explained, we do not know how much the \ili{Tsez}, \ili{Khwarshi} and \ili{Hunzib} texts have been edited). The \ili{Bezhta} texts used for this paper have been written down and no oral versions exist. This might explain why we have only relatively few examples from \ili{Bezhta} in which the perfective \isi{converb} always occurs in the bridging clause.

\subsection{Bridging constructions, repetition, and predicate doubling}
\label{ssec:Bridging constructions, repetition, and predicate doubling}
A problem we encountered when analyzing bridging constructions is keeping them apart from simple \isi{repetition} of clauses. For instance, \refex{ex:24ab} has been uttered in a \isi{procedural text} that describes the preparation of the Daghestanian national dish khinkal (a type of dumplings). The speaker repeats verbatim one clause with a short break between the two utterances. The example resembles \refex{ex:20ab} below, but in contrast to \refex{ex:20ab}, both clauses in \refex{ex:24ab} are main clauses containing imperative verb forms as all other main clauses in the texts. It is probable that the speaker who uttered \refex{ex:24ab} repeated the sentence because she was concentrating on narrating all individual actions in the correct order and the \isi{repetition} of the clause gave her a little bit more time to prepare the next utterances. As can be seen in \refex{ex:24b}, she also repeats a preposition.



\begin{exe}
	\ex	\label{ex:24ab} 
	\langinfo{Hinuq}{}{Forker, unpublished data}
	\begin{xlist}
		\ex	\label{ex:24a}
		\gll	xokʼo  b-uw-a   b-aqʼe-yo   atʼ=no   r-ux!\\
			khinkal(\textsc{ii}) \textsc{iii}-make-\textsc{inf} \textsc{iii}-must-\textsc{cond} flour=\textsc{add}  \textsc{v}-take\\
		\glt	\sqt{If you have to prepare khinkal, take flour!}

		\ex	\label{ex:24b}
		\gll	atʼ=no   r-ux!   kʼotʼo-ma    teɬer, teɬer čiyo=n    kur!         soda=n    kur!\\
			flour=\textsc{add}  \textsc{v}-take plate-\textsc{in} into into salt=\textsc{add} throw soda=\textsc{add}  throw\\
		\glt	\sqt{Take flour! Pour (lit. throw) salt into, into a plate! Pour soda!} 
	\end{xlist}
\end{exe}


Example \refex{ex:25ab} contains another \isi{repetition} of a \isi{main clause} that could have been used by the speaker as a \isi{stylistic device} to indicate intensity. Again the clauses resemble bridging constructions, but without the morphosyntactic structure of \isi{main clause} followed by converbal clause that we have identified in \refsec{sec:Formal characteristics}.


\ea\label{ex:25ab}
\langinfo{Hunzib}{}{\citealt[][257]{vandenBerg.1995}}\\
\gll \textbf{e͂ƛ'e-n} 	\textbf{lo} 			bəd 	wazir,  		\textbf{e͂ƛ'e-n} 	\textbf{lo} 			əgi-do 		a͂q'-oɬ m-ɨqə-k'-ən 			gudo 		m-uχe-n, 				lači=n 				r-αhu-n  ƛ'odo-s, 		hə͂s	 b-ɨqː'u 	ho͂go 		b-oχče-n, 		ho͂go-li-i=n 			e͂du k'arƛe-k'-en 	hadeʔeče-n 	sɨd 		bač-do 	raʕal-li-ƛ' 		gəl-ən 	lo \\
go-\textsc{cvb}	be.\textsc{prs.i}	this	advisor(\textsc{i})	go-\textsc{cvb}	be.\textsc{prs.i}	there-\textsc{dir}	come-\textsc{ant.cvb} \textsc{iv}-catch-\textsc{caus-cvb}	hen(\textsc{iv})	\textsc{iv}-slaughter-\textsc{cvb}		clothes(\textsc{v})=\textsc{add}		\textsc{v}-take-\textsc{cvb} above-\textsc{abl}	one	\textsc{iv}-big	coat(\textsc{iv})	\textsc{iv}-take-\textsc{cvb}		coat-\textsc{obl}-\textsc{in=add}	inside twirl-\textsc{caus-cvb}	be.slow-\textsc{cvb}	one.\textsc{obl}	rock-\textsc{ins}	edge-\textsc{obl-spr}	put-\textsc{cvb}	be.\textsc{prs.i}\\
\glt \sqt{The advisor went and he went and when he arrived there, he caught a hen and killed it, 	he took the (boy's) outer clothes off and took a furcoat and he wrapped the boy in the coat and put him on the edge of the rock.} 
\z


In example \refex{ex:20ab} the first clause is a converbal clause with the reduplicated perfective \isi{converb}. It is followed by another clause with the same predicate inflected as \isi{narrative} \isi{converb}. The construction looks similar to bridging constructions because of the identical predicates, but the two clauses slightly differ. The first converbal clause lacks any arguments, contains only a temporal adjunct and is verb-final. The second converbal clause, by contrast, contains the object and the verb occurs in the clause-initial position. However, because both clauses are converbal clauses, the example does not adhere to our definition of bridging constructions in \ili{Tsezic} and is therefore analyzed as \isi{repetition}.

\begin{exe}
	\ex	\label{ex:20ab} 
	\langinfo{Hinuq}{}{Forker, unpublished data}\\
			\gll	[ocʼera  ocʼera ɬera    minut-ma    r-exir-an  r-exir-no], [b-exir-no  haw  pulaw],  hoboy hezodoy kʼotʼo-ma   gotʼ-no  qʼidi=n    b-iči-n, ga\\
			ten.\textsc{obl}  ten.\textsc{obl} five.\textsc{obl} minute-\textsc{in}  \textsc{v}-cook-\textsc{red} \textsc{v}-cook-\textsc{cvb} \textsc{iii}-cook-\textsc{cvb} this pilaw(\textsc{iii})  then  then    plate-\textsc{in}    pour-\textsc{cvb} down=\textsc{add}  \textsc{hpl}-sit-\textsc{cvb} drink.\textsc{imp} \\
		\glt	\sqt{Cooking it for 10--15 minutes, and having cooked the pilaw, then pour it into plates, sit down and eat (lit. drink) it.}
\end{exe}

\ili{Hinuq}, \ili{Khwarshi} and \ili{Bezhta} also have constructions in which the predicate is doubled. The first occurrence of the predicate occurs in the infinitive or perfective \isi{converb} followed by the additive particle or another particle. The second occurrence of the predicate can also have the form of the perfective \isi{converb} or it is used as \isi{finite} verb and inflected for the appropriate tense. These constructions can express intensity, prolonged duration, emphasis, predicate \isi{topicalization} and sometimes polarity focus (\citealt{Maisak.2010}; \citealt{Forker.2015}). The \ili{Bezhta} example in \refex{ex:26ab} can be paraphrased with \sqt{As for coming, people do not come here}. Another instance of predicate doubling is the first \isi{converb} clause in \refex{ex:20ab}.

\ea\label{ex:26ab} 
\langinfo{Bezhta}{}{unpublished data, courtesy of M. Khalilov}\\
\gll bekela-a-qa      hiyabačʼe-na  hoƛoʔ ädäm       o͂qʼ-an=na o͂qʼ-aʔa-s \\
snake-\textsc{pl}-\textsc{poss} fear.\textsc{pl}-\textsc{cvb} here    person come-\textsc{inf=add}   come-\textsc{neg-prs}\\
\glt \sqt{Because of fear for snakes people do not come here.}
\z




\section{Bridging constructions in other Nakh-Daghestanian languages}
\label{sec:Bridging constructions in other Nakh-Daghestanian languages}
Not only \ili{Tsezic} languages but also other languages of the \ili{Nakh-Daghestanian} language family use bridging constructions. One of those languages is \ili{Chirag Dargwa}, a member of the Dargwa (or Dargi) sub-branch. \refex{ex:27ab} illustrates that \ili{Chirag Dargwa} uses the same strategy that we already saw in \ili{Tsezic} languages. The reference clause is a \isi{main clause} in the past resultative tense while the bridging construction is again a non-main converbal clause. Additionally, there is a change in the word order. The reference clause has VS constituent order because it introduces new referents (as it was explained for the \ili{Khwarshi} example in \refex{ex:6ab}). The bridging clause is verb-final because this is the preferred order for adverbial clauses and for clauses with neutral information structure.

\begin{exe}
	\ex	\label{ex:27ab}
\langinfo{Chirag Dargwa}{}{D. Ganenkov, p.c.} 
	\begin{xlist}
		\ex	\label{ex:27a}
		\gll	\underline{k’aˤ}	\underline{\smash{q}’ilae}	\underline{ʡaši-l-i}	\underline{a\smash{g}-ur-re}	\underline{niš=ra} \underline{rusːi=ra}\\
			\textsc{dem.up}	Qilae	caraway-\textsc{obl-spr}	go.\textsc{pfv-aor-res.3}	mother=\textsc{add} girl=\textsc{add}\\
		\glt	\sqt{A mother and a daughter went there to Qilae for caraway.}

		\ex	\label{ex:27b}
		\gll	\textbf{niš=ra}	\textbf{rusːi=ra}	\textbf{ʡaši-l-i}	\textbf{ag-ur-sːaħ},	[…] q’ʷala	d-arq’-ib-le	itː-a-d	ʡaše \\
			mother=\textsc{add}	girl=\textsc{add}	 caraway-\textsc{obl-spr} 	go.\textsc{pf-aor-temp} \phantom{x} <collect>	\textsc{n.pl}-do.\textsc{pfv-aor-res.3}	\textsc{dem.dist-pl-erg}	caraway\\
		\glt	\sqt{When the mother and the daughter went for caraway, […] they collected the caraway.} 
	\end{xlist}
\end{exe}

In example \refex{ex:28ab} from \ili{Agul}, a language of the Lezgic sub-branch, the main verb of the bridging clause is marked by a temporal \isi{converb} while the verb in the reference clause is \isi{finite} and bears the aorist suffix. 

\begin{exe}
	\ex	\label{ex:28ab}
\langinfo{Agul}{}{\citealt[][134]{Maisak.2014}} 
	\begin{xlist}
		\ex	\label{ex:28a}
		\gll	aχira 		\underline{\smash{χ}.i-s} 				\underline{\smash{qaχ}.i-naw} 		\underline{mi} 		\underline{bäʕž}\\
			finally 	leave.\textsc{inf-inf}  	start.\textsc{pfv-aor} 	\textsc{dem.m} 	friend\\
		\glt	\sqt{The friend was about to go.}

		\ex	\label{ex:28b}
		\gll	\textbf{χ.i-s} 				\textbf{qaχ.a-gana} 	\textbf{mi} 		ruš.a-s 			raqq.u-naw 	p.u-naw \\
			leave.\textsc{inf-inf} 	start.\textsc{pfv-temp} 	\textsc{dem.m}  	daughter-\textsc{dat} 	see.\textsc{pfv-aor} 	say-\textsc{aor}\\
		\glt	\sqt{When he started to go, the girl saw him and said...} 
	\end{xlist}
\end{exe}


In \ili{Tsova-Tush}, one of the three Nakh languages, the use of converbs is the primary strategy to express \isi{recapitulative linkage}. Bridging constructions can also be found regularly in \ili{Chechen} (Molochieva, p.c.).

\begin{exe}
	\ex	\label{ex:29ab} 
	\langinfo{Tsova-Tush}{}{\citeauthor{ecling}}
	\begin{xlist}
		\ex	\label{ex:29a}
		\gll	d-ax-en, 	\underline{xi} 			\underline{meɬ-or=e}\\
			\textsc{ii}-go-\textsc{aor}, 	water 	drink.\textsc{ipfv-pst=add}\\
		\glt	\sqt{They went off and drank water.}

		\ex	\label{ex:29b}
		\gll	\textbf{xi} 			\textbf{meɬ-oš} 					o 		maq'vlen 		oqar 		c'omal 	eg-b-ie͂  ču, 	me 		ču-toħ-y-it-ra-lŏ \\
			water		drink.\textsc{ipfv-sim.cvb}	that	Makvala.\textsc{dat}	\textsc{3pl.erg}	drug(\textsc{v})	mix-\textsc{v}-do.\textsc{pfv.aor} in		\textsc{comp}		\textsc{pvb}-sleep-\textsc{ii-caus-pst-evid} \\
		\glt	\sqt{While drinking they mixed drugs for that Makvala to make her fall asleep.}
	\end{xlist}
\end{exe}

Due to the lack of data we cannot judge if some sub-branches of the \ili{Nakh-Daghestanian} language family such as \ili{Tsezic} show a larger preference for bridging constructions than others (e.g., \ili{Lak}). Furthermore, except for the \ili{Tsezic} languages we do not have examples of \isi{summary linkage} or \isi{mixed linkage}, and all examples \refex{ex:27ab}--\refex{ex:29ab} contain specialized temporal converbs in the bridging clause and not general converbs. It seems reasonable to assume that \isi{narrative} traditions and genres largely overlap among the \ili{Nakh-Daghestanian} peoples such that from a functional perspective we would expect to find bridging constructions across the same types of narrations (traditional fictional narratives) and within the same types of (oral) performance  (as suggested in \ili{Matsigenka}, see \citealt{chapters/02Emlen} [this volume]).


\section{Conclusion}
\label{sec:Conclusion}
Bridging constructions are a common feature in narratives of \ili{Nakh-Daghestanian} languages. In this paper, we focused on the \ili{Tsezic} languages, but bridging constructions seem to exist in most, if not all, branches of the \ili{Nakh-Daghestanian} language family.

We defined bridging constructions as bipartite consisting of a main reference clause followed by a subordinate bridging clause. The bridging clause expresses adverbial \isi{subordination} and is marked by a variety of general or specialized temporal converbs. In \ili{Tsezic}, bridging constructions instantiate \isi{recapitulative linkage} as well as \isi{summary linkage}, although the latter is not very frequent. The main functions are stylistic rather than grammatical. They are stylistic devices of traditional narratives and represent a specific type of \isi{parallelism}, which is characteristic of oral performances. In addition, \ili{Tsezic} bridging constructions are repeatedly used to indicate episode changes in narration, which can but need not be accompanied by switches of subject referents or locations. More research is required in order to explore how bridging constructions relate to other forms of \isi{repetition} and \isi{parallelism} such as predicate doubling.
 

 \section*{Appendix}
 \setcounter{equation}{0}
 \exewidth{(A23)}
A \ili{Hunzib} story told by Džamaludin Atranaliev from Stal’skoe (\citealt[154--157]{vandenBerg.1995})  about a mother and a father who were frequently ill, both of them claiming to want to die first so the other one could take care of the son. The excerpt sets in right after the parents discuss the probable looks of Malakulmawt, the angel of death, to which their son replies that he looks like a plucked cock.

\begin{exe}
\exi{(A1)}
\gll əg-ra bowαž-er m-ac’-oɬ, əg-ra m-učαχ-αšun bed ože gišo-ke-n e͂ƛ’e-n m-ɨqə-k’-en žide-s b-iʔer ħeleku=n \underline{o\smash{g}u} \underline{m-oƛ'ak’-en} \underline{lo} \\
that-\textsc{pl} believe.\textsc{pl-pst.ptcp} \textsc{hpl}-see-\textsc{ant.cvb} that-\textsc{pl} \textsc{hpl}-slumber-\textsc{imm.ant} then boy(\textsc{i}) outside-\textsc{inch-cvb} go.\textsc{i}-\textsc{cvb} \textsc{iv}-find-\textsc{caus-cvb} self.\textsc{obl.pl-gen} \textsc{iv}-small cock(\textsc{iv})=\textsc{add} that \textsc{iv}-pluck-\textsc{cvb} be.\textsc{prs.iv}\\
\glt \sqt{When he saw that they believed him, the boy went out, as soon as they fell asleep, caught their own little cock and plucked it.}
\end{exe}

\begin{exe} 
\exi{(A2)}
\gll \textbf{m-oƛ'ak’-en} hi͂ja-do=n b-əc’-əru səsəq’an pode=n=žun hade<b>eče-n e͂du m-ije-n lo oɬu-l ogu buƛii \\
\textsc{iv}-pluck-\textsc{cvb} blood.\textsc{obl-ins=add} \textsc{iv}-be.filled-\textsc{pst.ptcp} some feather(\textsc{iv)=add}=with be.slow<\textsc{iv}>-\textsc{cvb} inside \textsc{iv}-send-\textsc{cvb} be.\textsc{prs.iv}  that.\textsc{obl-erg} that home\\
\glt \sqt{Having plucked it, covered with blood, some feathers left, he let it carefully into the house.}
\end{exe}

\begin{exe}
\exi{(A3)}
\gll bed-do ogu k’ok’ol-eru m-oƛ’ak’-eru taχ-li-ƛ ƛɨrə m-eƛ’e-n b-eče-n lo\\
then-\textsc{dir} that hurt-\textsc{pst.ptcp} \textsc{iv}-pluck-\textsc{pst.ptcp} ottoman-\textsc{obl-sub} under \textsc{iv}-go-\textsc{cvb} \textsc{iv}-stay-\textsc{cvb} be.\textsc{prs.iv}\\
\glt \sqt{Then it, being mauled and plucked, went and sat under the ottoman.}
\end{exe}

\begin{exe}
\exi{(A4)}
\gll \underline{sɨd} \underline{zaban-li-i} \underline{ə\smash{g}i-s} \underline{bed} \underline{\smash{g}išo-ke-n} \underline{lo}\\
one.\textsc{obl} time-\textsc{obl-in} there-\textsc{abl} then outside-\textsc{inch-cvb} be.\textsc{prs.iv}\\
\glt \sqt{At one point, it came out from there.}
\end{exe}

\begin{exe}
\exi{(A5)}
\gll \textbf{gišo-ke-n} b-αƛƛe m-aq’e-n zuq’u-n lo qoqo-o ħeleku\\
outside-\textsc{inch-cvb} \textsc{iv}-middle \textsc{iv}-come-\textsc{cvb} be-\textsc{cvb} be.\textsc{prs.iv} house-\textsc{in} cock(\textsc{iv})\\
\glt \sqt{It came out, the cock came into the middle of the room.}
\end{exe}

\begin{exe}
\exi{(A6)}
\gll \underline{deno} \underline{t’uwαt’-en} \underline{lo} \underline{\smash{q}’anu=n} \underline{ə\smash{g}-ra} \underline{oɬu-l} \underline{\smash{qoqoq}o} \underline{ƛe} \underline{nɨs-oɬ}\\
back throw.\textsc{pl-cvb} be.\textsc{prs.hpl} two=\textsc{add} that-\textsc{pl} that.\textsc{obl-erg} \textsc{interj} \textsc{quot} say-\textsc{ant.cvb} \\
\glt \sqt{They woke up when it crowed.}
\end{exe}

\begin{exe}
\exi{(A7)}
\gll \textbf{deno} \textbf{t’uwαt’-oɬ} ogu bed tišo, ʕali-ɬ-do-s beddo=n m-uχe-n ʕajšat-i-ɬ-do m-eƛ’e-n lo ogu, art’o j-uh-a j-at’ə-r-o-ɬ-do\\
back throw.\textsc{pl-ant.cvb} that then over.there Ali-\textsc{cont-dir-abl} back=\textsc{add} \textsc{iv}-turn-\textsc{cvb} Ayshat(\textsc{ii})-\textsc{obl-cont-dir} \textsc{iv}-go-\textsc{cvb} be.\textsc{prs.iv} that before \textsc{ii}-die-\textsc{inf} \textsc{ii}-want-\textsc{pst.ptcp-obl-cont-dir} \\
\glt \sqt{When they woke up, the cock went across (the room) from Ali, having turned to Ayshat, to her who wanted to die first.}
\end{exe}

\begin{exe}
\exi{(A8)}
\gll žini-ɬ-do m-aq’e-č m-ac’-oɬ ħeleku, “bodu ħeleku Malakulmawt lo” ƛe gič’-en, hi͂č’e-ru oɬu-l, ʕali-ɬ-do “kiš” ƛe n-ac’əj nɨsə-n, ʕali-ɬ-do “kiš” ʕali-ɬ-do “kiš”\\
self.\textsc{obl-cont-dir} \textsc{iv}-come-\textsc{icvb} \textsc{iv}-see-\textsc{ant.cvb} cock(\textsc{iv}) this cock(\textsc{iv}) Malakulmawt be.\textsc{prs.iv} \textsc{quot} think-\textsc{cvb} fear-\textsc{pst.ptcp} that.\textsc{obl-erg} Ali-\textsc{cont-dir} \textsc{interj} \textsc{quot} \textsc{v}-appear say-\textsc{cvb} Ali-\textsc{cont-dir} \textsc{interj} Ali-\textsc{cont-dir} \textsc{interj} \\
\glt \sqt{When she saw it coming, thinking that the cock was Malakulmawt, she said, frightened, "Shoo!" to Ali.}
\end{exe}

\begin{exe}
\exi{(A9)}
\gll  bed	ħeleku	\underline{deno}	\underline{m-u\smash{q}'-o\smash{y}s-i?}\\
then	cock(\textsc{iv})	back	\textsc{iv}-turn-\textsc{fut.neg-int} \\
\glt \sqt{Would not the cock then turn around?}
\end{exe}

\begin{exe}
\exi{(A10)}
\gll  \textbf{bed} 	\textbf{deno}		\textbf{m-uq'e-n}		\underline{ʕali-ɬ-do}			\underline{nuu-n}			\underline{lo}\\
then	back		\textsc{iv}-turn-\textsc{cvb}	Ali-\textsc{cont-dir}	come-\textsc{cvb}	be.\textsc{prs.iv} \\
\glt \sqt{Then having turned, it went to Ali.}
\end{exe}

\begin{exe}
\exi{(A11)}
\gll  \textbf{ʕali-ɬ-do} \textbf{nuw-oɬ}, “ʕajšat-i-ɬ-do kiš, ʕajšat-i-ɬ-do kiš” ƛe nɨsə-n ʕali-lo-n b-oc’-on lo ogu \\
Ali-\textsc{cont-dir} come-\textsc{ant.cvb}  Ayshat-\textsc{obl-cont-dir} \textsc{interj} Ayshat-\textsc{obl-cont-dir} \textsc{interj} \textsc{quot} say-\textsc{cvb} Ali-\textsc{erg=add} \textsc{iv}-chase-\textsc{cvb} be.\textsc{prs.iv} that  \\
\glt \sqt{When it came to Ali, Ali chased it away, saying "Shoo!" to Ayshat.}
\end{exe}

\begin{exe}
\exi{(A12)}
\gll  deno m-eƛ’e-n  beddo m-eƛ’e-n maha-a-ƛ’ žoʁ-i-i-ƛ’ tuwαc’ə-n ože=n maduhanɬi=n zuq’un lo\\
back \textsc{iv}-go-\textsc{cvb} back \textsc{iv}-go-\textsc{cvb} courtyard-\textsc{in-trans} window-\textsc{obl-in-trans} look.\textsc{pl-cvb} boy=\textsc{add} neighbours=\textsc{add} be.\textsc{cvb} be.\textsc{prs.hpl}  \\
\glt \sqt{While it went back and forth, the boy and the neighbours were looking at them from the courtyard through the window.}
\end{exe}

\begin{exe}
\exi{(A13)}
\gll  ʕadam-la zuq’un lo ɬejaʔe-č əg-ra-ƛ’\\
person-\textsc{pl} be.\textsc{cvb} be.\textsc{prs.hpl} laugh.\textsc{pl-icvb} that-\textsc{pl-spr}  \\
\glt \sqt{The people were laughing at them.}
\end{exe}

\begin{exe}
\exi{(A14)}
\gll  əgaa-s žo r-αqu-n li\\
so-\textsc{gen} thing(\textsc{v}) \textsc{v}-happen-\textsc{cvb} be.\textsc{prs.v}  \\
\glt \sqt{Such a thing happened.}
\end{exe}

\section*{Abbreviations}
\begin{multicols}{2}
\begin{tabbing}
\textsc{purp.cvb} \= immediate anterior converb\kill
\textsc{1sg} \> first person singular\\
\textsc{2sg} \> second person singular\\
\textsc{3sg}  \> third person singular\\
\textsc{i-v} \> gender\\
\textsc{abl} \> ablative case\\
\textsc{ad} \> adessive case\\
\textsc{add} \> coordinating enclitic\\
\textsc{all} \> allative case\\
\textsc{ant.cvb} \> anterior converb\\
\textsc{aor} \> aorist\\
\textsc{apud} \> apudessive case\\
\textsc{caus} \> causative\\
\textsc{cntr} \> contrastive\\
\textsc{com} \> comitative\\
\textsc{comp} \> complementizer\\
\textsc{cond} \> conditional converb\\
\textsc{cont} \> contact case\\
\textsc{cvb} \> perfective/narrative converb\\
\textsc{dat} \> dative\\
\textsc{def} \> definiteness\\
\textsc{dem} \> demonstrative\\
\textsc{dir} \> directional\\
\textsc{dist} \> distal\\
\textsc{emph} \> emphatic enclitic\\
\textsc{erg} \> ergative\\
\textsc{evid} \> evidentiality\\
\textsc{fut} \> future tense\\
\textsc{gen} \> genitive\\
\textsc{gen1} \> first genitive\\
\textsc{gen2}  \> second genitive\\
\textsc{hpl} \> human plural\\
\textsc{imm.ant} \> immediate anterior converb\\
\textsc{imp} \> imperative\\
\textsc{in} \> in case\\
\textsc{inch} \> inchoative\\
\textsc{inf} \> infinitive\\
\textsc{ins} \> instrumental\\
\textsc{int} \> interrogative particle\\
\textsc{inter}  \> inter case\\
\textsc{interj} \> interjection\\
\textsc{ints} \> intensifier\\
\textsc{icvb} \> imperfective converb\\
\textsc{lat} \> lative case\\
\textsc{n} \> neuter singular\\
\textsc{neg} \> negation\\
\textsc{npl} \> non-human plural\\
\textsc{obl} \> oblique stem marker\\
\textsc{pfv} \> perfective\\
\textsc{pl} \> plural\\
\textsc{poss} \> possessive\\
\textsc{proh} \> prohibitive\\
\textsc{prs} \> present tense\\
\textsc{pst.uw} \> unwitnessed past tense\\
\textsc{pst.wit} \> witnessed past tense\\
\textsc{ptcp} \> participle\\
\textsc{purp.cvb} \> purposive  converb\\
\textsc{quot} \> quotative\\
\textsc{red} \> reduplication\\
\textsc{res} \> resultative\\
\textsc{sg} \> singular\\
\textsc{sim.cvb} \> simultaneous converb\\
\textsc{spr} \> super case\\
\textsc{sub} \> sub case\\
\textsc{temp} \> temporal  converb\\
\textsc{trans} \> translative\\
\textsc{up} \> located above speaker\\
\textsc{vers} \> versative
\end{tabbing}
\end{multicols}

%%%%%%%%%%%%%%%%%%%%%%%%%%%%%%%%%

\sloppy

\printbibliography[heading=subbibliography,notkeyword=this]
%{\sloppy
%\printbibliography[heading=subbibliography,notkeyword=this]
%}
\end{document}

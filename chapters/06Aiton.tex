\documentclass[output=paper]{LSP/langsci} 
\author{
   Grant Aiton\affiliation{University of Alberta}
}
\title{The form and function of bridging constructions in Eibela discourse}
%epigram
\abstract{Discourse in Eibela utilizes extensive repetition and summarization of events as a means of bridging discourse episodes. These bridging constructions consist of a main reference clause at the end of a unit of discourse, which is immediately referenced by a non-main bridging clause at the commencement of the following unit of discourse. Bridging clauses may be formed from two types of non-main clause, namely medial clauses initiating a clause chain and topic clauses which are embedded within another medial or final clause. Differing units of discourse are often accompanied by differing forms of bridging construction, with clause chain boundaries featuring verbatim repetition of clauses, and larger paragraphs being bounded by bridging clauses utilizing anaphoric predicates. Bridging constructions have been previously shown to serve various functions in Papuan languages, including thematic continuity, reference tracking, and event sequencing, which will also be illustrated in the current discussion of bridging constructions in Eibela.}
\maketitle
%-------------------------

\begin{document}

\section{Introduction and background} 
\label{AiIntroduction}
Eibela, also referred to as Aimele (Ethnologue code: AIL), has approximately 300 speakers living primarily in Lake Campbell, Western Province, Papua New Guinea. The genetic affiliation of Eibela has not been thoroughly investigated, but it is likely that it belongs to the proposed Trans-New Guinea Phylum, of the central and South New Guinea stock, since this is the classification given to the closely related language Kaluli by \citet{wurm78} and \citet{voorhoeve68}. A lower level classification is given as the Bosavi language family in \citet{shaw86}. The data for this paper is drawn from a corpus of approximately 17 hours of transcribed speech from a variety of genres, including narratives, procedurals, myths, sermons, discourse, and songs, which is available online in the Endangered Languages Archive \citep{Aiton.2016}. This corpus is the result of approximately 13 months of immersive fieldwork in Lake Campbell and Wawoi Falls in Western Province, Papua New Guinea. Since bridging constructions are a phenomenon of discourse organization, they predominantly occur in long stretches of speech from a single speaker, and the examples in this chapter are therefore drawn from monologues, including narratives, myths, and procedural descriptions. An extended excerpt from a monologue is provided in the appendix. The text chosen for the appendix is considered by the author to be representative of personal narratives in terms of event structure and the usage of bridging constructions, and where possible claims made in the prose of this chapter are supported by examples from the appendix so that the reader may view these clauses in the context of a larger discourse.

Discourse in Eibela utilizes frequent repetition and summarization of events as a means of bridging discourse episodes. These bridging constructions consist of a main reference clause at the end of a unit of discourse, which is immediately reiterated by a repetition in a non-main bridging clause at the commencement of the following unit of discourse. This paper offers an extensive description of this phenomenon in Eibela, but first a basic introduction some aspects of Eibela is warranted. The canonical constituent order for Eibela is SV in intransitive clauses and AOV in transitive clauses, though other constituent orders are possible. Constituents which are prominent or topical are often omitted from clauses completely. Morphology is exclusively suffixing, with complex verbal morphology for tense, aspect, mood, and evidentiality, and optional ergative-absolutive case marking on noun phrases in core argument positions \citep[see][]{Aiton.2014}. Word classes include open classes of nouns, verbs, and adverbs, and closed classes of adjectives, demonstratives, postpositions, verbal particles, and quantifiers.

Predicates in Eibela can be formed by lexical roots of nearly any word class, although only verbs may be inflected by the full range of tense, aspect, mood, and evidentiality suffixes. Complex inflectional classes of verbs feature various patterns of stem alternations and suppletive tense forms, as well as complex predicates consisting of multiple verbal roots forming a single predicate. 

\begin{exe}
\ex \label{Aiex:01}
\gll [agɛ 	ɸɛɸɛ-jaː]\textsubscript{s} [ɛna]\textsubscript{x} [dobosuwɛ]\textsubscript{x} \textbf{[tɛ} 	\textbf{aːnɛ]}\textsubscript{pred}\\
dog	skinny-\textsc{abs}	there	underneath	go.down 	go\textsc{:pst}\\
\glt \sqt{The skinny dog went down underneath there.}\\
\end{exe}


\begin{exe}
\ex \label{Aiex:02}
\gll [sobolo-wa]\textsubscript{s} \textbf{[tɛbɛ} 	 \textbf{do-wa]}\textsubscript{pred}\\
plane-\textsc{abs}	land	\textsc{stat}-\textsc{pst}\\
\glt \sqt{A plane has landed.}\\
\end{exe}



These complex predicates may take the form of serial verb constructions as in (\ref{Aiex:01}), or auxiliary constructions, as in (\ref{Aiex:02}). In these constructions, only the final verbal root is inflected for predicate categories such as tense, aspect, mood, and evidentiality.

Eibela clauses may be linked together into clause chains, which include several medial clauses culminating in a fully inflected final clause. Clauses in examples will be labeled in subscript to show whether they are a final or a medial clause. In medial clauses, the different-subject marking suffix \textit{‑bi} may be used to show that the subject of the medial clause differs from the subject of the main clause, as seen in example (\ref{Aiex:03ab}).

%example3
\begin{exe}
\ex \label{Aiex:03ab}
\begin{xlist}
\ex \label{Aiex:03a}
\gll [nɛ ɛja-jaː mumunɛ ɛlɛbɛ la-\textbf{bi}]\textsubscript{medial}\\
\textsc{1}:\textsc{sg} father-\textsc{abs} \textsc{name} head be-\textsc{ds}\\
\glt \sqt{‎‎My father was at the head of Mulume creek, and…}\\
\ex \label{Aiex:03b}
\gll[saːgoi ɛjalɛ motuwɛ ɛjalɛ gɛdajoɸa sɛdɛ hɛna mi-jaː]\textsubscript{final}\\
‎\textsc{name} \textsc{coord}:\textsc{du} \textsc{name} \textsc{coord}:\textsc{du} tree.trunk:\textsc{abs} hit \textsc{dur} come-\textsc{pst}\\
\glt \sqt{‎‎‎‎Sagoi and Motuwe came while beating tree trunks (so their approach would be heard).}\\
\end{xlist}
\end{exe}

In this example, the subject of the medial clause in (\ref{Aiex:03a}) is \textit{nɛ ɛjaja} `my father', who is described as being at a location, whereas in the final clause (\ref{Aiex:03b}) the subject is the coordinated noun phrase \textit{saːgoi ɛjalɛ motuwɛ ɛjalɛ} `Saːgai and Motuwe', who are coming while hitting trees. Clauses and noun phrases may additionally be morphologically topicalized, in which case, different subject marking functions in much the same way as in example (\ref{Aiex:03ab}), as can be seen in (\ref{Aiex:04ac}) where the verb in the topic clause is suffixed by \textit{-bi} since its subject differs from that of the main predicate.

%example4
\begin{exe}
\ex \label{Aiex:04ac}
\begin{xlist}
\ex \label{Aiex:04a}
\gll [[na	no-wa ɛimɛ ka aɡlɛ-si]\textsubscript{medial} kɛkɛkɛ]\textsubscript{final}\\
animal	‎\textsc{indf}-‎\textsc{abs} quickly ‎\textsc{foc} laugh-‎\textsc{med}:‎\textsc{pfv} laugh:‎\textsc{ideo}\\
‎‎‎\glt \sqt{The other animals were already laughing.}\\
\ex \label{Aiex:04b}
\gll [no-wɛ-mi=jaː	ɛimɛ	ka	aɡlɛ-\textbf{bi}=jaː]\textsubscript{topic}\\
‎\textsc{indf}-‎\textsc{loc}-‎\textsc{assoc}=‎\textsc{top} already ‎\textsc{foc} laugh-‎\textsc{ds}=‎\textsc{top}\\
‎‎‎\glt \sqt{Another one was already laughing, then...}\\
\ex \label{Aiex:04c}
\gll [[no	wɛ	aːɡɛ	kɛɡa=jaː]\textsubscript{topic}	wɛ	suwɛ	da-li	lɛ-ki	wɛ	dɛdɛ	laː-\textbf{bi}]\textsubscript{final}\\
\textsc{contr} this dog bony=‎\textsc{top} this inside lie-‎\textsc{sim} be-‎\textsc{cont} this hear be-‎\textsc{ds}\\
\glt \sqt{This one, this bony dog who was still inside was listening to this.}\\
\end{xlist}
\end{exe}

A direct contrast of these two usages of the suffix \textit{-bi} is shown in example (\ref{Aiex:04ac}). In (\ref{Aiex:04b}) topic clause has a different subject from the following main clause, and therefore bears the different subject marker. The subject of the topic clause is a pig, who is laughing at the dogs in a folk tale, while the subject of the main clause is one of the dogs, who is covertly listening. In (\ref{Aiex:04c}) the different subject marker appears in the main clause as well, specifying an unexpected or non-topical subject for this clause, where the dog is an unexpected introduction into the story. This use of the different subject marker in a main clause may be interpreted as a kind of desubordination, is which a clause with the morphological form of a non-main clause is functionally and syntactically independent \citep{evans07}.

%vvvvvvvvvvvgggggggggggggggggggreview setion number
With this introduction to Eibela morphosyntax in mind, the bridging clauses described in
subsequent sections may be formed from two types of non-main clause, namely medial clauses
initiating a clause chain and topic clauses which are embedded within another medial or final clause.
Differing units of discourse are often accompanied by differing forms of bridging construction, with
clause chain boundaries featuring verbatim repetition of clauses, and larger paragraphs being bounded
by bridging clauses utilizing anaphoric predicates. Bridging constructions have been previously shown to
serve various functions in Papuan languages, including thematic continuity, reference tracking, and
event sequencing, which will also be illustrated in the current discussion of bridging constructions in
Eibela. The morphosyntax of clause-chaining and clause topicalization strategies will be further
discussed in \refsec{Ailinkingtopic} below. The use of these clause linking devices in bridging constructions will be shown in
\refsec{AiFormal.aspects}, and finally, the semantics and function of bridging constructions will be explored in \refsec{AiDiscourse}, including
discourse organization, temporal anchoring, causation, and argument tracking.

\section{Clause linking and topic clauses} 
\label{Ailinkingtopic}
Two clause linking strategies are relevant to the current discussion of bridging constructions in
Eibela: clause chaining and topicalization. A clause will be assumed to include a predicate and all
arguments of that predicate, though topical or given arguments may often be elided. Clause chaining
consists of a series of at least two clauses, and sometimes several, which describe a series of related
events. A clause chain will be an important unit of Eibela discourse throughout this paper. Topicalization
is a feature of a complex clause whereby a single non-main clause or noun phrase appears immediately
before a clause and functions as the topic or reference point of the following clause.

\subsection{Clause linking} 
\label{Ailinking}
Clause chaining is a form of clause linking where one or more non-main clauses with limited
inflection appear in a sequence, or chain, and the full inflection of tense aspect and mood is expressed
on the final main clause of the chain \citep[][374--376]{longacre07}. For example, in the short clause chain shown
in examples (\ref{Aiex:App23}--\ref{Aiex:App24}) of the appendix, the first medial non-main clause includes the predicate \textit{hɛnaː} \textit{disi}, which is not specified for
tense, and is suffixed by the perfective clause chaining morpheme \textit{-si}. Tense specification is only provided on the verb of the final main clause, \textit{muːduː} `washed' in (\ref{Aiex:App24}). Clause chaining structures have previously been described as something intermediate
between coordinate and subordinate clause linking or labeled as ``coordinate but dependent''
\citep{haiman83}  or ``cosubordinate'' \citep{valin84}.

The two clause linkers \textit{=nɛgɛː} and \textit{-si} are more or less synonymous with no obvious
distributional differences. The aspectual difference represented by the glossing as imperfective for
\textit{=nɛgɛː}, and perfective for \textit{-si}, reflects a tendency rather than a strict correspondence. The enclitic \textit{=nɛgɛː}
more frequently occurs with ongoing events that will still be co-occurring along with the subsequently
described events, while the suffix \textit{-si} more often occurs with perfective events which are completed and
then followed by a consecutive event. 

An additional chaining enclitic \textit{=ki} may be used for ongoing or persisting events, as in (\ref{Aiex:05}) and (\ref{Aiex:06af}). This is used for ongoing imperfective events which continue up until the occurrence of the
following clause. The continuous enclitic \textit{=ki} is aspectually similar to the imperfective enclitic \textit{=nɛgɛː}, but differs in usage
primarily in that \textit{=ki} represents stative, repetitive, or unchanging event structures, whereas \textit{=nɛgɛː} is
often used for processes or telic events. Non-verbal predicates may be used in clause chaining
constructions, but must be accompanied by a verbal auxiliary in non-main clauses as seen in (\ref{Aiex:06af}).


\begin{exe}
\ex \label{Aiex:05}
\gll [sɛnɛ=\textbf{ki}]\textsubscript{medial} [aːmi makiso-wa ɛ-saː-bi]\textsubscript{final}\\
stay=\textsc{cont} \textsc{dem}:\textsc{assoc} visitor-\textsc{abs}	do-\textsc{3}:\textsc{vis}-\textsc{ds}\\
\glt \sqt{We were living there and a visitor did that (came).}\\
\end{exe}

\begin{exe}
\ex \label{Aiex:06af}
\begin{xlist}
\ex \label{Aiex:06a}
\gll [ɛjaːgɛ dɛmɛ di-sɛnɛ waːlɛ-mɛna]\textsubscript{final}\\
butterfly do do-\textsc{nmlz} tell-\textsc{fut}:\textsc{non}.\textsc{3}\\
\glt \sqt{I will tell about what butterflies do.}\\
\ex \label{Aiex:06b}
\gll [ɛjaːgɛ	do-si=ki]\textsubscript{medial} \underline{\smash{[uʃu]}}\textsubscript{final}\\
butterfly \textsc{stat}-\textsc{med}:\textsc{pfv}=\textsc{cont} egg\\
\glt \sqt{There being a butterfly then there is an egg.}\\
\ex \label{Aiex:06c}
\gll \textbf{[uʃu}	\textbf{do-si=ki]}\textsubscript{medial}	\underline{\smash{[kɛkɛbɛaːnɛ]}}\textsubscript{final}\\
egg	\textsc{stat}-\textsc{med}:\textsc{pfv}=\textsc{cont} caterpillar\\
\glt \sqt{There being an egg then there is a caterpillar.}\\
\ex \label{Aiex:06d}
\gll \textbf{[kɛkɛbɛaːnɛ} \textbf{do-si=ki]}\textsubscript{medial} \underline{\smash{[kokoːno]}}\textsubscript{final}\\
caterpillar \textsc{stat}-\textsc{med:}\textsc{pfv}=\textsc{cont} pupae\\
\glt \sqt{There being caterpillar then there is a pupae.}\\
\ex \label{Aiex:06e}
\gll \textbf{[kokoːno} \textbf{do-si=ki]}\textsubscript{medial} \underline{\smash{[ɛja:gɛ]}}\textsubscript{final}\\
pupae \textsc{stat}-\textsc{med}:\textsc{pfv}=\textsc{cont} butterfly\\
\glt \sqt{There being a pupae then there is a butterfly.}\\
\ex \label{Aiex:06f}
\gll [ɛjaːgɛ	maːna	wa	kam]\textsubscript{final}\\
butterfly	behavior:\textsc{abs} \textsc{dir}	finish\\
\glt \sqt{The (story of) butterfly behavior is finished.}\\
\end{xlist}
\end{exe}

Every line given in (\ref{Aiex:06af}) is a clause chain, and the main clauses (\ref{Aiex:06c} to \ref{Aiex:06f}) each begin with a non-main medial
clause (shown in bold) which repeats the proposition of the preceding main clause. When the nominal
predicate is the predicate of a main clause, no auxiliary is needed, but in non-main clauses, the clause
linking morphology may only appear with a verbal auxiliary being appended to the nominal predicate to
for a complex predicate.

\subsection{Topicalization} 
\label{AiTopicalization}
Topicalization is a general process of identifying some concept as the topic or theme of a clause.
In Eibela, this is accomplished by means of left dislocated clause position and the enclitic \textit{=jaː}. \citet{Aiton.2014} summarizes the use of topicalized noun phrases and clause arguments in Eibela argument
structure, such as the example given in (\ref{Aiex:07ab}).

%ex7
\begin{exe}
\ex \label{Aiex:07ab}
\begin{xlist}
\ex \label{Aiex:07a}
\gll \textbf{[[sɛinaːbiː=jaː]}\textsubscript{topic}	gomoːlo-wɛː	hojɛ-kɛː	hɛnaː-gɛnɛː]\textsubscript{medial}\\
tree.kangaroo=\textsc{top}	\textsc{name}-\textsc{erg}	hunt-\textsc{iter}	go-\textsc{med}:\textsc{ipfv}\\
\glt \sqt{Tree kangaroos, Gomoolo had gone hunting (for those animals)…}\\
\ex \label{Aiex:07b}
\gll [olaː	ka	laː]\textsubscript{final}\\
shoot:\textsc{pst}	\textsc{foc}	\textsc{def}\\
\glt \sqt{...and (he) had shot one (a tree kangaroo).}\\
\end{xlist}
\end{exe}

The current discussion will not further discuss topical noun phrases, and will focus on the occurrence of clauses as the topic of a subsequent main clause.
	A clause is presented in the topic position to provide a conceptual point of reference for the event described in the main clause. When the topic is a clause, as in example (\ref{Aiex:08}), the clause is followed by the topic-marking enclitic \textit{=jaː} and precedes the main clause.

%ex8
\begin{exe}
\ex \label{Aiex:08}
\gll [\textbf{[nɛ}	\textbf{ɛsɛ} \textbf{no-wa}	\textbf{oɡɛ}	\textbf{di=jaː]}\textsubscript{topic}	ɸiliː-nɛ]\textsubscript{final}\\
1:\textsc{sg}	string.bag	\textsc{indf}-\textsc{abs}	pick.up	take=\textsc{top}	ascend-\textsc{pst}\\
\glt \sqt{Taking another bag, I went up.}\\
\end{exe}

The semantic relationship between the topic clause and main clause is rather vague. In (\ref{Aiex:08}), the intended meaning is that the speaker primarily intended to take his string bag somewhere, and in order to do this, he walked uphill. In future time contexts, a topic clause can produce a conditional reading, as in (\ref{Aiex:09}).

%ex9
\begin{exe}
\ex \label{Aiex:09}
\gll [\textbf{[ɡɛ}	\textbf{soːwa}	\textbf{suɡuːluː-mɛnaː=jaː]}\textsubscript{topic}	ɛːlɛmɛːntɾiː	tiːsa-jaː	kɛlɛ-maː]\textsubscript{final}\\
2:\textsc{sg}	child	attend.school-\textsc{fut}=\textsc{top}	elementary	teacher-\textsc{abs}	find-\textsc{imp}\\
\glt \sqt{If your children are to go to school, then find a teacher!}\\
\end{exe}

A conditional meaning as in (\ref{Aiex:09}) could simply be paraphrased as an intentional meaning, i.e. `Find a teacher in order to ensure that your children attend school.'
	This intentional/conditional meaning cannot be taken for granted however. Instead it seems to be an incidental result of the topic clause’s role as a prominent and given piece of information \citep{haiman.1978}. In both (\ref{Aiex:08}) and (\ref{Aiex:09}), the topic clause refers to previously mentioned information which is a prominent and ongoing topic of the narrative. The role of the main clause is then to expand upon the given topic and provide new information which has not yet been presented. For example, in clause (\ref{Aiex:App26}) of the appendix, the events of the topic clause and main clause are sequential, with the topic clause clearly preceding the events of the main clause, and no intentional interpretation is possible. When a clause appears as a topic, the topicalized clause reiterates familiar or already mentioned information as a reference point for new information which is introduced in the following main clause. This results in the bridging constructions which will be discussed in greater detail in \refsec{AiFormal.aspects}.

\subsection{Topicalized medial clauses} 
\label{AiTopicalized.medial}
Interestingly, these two strategies of clause linking, chaining and topicalization, may co-occur. The perfective clause linking suffix \textit{-si} may be used in a topicalized clause to provide specific aspectual information, as in example (\ref{Aiex:10ab}). In the example (\ref{Aiex:10b}), the clause \textit{nɛ bɛdɛsijaː} is presented with both the clause linking suffix \textit{-si} and the topicalizing enclitic \textit{=jaː}.

%ex10
\begin{exe}
\ex \label{Aiex:10ab}
\begin{xlist}
\ex \label{Aiex:10a}
\gll [kosuwa-jaː	ja	ɡiɡɛ	di	bɛda-nɛ]\textsubscript{medial}\\
cassowary-\textsc{abs}	come	make.noise	\textsc{pfv}	hear:\textsc{pst}-\textsc{med}:\textsc{ipfv}\\
\glt \sqt{I heard a cassowary come and make noise.}\\
\ex \label{Aiex:10b}
\gll [\textbf{[nɛ}	\textbf{bɛdɛ-si=jaː]}\textsubscript{topic}	ma	bobo]\textsubscript{final}\\
1:\textsc{sg}	hear-\textsc{med}:\textsc{pfv}=\textsc{top}	\textsc{neg}	real\\
\glt \sqt{I heard that, and (I thought) it was not real (i.e. a spirit).}\\
\end{xlist}
\end{exe}

In this construction, the clause linking suffix \textit{-si} provides aspectual information regarding the timing of the topic with respect to the main clause. Specifically, the topic and main clause are consecutive events, where the topic clause is a perfective event occurring immediately prior to the main clause.
	In addition to these semantic and functional considerations, topical clauses containing an auxiliary within the predicate require the clause linking suffix  \textit{-si}. This is true even if the aspectual information provided by the suffix  \textit{-si} is redundant as in (\ref{Aiex:11b}).

%ex11
\begin{exe}
\ex \label{Aiex:11ab}
\begin{xlist}
\ex \label{Aiex:11a}
\gll [aːmi	dɛɸija-ɸɛi]\textsubscript{final}\\
\textsc{dem}:\textsc{assoc}	measure-\textsc{hypoth}:\textsc{comp}\\
\glt \sqt{‎‎(The other sleeping space being made like this,) measure there.}\\
\ex \label{Aiex:11b}
\gll [\textbf{[ɛ}	\textbf{di-si=jaː]}\textsubscript{topic}	hɛnaː-nɛː]\textsubscript{medial}	[isi-jaː	kodu-mɛi]\textsubscript{final}\\
do	\textsc{pfv}-\textsc{med}:\textsc{pfv}=\textsc{top}	\textsc{dur}-\textsc{med}:\textsc{ipfv}	post-\textsc{abs}	cut-\textsc{hypoth}\\
\glt \sqt{‎‎That being done, go and cut the posts.}\\
\end{xlist}
\end{exe}

In example (\ref{Aiex:11ab}), the auxiliary \textit{di} specifies a perfective aspect, and in this context, the aspectual overtones of the suffix \textit{-si} are redundant. In contrast, the auxiliary \textit{hɛnaː} is used for continuing durative action, which is incompatible with the perfective aspect which often corresponds to the clause linker \textit{-si}.

\section{Formal aspects of bridging construction in Eibela} 
\label{AiFormal.aspects}
In this section the form of bridging constructions in Eibela will be examined and shown to fall into two types of bridging construction: Recapitulative Linkage and Summary Linkage. The general notion of a bridging construction, along with these two sub-types of bridging construction, is thoroughly explained in \citeauthor{guerin18} (this volume), and this section will follow the same terminology and conventions except where noted. These notational conventions will include underlining the reference clause and displaying in bold the bridging clause in a bridging construction. This section will include the presentation and definition of key terms and concepts involved in the realization of bridging constructions as realized in Eibela, and the ways in which clause chaining and topical clauses form linking structures in Eibela discourse. 


\subsection{Overview of bridging constructions} 
\label{AiOverview}
The type of bridging constructions examined in this paper is confined to non-main clauses, including medial and topical clauses, which repeat or summarize a previous element of the discourse (\citealt{devries.2005,devries.2006,dixon09}; \citealt[][382--383]{Thompson.et.al.2007}). If example (\ref{Aiex:11ab}) is again considered, it is apparent that the topical clause in (\ref{Aiex:11b}) is a repetition of the main clause in (\ref{Aiex:11a}). In the discussion of these sorts of repetitions, it will be useful to refer to the original clause, as in (\ref{Aiex:11a}), as the reference clause, while the repetition, as in (\ref{Aiex:11b}), will be referred to as the bridging clause as presented in chapter one of this volume. A reference clause is most often a final main clause, but as seen by the medial clause in (\ref{Aiex:11a}), this is not always the case. Additionally, a reference clause need not be a main clause with a verbal predicate, as evidenced by the nominal predicates involved in the bridging constructions in example (\ref{Aiex:06af}). A bridging clause on the other hand may be either a medial non-main clause, or an embedded topic clause, as seen in the topic clause forming a bridging clause in (\ref{Aiex:11b}).

\subsection{Recapitulative linkage} 
\label{AiRecapitulative}
The form of the bridging clause may broadly be described as either recapitulation or summarizing. Recapitulative linkage refers to a bridging clause with a predicate which is synonymous or identical to the predicate of the reference clause. In contrast, summary linkage refers to a bridging clause with a generic or anaphoric verb which makes reference to the same event as the reference clause. All of the examples given thus far fall into the category of recapitulation. In these examples, much of the lexical content and argument structure from the reference clause is repeated in the bridging clause, as illustrated in clauses (\ref{Aiex:App26}--\ref{Aiex:App27}) of the appendix where the predicate and object of the reference clause is repeated in the bridging clause, and only the case-marking and verbal inflection differs.

In addition to very close repetitions of vocabulary like the example seen in (\ref{Aiex:App26}--\ref{Aiex:App27}) of the appendix, recapitulative linkage may also include substitutions in the reference clause as described in \citeauthor{guerin18} (this volume). This may be due to differing word choices which may slightly alter the proposition by including more or less information than the reference clause, or the inclusion or exclusion of clause constituents. Of course the bridging clause and the reference clause must by definition describe the same event, but the use of synonyms or the choice to include or exclude certain details may alter the information load of the bridging clause relative to the reference clause.

In instances where a synonym or near synonym is used, the predicates may differ in their precise meaning, and therefore offer differing perspective on an event. For example, in example (\ref{Aiex:12ad}), the reference clause in (\ref{Aiex:12a}) and the bridging clause in (\ref{Aiex:12b}) both refer to the same event, namely the act of whittling a strip of vine so that it is thin and smooth and can be used as a fine cord in construction.

%ex12
\begin{exe}
\ex \label{Aiex:12ad}
\begin{xlist}
\ex \label{Aiex:12a}
\gll \underline{[sɛːli}	\underline{gaːlɛ-mɛi]}\textsubscript{final}\\
properly	shave.thin-\textsc{hypoth}\\
\glt \sqt{(You) should shave it properly}\\
\ex \label{Aiex:12b}
\gll \textbf{[sɛli}	\textbf{ɛmɛlɛ-si]}\textsubscript{medial}\\
properly	make.flat-\textsc{med}:\textsc{pfv}\\
\glt \sqt{Flatten it properly (by shaving) and then…}\\
\ex \label{Aiex:12c}
\gll [[gaːjɛ-liːː	gaːlɛ	di=jaː]\textsubscript{topic}	ɸogono	di-si]\textsubscript{medial}\\
shave.thin-\textsc{sim}:\textsc{dur}	shave.thin	\textsc{pfv}=\textsc{top}	other.side	\textsc{pfv}-\textsc{med}:\textsc{pfv}\\
\glt \sqt{Keep shaving it thin, when it’s shaved thin, take the other side, and then…}\\
\ex \label{Aiex:12d}
\gll [mɛːgi	ɛna	gudɛː-kɛi	ɸiliː-mɛi]\textsubscript{final}\\
rope	\textsc{dem}	wrap-\textsc{inst}	ascend-\textsc{hypoth}\\
\glt	\sqt{(You) should wrap the rope going up.}\\
\end{xlist}
\end{exe}

The reference clause and bridging clause use different verbs to predicate the event however, and in doing so, they each present a different aspect of the action being described. Initially, the verb  \textit{gaːlɛ} is used in (\ref{Aiex:12a}) and describes the act of whittling or shaving thin strips of material off of an item with a knife. The bridging clause in (\ref{Aiex:12b}) then describes the same action, but uses the predicate \textit{ɛmɛlɛ} meaning `to level' or `to make flat'. This word choice describes the intention or goal of the event in the bridging clause and complements the description of the method described in the reference clause. In this way, the two clauses taken together present a more complete description of the event than either clause taken alone.
	Elements of the reference clause are also routinely omitted in bridging clauses, as noted in \citeauthor{guerin18} (this volume). This isn’t particularly surprising in Eibela since backgrounded arguments are often elided in all Eibela clause types. A given argument is typically elided when it is readily predictable from the context. Additionally, a complex noun phrase in the reference clause may be repeated in a simplified form as in line (\ref{Aiex:App22}) of the appendix where \textit{baːkɛlɛ duna} `bush turkey nest' is reduced to the simpler form \textit{baːkɛlɛ} `bush turkey (nest)' in the bridging clause seen in line (\ref{Aiex:App23}). Elements of a bridging clause are obviously very predictable given their repetitive nature, and omitting arguments, or elements of complex arguments, is simply a means of back-grounding known information which has less prominence within the discourse.


In cases where the reference clause contains a topic, the topic is also omitted from the repetition in the bridging clause, as in example (\ref{Aiex:App48}--\ref{Aiex:App50}) of the appendix. The bridging clause makes reference to only the main clause of this final clause of the clause-chain, and does not repeat the embedded topic \textit{hanɛ sɛja} `river shore' or the preceding medial clause \textit{hɛnaːnɛgɛː} `went and...'. In summary, recapitulative linkage is a repetition of lexical elements from the reference clause. These can be exact repetitions of the same lexical items, or may be semantically related terms with the same predicative or argument reference. The repeated bridging clauses are typically reduced relative to the previous reference clause and tend to include only the predicate and highlighted arguments, while less prominent elements are reduced or omitted. The function and motivation for choosing particular clause elements to be repeated in a bridging clause will be further explored in \refsec{AiDiscourse}.


\subsection{Summary linkage} 
\label{AiSum.linkg}
Summary linkage differs from recapitulative linkage in that the predicate of the bridging clause utilizes a generic verb to refer to a preceding event rather than repeated lexical items. In Eibela, this can take several forms, including the light verb \textit{ɛ} `do', the demonstrative verb \textit{wogu} `do thus', or the durative auxiliary verb \textit{hɛnaː}. In contrast to recapitulative linkage, the bridging clause in summary linkage is always preceded by a final clause. In recapitulative bridging, the preceding reference clause may be either a final or medial clause. This means that summary linkage in Eibela is always the first part of a new clause chain or complex clause. As with recapitulative linkage, the bridging clause may take the form of either a medial clause or topic clause.

\subsubsection{\textit{ɛ} `do'} 
\label{Ailightverb.do}
	The light verb \textit{ɛ} is by far the most common summary linkage strategy. It does not occur with any tense morphology, but a variety of aspectual and conjunctive enclitics, including switch reference, perfectivity, and completion. The reference of \textit{ɛ} `do' is non-specific and general. In (\ref{Aiex:App29ab}) of the appendix, the topic clause \textit{ɛbija} ‘do’ makes reference to the preceding final clause, \textit{ɛimɛ oːɸa aːnɛ} `The sun set'.
Bridging clauses formed with \textit{ɛ} are commonly medial clauses, as in (\ref{Aiex:13ac}), or topic clauses as in
(\ref{Aiex:14ab}). In these cases, the bridging clause is an introductory dependent of a larger complex clause or clause chain. In (\ref{Aiex:13ac}), the summary bridging clause in (\ref{Aiex:13b}) forms the initial medial clause of a short chain of three clauses.

%ex13
\begin{exe}
\ex \label{Aiex:13ac}
\begin{xlist}
\ex \label{Aiex:13a}
\gll \underline{\smash{[aːmi}}	\underline{\smash{ɛna}}	\underline{\smash{bɛː-ɸɛi]}}\textsubscript{final}\\
\textsc{dem:assoc}	\textsc{dem:abs}	put.on-\textsc{hypoth:comp}\\
\glt \sqt{Then put it on there.}\\
\ex \label{Aiex:13b}
\gll \textbf{[ɛ}	\textbf{di-si]}\textsubscript{medial}\\
	do	\textsc{pfv-med:pfv}\\
\glt \sqt{Do that and then…}\\
\ex \label{Aiex:13c}
\gll	[ɛna	mɛgi	ɛna	adlɛ-lɛ-si]\textsubscript{medial}	[taːlɛ=ta]\textsubscript{final}\\
	\textsc{dem}	rope	\textsc{dem:abs}	tie.on-\textsc{sim-med:pfv}	finish=\textsc{atel}\\
\glt \sqt{…then tie that rope on there and finish.}\\
\end{xlist}
\end{exe}

Similarly, the non-main clause in (\ref{Aiex:14b}) is the topic of the following main clause. 

%ex 14
\begin{exe}
\ex \label{Aiex:14ab}
\begin{xlist}
\ex \label{Aiex:14a}
\gll \underline{\smash{[usaja}}	\underline{ka}	\underline{\smash{ja}}	\underline{di}]\textsubscript{final}\\
	\textsc{name}	\textsc{foc}	came	marry\\
\glt	\sqt{Usaja came and married her.}\\
\ex \label{Aiex:14b}
\gll [\textbf{[ɛ=ta-bi=jaː]}\textsubscript{topic}	ɛgɛ-jaː	ugɛi	ɛna	aːmi	mi-jaː-bo]\textsubscript{final}\\
	be-\textsc{atel-ds=top}	someone-\textsc{abs}	\textsc{name}	that\textsc{:abs}	\textsc{dem:assoc}	come\textsc{-pst-infer}\\
\glt \sqt{He was doing that, so this guy, this Ugei came there.}\\
\end{xlist}
\end{exe}

The main difference between the uses seen in (\ref{Aiex:13b}) and (\ref{Aiex:14b}) is the scope of the bridging clause’s dependency, either as a constituent of a single following main clause, as with the topical function in (\ref{Aiex:14ab}), or a component in a series of medial clauses forming a clause chain as in (\ref{Aiex:13ac}).

\subsubsection{\textit{wogu} `do thus'} 
\label{Aidemonverb.wogu}
The demonstrative verb \textit{wogu} (commonly reduced to \textit{o} or \textit{ogu}) functions very similarly to the semantically light verb \textit{ɛ} with regard to bridging constructions, except that the reference of the demonstrative verb must be a specific event. A reference event is either an exophoric reference (e.g. `doing that' where the event is in progress and may be seen), or an event described immediately previously. In a bridging role, \textit{wogu} does not present any tense, absolute aspect, mood, or evidentiality morphology, and is limited to clause-linking morphology such as relative aspect, topicalization, and switch reference. This results in a slightly more morphologically deficient predicate than \textit{ɛ}. A prominent semantic difference is that \textit{wogu} is more limited with regard to its scope of reference, whereas \textit{ɛ} may reference an entire discourse episode or state of affairs. For example, in (\ref{Aiex:15ae}) there are multiple instances of \textit{wogu} bridging clauses which specifically reference the immediately preceding clause. Bridging clauses with the demonstrative verb \textit{wogu} may take the form of topic clauses as in (\ref{Aiex:15b}), and medial clauses as in (\ref{Aiex:15d}--\ref{Aiex:15e}).

%ex15
\begin{exe}
\ex \label{Aiex:15ae}
\begin{xlist}
\ex \label{Aiex:15a}
\gll \underline{\smash{[isa-jaː}}	\underline{tila}	\underline{bu-saː-bi}]\textsubscript{final}\\
ground-\textsc{abs}	descend	impact-\textsc{vis:3-ds}\\
\glt	‘‎\sqt{They continued struggling and fell to the ground.}\\
\ex \label{Aiex:15b}
\gll	\textbf{[[wogu-bi=jaː]}\textsubscript{topic}	bɛda=nɛgɛː]\textsubscript{medial}	[aːmi	kolu-wa	wɛlɛ-saː-bi]\textsubscript{final}\\
do.thus-\textsc{ds=top}	see-\textsc{med:ipfv}	\textsc{dem:assoc}	man-\textsc{abs}	shout-\textsc{3:vis-ds}\\
\glt	\sqt{‎‎They did that and then I saw (Hauwa) call to the men.}\\
\ex \label{Aiex:15c}
\gll	\underline{[dobuwɛ-joːː}	\underline{ɛ-saː-bi]}\textsubscript{final}\\
\textsc{name-voc}	do-\textsc{vis:3-ds}\\
\glt	\sqt{He said, ``Dobuwe!''}\\
\ex \label{Aiex:15d}
\gll	\textbf{[wogu-bi]}\textsubscript{medial}	\underline{[bɛda-lolu=wa}	\underline{waːː]}{final}\\
do.thus-\textsc{ds}	see\textsc{ːpst-comp=top}	wah!\\
\glt	\sqt{He did that and I saw them go ``whaa!''}\\
\ex \label{Aiex:15e}
\gll	\textbf{[o-si=ki]}\textsubscript{medial}	[ja-bi]\textsubscript{final}\\
do.thus\textsc{-med:pfv=cont}	come-\textsc{ds}\\
\glt	\sqt{I did that (saw them) and they came.}\\
\end{xlist}
\end{exe}

As seen in (\ref{Aiex:15e}), and (\ref{Aiex:16b}), in topic and medial positions, the two reduced forms of \textit{wogu} (\textit{o} and \textit{ogu}) are commonly used in free alternation.

%ex16
\begin{exe}
\ex \label{Aiex:16ab}
\begin{xlist}
\ex \label{Aiex:16a}
\gll \underline{[ɡɛː}	\underline{hɛːɡa-jaː}	\underline{ɛ-saː]}\textsubscript{final}\\
\textsc{2}:sg	how\textsc{:pst-inter:non.prs}	say-\textsc{3:vis}\\
\glt \sqt{He said ``What happened to you?''.}\\
\ex \label{Aiex:16b}
\gll \textbf{[oɡu}	\textbf{bɛda]}\textsubscript{medial}	[nɛ	ɛnɛbɛ	wɛ	dɛːja	wɛ	kɛi]\textsubscript{final}\\
do.thus	\textsc{cons}	\textsc{1:sg}	leg	this	swollen	this	\textsc{asser}\\
\glt \sqt{‎‎He did (said) that, so (I said) ``My leg is swollen, this one.''}\\
\end{xlist}
\end{exe}

This reduction does not occur when \textit{wogu} is used as the main predicate of the clause, and is a prominent feature of topical and medial bridging clauses formed with \textit{wogu}.


\subsubsection{\textit{hɛnaː} `durative'} 
\label{Aiaspect.hena}
The durative auxiliary \textit{hɛnaː} may also be used as the predicate of a bridging clause, as shown in (\ref{Aiex:17c}). Like wogu, there is no tense, aspect, mood, or evidentiality inflection in topic or medial clauses predicated by durative \textit{hɛnaː}. Additionally, the auxiliary \textit{hɛnaː} cannot appear as the final predicate in a final clause. 

%ex17
\begin{exe}
\ex \label{Aiex:17ac}
\begin{xlist}
\ex \label{Aiex:17a}
\gll	[ɛimɛ	oɡa	ɛ	ɡɛ-mɛna=ta]\textsubscript{medial}	\underline{\smash{[holo}}	\underline{\smash{anɛ-obo]}}\textsubscript{final}\\
already	pandanus	seedling	plant-\textsc{fut=atel}	\textsc{dem:up}	go:\textsc{pst-infer}\\
\glt ‎\sqt{‎He had already gone up there to plant pandanus seeds.}\\
\ex \label{Aiex:17b}
\gll	\textbf{[[oɡu-bi=jaː]}\textsubscript{topic}	\underline{nɛ}	\underline{\smash{nɛ-ɸɛni}}	\underline{ɛna}	\underline{\smash{ja}}	\underline{\smash{di]}}\textsubscript{final}\\
do.thus\textsc{-ds=top}	\textsc{1:sg}	\textsc{1:sg-}alone	still	here	\textsc{pfv}\\
\glt \sqt{He did that, I was still alone here.}\\
\ex \label{Aiex:17c}
\gll	\textbf{[[hɛnaː-si=jaː]}\textsubscript{topic}	si-jaː]\textsubscript{final}\\
\textsc{dur-med:pfv=top}	move.around\textsc{-pst}\\
\glt	\sqt{That being the case, I was wandering around here.}\\
\end{xlist}
\end{exe}


Other auxiliaries must be preceded by the dummy verb \textit{ɛ} (e.g. (\ref{Aiex:13b})), and the independence of \textit{hɛnaː} as a predicate is unique among auxiliaries. Semantically, \textit{hɛnaː} specifies an ongoing action or continuing state, and originates from a verb meaning `to go'.

Similarly to \textit{wogu}, in medial clauses \textit{hɛnaː} is often reduced, in this case to \textit{naː}, as shown in (\ref{Aiex:18a}).

%ex18
\begin{exe}
\ex \label{Aiex:18ab}
\begin{xlist}
\ex \label{Aiex:18a}
\gll [ɛ-ɸɛija]\textsubscript{medial}	[naː-si]\textsubscript{medial}\\
do-\textsc{prf}	\textsc{dur-med:pfv}\\
\glt \sqt{That had happened and then…}\\
\ex \label{Aiex:18b}
\gll [nɛ	ɛna	hodosu-wɛ=mi]\textsubscript{medial}\\
\textsc{1:sg}	still	small\textsc{-loc=assoc}\\
\glt \sqt{when I was still small…}\\
\end{xlist}
\end{exe}

This reduction occurs only in bridging constructions such as the example in (\ref{Aiex:18a}). The primary difference between \textit{ɛ} ‘do’, \textit{wogu} ‘do thus’, and \textit{hɛnaː} ‘continue doing’ is a semantic contrast. \textit{ɛ} ‘do’ has no substantive semantic content, and makes reference to an indefinite stretch preceding discourse while providing a verb stem for clause-linking morphology. \textit{wogu} ‘do thus’ on the other hand makes definite reference to a specific event which immediately precedes the bridging clause, or is clear from the extra-linguistic context. Finally, \textit{hɛnaː} ‘continue doing’, has a prominent aspectual meaning of durativity, and references a definite immediately preceding event. More on the discourse roles of bridging constructions follows in \refsec{AiDiscourse}.

\section{Discourse functions of bridging constructions} 
\label{AiDiscourse}
Bridging constructions are found to have several functions within a discourse, including frame-setting, argument tracking, showing temporal relations between clauses, and defining discourse episodes. In general, these functions revolve around establishing a given frame of reference, and then situating new information within this frame of reference. \citet{prince81} presents a relevant discussion in which given entities may be thought of as ``hooks'' for new information, and that the given information therefore provides a sentential anchor for additional information. This anchor provided by the bridging clause may establish information such as a temporal setting, the participants involved, or the relevance of events to one another with regard to reasons, causes, and effects. This information then helps the hearer to integrate the subsequent new information in the broader discourse and therefore promotes textual cohesion.

In this analysis, two levels of discourse organization become apparent. A larger series of related events is broken into episodes, while the entire series of related events forms a cohesive unit within a larger discourse. This larger unit will be referred to as the paragraph \citep[corresponding to the idea of a paragraph in][372]{Thompson.et.al.2007}, and the constituent parts will be referred to as episodes. Episodes are made up of one or more clause chains, and the formal realization of these discourse units is the preference for recapitulative linkage at episode boundaries, and summary linkage at paragraph boundaries. The use of bridging constructions in discourse organization to define two levels of discourse is discussed in greater detail in \citet{Aiton.2015}.

\subsection{Discourse organization} 
\label{Aidiscourse.org}
Bridging constructions occur at a boundary between discourse episodes. It is a way of reiterating and summarizing the conclusion of a series of events, and then highlighting the relationship of the following episode to the previous events (see \citealt{devries.2005}’s discussion of thematic continuity and discontinuity). In Eibela narratives, the identity of these two discourse units is often defined by the type of bridging clause that is used, and that these distinctions will result in different types of bridging constructions having differing discursive functions. Two representative examples will be discussed in the text below, and additional examples may be seen in the final appendix of this chapter.

	For example, in the example given in (\ref{Aiex:19a}--\ref{Aiex:19c}), there is a significant shift between a description of an event in the distant past, when the speaker burned himself as a child, and a description of the present state of affairs, when the speaker shows the scar that is currently present due to these past events. The summary linkage in (\ref{Aiex:19c}) appears at the end of a text, and marks the end of the final paragraph of the narrative, and the beginning of a metatextual commentary on the narrative as a whole rather than a single identifiable reference clause. This transition both marks a shift in temporal reference and highlights the semantic relationship between the paragraphs.
	
	%ex19
	\begin{exe}
\ex \label{Aiex:19af}
\begin{xlist}
\ex \label{Aiex:19a}
\gll [gulu	tila=nɛgɛː]\textsubscript{medial}\\
knee	descend\textsc{=med:ipfv}\\
\glt \sqt{This knee was down and then…}\\
\ex \label{Aiex:19b}
\gll [dɛ	ɛna	ka	gɛ-ɸɛija]\textsubscript{medial}\\
fire	that	\textsc{foc}	burn\textsc{-prf}\\
\glt \sqt{It was burned on that fire.}\\
\ex \label{Aiex:19c}
\gll \textbf{[ɛ-ɸɛija]}\textsubscript{medial}	[umoko	wɛ	daː	ko]\textsubscript{final}\\
do\textsc{-prf}	scar	this	exist	\textsc{dem:pred}\\
\glt \sqt{That happened and this is the scar.}\\
\ex \label{Aiex:19d}
\gll \textbf{[ɛ-ɸɛija]}\textsubscript{medial}	[nana	la	babalɛ	do-wa]\textsubscript{final}\\
do\textsc{-prf}	\textsc{1:sg:p}	\textsc{def}	not.know	\textsc{stat-pst}\\
\glt \sqt{That happened and I didn't know (about it).}\\
\ex \label{Aiex:19e}
\gll \textbf{[ɛ-ɸɛija]}\textsubscript{medial}	[ka	nɛ	ɛja	ɛ	waːlɛ	bɛda]\textsubscript{medial}\\
do-\textsc{perf}	\textsc{foc}	\textsc{1:sg}	father	\textsc{3:sg}	tell	\textsc{cons}\\
\glt \sqt{That happened, and my father, he told (me about it) so…}\\
\ex \label{Aiex:19f}
\gll [nɛ	ɛna	dɛda]\textsubscript{final}\\
\textsc{1:sg}	\textsc{dem}	understand\textsc{:pst}\\
\glt \sqt{I know about that (story).}\\
\end{xlist}
\end{exe}

	While the excerpt in example (\ref{Aiex:19af}) isn’t long enough to show individual episodes in the initial paragraph, a larger example drawn from the appendix shows a long series of events is broken into four discourse episodes which describe three stages of a narrative and a final episode marking the coda of the paragraph. In the first episode beginning in line (\ref{Aiex:App37}) of the appendix, the protagonists decide to attack a pig that was unexpectedly encountered. In the second episode, (\ref{Aiex:App38}) of the appendix, the protagonists are attacking the pig without successfully killing it. Then in (\ref{Aiex:App41}--\ref{Aiex:App43}) the speaker steps into the assault and successfully kills the pig. The bridging clauses in (\ref{Aiex:App38}) and (\ref{Aiex:App41}) signal a transition between these three distinct episodes in the narrative. Finally, another instance of summary linkage in (\ref{Aiex:App44}) of the appendix references the entire series of events and is followed by a finale of sorts which describes the final result of the entire narrative.
	
	In the lines (\ref{Aiex:App37}--\ref{Aiex:App43}) of the appendix, the entire sequence constitutes one paragraph. This paragraph is divided into four episodes in total, with the first three episodes describing the events that occurred, and the final episode providing a summary and result of the whole paragraph. Whereas the bridging constructions in (\ref{Aiex:App38}) and (\ref{Aiex:App41}) of the appendix, reference only the immediately preceding event, the final example of summary linkage references the entire series of events and comments on the result of the entire paragraph. This shows two levels of discourse organization, which are associated with different types of bridging construction. Individual events form episodes, which are linked to other episodes describing related event by means of recapitulative linkage. A series of episodes linked by recapitulative linkage may then form a paragraph. An instance of summary linkage at the termination of a paragraphmay then present a conclusion or commentary, which is presented in relation to the entire series of linked episodes.
	
	The same pattern can be seen in procedural texts, where a series of steps constitute a larger coherent stage in the project. Example (\ref{Aiex:20ai}) is a continuation of the process described in example (\ref{Aiex:12ad}) in which the speaker is describing the process of making a head dress. The paragraph from (\ref{Aiex:20a}) to (\ref{Aiex:20i}) describes how to wrap the frame of the head dress in vine cord and then inserting feathers into the cord. Each individual step is part of the larger task of wrapping the head dress and inserting feathers into the cord, and the paragraph is brought to a conclusion by the concluding episode in (\ref{Aiex:20h}) which is introduced by summary linkage.
	%ex20
	\begin{exe}
\ex \label{Aiex:20ai}
\begin{xlist}
\ex \label{Aiex:20a}
\gll \underline{[aːmi}		\underline{kowɛːgɛ-si]}\textsubscript{medial}	\underline{[ɸiliː-mɛi]}\textsubscript{final}\\
\textsc{pro:assoc}	weave.together\textsc{-med:pfv}	ascend\textsc{-hypoth}\\
\glt \sqt{Then weave (the strands) going up.}\\
\ex \label{Aiex:20b}
\gll [aːnɛ-kɛi	ɡo=taː]\textsubscript{medial}\\
two-\textsc{inst}	meet\textsc{=tel}\\
\glt \sqt{The two ends are joined together.}\\
\ex \label{Aiex:20c}
\gll	\textbf{[[kowɛːɡɛ-si} \textbf{ɸiliː=jaːː]}\textsubscript{topic}	taːlɛ=taː	di-si]\textsubscript{medial}\\
weave.together\textsc{-med:pfv}	ascend\textsc{=top:dur}		finish\textsc{=tel}	\textsc{pfv-med:pfv}\\
\glt \sqt{Having woven (the strands) together, then that’s finished.}\\
\ex \label{Aiex:20d}
\gll \underline{\smash{[aːmi}}	\underline{\smash{mɛgi}}	\underline{no-wa}	\underline{la}	\underline{\smash{gaːlɛ-mɛi]}}\textsubscript{final}\\
\textsc{dem:assoc}	rope	another\textsc{-abs}	\textsc{def:abs}	shave.thin\textsc{-hypoth}\\
\glt \sqt{Then shave thin another piece of rope.}\\
\ex \label{Aiex:20e}
\gll [[mɛgi	no=wa]\textsubscript{topic}	abo	bu	solu-mɛi]\textsubscript{final}\\
rope	another\textsc{=top}	bird	quill	put.in\textsc{-hypoth}\\
\glt \sqt{Then push bird quills into the other rope.}\\
\ex \label{Aiex:20f}
\gll \textbf{[[mɛgi}	\textbf{no=wa}	\textbf{gaː=jaː]}\textsubscript{topic}	la-bi-no	di-si]\textsubscript{medial}\\
rope	another\textsc{=abs}	shave.thin\textsc{=top}	exist\textsc{-ds-irr}	\textsc{pfv-med:pfv}\\
\glt \sqt{The other shaved rope is there, so…}\\
\ex \label{Aiex:20g}
\gll [aːmi	ɛna	bɛːɸɛi]\textsubscript{final}\\
\textsc{dem:assoc}	\textsc{dem:abs}	put.on\textsc{:hypoth}\\
\glt \sqt{Then put it on there.}\\
\ex \label{Aiex:20h}
\gll \textbf{[ɛ	di-si]}\textsubscript{medial}	[ɛna	mɛgi	ɛna	adlɛ-li-si]\textsubscript{medial}	[taːlɛ=ta]\textsubscript{final}\\
be	\textsc{pfv-med:pfv}	\textsc{dem}	rope	\textsc{dem:abs}	tie.on\textsc{-sim-med:pfv}	finish\textsc{=atel}\\
\glt \sqt{That's done, and then tie that rope on there and finish.}\\
\ex \label{Aiex:20i}
\gll	[no-wa	la	wogu-mɛi]\textsubscript{final}\\
other\textsc{-abs}	\textsc{def}	do.thus\textsc{-hypoth}\\
\glt \sqt{Do the other one like that.}\\
\end{xlist}
\end{exe}

	The final line in (\ref{Aiex:20i}) describes a new series of events in the discourse and constitutes a separate and distinct stage in the construction of the head dress. Another detail of note in the extract is that the instances of recapitulation bridging at episode boundaries within the paragraph are not contiguous with the reference clause that they refer to. Instead the bridging clauses seems to precede a paraphrase of the immediately preceding clause. It is possible that the speaker is self-correcting to repeat a clause with the addition of a bridging clause referring to the preceding event for additional clarity.
	

	The concluding episode of a paragraph, such as (\ref{Aiex:20h}), is prototypically marked by a summary linkage clause utilizing the light verb \textit{ɛ}, which references the events of the entire paragraph. In some cases, summary linkage can introduce commentary of a much larger discourse unit such as an entire narrative. In (\ref{Aiex:19af}) a speaker is commenting on a story he has just completed which describes events from his childhood. He is explaining how he came to know the story and the lasting scar that resulted. In this example, the summary linkage clauses in (\ref{Aiex:19c}), (\ref{Aiex:19d}), and (\ref{Aiex:19e}) all reference the entire narrative and offer concluding remarks on the story. Bridging constructions are a way to signal a shift in an episode and perspective, while maintaining a clear sentential link between related episodes. 
	
\subsection{Temporal relations} 
\label{AiTemporal}
One of the most straight-forward functions of bridging constructions is to repeat the reference clause with the addition of a morpheme which specifies relative aspect. These morphemes specify the temporal relationships between the main clause and the bridging clause, and in so doing, specify the temporal relationship between two stretches of discourse. The first example is beginning a new clause chain with a bridging clause consisting of a medial clause using the perfective linker \textit{-si}, either specifying a completed perfective event, or in conjunction with the simultaneous action suffix \textit{-li}. When used to describe a completed perfective event, as in (\ref{Aiex:20h}), this represents an immediately preceding completed action followed by a subsequent action. When combined with the simultaneous event suffix \textit{-li}, the bridging clause specifies that the preceding event is still in progress when the following events in the clause chain occur, as in solalisi `peeling' in line (\ref{Aiex:App27}) of the appendix. When describing an ongoing state rather than a telic event, a bridging clause may present the enclitic \textit{=ta}, which specifies that the state continues during the following events of the following discourse episode, which is seen in \textit{taː doːtaː} `having crossed' in line (\ref{Aiex:App32}) of the appendix. A final example is the perfect aspect suffix \textit{-ɸɛija}, which specifies a completed event, the result of which is still relevant to the ensuing discourse, as seen prominently in the bridging clauses in (\ref{Aiex:19c}--\ref{Aiex:19e}).

\subsection{Causal relations} 
\label{AiCausal}
The consequential auxiliary \textit{bɛda} specifies a consequential relationship rather that a temporal one. In a bridging clause utilizing \textit{bɛda}, the events of the previous discourse episode are represented as the cause of the subsequent events. For example, in line (\ref{Aiex:App50}) of the appendix, the final event of the previous series of events, i.e., the setting of the sun, is presented as the event which initiates the following series of events, i.e., the decision to leave. Similarly, in line (\ref{Aiex:App41}) of the appendix the events preceding the reference clause, a failed attempt to kill a pig, is presented as the cause of the events following the bridging clause, i.e. another attempt to kill the pig. By making reference to previous discourse with summary linkage with the addition of a consequential auxiliary (or one of the relative aspect markers discussed above) the relevance of the reference clause, and the previous series of events, to the subsequent series of events is made explicit.
	
\subsection{Argument tracking} 
\label{Aitracking}
Another way that bridging constructions situate new information within an ongoing discourse is to specify the participants involved. The first construction serving this function is a bridging clause displaying a change in subject by means of the different subject morpheme \textit{-bi}. The usage of the different subject marker differs in function between main clauses and non-main clauses. In main clauses, an unexpected or non-topical subject will also necessitate the different subject marker, as in (\ref{Aiex:15a}) and (\ref{Aiex:15c}) where the different subject marker is used on the predicate of a main clause. In non-main clauses, the different subject marker specifies that the subject of the non-main clause differs from the following main clause. For example, in line (\ref{Aiex:App29a}) of the appendix the anaphoric form ɛbijaː also specifies a change in subject, from `the sun' in the preceding reference clause `the sun was setting' to the narrator in following clause `(I) finished peeling the \textit{owaːlo} bark'. The excessive and perhaps redundant switch reference marking in (\ref{Aiex:15ae}) may be a way of emphasizing the shift in participant reference and further clarifying the relevant arguments for each clause. In (\ref{Aiex:15ae}), for example, four different participants are referenced, which might contribute to confusion regarding the roles that each person or group in playing in the individual clauses.

\section{Summary} 
\label{AiSumm}
To conclude, bridging constructions in Eibela are formed through two syntactic clause-linking strategies, topicalization and clause chaining. These bridging constructions may be further described as either summary linkage, which utilized one of three different anaphoric verbs to form the bridging clause, or recapitulative linkage, which repeats the lexical material of the reference clause. Summary linkage using the verb \textit{wogu} `do thus' or the aspect-marking verb \textit{hɛna:} `continue doing' has definite reference to the immediately preceding reference clause, while the pro-verb \textit{ɛ} `do' makes indefinite reference to preceding discourse. Recapitulative linkage repeats elements of the reference clause as a non-main bridging clause, but may omit or substitute elements. 

Discourse organization is also shown to feature two levels of discourse which coincide with the usage of recapitulative linkage and summary linkage. Individual events form smaller units of discourse, here referred to generically as episodes, which may be combined with related events by means of bridging constructions to form larger units of discourse, here referred to as paragraphs. These two discourse units are formally distinguished in Eibela by using recapitulative linkage at episode boundaries to show that a subsequent episode is related to the previous episode, while summary linkage at the end of a series of related episodes may assert that a proposition is relevant to the entire series of episodes rather than only the immediately preceding event. A similar pattern may be found in the closely related language Kasua, which likewise favors the use of summary linkage at the beginning of a ``new thematic paragraph'' \citep[][24]{logan08}.

Bridging constructions may be found with similar form and function in other languages of Papua New Guinea, and the patterns observe in Eibela may represent a general regional trend. \citet[][]{Jendraschek09} observes that bridging constructions allow for switch reference marking between discourse units that would not otherwise be possible, and therefore contributes to reference tracking in the Iatmul language. He also observes that languages which feature prominent use of bridging constructions generally do not feature a native class of conjunctions, and that bridging constructions may be serving the same functional role of a conjunction in linking independent clauses. This follows from \citet[][367]{devries.2005} and \citet[][374--375]{longacre07}, which argue that languages of Papua New Guinea tend to avoid noun phrases and argument anaphors as a means of referent tracking, and instead rely on verbal morphology and switch reference marking in dependent (or cosubordinate) clauses. Bridging linkage may therefore be a general coordination strategy for those languages which feature rich verbal morphology, and a tendency to use fewer overt arguments in discourse.

Bridging constructions in Eibela provide varying ways of reiterating previous discourse before presenting new information. This can be viewed as form of topic setting, where a frame of reference is established by a bridging clause which then serves as the basis for subsequent events. The frame of reference defined by the bridging clause will therefore define the relevance of the following main clause. In the case of a medial clause functioning as a bridging clause, the frame of reference can be relevant to an entire clause chain. Bridging clauses formed by a topic clause, on the other hand, typically provide a frame of reference for a single following main clause. Finally, this topic setting role may be viewed as a means of assisting in reference tracking through verbal switch reference morphology, and coordinating independent clauses or clause chains in discourse where there is no native class of coordinating conjunctions.



\section*{Appendix}
 \setcounter{equation}{0}
This appendix provides an extended excerpt from a narrative told by Edijobi Hamaja, an adult
female speaker of Eibela who resides in Lake Campbell, while she describes a bush walk. Bridging
constructions are labeled throughout using the familiar notation of underlined text for reference clauses
and bold text for bridging clauses.

\begin{exe}
\ex \label{Aiex:App21}
\gll [[jaː-nɛː]\textsubscript{pred}]\textsubscript{medial}\\
come-\textsc{med}:\textsc{ipfv}\\
\glt \sqt{(I) came and…}\\
\end{exe}

\begin{exe}
\ex \label{Aiex:App22}	
\gll \underline{[[baːkɛlɛ}	\underline{duːna]}\textsubscript{o}	\underline{[dɛlaː]}\textsubscript{pred}]\textsubscript{final}\\
bush.turkey	nest:\textsc{abs}	dig:\textsc{pst}\\
‎\glt \sqt{(I) dug into a bush turkey nest.}\\
\end{exe}

\begin{exe}
\ex \label{Aiex:App23}
\gll \textbf{[[baːkɛlɛ]}\textsubscript{o}	\textbf{[dɛlaː]}\textsubscript{pred}]\textsubscript{final}	\textbf{[[hɛnaː}	\textbf{di-si]}\textsubscript{pred}]\textsubscript{medial}\\
bush.turkey	dig:\textsc{pst}		\textsc{dur}	\textsc{pfv}-\textsc{med}:\textsc{pfv}\\
‎‎\glt \sqt{(I) continued to digging into the bush turkey (nest) and then...’}\\
\end{exe}

\begin{exe}
\ex \label{Aiex:App24}
\gll [[tilaː]\textsubscript{pred}	\underline{\smash{[haːnaː]}}\textsubscript{o}	\underline{\smash{[muːduː]}}\textsubscript{pred}]\textsubscript{final}\\
descend	water:\textsc{abs}	wash:\textsc{pst}\\
\glt \sqt{(I) went down and washed.}\\
\end{exe}

\begin{exe}
\ex \label{Aiex:App25}
\gll \textbf{[[[haːnaː]}\textsubscript{o}	\textbf{[muːluː-wɛː]}\textsubscript{pred}]\textsubscript{x}	\textbf{[hɛnaː}	\textbf{di-si]}\textsubscript{pred}]\textsubscript{medial}\\
water:\textsc{abs}	wash-\textsc{loc}	\textsc{dur}	\textsc{pfv}-\textsc{med}:\textsc{pfv}\\
\glt \sqt{(I) finished washing and then…}\\
\end{exe}

\begin{exe}
\ex \label{Aiex:App26}	
\gll [[ɸiliː-nɛ:=jaː]\textsubscript{topic}	\underline{\smash{[owaːlo-waː]}}\textsubscript{o}	\underline{\smash{[solaː}}	\underline{\smash{di]}}\textsubscript{pred}]\textsubscript{final}\\
ascend-\textsc{pst}=\textsc{top}	tree.type-\textsc{abs}	peel.bark	\textsc{pfv}\\
\glt \sqt{(I) went up and peeled bark strips from an owaːlo tree.}\\
\end{exe}

	
\begin{exe}
\ex \label{Aiex:App27}	
\gll \textbf{[[owaːlo]}\textsubscript{o}	\textbf{[solaː-liː-si]}\textsubscript{pred}]\textsubscript{medial}\\
tree.type	peel.bark-\textsc{sim}-\textsc{med}:\textsc{pfv}\\
‎‎\glt \sqt{While was peeling bark off a owaːlo tree…}\\
\end{exe}

\begin{exe}
\ex \label{Aiex:App28}
\gll [[bɛdaː-loːlu=waː]\textsubscript{topic}	\underline{\smash{[ɛimɛ]}}\textsubscript{x}	\underline{\smash{[oːɸaː]}}\textsubscript{s}	\underline{\smash{[aːnɛː]}}\textsubscript{pred}]\textsubscript{final}\\
see-\textsc{ass.ev}=\textsc{top}	already	sun:\textsc{abs}	go:\textsc{pst}\\
\glt \sqt{I saw that the sun was already setting.}\\
\end{exe}

\begin{exe}

\ex \label{Aiex:App29ab}
\begin{xlist}
\ex \label{Aiex:App29a}
\gll \textbf{[ɛ-biː=jaː]}\textsubscript{topic}	[[owaːlo-waː]\textsubscript{o}	[solaː	hɛnɛ	di-si=jaː]\textsubscript{pred}]\textsubscript{topic}\\
do-\textsc{ds}=\textsc{top}	tree.type-\textsc{abs}	peel.bark	\textsc{dur}	\textsc{pfv}-\textsc{med}:\textsc{pfv}=\textsc{top}\\
\glt \sqt{It was doing that, so (I) finished peeling the owaːlo bark and then…}\\
\ex \label{Aiex:App29b}
\gll [[hɛnaː]\textsubscript{pred}	[toːɡolɛː]\textsubscript{x}	[ɛːsaː	kaː]\textsubscript{o}	[oːɡɛː	di]\textsubscript{pred}]\textsubscript{final}\\
go	road:\textsc{loc}	bilum:\textsc{abs}	\textsc{foc}	carry.bilum	\textsc{pfv}\\
\glt \sqt{(I) went to the road and picked up my bilum (string bag).}\\
\end{xlist}
\end{exe}

\begin{exe}
\ex \label{Aiex:App30}
\gll [[\textit{[oːkɛ]}\textsubscript{x}	[dijaː	ti-nɛː=jaː]\textsubscript{pred}]\textsubscript{topic}	[jaː-nɛː]\textsubscript{pred}]\textsubscript{medial}\\
okay	hold	descend-\textsc{pst}=\textsc{top}	\textsc{dir:ven}-\textsc{med}:\textsc{ipfv}\\
\glt \sqt{(I) was coming down carrying (the bilum) and…}\\
\end{exe}

\begin{exe}
\ex \label{Aiex:App31}
\gll \underline{[[oːlonaː]}\textsubscript{o}	\underline{[taː-nɛː]}\textsubscript{pred}]\textsubscript{final}\\
\textsc{name}	cross-\textsc{pst}\\
‎‎\glt \sqt{I crossed the Oːlonaː.}\\
\end{exe}

\begin{exe}
\ex \label{Aiex:App32}
\gll \textbf{[[[oːlonaː]}\textsubscript{o}	\textbf{[taː}	\textbf{doː-taː]}\textsubscript{pred}]\textsubscript{x}	[noːloː	hoːnoː]\textsubscript{pred}]\textsubscript{final}\\
\textsc{name}	cross-\textsc{tel}	\textsc{stat}-\textsc{tel}	other.side	\textsc{dem}:\textsc{lvl}\\
\glt \sqt{‎‎I was on that other side having crossed the Oːlonaː.}\\
\end{exe}

\begin{exe}
\ex \label{Aiex:App33}
\gll [[hɛnaːː]\textsubscript{pred}]\textsubscript{medial}\\
go:\textsc{dur}\\
\glt \sqt{We were going and…}\\
\end{exe}

\begin{exe}
\ex \label{Aiex:App34}
\gll [[jɛː-si	dɛnɛ	baːlɛ]\textsubscript{x}	\underline{\smash{[kɛː-jaː}}	\underline{\smash{kaː]}}\textsubscript{o}	\underline{\smash{[hoːdɛ-si]}}\textsubscript{pred}]\textsubscript{medial}\\
come-\textsc{pl}	\textsc{prog}	\textsc{coord}	pig-\textsc{abs}	\textsc{foc}	bark-\textsc{med}:\textsc{pfv}\\
\glt \sqt{While we were coming, (the dogs) were barking at a pig and then…}\\
\end{exe}

\begin{exe}
\ex \label{Aiex:App35}
\gll \textbf{[[[kɛː-jaː]}\textsubscript{o}	\textbf{[hoːdɛ-bi=jaː]}\textsubscript{pred}]\textsubscript{topic}	[kaliːjaː]\textsubscript{s}	[ɛ-taː]\textsubscript{pred}]\textsubscript{final}\\
pig-\textsc{abs}	bark-\textsc{ds}=\textsc{top}	wallaby	do-\textsc{tel}\\
‎\glt \sqt{We thought the dogs barking at a pig was (actually) a wallaby.}\\
\end{exe}

\begin{exe}
\ex \label{Aiex:App36}
\gll [[hɛnɛ-si	dɛnɛ	baːlɛ]\textsubscript{x}	[kɛː	kaː]\textsubscript{o}	[hoːdɛ=jaː	laː-biː=jaː]\textsubscript{pred}]\textsubscript{topic}	[kaː]\textsubscript{pred}\\
go-\textsc{pl}	\textsc{prog}	\textsc{coord}	pig	\textsc{foc}	bark=\textsc{top}	exist-\textsc{ds}=\textsc{top}	\textsc{foc}\\
‎\glt \sqt{While we were going the dogs were there barking at a pig.}\\
\end{exe}

\begin{exe}
\ex \label{Aiex:App37}
\gll [[[kɛː	ɛnaː]\textsubscript{o}	\underline{\smash{[soboː.oːnoː-kɛi]}}\textsubscript{x}	\underline{\smash{[sɛbɛːnaː-taː]}}\textsubscript{pred}]\textsubscript{x}	\underline{\smash{[kaː}}	\underline{\smash{hɛnɛ-saː]}}\textsubscript{pred}]\textsubscript{final}\\
pig	that:\textsc{abs}	ax-\textsc{inst}	hit:\textsc{n.sg.a}:\textsc{purp}-\textsc{tel}	\textsc{foc}	go-\textsc{pl}:\textsc{pst}\\
‎\glt \sqt{We went to hit that pig with an ax anyway.}\\
\end{exe}

\begin{exe}
\ex \label{Aiex:App38}
\gll \textbf{[[soboː.oːnoː-kɛi]}\textsubscript{x}	\textbf{[sɛdaː-loːlu]}\textsubscript{pred}]\textsubscript{medial}\\
ax-\textsc{inst}	hit:\textsc{n.sg.a}-\textsc{ass.ev}\\
\glt \sqt{In hitting it with the ax…}\\
\end{exe}

\begin{exe}
\ex \label{Aiex:App39}
\gll \underline{\smash{[[moɡaːɡɛ-li}}	\underline{\smash{sɛdɛ-si]}}\textsubscript{pred}]\textsubscript{medial}\\
bad-\textsc{sim}	hit:\textsc{n.sg.a}-\textsc{med}:\textsc{pfv}\\
\glt \sqt{We hit it badly and then…}\\
\end{exe}

\begin{exe}
\ex \label{Aiex:App40}
\gll [[ɸoːsɛː	kiː-jɛː]\textsubscript{pred}]\textsubscript{final}\\
back:\textsc{loc}	bone-\textsc{loc}\\
\glt \sqt{(It was) on the backbone (that we hit it).}\\
\end{exe}

\begin{exe}
\ex \label{Aiex:App41}
\gll \textbf{[[ɛ=bɛdaː-nɛː]}\textsubscript{pred}]\textsubscript{medial}\\
do=\textsc{cons}-\textsc{med}:\textsc{ipfv}\\
\glt \sqt{We did that so…}\\
\end{exe}

\begin{exe}
\ex \label{Aiex:App42}
\gll [[mi-jɛː=jaː]\textsubscript{topic}	[soːboː-kɛi]\textsubscript{x}	[jaː	doː-si]\textsubscript{pred}]\textsubscript{medial}\\
come-\textsc{pst}=\textsc{top}	knife-\textsc{inst}	\textsc{dir:ven}	\textsc{stat}-\textsc{med}:\textsc{pfv}\\
\glt \sqt{I came there with the knife, and then}\\
\end{exe}

\begin{exe}
\ex \label{Aiex:App43}
\gll \underline{[[kɛː}	\underline{ɛnaː]\textsubscript{o}}	\underline{[kaː}	\underline{oːlaː]}\textsubscript{pred}]\textsubscript{final}\\
pig	\textsc{dem}:\textsc{abs}	\textsc{foc}	shoot:\textsc{pst}\\
\glt \sqt{I stabbed the pig.}\\
\end{exe}

\begin{exe}
\ex \label{Aiex:App44}
\gll \textbf{[[lɛ}	\textbf{hɛnaː]}\textsubscript{pred}]\textsubscript{medial}\\
do	\textsc{dur}\\
\glt \sqt{I did that then…}\\
\end{exe}

\begin{exe}
\ex \label{Aiex:App45}
\gll \underline{\smash{[[kɛː-jaː]}}\textsubscript{s}	\underline{\smash{[kaː}}	\underline{\smash{ɡuːduː-saː-bi]}}\textsubscript{pred}]\textsubscript{final}\\
pig-\textsc{abs}	\textsc{foc}	die-3:\textsc{dr}-\textsc{ds}\\
\glt \sqt{that pig died.}\\
\end{exe}

\begin{exe}
\ex \label{Aiex:App46}
\gll \textbf{[[kɛː-jaː]}\textsubscript{s}	\textbf{[ɡuːduː}	\textbf{hɛnaː}	\textbf{doː-si]}\textsubscript{pred}]\textsubscript{medial}\\
pig-\textsc{abs}	die	go	\textsc{stat}-\textsc{med}:\textsc{pfv}\\
\glt \sqt{The pig had died, and then…}\\
\end{exe}

\begin{exe}
\ex \label{Aiex:App47}
\gll [[joːlaː]\textsubscript{pred}]\textsubscript{final}\\
butcher:\textsc{pst}\\
\glt \sqt{(We) butchered (it).}\\
\end{exe}

\begin{exe}
\ex \label{Aiex:App48}
\gll [[hɛnaː-nɛː]\textsubscript{pred}]\textsubscript{medial}\\
go-\textsc{med}:\textsc{ipfv}\\
\glt \sqt{We went and…}\\
\end{exe}

\begin{exe}
\ex \label{Aiex:App49}
\gll [[haːnɛ	sɛː=jaː]\textsubscript{topic}	\underline{\smash{[kaː}}	\underline{soːloː}	\underline{\smash{di]}}\textsubscript{pred}]\textsubscript{final}\\
river	beach=\textsc{top}	\textsc{foc}	darken	\textsc{pfv}\\
\glt \sqt{It got dark, at the riverside.}\\
\end{exe}

\begin{exe}
\ex \label{Aiex:App50}
\gll [\textbf{[[soːlo}	\textbf{di=jaː]}\textsubscript{pred}]\textsubscript{topic}	[bɛdaː=nɛgɛː]\textsubscript{pred}]\textsubscript{medial}\\
become.dark	\textsc{pfv}=\textsc{top}	\textsc{cons}-\textsc{med}:\textsc{ipfv}\\
\glt \sqt{It had gotten dark, so…}\\
\end{exe}

\begin{exe}
\ex \label{Aiex:App51}
\gll \underline{\smash{[[kaː}}	\underline{\smash{taː=nɛgɛː]}}\textsubscript{pred}]\textsubscript{final}\\
\textsc{foc}	cross-\textsc{med:ipfv}\\
\glt \sqt{We still crossed.}\\
\end{exe}

\begin{exe}
\ex \label{Aiex:App52}
\gll \textbf{[[haːnɛ}	\textbf{waːwi-jaː]}\textsubscript{o}	\textbf{[kaː}	\textbf{taːlɛ-si]}\textsubscript{pred}]\textsubscript{medial}\\
river	name-\textsc{abs}	\textsc{foc}	cross-\textsc{med}:\textsc{pfv}\\
\glt \sqt{We crossed the Waːwi river and then…}\\
\end{exe}

\section*{ Abbreviations}

\textsc{:} portmanteau,
\textsc{-} affix boundary,
\textsc{=} clitic boundary,
\textsc{1} 1st person,
\textsc{2} 2nd person,
\textsc{3} 3rd person,
\textsc{a} transitive subject,
\textsc{abs} absolutive,
\textsc{ass.ev} associated event,
\textsc{asser} assertion,
\textsc{assoc} associative,
\textsc{atel} atelic,
\textsc{comp} complement clause,
\textsc{compl} completive,
\textsc{cons} consequence,
\textsc{cont} continuous/continuative,
\textsc{contr} contrastive,
\textsc{coord} coordinator,
\textsc{def} definite,
\textsc{dem} demonstrative,
\textsc{dir} directional,
\textsc{ds} different subject,
\textsc{dr} direct,
\textsc{dur} durative,
\textsc{erg} ergative,
\textsc{foc} focus,
\textsc{fut} future,
\textsc{hypoth} hypothetical,
\textsc{ideo} ideophone,
\textsc{imp} imperative,
\textsc{indf} indefinite,
\textsc{infer} inferred,
\textsc{ins} instrumental,
\textsc{inter} interrogative,
\textsc{ipfv} imperfective,
\textsc{irr} irrealis,
\textsc{iter} iterative,
\textsc{loc} locative,
\textsc{lvl} same elevation,
\textsc{med} medial,
\textsc{n} not,
\textsc{neg} negation,
\textsc{nmlz} nominaliser,
\textsc{non} non,
\textsc{p} patient,
\textsc{pfv} perfective,
\textsc{pl} plural,
\textsc{pred} predicative,
\textsc{prf} perfect,
\textsc{prog} progressive,
\textsc{prs} present,
\textsc{pst} past,
\textsc{purp} purposive,
\textsc{sg} singular,
\textsc{sim} simultaneous,
\textsc{stat} stative,
\textsc{top} topic,
\textsc{up} higher elevation,
\textsc{ven} venitive

\section*{Acknowledgements}
The author wishes to thank the participants of a two-day workshop entitled ``Bridging Linkage in Cross-linguistic Perspective'' that was organized at the Cairns Institute (James Cook University, Australia), on 25--26 February 2015 by Valérie Guérin and Simon Overall. Without the organization of this workshop and the insightful comments of its participants, this chapter would not have been written. Additional comments from two anonymous reviewers and the editor, Valérie Guérin, were greatly appreciated and substantially improved the quality of this publication.

%%%%%%%%%%%%%%%%%%
%%%%%%%%%%%%%%%%%%
%%%%%%%%%%%%%%%%%%% a line with a % sign before is not shown when compiled- we can leave notes 
%%%%%%%%%%%%%%%%%%
%%%%%%%%%%%%%%%%%%% to start, go to your word document, separate out all examples on a new doc, remove all formatting so you get only plain text utf 8 [no bold, no caps, etc]. then copy the words into the slots and change the commands as you see fit. a command starts with \ and its domain is inside {}.
%%%%%%%%%%%%%%%%%%
%%%%%%%%%%%%%%%%%%\begin{exe}
%%%%%%%%%%%%%%%%%%\ex \label{Aiex01}
%%%%%%%%%%%%%%%%%%\gll [agɛ 	ɸɛɸɛ-jaː]\textsubscript{s} [ɛna]\textsubscript{x} [dobosuwɛ]\textsubscript{x} [tɛ 	aːnɛ]\textsubscript{pred}\\
%%%%%%%%%%%%%%%%%%dog	skinny-\textsc{abs}	there	underneath	go.down 	go\textsc{:pst}\\
%%%%%%%%%%%%%%%%%%\glt \sqt{The skinny dog went down underneath there.}\\
%%%%%%%%%%%%%%%%%%\end{exe}
%%%%%%%%%%%%%%%%%%%with this type of example, i cannot use subscript and small caps or caps :-(
%%%%%%%%%%%%%%%%%%
%%%%%%%%%%%%%%%%%%
%%%%%%%%%%%%%%%%%%%the \label{Aixxx} is what you will call in the body of the text. ....As Aiton has shown in (\ref{Aiex01}), the skinny dog....i suggest you keep the same labels as  the numbers in the text, so it is easy to refer to the examples while writing in latex and following the text in your chapter. so in your chapter, your current example (10a) should be \label{Aiex10a}
%%%%%%%%%%%%%%%%%%
%%%%%%%%%%%%%%%%%%%\textbf{tab.tom} = bold; \underline{nyob} = underline; \textsc{sg}= small caps;\textit{you}= italics; \\ at the end of a line = go to the next line; \sqt{} = add quotes; \gll = source langauge; \glt = english translation; () and [] as is
%%%%%%%%%%%%%%%%%%
%%%%%%%%%%%%%%%%%%

%%%%%%%%%%%%%%%%%%
%%%%%%%%%%%%%%%%%%
%%%%%%%%%%%%%%%%%%\begin{exe}
%%%%%%%%%%%%%%%%%%\ex \label{Aiex:10ab}
%%%%%%%%%%%%%%%%%%\begin{xlist}
%%%%%%%%%%%%%%%%%%\ex \label{Aiex:10a}
%%%%%%%%%%%%%%%%%%\gll thiab  \underline{tau} \underline{nyob} \underline{tos},\\
%%%%%%%%%%%%%%%%%%and \textsc{pfv} stay wait\\
%%%%%%%%%%%%%%%%%%\glt \sqt{… and (he) stayed (there and) waited,}\\
%%%%%%%%%%%%%%%%%%\ex \label{Aiex:10b}
%%%%%%%%%%%%%%%%%%\gll \textbf{thaum} \textbf{nws} \textbf{tab.tom} \textbf{mus} \textbf{nyob} \textbf{tos} ces …\\     	      
%%%%%%%%%%%%%%%%%%     when 3\textsc{sg} just go stay wait and.then\\
%%%%%%%%%%%%%%%%%%\glt \sqt{(and) when he had just gone to stay (there) and wait, then...} 
%%%%%%%%%%%%%%%%%%\end{xlist}
%%%%%%%%%%%%%%%%%%\end{exe}
%%%%%%%%%%%%%%%%%%
%%%%%%%%%%%%%%%%%%
%%%%%%%%%%%%%%%%%%\begin{exe}
%%%%%%%%%%%%%%%%%%\ex \label{Aiex:2ac}
%%%%%%%%%%%%%%%%%%\begin{xlist}
%%%%%%%%%%%%%%%%%%\ex \label{Aiex:2a}
%%%%%%%%%%%%%%%%%%\gll \underline{Lub} \underline{sij.hawm} \underline{ntawm} \underline{neeg} \underline{khiav} \underline{coob}     \underline{heev} \underline{mas}.        \\
%%%%%%%%%%%%%%%%%%\textsc{clf} time that person run be.many very \textsc{ip}\\
%%%%%%%%%%%%%%%%%%\glt \sqt{(At) that time there were very many people fleeing.}\\
%%%%%%%%%%%%%%%%%%\ex \label{Aiex:2b}
%%%%%%%%%%%%%%%%%%\gll   \textbf{Yog} \textbf{li} \textbf{ntawd},\\
%%%%%%%%%%%%%%%%%%\textsc{cop} like that\\
%%%%%%%%%%%%%%%%%%\glt \sqt{that being the case,}\\
%%%%%%%%%%%%%%%%%%\ex \label{Aiex:2c}
%%%%%%%%%%%%%%%%%%\gll lawv  thiaj hais tias ua  peb puas yog neeg nyob nram   tiag.\\     	      
%%%%%%%%%%%%%%%%%%     3\textsc{pl} then say \textsc{comp} do 1\textsc{pl} \textsc{q} \textsc{cop} person live place.down level.place\\
%%%%%%%%%%%%%%%%%%\glt \sqt{they then asked whether we were people (who) lived down (in Vientiane)}
%%%%%%%%%%%%%%%%%%\end{xlist}
%%%%%%%%%%%%%%%%%%\end{exe}
%%%%%%%%%%%%%%%%%%
%%%%%%%%%%%%%%%%%%






\printbibliography[heading=subbibliography,notkeyword=this] 
\end{document}
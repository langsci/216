 

% \newcommand{\noop}[1]{}
\newcommand{\sqt}[1]{`#1'}
\newcommand{\refex}[1]{(\ref{#1})}
\newcommand{\reftab}[1]{\tabref{#1}}
\newcommand{\refsec}[1]{\sectref{#1}}
\newcommand{\dqt}[1]{``#1''}

%VALERIE: code to make example in italics
\renewcommand{\exfont}{\itshape}
%---------------------------------------

%VALERIE: code to properly show Angela's examples

%\newcounter{TranscriptLineCounter}[xnumi] % Line counter resets when there is a new gb4e environment.
%
%\newenvironment{Transcript}[2]{%
%\let\oldexfont\exfont% save current exfont
%\renewcommand{\exfont}{\upshape}
%\ea
%\stepcounter{TranscriptLineCounter}%Start with 1, not 0
%\begin{tabbing}
%00 \= #1: \= \hspace{1em} \= #2 \kill}
%{\end{tabbing}
%\z
%\let\exfont\oldexfont% reset exfont to previous user setting
%}
%
%\newcommand{\TLine}[3][]{%
%\ifnum\value{TranscriptLineCounter}<10 0\fi\arabic{TranscriptLineCounter} \> \def\temp{#2}\ifx\temp\empty\else#2:\fi \> #1 \> \begin{minipage}[t]{\linewidth}#3\end{minipage}%
%\stepcounter{TranscriptLineCounter}%
%}
%---------------------------------------

%VALERIE: REVISED code to properly show Angela's examples

\newcounter{TranscriptLineCounter}[xnumi] % Line counter resets when there is a new gb4e environment.
\newlength{\TranscriptGlossLength}% Lengths to determine the width of the last column
\newlength{\TCollectedLength}% A temp length that collects variable widths

\newenvironment{Transcript}[3][{>}{>}]{%
\let\oldexfont\exfont% save current exfont
\renewcommand{\exfont}{\upshape}
\ea
\settowidth{\TCollectedLength}{00 #2: #1 }
\setlength{\TranscriptGlossLength}{\linewidth - \TCollectedLength}
\stepcounter{TranscriptLineCounter}%Start with 1, not 0
\begin{tabbing}
00 \= #2: \= #1 \= #3 \kill}
{\end{tabbing}
\z
\let\exfont\oldexfont% reset exfont to previous user setting
}

\newcommand{\TLine}[3][]{%
\ifnum\value{TranscriptLineCounter}<10 0\fi\arabic{TranscriptLineCounter} \> \def\temp{#2}\ifx\temp\empty\else#2:\fi \> #1 \> \begin{minipage}[t]{\TranscriptGlossLength}#3\end{minipage}%
\stepcounter{TranscriptLineCounter}%
}
%---------------------------------------
%valerie
\newcommand{\faVenus}{♀} 
\newcommand{\faMars}{♂}

\newcommand{\aREF}[1]{(A\ref{#1})}

\makeatletter
\let\theauthor\@author %store in more accessible format
\makeatother 
 
\newcommand{\lsFirstAuthorFullName}{}%temporary, will be overwritten
\newcommand{\lsFirstAuthorFirstName}{}%temporary, will be overwritten
\newcommand{\lsFirstAuthorLastName}{}%temporary, will be overwritten
\newcommand{\lsFirstAuthorString}{\lsFirstAuthorLastName, \lsFirstAuthorFirstName} %can be customized in localmetadata.tex
\newcommand{\lsNonFirstAuthorsString}{}  %default, will be overwritten iff more than one author
\newcommand{\lsImpressionCitationAuthor}{\lsFirstAuthorString \lsNonFirstAuthorsString}


\renewcommand{\and}{NONLASTAND} %expand for easier checking. Might need to be undone later on
\renewcommand{\lastand}{LASTAND} %expand for easier checking

\IfSubStr{\theauthor}{NONLASTAND}{%2+authors
  \renewcommand{\lsFirstAuthorFullName}{\StrBefore{\theauthor}{\and }}
  \renewcommand{\lsFirstAuthorFirstName}{\StrBefore{\theauthor}{ }} 
  \renewcommand{\lsFirstAuthorLastName}{\StrBetween{\theauthor}{ }{\and }}
  \renewcommand{\lsNonFirstAuthorsString}{\and\StrBehind{\theauthor}{\and }} 
  }{%else
    \IfSubStr{\theauthor}{LASTAND}{%less than two authors, more than one
    \renewcommand{\lsFirstAuthorFullName}{\StrBefore{\theauthor}{\lastand }}
    \renewcommand{\lsFirstAuthorFirstName}{\StrBefore{\theauthor}{ }}
    \renewcommand{\lsFirstAuthorLastName}{\StrBetween{\theauthor}{ }{\lastand }}
    \renewcommand{\lsNonFirstAuthorsString}{\lastand\StrBehind{\theauthor}{\lastand }} 
    }{%else exactly one author
      \renewcommand{\lsFirstAuthorFirstName}{\StrBefore{\theauthor}{ }}
      \renewcommand{\lsFirstAuthorLastName}{\StrBehind{\theauthor}{ }}
      }
    }   

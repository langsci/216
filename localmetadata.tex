%%%%%%%%%%%%%%%%%%%%%%%%%%%%%%%%%%%%%%%%%%%%%%%%%%%%
%%%                                              %%%
%%%                 Metadata                     %%%
%%%          fill in as appropriate              %%%
%%%                                              %%%
%%%%%%%%%%%%%%%%%%%%%%%%%%%%%%%%%%%%%%%%%%%%%%%%%%%%

\title{Bridging constructions}  %look no further, you can change those things right here.
%\subtitle{A cross-linguistic study of \...}
% \BackTitle{} % Change if BackTitle != Title
\BackBody{Many descriptive grammars report the use of a linguistic pattern at the interface between discourse and syntax which is known generally as \textit{tail-head linkage}. This volume takes an unprecedented look at this type of linkage across languages and shows that there exist three distinct variants, all subsumed under the hypernym \textit{bridging constructions}. The chapters highlight the defining features of these constructions in the grammar and their functional properties in discourse. The volume reveals that:
\begin{itemize}
\item Bridging constructions consist of two clauses: a reference clause and a bridging clause. Across languages, bridging clauses can be subordinated clauses, reduced main clauses, or main clauses with continuation prosody.
\item Bridging constructions have three variants: \textit{recapitulative linkage}, \textit{summary linkage} and \textit{mixed linkage}. They differ in the formal makeup of the bridging clause. 
\item In discourse, the functions that bridging constructions fulfil depend on the text genres in which they appear and their position in the text. 
\item If a language uses more than one type of bridging construction, then each type has a distinct discourse function.
\item Bridging constructions can be optional and purely stylistic or mandatory and serve a grammatical purpose.
\item Although the difference between bridging constructions and clause repetition can be subtle, they maintain their own distinctive characteristics. 
\end{itemize}}
%\dedication{Change dedication in localmetadata.tex}
%\typesetter{Change typesetter in localmetadata.tex}
%\proofreader{Change proofreaders in localmetadata.tex}
\author{Valérie Guérin} %use this field for the volume editors

\renewcommand{\lsISBNdigital}{978-0-000000-00-0}                     
\renewcommand{\lsSeries}{sidl}  
\renewcommand{\lsSeriesNumber}{}
\renewcommand{\lsID}{216}
